\chapter*{Синтаксические и~семантические свойства клитики~\textit{=и} в~шугнанском языке}
\addcontentsline{toc}{chapter}{\textit{Д.~Чистякова}. \textbf{Синтаксические и~семантические свойства клитики \textit{=и} в~шугнанском языке}}
\setcounter{section}{0}
\chaptermark{Синтаксические и~семантические свойства клитики~\textit{=и}…}
\label{chapter-chist-clitic}

\begin{customauthorname}
Дарья Чистякова
\end{customauthorname}

\begin{englishtitle}
\i{Syntactic and semantic features of the clitic =\textit{i} in Shughni\\{\small Daria Chistiakova}}
\end{englishtitle}

\begin{abstract}
В статье на материале корпусного исследования и элицитированных примеров описывается факультативный энклитический показатель 3-го лица единственного числа =\i{и} в шугнанском языке (памирские языки ‹~восточноиранская группа иранских языков), появляющийся при переходных глаголах в прошедших временах. В большинстве случаев этот показатель занимает ваккернагелевскую позицию после первой полной составляющей, однако может занимать и более дальнюю от начала клаузы позицию в зависимости от действия ритмико-синтаксических барьеров. В статье описаны два барьера, влияющие на позицию энклитики. Первый тип барьера (подлежащее в первой позиции) обусловлен синтаксическими факторами, действует только на показатель {\sc 3sg} =\i{и} и постепенно исчезает в современном языке. Второй тип барьера (обстоятельство в первой позиции) зависит от коммуникативной ситуации и соблюдается факультативно как в текстах первой половины XX~века, так и в современном языке. Также описываются взаимодействие барьеров при одновременном появлении и частотность барьеров как в современном языке, так и в текстах 1915–1949~годов.
\end{abstract}

\begin{keywords}
энклитики, ритмико-синтаксические барьеры, шугнанский язык, памирские языки, ваккернагелевская позиция.
\end{keywords}

\vfill

\begin{eng-abstract}
This article describes the optional 3\textsuperscript{rd} person singular enclitic =\i{i} in Shughni (the Pamir group of the Eastern branch of the Iranian languages), which appears with transitive verbs in the past tenses. The study is based on a corpus study as well as elicited examples. Generally, the enclitic =\i{i} occupies the Wackernagel position after the first constituent, but can shift due to rhythmic-syntactic barriers. Two barriers that affect the position of the enclitic are described. The first type of barrier (the subject in the first position) is caused by syntactic factors, it affects only on the enclitic 3Sg =\i{i} and is gradually disappearing in the modern language. The second type of barrier (the adverbial in the first position) depends on the communicative situation and is observed optionally both in the texts of the first half of the XX\textsuperscript{th} century and in the modern language. This study also describes the interaction of barriers, if they appear simultaneously, and the frequency of barriers both in the modern language and in the texts of 1915–1949.
\end{eng-abstract}

\begin{eng-keywords}
enclitics, rhythmic-syntactic barriers, Shughni, Pamir languages, Wackernagel position.
\end{eng-keywords}

\begin{acknowledgements}
Я бы хотела поблагодарить за помощь в работе над статьей моего научного руководителя Романа Витальевича Ронько, руководителей нашей памирской экспедиции Владимира Александровича Плунгяна и Екатерину Владимировну Рахилину, а также Степана Михайлова за его ценные советы, Александра Сергиенко, собравшего для меня часть данных, и анонимных рецензентов за комментарии к статье. Публикация подготовлена в результате проведения исследования (проекта №~22-00-034) в рамках Программы «Научный фонд Национального исследовательского университета “Высшая школа экономики” (НИУ~ВШЭ)» в 2022~г.
\end{acknowledgements}

\begin{initialprint}
\fullcite{chistiakova2022_clitic}\end{initialprint}

\vfill

\section{Введение} \label{clit-intro}

Настоящая статья посвящена позиции подвижного показателя 3-го лица единственного числа в шугнанском языке. Лицо и число глагола в шугнанском языке в прошедших временах выражаются при помощи подвижных аналитических показателей, которые являются ваккернагелевскими энклитиками и занимают вторую позицию после первой полной составляющей. Единственное исключение составляет показатель 3-го лица единственного числа, чья позиция в предложении может варьироваться от второй до четвёртой.

Шугнанский язык не имеет письменной традиции, в данной работе мы обращаемся к транскрипции, используемой И.~И.~Зарубиным в текстах, записанных им в 1915–1927~годах\fn{В настоящем переиздании транскрипция была заменена на кириллическую (см.~\hyperref[chapter-intro]{введение}) — \i{прим.~переиздания}.}. Базовый порядок слов в шугнанском языке — SOV.

В существующих описаниях шугнанского языка говорится, что отделимые показатели лица и числа в прошедших временах примыкают энклитически к «первому ударному члену (или блоку) предложения» \parencite[237]{edelman_yusufbekov1999_shughni}, в другой работе \parencite[799]{edelman_dodykhudoeva2009_shughni} уточняется, что энклитики, как правило, присоединяются к первой составляющей клаузы. Иными словами, энклитика присоединяется не после первого ударного слова (строгий закон Ваккернагеля), а после первой полной составляющей (нестрогий закон Ваккернагеля). Ранее в восточноиранской группе языков клитики второй позиции, следующие нестрогому закону Ваккернагеля, были зафиксированы, например, для пушту \parencite[82]{tegey1977} и осетинского языков \parencite[157]{abaev1959}.

Уточняя правило выбора позиции, следует сказать, что в шугнанском языке энклитики не могут разрывать именные группы, но могут присоединяться к первой (именной) составляющей конструкций с лёгкими глаголами. Чтобы было удобнее ориентироваться в примерах, все составляющие, предшествующие энклитике, оформлены квадратными скобками.

\ex<exclit1>
\begingl
\gla {[}инҷу̊м{]}=\b{и} чӯд//
\glc вещи={\sc 3sg} делать.{\sc pst}//
\glft ‘всё подготовил’ \trailingcitation{\parencite[20]{zarubin1960}}//
\endgl \xe

\ex<exclit2>
\begingl
\gla {[}салу̊м{]}=\b{и} чӯд//
\glc приветствие={\sc 3sg} делать.{\sc pst}//
\glft ‘он поздоровался’ \trailingcitation{\parencite[54]{zarubin1960}}//
\endgl \xe

В шугнанском языке нет ограничений на часть речи опорного слова для клитики: она может присоединяться к именам, местоимениям, наречиям, глаголам и послелогам, замыкающим именную группу.

Если опорное слово оканчивается на гласный, перед энклитикой появляется протетический согласный /\i{й}/:

\ex<exclit3>
\begingl
\gla {[}Wи ғидорā{]}=\b{йи} зох̌т ху-рд.//
\glc {\sc d3.m.sg.o} кувшин={\sc 3sg} брать.{\sc pst} {\sc refl-dat}//
\glft ‘Тот кувшин взял себе.’ \trailingcitation{\parencite[72]{zarubin1960}}//
\endgl \xe

\ex<exclit4>
\begingl
\gla {[}тамошо{]}=\b{йен} чӯд//
\glc наблюдение={\sc 3pl} делать.{\sc pst}//
\glft ‘посмотрели’ \trailingcitation{\parencite[59–60]{zarubin1960}}//
\endgl \xe

Д.~К.~Карамшоев выделяет в шугнанском языке баджувский диалект и шахдаринский говор (см.~\parencite[5]{karamshoev1988}). В грамматике баджувского диалекта \parencite[152]{karamshoev1963} отмечается различие в распределении между показателем 3-го лица единственного числа =\i{и} и другими аналитическими показателями лица и числа. В баджувском диалекте показатель {\sc 3sg} не может присоединяться к первой составляющей, если она выражена подлежащим\fn{«Показатель -\i{и} не может присоединиться к первому члену-подлежащему, в то время как остальные показатели всегда присоединяются к нему» \parencite[152]{karamshoev1963}.}. Показатель {\sc 3sg} обладает большей подвижностью и присоединяется к дополнению или к сказуемому, в то время как остальные показатели чаще присоединяются к первой составляющей, чаще всего выраженной подлежащим (так как базовый порядок слов — SOV). Можно сказать, что в баджувском диалекте действует правило барьера, запрещающее показателю {\sc 3sg} присоединяться к подлежащему.

Пример стандартной позиции лично-числового показателя\fn{В примере (\gethref{exclit6}) использована форма \i{биδовд}, которая в словаре Карамшоева [\cite*{karamshoev1988}] обозначена как форма Презенса 3-го~лица единственного числа женского рода, но предложение переведено в прошедшем времени (в источнике: \i{худ ба худ пӯшида шуданд}); возможно, это ошибка автора грамматики, Тупчи~Бахтибекова — \i{прим.~переиздания}.}:

\ex<exclit5>
\begingl
\gla {[}Wуз{]}=\b{ум} wи ба-дил на-чӯд.//
\glc {\sc pron.1sg=1sg} {\sc d3.m.sg.o} {\sc all}-сердце {\sc neg}-делать.{\sc pst}//
\glft ‘А я не сделал по-его.’ \trailingcitation{\parencite[76]{zarubin1960}}//
\endgl \xe

\ex<exclit6>
\begingl
\gla {[}Wи цем-ен{]}=\b{ен} худ ба худ-аθ биδовд.//
\glc {\sc d3.m.sg.o} глаз-{\sc pl=3pl} сам {\sc all} сам-{\sc adv} закрыться.{\sc prs.f.3sg}//
\glft ‘Его глаза сами по себе закрылись.’ \trailingcitation{\parencite[38]{bakhtibekov1979}}//
\endgl \xe

Пример с перемещением клитики:

\ex<exclit7>
\begingl
\gla {[}Йā ру̊пц{]} {[}лу̊д{]}=\b{и}: <…>//
\glc {\sc d3.f.sg} лиса сказать.{\sc pst=3sg} ~//
\glft ‘Лисица сказала: <…>’ \trailingcitation{\parencite[78]{zarubin1960}}//
\endgl \xe

Рассмотрим некоторые термины, необходимые для дальнейшего рассуждения. В работе \parencite[6]{zwicky1977} выделяются два типа клитик: простые и специальные. Под простыми клитиками понимаются сокращённые безударные формы слов, которые возникают при определённых условиях, или «сокращённые морфологические компоненты» \parencite[132]{nikolaeva2008}. Специальные клитики также не имеют собственного ударения, но их поведение в предложении определяется характерными лишь для них дистрибутивными правилами и синтаксическими принципами размещения. С точки зрения данной классификации изучаемая нами энклитика 3-го лица единственного числа =\i{и} относится к специальным, и её поведение должно описываться определёнными правилами, которые мы попробуем охарактеризовать.

В нашей работе мы будем обращаться к понятию ритмикосинтаксических барьеров — составляющих, которые могут влиять на расположение клитик. Анализ с использованием ритмико-синтаксических барьеров применял А.~А.~Зализняк при описании системы энклитик в древнерусском языке \parencite{zalizniak2008}. В своём исследовании А.~А.~Зализняк утверждал, что, согласно закону Ваккернагеля, клитика всегда занимает вторую позицию, однако существуют также дополнительные ограничения, в результате действия которых «начальная часть клаузы может быть как бы отчленена» \parencite[48]{zalizniak2008}, и отсчёт составляющих начинается после неё. Эти частные правила, модифицирующие закон Ваккернагеля, устанавливают новую точку «условного начала», которую Зализняк называет ритмико-синтаксическим барьером. Под барьером понимается «синтаксическая вершина или полная группа, добавление которой в состав предложения меняет позиции отдельной клитики или цепочки клитик» \parencite[387]{zimmerling2013}.

Выделяют обязательные, полуобязательные и факультативные барьеры. Обязательный барьер реализуется всегда и определяется чёткими правилами. Факультативный барьер связан скорее с коммуникативной ситуацией, он может появляться или отсутствовать в зависимости от намерения говорящего выделить что-либо в речи; существуют семантические и синтаксические факторы, повышающие вероятность возникновения факультативного барьера, но нет формальных правил, детерминирующих его появление. Понятие полуобязательного барьера неактуально в данной статье, поэтому оно не рассматривается \parencite[54]{zalizniak2008}.

В ходе нашего исследования были выявлены и описаны два барьера. Для удобства было решено ввести в оформление дополнительные обозначения: если составляющая является барьером, она отмечена подписью «\i{Б}» с номером барьера. В примере ниже первая полная составляющая \i{{[}йу ешу̊н{]}} является барьером~I.

\ex<exclit72>
\begingl
\gla {[}Йу ешу̊н{]}\textsubscript{\b{Б1}} {[}дуо{]}=\b{йи} чӯд.//
\glc {\sc d3.m.sg} ишан молитва={\sc 3sg} делать.{\sc pst}//
\glft ‘Ишан прочитал молитву.’ \trailingcitation{\parencite[62]{zarubin1960}}//
\endgl \xe

Далее в тексте статьи будут описаны данные, на основе которых было проведено исследование, в \hyperref[clit-position]{третьем разделе} мы перечислим позиции, которые может занимать подвижный показатель 3-го лица единственного числа, а в \hyperref[clit-barone]{четвёртом} и \hyperref[clit-bartwo]{пятом} разделах мы обратимся к описанию двух барьеров, влияющих на расположение энклитики. В \hyperref[clit-distrib]{шестом разделе} мы рассмотрим, как данные барьеры могут влиять на другие энклитики, и какие есть закономерности в появлении барьеров в зависимости от времени создания текста.

\section{Материал исследования} \label{clit-data}

Основными материалами для исследования послужили данные, собранные методом элицитации в городе Хорог и в близлежащих кишлаках в ходе экспедиций на Памир в 2018–2019~годах. Другими источниками данных послужили тексты, записанные в 1915–1917 и 1927~годах. И.~И.~Зарубиным [\cite*{zarubin1960}] и собранные в 1948–1949 годах В.~С.~Соколовой (цит. по~\parencite{pakhalina1969_pamir}). Также использовались данные словарей [\cite{karamshoev1988}; \cite*{karamshoev1991}; \cite*{karamshoev1999}; \cite{zarubin1960}], грамматика баджувского диалекта шугнанского языка \parencite{karamshoev1963}, существующие очерки шугнанского языка и описания памирских языков \parencites{edelman_yusufbekov1999_shughni}{edelman_dodykhudoeva2009_shughni}{pakhalina1969_pamir}{bakhtibekov1979}.

Всего в ходе элицитации были опрошены девять информантов, семь женщин и двое мужчин. Возраст информантов варьируется от 18 до 50~лет.

Анкеты семи информантов были собраны в 2019~году во время экспедиции на Памир, у каждого из них были собраны две анкеты, одна из которых представляла собой связный текст, а вторая состояла из набора предложений. Запись одного текста была сделана в Душанбе в феврале 2020~года с двумя информантами, расшифровка проводилась с носителями шугнанского в Москве.

На данный момент рассмотрено 539~примеров, из которых 228 были собраны путем элицитации во время экспедиций 2018 и 2019~годов и в 2020~году. Ещё 311~примеров взяты из текстов, записанных в 1915–1917~годах \parencite{zarubin1960} (289~примеров) и в 1948–1949~годах \parencite{sokolova1953} (22~примера). Таким образом, можно говорить о двух синхронных «срезах» языка, между которыми прошло почти сто лет.

\section{Позиция показателя {\sc 3sg}} \label{clit-position}

И в текстах, записанных в начале XX~века, и в современных записях были примеры, демонстрирующие, что позиция показателя {\sc 3sg} =\i{и} может отличаться от позиций всех остальных лично-числовых клитик. Во-первых, употребление всех лично-числовых показателей, кроме показателя {\sc 3sg} =\i{и}, является строго обязательным. Напротив, подвижный показатель {\sc 3sg} =\i{и} используется нерегулярно, он факультативно появляется при переходных глаголах и опускается при непереходных. Это почти единственный маркер переходности глаголов, который, однако, ведёт себя непоследовательно: он может присоединяться к таким вроде бы непереходным глаголам, как ‘чихнуть’ (\i{йу=\b{йи} пиршт} ‘он={\sc 3sg} чихнул’) или ‘смеяться’ (\i{йу ғиδā=\b{йи} шӣнч} ‘этот парень={\sc 3sg} засмеялся’) \parencite[236]{edelman_yusufbekov1999_shughni}, и опускаться при переходных глаголах.

Во-вторых, при употреблении показателя {\sc 3sg} часто нарушается закон Ваккернагеля, и энклитика =\i{и} вместо ожидаемой позиции после первой составляющей может занимать позицию после второй или даже третьей составляющей.

В элицитированных примерах в 47~случаях (20,5\%) энклитика отсутствовала, один пример был исключен как неоднозначный, а в 12~примерах (6,7\%) энклитика была смещена вправо.

\begin{table}[H]
 \centering
 \caption{Позиция энклитики в XX и XXI~веках}
 \smallskip
 \label{tab:clit1}
 \begin{tabular}{c|cc} \toprule
 время записи & \makecell{энклитика\\на 1-ой составляющей\\(вторая позиция)} & \makecell{энклитика правее\\1-ой составляющей} \\ \midrule
 \makecell{1915–1917,\\1948–1949~гг.} & 202 (65\%) & 109 (35\%) \\
 2019–2020~гг. & 168 (93,3\%) & 12 (6,7\%) \\ \bottomrule
 \end{tabular}
\end{table}

В приведённой таблице сравниваются данные из текстов, записанных И.~И.~Зарубиным в первой половине XX~века (верхняя строка), и данные, собранные при помощи элицитации в 2019–2020~годах (нижняя строка). Видно, что и в старых текстах, и в новых энклитика чаще занимает вторую позицию. Количество примеров, где энклитика по какой-либо причине смещается дополнительно вправо, в старых текстах значительно выше, чем в современных элицитированных примерах: в старых текстах примеры с перемещением клитики составляют 35\% от общего количества примеров с энклитикой, в новых — 6,7\%. По критерию ${\chi}$-квадрат различия между текстами 1915–1949 и 2019–2020~годов значимы.

Рассмотрим примеры с разными типами расположения энклитик.

В примерах (\gethref{exclit9}–\gethref{exclit11}) все лично-числовые показатели занимают стандартную вторую позицию после первой полной составляющей:

\ex<exclit9>
\begingl
\gla {[}Wуз{]}=\b{ум} деф йум-анд ача на-wӣн-т.//
\glc {\sc pron.1sg=1sg} {\sc d2.pl.o} {\sc d1.sg-loc} совсем {\sc neg}-видеть-{\sc pst}//
\glft ‘Я их там совсем не видел.’ \trailingcitation{\parencite[38]{bakhtibekov1979}}//
\endgl \xe

\ex<exclit10>
\begingl
\gla {[}Ца wахт{]}=\b{āм} мāш рӯбā ваδҷ?//
\glc {\sc subd} время={\sc 1pl} {\sc pron.1pl} лиса быть.{\sc pf.pl}//
\glft ‘А когда мы были лисицей?’ \trailingcitation{\parencite[64]{zarubin1960}}//
\endgl \xe

\ex<exclit11>
\begingl
\gla {[}ху моθ{]}=\b{и} вирух̌-т//
\glc {\sc refl} посох={\sc 3sg} ломать-{\sc pst}//
\glft ‘свой посох разломал’ \trailingcitation{\parencite[81]{zarubin1960}}//
\endgl \xe

Именные группы могут быть развёрнутыми, как, например, первые составляющие в примерах (\gethref{exclit12}–\gethref{exclit13}): “труп своей матери”, “руки своих тестя и тёщи”. Энклитика не разрывает их, но по правилу занимает позицию после первой полной составляющей.

\ex<exclit12>
\begingl
\gla {[}Ху нāн мурδа{]}=\b{йи} маркаб=ти саwор чӯд.//
\glc {\sc refl} мать труп={\sc 3sg} осёл={\sc sup} верхом делать.{\sc pst}//
\glft ‘Труп своей матери посадил на осла.’ \trailingcitation{\parencite[51]{pakhalina1969_pamir}}//
\endgl \xe

\ex<exclit13>
\begingl
\gla {[}Ху хӣх̌=ат хисур δус=тӣр{]}=\b{и} бā чӯд.//
\glc {\sc refl} тесть={\sc and2} тёща рука={\sc sup=3sg} поцелуй делать.{\sc pst}//
\glft ‘Своим тестю и тёще руки (буквально: руки своих тестя и тёщи) поцеловал.’ \trailingcitation{\parencite[73]{zarubin1960}}//
\endgl \xe

В примерах (\gethref{exclit14}–\gethref{exclit16}) энклитика {\sc 3sg} =\i{и} занимает позицию после двух полных составляющих:

\ex<exclit14>
\begingl
\gla {[}Йу ғиδā{]} {[}wи хадāр ризӣн{]}=\b{и} зох̌-т.//
\glc {\sc d3.m.sg} парень {\sc d3.m.sg.o} старший.{\sc f} дочь={\sc 3sg} брать-{\sc pst}//
\glft ‘Парень взял его старшую дочь.’ \trailingcitation{\parencite[51]{pakhalina1969_pamir}}//
\endgl \xe

\ex<exclit15>
\begingl
\gla {[}Бāд{]} {[}δорг{]}=\b{и} хуб лап тилāп-т…//
\glc потом дерево={\sc 3sg} достаточно много просить-{\sc pst}//
\glft ‘Он выпрашивает достаточно дерева…’ \trailingcitation{\parencite[20]{zarubin1960}}//
\endgl \xe

\ex<exclit16>
\begingl
\gla Атā {[}Йесо{]} {[}даwу̊м{]}=\b{и} чӯд: <…>//
\glc {\sc and3} Иисус начало={\sc 3sg} делать.{\sc pst} ~//
\glft ‘И Иисус начал: <…>’ \trailingcitation{Лк. 15:11, \parencite[54]{dodixudoev2001}}//
\endgl \xe

В примере (\gethref{exclit17}) энклитика занимает позицию после трех полных составляющих:

\ex<exclit17>
\begingl
\gla {[}Йā{]} {[}даδ{]} {[}wи колā{]}=\b{йи} вӯд=ху, <…>//
\glc {\sc d3.f.sg} тогда {\sc d3.m.sg.o} материя={\sc 3sg} нести.{\sc pst=and1} ~//
\glft ‘Тогда она вынесла материю…’ \trailingcitation{\parencite[66]{zarubin1960}}//
\endgl \xe

Очевидно, на расположение энклитики влияют дополнительные факторы. Наша гипотеза заключается в том, что все клитики, включая показатель 3-го лица единственного числа =\i{и}, являются ваккернагелевскими, однако в шугнанском языке существуют определённые барьеры, провоцирующие смещение энклитики 3-го лица единственного числа =\i{и} вправо. Видимо, в течение ХХ~века изменился характер этих барьеров, а вследствие этого изменилась и позиция энклитики.

\section{Барьер~I. Первая составляющая, выраженная подлежащим} \label{clit-barone}

\subsection{Тексты и элицитация} \label{clit-barone-texts}

Анализируя тексты, собранные И.~И.~Зарубиным в 1915–1917 и 1927~годах и В.~С.~Соколовой в 1948–1949~годах, мы заметили, что использование энклитики =\i{и} во всех найденных в них примерах соответствует ограничению, описанному Д.~К.~Карамшоевым для баджувского диалекта: возникает правило барьера, запрещающее энклитике =\i{и} присоединяться к первой составляющей, если она является подлежащим. Это правило барьера\fn{Далее в примерах мы будем использовать помету «\i{Б1}» для указания на этот барьер.} соблюдается в в работе \parencite{pakhalina1969_pamir} и в семи разных текстах, записанных с семью носителями \parencite{zarubin1960}.

\ex<exclit18>
\begingl
\gla {[}Му тāт{]}\textsubscript{\b{Б1}} {[}wам х̌итур муор{]}=\b{и} ~~~~~~~~~~~~~~~~~~~~~~~~~~~~~ пи ху хез тӣж-д.//
\glc {\sc pron.1sg.o} отец {\sc d3.f.sg.o} верблюд кольцо={\sc 3sg} ~ {\sc up} {\sc refl} {\sc apud} тянуть-{\sc pst}//
\glft ‘Мой отец потянул к себе верблюжье кольцо.’ \trailingcitation{\parencite[11]{zarubin1960}}//
\endgl \xe

В примере (\gethref{exclit18}) по закону Ваккернагеля энклитика должна была бы присоединиться к первой составляющей ‘мой отец’, но вместо этого показатель оказывается после второй полной составляющей ‘кольцо этого верблюда’. Точно такое же смещение вправо на одну составляющую наблюдается ниже в примерах (\gethref{exclit19}–\gethref{exclit21}):

\ex<exclit19>
\begingl
\gla {[}Йу{]}\textsubscript{\b{Б1}} {[}лап-аθ тилло{]}=\b{йи} вӯɣ̌ҷ.//
\glc {\sc d3.m.sg} много-{\sc adv} золото={\sc 3sg} нести.{\sc pf}//
\glft ‘Он принёс много золота.’ \trailingcitation{\parencite[298]{karamshoev1988}}//
\endgl \xe

\ex<exclit20>
\begingl
\gla {[}Wи ризӣн{]}\textsubscript{\b{Б1}} {[}пех̌с-т{]}=\b{и}…//
\glc {\sc d3.m.sg.o} дочь спросить-{\sc pst=3sg}//
\glft ‘Его дочь спросила…’ \trailingcitation{\parencite[76]{zarubin1960}}//
\endgl \xe

\ex<exclit21>
\begingl
\gla {[}Йу{]}\textsubscript{\b{Б1}} {[}лу̊д{]}=\b{и}: <…>//
\glc {\sc d3.m.sg} сказать.{\sc pst=3sg} ~//
\glft ‘Он сказал: <…>’ \trailingcitation{\parencite[51]{pakhalina1969_pamir}}//
\endgl \xe

\ex<exclit22>
\begingl
\gla {[}Йу мис{]}\textsubscript{\b{Б1}} {[}ик=дис=га йи ҷундор{]}=\b{и} кух̌-ч=ат…//
\glc {\sc d3.m.sg} тоже {\sc emph}=такой={\sc add} {\sc indef} баран={\sc 3sg} убить-{\sc pf=and2}//
\glft ‘Он тоже уже зарезал скотину…’ \trailingcitation{\parencite[69]{zarubin1960}}//
\endgl \xe

Несмотря на то, что в приведённых примерах энклитика присоединяется в основном к прямому дополнению, это не является строгим правилом: в качестве опорного слова могут выступать косвенные дополнения (\gethref{exclit23}), обстоятельства (\gethref{exclit24}), сказуемое (\gethref{exclit20}–\gethref{exclit21}).

\ex<exclit23>
\begingl
\gla {[}Худоwанд{]}\textsubscript{\b{Б1}} {[}ди одам-ард{]}=\b{и} дāкчӯҷ тусби-йен.//
\glc Господь {\sc d2.m.sg.o} человек-{\sc dat=3sg} дать.{\sc pf} чётки-{\sc pl}//
\glft ‘Творец дал тому человеку чётки.’ \trailingcitation{\parencite{shakarmamadov2005}}//
\endgl \xe

\ex<exclit24>
\begingl
\gla …{[}даδ{]}=\b{и} wи потх̌обачā ғêв анҷӯв-д.//
\glc затем={\sc 3sg} {\sc d3.m.sg.o} принц рот держать-{\sc pst}//
\glft ‘…тогда принцу рот закрыла.’ \trailingcitation{\parencite{shakarmamadov2005}}//
\endgl \xe

Описанный выше барьер влияет только на показатель 3-го лица единственного числа, остальные лично-числовые показатели свободно присоединяются к подлежащему. В примерах (\gethref{exclit25}–\gethref{exclit26}) видно, что, хотя в обеих клаузах первая составляющая выражена подлежащим, энклитика =\i{и} присоединяется после второй составляющей, а энклитики =\i{ум} (показатель {\sc 1sg}) и =\i{ен} (показатель {\sc 3pl}) — после первой:

\ex<exclit25>
\begingl
\gla {[}Йу{]}\textsubscript{\b{Б1}} {[}лу̊д{]}=\b{и}: «Wуз=\b{ум} wи пех̌с-т <…>».//
\glc {\sc d3.m.sg} сказать.{\sc pst=3sg} {\sc pron.1sg=1sg} {\sc d3.m.sg.o} спросить-{\sc pst} ~//
\glft ‘Он сказал: «Я его спросил <…>»’ \trailingcitation{\parencite[76]{zarubin1960}}//
\endgl \xe

\ex<exclit26>
\begingl
\gla {[}Чор=ат ɣ̌ин{]}=\b{ен} нах̌тойд тар ваҷ=ху…//
\glc муж={\sc and2} жена={\sc 3pl} выйти.{\sc pst.f/pl} {\sc eq} наружу={\sc and1}//
\glft ‘Муж и жена вышли наружу…’ \trailingcitation{\parencite[60]{zarubin1960}}//
\endgl \xe

\subsection{Случаи с нарушением барьера I} \label{clit-barone-cases}

В отличие от текстов первой половины XX~века, в элицитированных примерах энклитика в 93\%~случаев присоединяется к первой составляющей вне зависимости от того, каким членом предложения она выражена. В примерах (\gethref{exclit27}–\gethref{exclit29}) первая составляющая — это подлежащее, однако правило барьера не работает:

\ex<exclit27>
\begingl
\gla {[}Wи куд{]}=\b{и} Мукбилшо пуц пирен-т.//
\glc {\sc d3.m.sg.o} собака={\sc 3sg} Мукбилшо сын кусать-{\sc pst}//
\glft ‘Его собака укусила сына Мукбилшо.’ \trailingcitation{[элицитация, 2019]}//
\endgl \xe

\ex<exclit28>
\begingl
\gla {[}Махбуб{]}=\b{и} му қанфет-ен хӯд.//
\glc Махбуб={\sc 3sg} {\sc pron.1sg.o} конфета-{\sc pl} кушать.{\sc pst}//
\glft ‘Махбуб съел мои конфеты.’ \trailingcitation{[элицитация, 2019]}//
\endgl \xe

\ex<exclit29>
\begingl
\gla {[}Аҳмед{]}=\b{и} Саӣдā жӣwҷ чӯд.//
\glc Ахмед={\sc 3sg} Саида любить делать.{\sc pst}//
\glft ‘Ахмед полюбил Саиду.’ \trailingcitation{[элицитация, 2019]}//
\endgl \xe

При этом в речи некоторых носителей в точно таком же контексте барьер возникает (\gethref{exclit30}). Однако такая вариативность была редкой и непоследовательной, в другом контексте тот же самый носитель мог поместить клитику сразу после подлежащего.

\ex<exclit30>
\begingl
\gla {[}Аҳмед{]}\textsubscript{\b{Б1}} Саӣдā=\b{йи} жӣwч чӯд.//
\glc Ахмед Саида={\sc 3sg} любить делать.{\sc pst}//
\glft ‘Ахмед полюбил Саиду.’ \trailingcitation{[элицитация, 2019]}//
\endgl \xe

Таким образом, в текстах 1915–1948~годов во всех найденных примерах строго соблюдается правило барьера~I, запрещающее использование клитики после группы подлежащего. В современных текстах и элицитированных примерах барьер~I возникает в 4,7\% случаев от общего числа примеров, где первая составляющая — подлежащее. Можно сказать, что в современном языке этот барьер постепенно исчезает и в большинстве случаев (93,3\%) энклитика стремится располагаться после первой составляющей вне зависимости от того, выражена она подлежащим или другим членом предложения (ещё 2\% перемещений связаны с другим барьером).

\subsection{Барьер~I: эксперимент} \label{clit-barone-exp}

Мы решили дополнительно проверить, насколько часто возникает барьер в современном языке по сравнению с записями 1915–1917 и 1948–1949~годов. Для этого в ходе экспедиции в Таджикистан с двумя шугнанскими носителями был заново записан текст «Сказка о трёх братьях», впервые записанный В.~С.~Соколовой в 1948~году в Сталинабаде (современный Душанбе) в четырёх вариантах: шугнанском, рушанском, бартангском и сарыкольском (цит.~по~\parencite{pakhalina1969_pamir}).

Разумеется, записанный текст не идентичен версии 1948~года, часть предложений были построены иначе и не требовали наличия энклитики. Однако часть совпавших примеров подтвердила наши предположения: в современном языке правило барьера постоянно нарушается. Сравним примеры (\getfullhref{exclit31.a}) и (\getfullhref{exclit31.b}), первый из которых записан в 1948~году, а второй — в 2020~году.

Клауза в них начинается с обращения. В шугнанском языке обращение является обязательным барьером для всех энклитик, но в нашей статье мы не рассматриваем этот барьер подробнее. В примере (\getfullhref{exclit31.a}) правило барьера соблюдается, и энклитика занимает позицию после третьей составляющей (после двух барьеров), а в примере (\getfullhref{exclit31.b}) энклитика присоединяется к подлежащему, следовательно, правило барьера~I не работает.

\pex<exclit31>
\a<a> \begingl
\gla {[}Е вирод{]}, {[}ту х̌итур{]}\textsubscript{\b{Б1}} ~~~~~~~~~~~~~~~~~~~~~~~~~~~~~~~~~~~~~~~~~~~~~~~~~~~~~~~~~~~~~~~~~ {[}му шӣг{]}=\b{и} хӯд.//
\glc {\sc voc} брат {\sc pron.2sg} верблюд ~ {\sc pron.1sg.o} телёнок={\sc 3sg} кушать.{\sc pst}//
\glft ‘Эй, брат, твой верблюд съел моего телёнка.’ \trailingcitation{\parencite[50]{pakhalina1969_pamir}}//
\endgl
\a<b> \begingl
\gla {[}Е вирод{]}, {[}ту х̌итур{]}=\b{и} ~~~~~~~~~~~~~~~~~~~~~~~~~~~~~~~~~~~~~~~~~~~~~~~~~~~~~~~ {[}му шӣг-буц{]} хӯд.//
\glc {\sc voc} брат {\sc pron.2sg} верблюд={\sc 3sg} ~ {\sc pron.1sg.o} телёнок-детёныш.{\sc m} кушать.{\sc pst}//
\glft ‘Эй, брат, твой верблюд съел моего телёнка.’ \trailingcitation{[элицитация, 2020]}//
\endgl \xe

Сравним ещё несколько примеров, где в тексте 1948~года энклитика располагается на второй составляющей, а в современной элицитации — на первой:

\pex<exclit33>
\a<a> \begingl
\gla {[}Йу чорик{]}\textsubscript{\b{Б1}} {[}маркāб{]}=\b{и} δод қӣмб.//
\glc {\sc d3.m.sg} мужчина осёл={\sc 3sg} ударить.{\sc pst} камень//
\glft ‘Тот человек ударил осла камнем.’ \trailingcitation{\parencite[51]{pakhalina1969_pamir}}//
\endgl
\a<b> \begingl
\gla {[}Йу чорик{]}=\b{и} wи маркāб ~~~~~~~~~~~~~~~~~~~~~~~~~~~~~~~~ жӣр қати δоδҷ.//
\glc {\sc d3.m.sg} мужчина={\sc 3sg} {\sc d3.m.sg.o} осёл ~ камень {\sc com} ударить.{\sc pf}//
\glft ‘Тот человек осла камнем ударил.’ \trailingcitation{[элицитация, 2020]}//
\endgl \xe

\pex<exclit35>
\a<a> \begingl
\gla {[}Йу{]}\textsubscript{\b{Б1}} {[}лу̊д{]}=\b{и}…//
\glc {\sc d3.m.sg} сказать.{\sc pst=3sg}//
\glft ‘Он сказал: <…>’ \trailingcitation{\parencite[51]{pakhalina1969_pamir}}//
\endgl
\a<b> \begingl
\gla {[}Йу{]}=\b{йи} лу̊в-ҷ…//
\glc {\sc d3.m.sg=3sg} сказать-{\sc pf}//
\glft ‘Он сказал: <…>’ \trailingcitation{[элицитация, 2020]}//
\endgl \xe

Если сравнить данные только этих двух текстов 1948 и 2020~годов записи, то мы получим Таблицу~\ref{tab:clit2}, где есть 11~примеров с барьером~I в старом тексте и ни одного в новом. Общее количество примеров с использованием энклитики различается, поэтому приведён также процент, который составляет указанное число примеров с барьером от всех примеров с энклитикой в данном тексте.

\begin{sidewaystable}
\begin{table}[H]
 \centering
 \caption{«Сказка о трёх братьях» в двух записях}
 \smallskip
 \label{tab:clit2}
 \begin{tabular}{c|cccc} \toprule
 \multirow{2}{*}{\makecell{год\\записи}} & \multirow{2}{*}{\makecell{1-я составляющая\\\b{не} подлежащее $\rightarrow$\\клитика на 1-ой\\составляющей}} & \multicolumn{2}{c}{\makecell{1-я составляющая —\\подлежащее}} & \multirow{2}{*}{\makecell{всего\\примеров}} \\
 & & \makecell{барьер~I не соблюдается $\rightarrow$\\клитика на 1-й составляющей} & \makecell{барьер~I соблюдается $\rightarrow$\\клитика смещается} & \\ \midrule
 1948 & 9 (45\%) & 0 & 11 (55\%) & 20 \\
 2020 & 7 (43\%) & 9 (57\%) & 0 & 16 \\ \bottomrule
 \end{tabular}
\end{table}
\end{sidewaystable}

Можно видеть, что в тексте 1948~года записи в половине случаев возникал барьер, и энклитика присоединялась ко второй составляющей, если первая была выражена подлежащим, однако ни одного подобного случая нет в тексте 2020~года записи\fn{По критерию Фишера данные статистически значимы. Каждый столбец сравнивался с каждым, полученные значения были: $p = 0,0077$ для столбцов 1 и 2, $p = 0,0215$ для столбцов 1 и 3, $p < 0,0001$ для столбцов 2 и 3.}. Таким образом подтверждается гипотеза о том, что в современном языке барьер стал факультативным.

\section{Барьер~II. Обстоятельство, относящееся ко~всей клаузе} \label{clit-bartwo}

\subsection{Описание барьера} \label{clit-bartwo-desc}

Помимо барьера, связанного с позицией подлежащего, как в примерах из текстов, так и в элицитированных примерах возникает также барьер другого характера. Правило, запрещающее клитике присоединяться к подлежащему, мы условно назвали барьером~I, а правило барьера, которое рассматривается ниже, — барьером~II. Барьер~II возникает, если первая составляющая — это обстоятельство, относящееся по смыслу ко всей клаузе, причем в отличие от барьера~I, барьер~II был и остаётся факультативным. Данный тип барьера, видимо, можно считать характерным для разных языков: так, приводя пример факультативного барьера в древнерусском языке, А.~А.~Зализняк называет «начальное слово или словосочетание, относящееся по смыслу ко всей остальной части фразы в целом» \parencite[56]{zalizniak2008}, которое также соотносится по смыслу с предыдущей фразой. Обстоятельства-барьеры, которые будут рассмотрены нами ниже, соответствуют этому описанию, так как они связывают предложение с предыдущим контекстом и задают временные рамки для всей клаузы. Так как примеров со смещением энклитики вследствие появления барьера-обстоятельства было найдено мало\fn{12~случаев с перемещением энклитики из 28~контекстов, где барьер~II теоретически мог возникнуть; ср.~187~контекстов и 85~примеров со смещением клитики для барьера~I.}, в статье приведено только описание барьера без статистических данных.

Посмотрим на пример (\gethref{exclit37}), где в обеих клаузах, начинающихся с обстоятельства, энклитика пропускает первую составляющую и занимает позицию после второй составляющей\fn{Далее для обозначения второго барьера в примерах мы будем использовать нотацию «\i{Б2}».}:

\ex<exclit37>
\begingl
\gla {[}Бāд{]}\textsubscript{\b{Б2}} {[}поɣ̌ӡā{]}=\b{и} wам зох̌-т=ху, ~~~~~~~~~~~~~~~~~~~~~~~~~~~~~ {[}даδ{]}\textsubscript{\b{Б2}} {[}х̌ирн{]}=\b{и} wам маскā чӯд.//
\glc потом начисто={\sc 3sg} {\sc d3.f.sg.o} брать-{\sc pst=and1} ~ затем гладкий={\sc 3sg} {\sc d3.f.sg.o} масло делать.{\sc pst}//
\glft ‘Начисто его [масло] выбирает (=~достаёт из маслобойки), потом выравнивает это масло’ \trailingcitation{\parencite[45]{zarubin1960}}//
\endgl \xe

Так же, как и барьер~I, этот барьер в основном действует на клитику 3-го лица единственного числа. Те же самые обстоятельства становятся опорными словами в сочетании с другими лично-числовыми показателями, ср.~примеры (\gethref{exclit38}–\gethref{exclit40}):

\ex<exclit38>
\begingl
\gla {[}Во{]}=\b{йāм} раwу̊н сат.//
\glc опять={\sc 1pl} идущий стать.{\sc pst.f/pl}//
\glft ‘[Мы] опять пустились в путь’ \trailingcitation{\parencite[10]{zarubin1960}}//
\endgl \xe

\ex<exclit39>
\begingl
\gla {[}Даδ{]}=\b{ен} тойд тар ху чӣд-ен.//
\glc тогда={\sc 3pl} идти.{\sc pst.f/pl} {\sc eq} {\sc refl} дом-{\sc pl}//
\glft ‘Потом [они] разошлись по домам’ \trailingcitation{\parencite[67]{zarubin1960}}//
\endgl \xe

\ex<exclit40>
\begingl
\gla {[}Сойа-δêд=ард{]}=\b{āм} ар Δӣ-Рих̌у̊н фирӣп-т.//
\glc тень-упасть.{\sc inf=dat=1pl} {\sc down} Нижний-Рушан достигнуть-{\sc pst}//
\glft ‘К вечеру [мы] добрались до Нижнего Рушана’ \trailingcitation{\parencite{zarubin1960}}//
\endgl \xe

Однако надо отметить, что было найдено несколько примеров, где барьер~II влияет также на позицию показателя 3-го лица множественного числа =\i{ен}, ср.~пример (\gethref{exclit41}):

\ex<exclit41>
\begingl
\gla {[}Даδ{]}\textsubscript{\b{Б2}} {[}ҷӣдо{]}=\b{ен} сат.//
\glc затем отдельный={\sc 3pl} стать.{\sc pst.f/pl}//
\glft ‘Потом [они] разделились’ \trailingcitation{\parencite[75]{zarubin1960}}//
\endgl \xe

Примеров с передвижением =\i{ен} всего~4 на 351~случай употребления (1,1\%), но сама возможность подобного перемещения показывает разницу между барьером~I, который обусловлен синтаксическими правилами и не затрагивает никакие энклитики, кроме =\i{и}, и барьером~II, который зависит скорее от коммуникативной ситуации и факультативно может действовать на разные энклитики.

Однако поведение энклитики =\i{и} и в этом случае непоследовательно: рассмотрим примеры (\gethref{exclit42}–\gethref{exclit44}), где вышеописанные \i{даδ} и \i{во} становятся опорными словами для клитики, вместо того чтобы выступать в роли барьера:

\ex<exclit42>
\begingl
\gla {[}Во{]}=\b{йи} йод пи ниwенц хез.//
\glc опять={\sc 3sg} нести.{\sc pst} {\sc up} невеста {\sc apud}//
\glft ‘Потом понёс к невесте’ \trailingcitation{\parencite[72]{zarubin1960}}//
\endgl \xe

\ex<exclit43>
\begingl
\gla {[}Даδ{]}=\b{и} wам х̌ац зох̌-т.//
\glc потом={\sc 3sg} {\sc d3.f.sg.o} вода брать-{\sc pst}//
\glft ‘Потом он взял воду’ \trailingcitation{\parencite[72]{zarubin1960}}//
\endgl \xe

\ex<exclit44>
\begingl
\gla {[}Даδ{]}=\b{и} чêд зох̌-т.//
\glc потом={\sc 3sg} нож брать-{\sc pst}//
\glft ‘Потом взял нож’ \trailingcitation{\parencite[72]{zarubin1960}}//
\endgl \xe

Как и в случае с барьером~I, первая составляющая-обстоятельство может быть выражена именной группой:

\ex<exclit45>
\begingl
\gla {[}Йи меθ{]}\textsubscript{\b{Б2}} {[}δод{]}=\b{и} wи-рд ҷуwоб.//
\glc {\sc indef} день дать.{\sc pst=3sg} {\sc d3.m.sg.o-dat} ответ//
\glft ‘Однажды [имам] дал ему ответ’ \trailingcitation{\parencite[80]{zarubin1960}}//
\endgl \xe

При анализе данных примеров острее всего встает проблема омонимии, так как у некоторых обстоятельств есть формы-алломорфы, которые сложно отличить от тех же форм с энклитикой =\i{и}. Например, при элицитации были получены три разных варианта перевода одного предложения:

\pex<exclit46>
\a<a> \begingl
\gla {[}Соат=и панҷ=анд{]} {[}ар.чāй.ца{]} диви=ти ~~~~~~~~~~~~~~~~~~~~~~~~~~~~~~~~~~~~~~~~ туқ-туқ δод.//
\glc час={\sc ez} пять={\sc loc} кто\_то дверь={\sc sup} ~ стук-{\sc redup} ударить.{\sc pst}//
\endgl
\a<b> \begingl
\gla {[}Соат=и панҷ=анд{]}\textsubscript{\b{Б2}} {[}ар.чāй.ца{]}=\b{йи} диви=ти ~~~~~~~~~~~~~~~~~~~~~~~~~~~~~~ туқ-туқ δод.//
\glc час={\sc ez} пять={\sc loc} кто\_то={\sc 3sg} дверь={\sc sup} ~ стук-{\sc redup} ударить.{\sc pst}//
\endgl
\a<c> \begingl
\gla {[}Соат=и панҷ=анд{]}=\b{и} ар.чāй.ца диви=ти ~~~~~~~~~~~~~~~~~~~~~~~~~~~~~~~~~~~~~~~~~ туқ-туқ δод.//
\glc час={\sc ez} пять={\sc loc}=\b{?} кто\_то дверь={\sc sup} ~ стук-{\sc redup} ударить.{\sc pst}//
\glft ‘В пять часов кто-то постучал в дверь’ \trailingcitation{[элицитация, 2019]}//
\endgl \xe

Если в примере (\getfullhref{exclit46.a}) энклитика пропущена, а в примере (\getfullhref{exclit46.b}) помещается на вторую составляющую, то относительно примера (\getfullhref{exclit46.c}) было получено два комментария от носителей (такой вариант предложения был дан дважды): по одной версии, \i{и} (\i{панҷ=анд=и}) — это лично-числовая энклитика {\sc 3sg}, по другой — просто одна из форм локативного послелога \i{=анд(и)}. Видимо, мы не можем с уверенностью говорить здесь об однозначной интерпретации\fn{Данный пример не учитывался в основном подсчёте как неоднозначный.}.

Если в тексте появляются два барьера одновременно, энклитика присоединяется к третьей составляющей, пропуская и подлежащее (по правилу барьера~I), и обстоятельство (по правилу барьера~II)\fn{Таким образом, оба барьера являются кумулятивными. Кумулятивный барьер, возникая в предложении, не влияет на уже существующий барьер и просто смещает клитику дополнительно вправо. Привативный же барьер отменяет действие первого барьера, возвращая исходный порядок слов в клаузе \parencite[117]{zimmerling2013}.}. Порядок составляющих-барьеров в клаузе не имеет значения: в примере (\gethref{exclit49}) сначала появляется барьер~II, а потом барьер~I, в примерах (\gethref{exclit50}–\gethref{exclit51}) на первом месте оказывается барьер~I, а барьер~II на втором. Клитика во всех трёх случаях занимает третью позицию.

\ex<exclit49>
\begingl
\gla {[}Даδ{]}\textsubscript{\b{Б2}} {[}wи wазир{]}\textsubscript{\b{Б1}} {[}ди{]}=\b{йи} х̌уд.//
\glc тогда {\sc d3.m.sg.o} визирь {\sc d2.m.sg.o=3sg} слышать.{\sc pst}//
\glft ‘Тогда его везир услышал это’ \trailingcitation{\parencite[79]{zarubin1960}}//
\endgl \xe

\ex<exclit50>
\begingl
\gla {[}Йу{]}\textsubscript{\b{Б1}} {[}во{]}\textsubscript{\b{Б2}} {[}лу̊д{]}=\b{и}.//
\glc {\sc d3.m.sg} опять сказать.{\sc pst=3sg}//
\glft ‘Опять тот сказал: <…>’ \trailingcitation{\parencite[75]{zarubin1960}}//
\endgl \xe

\ex<exclit51>
\begingl
\gla {[}Йā{]}\textsubscript{\b{Б1}} {[}даδ{]}\textsubscript{\b{Б2}} {[}wи колā{]}=\b{йи} вӯд.//
\glc {\sc d3.f.sg} тогда {\sc d3.m.sg.o} материя={\sc 3sg} нести.{\sc pst}//
\glft ‘Тогда она вынесла материю’ \trailingcitation{\parencite[66]{zarubin1960}}//
\endgl \xe

\subsection{Барьер~II: цепочка энклитик} \label{clit-bartwo-chain}

Барьер~II, в отличие от барьера~I, может также влиять на позиции других энклитик. Ниже мы рассмотрим примеры с другой энклитикой, которая тоже может смещаться под действием барьера~II\fn{В этом подразделе частицы \i{х̌о} и \i{мис} анализируются как ваккернагелевские энклитики. Несмотря на то, что представленный анализ кажется нам верным, синтаксические и фонологические свойства этих частиц и их клитический статус нуждаются в дальнейшем изучении — \i{прим.~переиздания}.}.

Модальная частица \i{х̌о} (‘вероятно’, ‘очевидно’, ‘по-видимому’, ‘кажется’) внутри клаузы обычно располагается по закону Ваккернагеля после первой полной составляющей. Если в предложении возникают лично-числовые энклитики (в том числе энклитика =\i{и}), занимающие вторую позицию, то сперва присоединяются лично-числовые энклитики, а потом частица \i{х̌о}, см.~примеры (\gethref{exclit52}–\gethref{exclit53}).

\ex<exclit52>
\begingl
\gla {[}Йу{]}=\b{йи=х̌о} мāш қишлоқ қати но-балад.//
\glc {\sc d3.m.sg=3sg}=оказывается {\sc pron.1pl} кишлак {\sc com} {\sc neg}-знакомый//
\glft ‘Он, оказывается, не знаком с нашим кишлаком’ \trailingcitation{[элицитация]}//
\endgl \xe

Если же первая составляющая — это обстоятельство, частица \i{х̌о} смещается вправо, занимая позицию после второй составляющей, в данном случае, после подлежащего \i{ту} ‘ты’ с примыкающей к нему локальной энклитикой \i{мис} ‘тоже’.

\ex<exclit53>
\begingl
\gla {[}Пирwос{]}\textsubscript{\b{Б2}}=ат ту=мис=\b{х̌о} ~~~~~~~~~~~~~~~~~~~~~~~~~~~~ ар-у̊ вад.//
\glc прошлый.год={\sc 2sg} {\sc pron.2sg}=тоже=оказывается ~ {\sc down-d1} быть.{\sc pst.f/pl}//
\glft ‘В прошлом году, оказывается, и ты здесь была?’ \trailingcitation{\parencite[378]{karamshoev1991}}//
\endgl \xe

При этом правило барьера~I на частицу \i{х̌о} не распространяется. Как можно видеть из примеров (\gethref{exclit52}–\gethref{exclit53}), \i{х̌о} может присоединяться к подлежащему, занимая вторую позицию. А.~А.~Зализняк называет подобные случаи «разделением энклитик» – когда на разные типы энклитик барьер действует по-разному \parencite[54]{zalizniak2008}.

При этом барьер~II соблюдается не всегда, и в примере ниже можно увидеть, что и лично-числовая энклитика =\i{ет} [{\sc 2pl}], и частица \i{х̌о} сохраняют ваккернагелевскую позицию после первой составляющей, несмотря на то, что она выражена обстоятельством:

\ex<exclit54>
\begingl
\gla {[}Нур{]}=\b{ет=х̌о} лап занāт чӯд.//
\glc сегодня={\sc 2pl}=оказывается много тренировка делать.{\sc pst}//
\glft ‘Сегодня вы, оказывается, долго тренировались’ \trailingcitation{\parencite[537]{karamshoev1988}}//
\endgl \xe

Можно сделать вывод о том, что обстоятельства в первой позиции играют роль факультативных барьеров. Факультативный барьер «представляет собой просодическое выражение некоторого выделения того или иного звена фразы <…>. Такое выделение всегда предполагает соответствующую интенцию говорящего, а её, разумеется, может и не быть» \parencite[55]{zalizniak2008}. Интенция связана со смысловым подчёркиванием, и если в некоторых случаях возникновение барьера можно предсказать, то в относительно нейтральных контекстах выбор местоположения клитики зависит только от конкретного носителя. Кроме того, сфера действия факультативного барьера не ограничивается энклитикой =\i{и}, он также может влиять на позиции других клитик, что недопустимо для барьера~I.

\section{Распределение барьеров} \label{clit-distrib}

Таким образом, в шугнанском языке есть как минимум два барьера, которые модифицируют положение клитики в клаузе. Ниже приведена таблица, где подсчитаны все случаи перемещения энклитики вследствие появления барьеров. В элицитированных примерах в 47~случаях энклитика была опущена, эти примеры не учитываются в таблице ниже.

В первом столбце указано время записи текстов, данные расположены в хронологическом порядке. Во втором столбце приводится число примеров с энклитикой в предложениях, где нет барьеров, и где она занимает каноническую ваккернагелевскую позицию. Этот столбец нужен, чтобы исключить из рассмотрения примеры, где нет даже теоретической вероятности перемещения клитики.

В третьем и четвёртом столбцах рассмотрены примеры, где возникает барьер~I. В третьем столбце можно увидеть количество примеров, где правило барьера нарушается, и энклитика занимает вторую позицию, как если бы барьера не было (ни одного случая в текстах XX~века, 56,7\% примеров в элицитированных текстах), в четвёртом столбце — примеры, где барьер~I соблюдается, и энклитика смещается (обратная ситуация: 24,7\% в текстах 1915–1917~годов, 31,3\% в текстах 1948–1949~годов, но 2,7\% в текстах 2019–2020~годов).

\begin{sidewaystable}
 \centering
 \caption{Распределение клитики =\i{и} в рассмотренных примерах}
 \smallskip
 \label{tab:clit3}
 \begin{tabular}{c|ccccccc|c} \toprule
 \multirow{2}{*}{{\small\makecell{время\\записи}}} & \multirow{2}{*}{{\small\makecell{нет барьеров,\\клитика во\\второй позиции}}} & \multicolumn{2}{c}{{\small\makecell{1-я составляющая —\\подлежащее}}} & \multicolumn{2}{c}{{\small\makecell{1-я составляющая —\\обстоятельство}}} & \multirow{2}{*}{{\small\makecell{два\\барьера}}} & \multirow{2}{*}{{\small\makecell{барьер\\III?}}} & {\small итого} \\
 & & {\small\makecell{барьер~I\\нарушен}} & {\small\makecell{барьер~I\\соблюдается}} & {\small\makecell{барьер~II\\нарушен}} & {\small\makecell{ барьер~II\\соблюдается}} & & & \\ \midrule
 \makecell{1915–1917,\\1927} & 161 (62,1\%) & 0 & 64 (24,7\%) & 11 (4,2\%) & 7 (2,7\%) & 14 (5,4\%) & 2 (0,7\%) & 259 \\
 \makecell{1948–1949} & 29 (55,8\%) & 0 & 16 (31,3\%) & 1 (2\%) & 0 & 5 (9,8\%) & 1 (1,9\%) & 52 \\
 \makecell{2019–2020} & 61 (33,9\%) & 102 (56,7\%) & 5 (2,7\%) & 4 (2,2\%) & 5 (2,7\%) & 2 (1,1\%) & 1 (0,5\%) & 180 \\ \bottomrule
 \end{tabular}
\end{sidewaystable}

Таким же образом распределены примеры в пятом и шестом столбцах для предложений с барьером~II. В пятом столбце указано количество примеров с энклитикой во второй позиции, нарушающей правило барьера: 4,2\% в 1915–1917~годах, 2\% в 1948–1949~годах, 2,2\% в 2019–2020~годах. Можно видеть, что примеров с нарушением барьера~II примерно равное количество во все годы. В шестом столбце – количество примеров, где барьер~II соблюдается и энклитика смещается вправо: 2,7\% в текстах 1915–1917~годов, нет примеров в текстах Пахалиной 1948–1949~годов, 2,7\% в элицитации 2019–2020~годов.

В отдельном столбце приведены примеры, где появляются и соблюдаются одновременно два барьера: 5,4\% в 1915–1917~годах, 9,8\% в 1948–1949~годах, 1,1\% в 2019–2020~годах. Стоит отметить, что в текстах Пахалиной, несмотря на отсутствие примеров с одиночным барьером~II, правило барьера тем не менее действует, но во всех найденных примерах лишь в сочетании с барьером~I.

Также в отдельный столбец под названием «Барьер III?» вынесены несколько примеров, в которых энклитика смещается вправо, но это передвижение невозможно объяснить при помощи уже описанных барьеров. В примере ниже опорным словом для клитики становится третья составляющая (глагол-сказуемое), хотя по правилам она должна была присоединиться ко второй составляющей, прямому дополнению:

\ex<exclit55>
\begingl
\gla {[}Йу{]}\textsubscript{\b{Б1}} {[}wи{]} {[}зох̌-т{]}=\b{и}=ху…//
\glc {\sc d3.m.sg} {\sc d3.m.sg.o} брать-{\sc pst=3sg=and1}//
\glft ‘Тот его взял и…’ \trailingcitation{\parencite[37]{zarubin1960}}//
\endgl \xe

Таким образом, чтобы оценить общее количество примеров с энклитикой во второй позиции, можно сложить данные из первого столбца и из столбцов с данными о нарушении барьеров. Видно, что общая доля примеров с энклитикой во второй позиции для текстов 1915–1917~годов равна 66,3\% (62,1\% + 4,2\%), для текстов 1948–1949~годов — 57,8\% (55,8\% + 2\%), а для элицитированных примеров 2019–2020~годов — 92,8\% (33,9\% + 56,7\% + 2,2\%). Количество примеров с энклитикой на первой составляющей в современных текстах значительно выше за счёт того, что в них чаще нарушаются ритмико-синтаксические барьеры.

При этом стоит отметить разницу между барьером~I, который не нарушался ни разу в старых текстах, но практически не соблюдается в элицитированных примерах, и барьером~II, который и в текстах XX~века, и в современном языке нарушается и соблюдается с сопоставимой частотностью.

Можно предположить, что барьер~I, запрещающий клитике =\i{и} присоединяться к подлежащему, строго соблюдался в первой половине XX~века (энклитика смещается во всех 311~рассмотренных примерах) и стал факультативным в современном языке (возникает в 2,7\%~случаев из 180~примеров). В свою очередь, барьер~II, помещающий условное начало клаузы справа от обстоятельств, был и остаётся факультативным барьером, использование которого связано в первую очередь с коммуникативной ситуацией. Процент появления барьера~II в текстах XX~века не отличается от данных последних лет: 2,7\% в старых текстах, 2,7\% в новых. Барьер~I влияет только на положение клитики 3-го лица единственного числа; остальные клитики, согласно закону Ваккернагеля и в соответствии с существующими описаниями, занимают позицию после первой полной составляющей. Барьер~II может иногда возникать в том числе для клитики 3-го лица множественного числа =\i{ен}.

\section{Заключение} \label{clit-conclusion}

На основе рассмотренных примеров можно заключить, что все энклитические показатели лица и числа в шугнанском языке подчиняются закону Ваккернагеля и занимают позицию после первой полной составляющей (отсчёт составляющих начинается после барьера). В языке есть как минимум два барьера, которые влияют на расположение энклитики 3-го лица единственного числа =\i{и}. Вследствие возникновения барьеров условное начало клаузы смещается вправо, и энклитика присоединяется к первой составляющей после последнего барьера. В ходе работы были описаны два барьера и была подсчитана частота появления каждого из них в разных текстах.

Барьер~I носит синтаксический характер: он запрещает энклитике =\i{и} присоединяться к первой составляющей, если она выражена подлежащим. Во всех текстах, записанных в ХХ~веке (в 1915–1917 и 1948–1949~годах), этот барьер строго соблюдается. Однако в современном языке этот барьер постепенно утрачивается, основным правилом становится закон Ваккернагеля: чем бы ни была выражена первая составляющая, энклитика стремится присоединиться к ней. Это видно в процентном соотношении: в старых текстах на все случаи появления энклитики приходится 24,7\%~примеров с передвижением, а в новых текстах этот процент составляет 2,7\%.

Барьер~II является коммуникативным: он сдвигает условное начало клаузы вправо, помещая его после вводных слов, наречий и обстоятельств времени, относящихся по смыслу ко всей клаузе (наиболее частотные примеры обстоятельств-барьеров в текстах: =\i{даδ} ‘тогда’ и =\i{бāд} ‘потом’). Это факультативный барьер, появление которого зависит в первую очередь от коммуникативной ситуации, он легко нарушался раньше и нарушается сейчас (встречается в 2,7\%~случаев в старых текстах и ровно так же в 2,7\% — в новых). Этот барьер затрагивает также позицию как минимум одной другой энклитической частицы, =\i{х̌о} ‘оказывается’, и как минимум одного энклитического показателя 3-го лица множественного числа, =\i{ен}, смещая их вправо от базовой ваккернагелевской позиции.

Можно отметить, что барьер~II типологически схож с коммуникативными барьерами, которые А.~А.~Зализняк выделяет для древнерусских энклитик \parencite[98]{zalizniak2008}. В качестве барьеров в древнерусском языке также выступали обстоятельства, которые часто выражались наречиями, а смещение клитики, как и в рассмотренных шугнанских примерах, было факультативным.

В шугнанском языке в одном предложении может быть два барьера одновременно (в любом порядке), в результате чего энклитика будет занимать позицию после третьей составляющей. Также были обнаружены четыре примера, где энклитика смещается ещё дальше вправо, но характер этого передвижения остается невыясненным — возможно, это следы других барьеров, которые пока не удалось описать.
