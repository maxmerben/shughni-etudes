\chapter*{К~изучению шугнанского языка: ретроспектива и~перспектива}
\addcontentsline{toc}{chapter}{\textit{В.~Плунгян}. \textbf{К~изучению шугнанского языка: ретроспектива и~перспектива}}
\setcounter{section}{0}
\chaptermark{К~изучению шугнанского языка: ретроспектива и перспектива}
\label{chapter-plun-retro}

\begin{customauthorname}
Владимир Плунгян
\end{customauthorname}

\begin{englishtitle}
\i{The study of Shughni: the past and the future\\{\small Vladimir Plungian}}
\end{englishtitle}

\begin{abstract}
В статье предлагается краткий обзор истории изучения и типологического своеобразия шугнанского языка и даётся общая характеристика исследовательского проекта НИУ~ВШЭ, связанного с полевой работой на Памире и созданием компьютерных инструментов для развития памирских исследований.
\end{abstract}

\begin{keywords}
памирские языки, шугнанский язык, полевая лингвистика, корпусная лингвистика.
\end{keywords}

\begin{eng-abstract}
The paper provides a brief overview of the history of Shughni studies and summarizes cross-linguistically interesting properties of Shughni (and other Pamir languages). Then, it elaborates on a current research project of the HSE University, which focuses on fieldwork in the Pamir region and computer-assisted tools for promoting Pamir linguistics investigations.
\end{eng-abstract}

\begin{eng-keywords}
Pamir languages, Shughni, field linguistics, corpus linguistics.
\end{eng-keywords}

\begin{initialprint}
\fullcite{plungian2022}\end{initialprint}

В настоящем выпуске «Вестника РГГУ» читатель найдет три статьи, посвящённые анализу грамматических особенностей шугнанского языка\fn{Оригинальная статья открывает раздел выпуска~№5 журнала «Вестник РГГУ. Серия “Литературоведение. Языкознание. Культурология”» за 2022~год, посвящённый шугнанскому языку.}. Это «Вариативность в употреблении указательных местоимений в шугнанском языке» [\hyperref[chapter-badeev-demon]{Бадеев 2022}], «Синтаксические и семантические свойств клитики =\i{и} в шугнанском языке» [\hyperref[chapter-chist-clitic]{Чистякова 2022b}] и «Посессивные конструкции с местоимениями в шугнанском языке [\hyperref[chapter-ronko-poss]{Ронько 2022}]. Статьи объединены не только тематически~— они создавались в рамках одного исследовательского проекта, который имеет уже достаточно длительную историю (и еще более длительную предысторию); как история, так и предыстория заслуживают несколько более подробного рассказа, который и последует ниже.

Шугнанский язык~— один из так называемых памирских языков иранской группы, самый крупный по числу говорящих (около~100 тысяч человек по недавним оценкам, ср.~\parencites[225]{edelman_yusufbekov1999_shughni}[7–9]{kalandarov2004}\fn{Численность говорящих на памирских языках в Таджикистане можно приблизительно оценить в 130–150~тысяч человек \parencites{dodykhudoeva1999}{kalandarov2018}; общую численность говорящих~— в 250–300~тысяч.}) и социолингвистически наиболее значимый (в частности, он доминирует в столице Бадахшана Хороге). Термин «памирский» в научном отношении не вполне строгий (хотя он широко употребителен как среди специалистов, так и среди самих носителей, чаще идентифицирующих себя, особенно во внешнем окружении, именно как «памирцев», тадж.~\i{помирӣ}); он в большей степени имеет ареальные и историко-культурные коннотации, чем генеалогические. Географически речь прежде всего идёт о компактном высокогорном регионе так называемого Западного Памира, где население сосредоточено в основном по долинам рек, образующим естественные ареалы; все они являются притоками Пянджа, по которому проходит значительная часть границы между Афганистаном и Таджикистаном. Важнейшие из этих притоков (с севера на юг)~— реки Вандж, Язгулем, Бартанг (Мургаб), Гунт, Шахдара, Богушдара и Памир. Административно это Ванджский, Рушанский, Шугнанский, Рошткалинский и Ишкашимский районы Горно-Бадахшанской автономной области Таджикистана и несколько граничащих с Пянджем восточных районов афганской провинции Бадахшан (Шигнанский, Ишкашимский, Ваханский и другие). Более периферийные (и более гетерогенные в языковом отношении) ареалы памирских языков находятся в западной части Китая (Сарыкол), а также в горных долинах Гиндукуша на территории афганского Бадахшана и пакистанского Читрала.

В генеалогическом плане языки Западного Памира существенно менее гомогенны, и вопрос об их внутренней классификации (как, впрочем, и о классификации иранских языков в целом) продолжает оставаться дискуссионным; из последних работ на эту тему см.~в особенности \parencites{wendtland2009}{korn2016_tree}{korn2019}, а также \parencite{edelman_dodykhudoeva2009_shughni} и \parencite{edelman2021}. Не вдаваясь в детали, отметим, что все исследователи выделяют в качестве бесспорного генеалогического единства шугнано-рушанский кластер (с шугнанским и примыкающими к нему с севера более малочисленными баджувским, рушанско-хуфским и бартангско-рошорвским, а также территориально отдалённым, но генеалогически очень близким сарыкольским); между идиомами этого кластера возможно частичное взаимопонимание. В достаточно близком родстве с языками шугнано-рушанского кластера находятся язгулямский \parencite{sokolova1967} и вымерший старованджский, занимавший наиболее северное положение на территории Бадахшана и не позднее начала XX~века вытесненный таджикскими говорами Дарваза \parencite{lashkarbekov2008}. Генетические отношения других языков и диалектных групп Западного Памира и Гиндукуша (ваханского, ишкашимско-сангличского и йидга-мунджанского) между собой и по отношению к шугнанско-язгулямской подгруппе~— при их несомненном ареально-типологическом сходстве\fn{Следует отметить и значительную культурно-религиозную общность памирских народов~— в частности, яркой особенностью памирцев является исповедуемый ими исмаилизм, играющий чрезвычайно важную роль в формировании местной идентичности, культуры и быта \parencite{kalandarov2006}.}~— остаются не до конца ясными (не все современные исследователи даже включают их в общую юго-восточную ветвь иранских языков, как это было традиционно принято в классификациях середины XX~века).

В типологическом отношении языки шугнанско-язгулямской подгруппы характеризуются целым рядом интересных особенностей. Прежде всего следует отметить достаточно сложную сегментную фонетику (с богатым консонантизмом и нетривиальным вокализмом) и чрезвычайно разветвлённую систему вокалических и консонантных чередований, играющую ключевую роль в выражении большинства грамматических категорий имени и глагола. В именной сфере заслуживает упоминания категория грамматического рода (отсутствующая в большинстве иранских языков), в глагольной~— развитие категории эвиденциальности на базе перфектоидных форм. Немалой сложностью отличается система указательных местоимений, сохраняющая наряду с противопоставлением по роду двухпадежное склонение \parencites{payne1989}{arkadiev2006} и трёхчастный дейксис. К дейктическим категориям может также добавляться противопоставление по «уровню» (англ.~\i{elevation}, ср.~\parencite{forker2019}), требующее различать положение ориентира выше, ниже или на уровне говорящего; оно проявляется прежде всего в выборе пространственных предлогов. В области морфосинтаксиса особенно интересно поведение энклитик, маркирующих лицо и число подлежащего в формах прошедших времён глагола (формы настоящего времени синтетические, хотя в большинстве случаев для выражения лица и числа подлежащего они используют этимологически те же показатели); правила употребления энклитик пока остаются не полностью понятыми. Как можно заметить, в предлагаемых ниже статьях многие из этих нетривиальных типологических особенностей исследуются на шугнанском материале.

Несмотря на большой интерес, который данные памирских языков представляют как для иранистики, так и для общей и ареальной типологии, эти языки нельзя назвать хорошо изученными (особенно по сравнению с другими иранскими языками). Прежде всего это объясняется их географической удалённостью и труднодоступностью, усугубленной политической нестабильностью данного региона в целом (в последнее время, к сожалению, последний фактор усиливается). Тем не менее история изучения этих языков начинается в конце XIX~века и отмечена целым рядом имен выдающихся исследователей. К первым крупным лингвистам, обратившим серьезное внимание на памирские языки, следует отнести норвежского ученого Георга Моргенстьерне (1892–1978) и российского ученого Ивана Ивановича Зарубина (1887–1964); оба начали масштабные полевые исследования в первые десятилетия XX~века, оба были специалистами широкого профиля, интересовавшимися не только языками, но и этнографией, историей и культурой народов данного региона (интересные детали о начале полевой работы Зарубина на Памире см.~также в \parencite{steblin_kamenskiy1993}). Последующие поколения иранистов во многом опирались на их опыт и собранные ими материалы; в их числе можно назвать В.~С.~Соколову (1916–1993), Т.~Н.~Пахалину (1928–1995), Д.~К.~Карамшоева (1932–2007)\fn{Уроженец бадахшанского Баджува и непосредственный ученик И.~И.~Зарубина, Д.~К.~Карамшоев описал свой родной баджувский диалект шугнанского языка \parencite{karamshoev1963}; эта работа до сих пор остаётся самой подробной и полной грамматикой идиома из состава шугнано-рушанского кластера. Помимо ряда специальных исследований, Д.~К.~Карамшоев занимался составлением выдающегося во многих отношениях трёхтомного шугнанско-русского словаря [\cite{karamshoev1988}, \cite*{karamshoev1991}, \cite*{karamshoev1999}]; в этом словаре примечателен прежде всего богатейший иллюстративный материал, основанный на многолетних полевых записях сотрудников отдела памироведения Института языка и литературы им.~Рудаки в Душанбе (с 1991 года~— в составе Института гуманитарных наук в Хороге).}, а также представителей более молодой плеяды~— И.~М.~Стеблина-Каменского (1945–2018) и Ш.~П.~Юсуфбекова (1962–2018). В настоящее время продолжают активно работать Д.~И.~Эдельман, которой памироведение (и иранистика в целом) обязана очень многим, М.~М.~Аламшоев, Л.~Р.~Додыхудоева и ряд других исследователей России и Таджикистана; из работ последнего времени, выполненных европейскими иранистами, можно отметить две замечательные диссертации \parencite{kim2017} и \parencite{obrtelova2019_text}\fn{В переиздании также дополнительно отметим следующие современные работы западных исследователей: \parencites{arlund2006}{barie2009}{hughes2011}{beck2013}{karvovskaya2013}{muller2015}{palmer2016}{sangregory2018} и, разумеется, новую шугнанскую грамматику К.~Паркера [\cite*{parker2023}] — \i{прим.~переиздания}.}.

Однако для наших полевых исследований шугнанского языка наиболее важным был как будто бы незначительный на этом общем  фоне эпизод, но без которого, очевидно, не сформировался бы наш интерес к Памиру. Речь идёт об одной из экспедиций А.~Е.~Кибрика (совместно с Б.~Ю.~Городецким) на Памир, в шугнанское селение Дебаста Шугнанского района, на реке Гунт. Она была предпринята в июле-августе 1969~года; помимо многочисленных студентов Отделения теоретической и прикладной лингвистики МГУ, в ней приняли участие (тогда уже хорошо известная профессиональная иранистка) Д.~И.~Эдельман и (тогда ещё аспирант) И.~М.~Стеблин-Каменский. Это была только третья по счету из 45~экспедиций А.~Е.~Кибрика, когда он ещё не вполне определился с выбором ареала и методики исследований. Памир привлек его, видимо, экзотичностью и труднодоступностью местных языков, но продолжения экспедиция не получила: сказалась чрезвычайная отдаленность этих мест (в те годы основная дорога в Бадахшан шла с северо-востока, через Ош и Мургаб, по старому Памирскому тракту, и путь был очень сложным и долгим); а в дальнейшем научные интересы А.~Е.~Кибрика оказались, как известно, почти безраздельно связаны с Дагестаном. Тем не менее, впечатления от этой поездки, единственной в своем роде, были яркими~— я хорошо помню рассказы её участников о Памире и спустя десять лет, и много позже. Памир представал в них как чрезвычайно далекий, но зачаровывающий мир; он казался одновременно и недоступным, и притягательным. В то же время эта поездка принесла и некоторые научные результаты: наиболее глубокими и интересными, пожалуй, были публикации \parencite{belikov1972} и \parencite{muravieva1975}. Первая из них~— краткие тезисы, но в них содержатся плодотворные идеи относительно семантики указательных местоимений в шугнанском (напомним, что это одна из самых сложных областей шугнанской грамматики); неслучайно некоторые из наблюдений В.~И.~Беликова были позднее использованы и в монографии \parencite{yusufbekov1998}, и в публикуемой ниже статье А.~О.~Бадеева [\hyperref[chapter-badeev-demon]{2022}]. Вторая публикация~— более развёрнутая статья, где едва ли не впервые предпринимается попытка систематически описать морфонологию чередований в основах шугнанского глагола (как мы отмечали выше, эта часть шугнанской грамматики также поражает своей сложностью).

«Памирский след», таким образом, хранился в памяти участников экспедиции и выпускников ОСиПЛ МГУ, но долгое время казалось, что этот эпизод так и останется единичным (небольшая экспедиция, предпринятая под руководством Б.~Ю.~Городецкого на следующий год, не принесла заметных результатов). Тем не менее времена менялись, и причудливое стечение обстоятельств (было ли это только волей случая?) в конце концов привело нас прямо на Памир~— тот самый Памир, куда мы со студенческих лет мечтали попасть. Началось всё со знакомства с носителями шугнанского языка в Москве (где сегодня удачным образом присутствует большая памирская диаспора) и коллегами из Российско-таджикского славянского университета в Душанбе, а закончилось регулярными поездками в Хорог и его окрестности (в Шугнанском и Рошткалинском районах), первая из которых произошла в августе 2018~года. Полевые исследования шугнанского языка удалось осуществить в рамках учебных и исследовательских программ Школы лингвистики и Института классического востока и античности НИУ ВШЭ; в числе руководителей этих экспедиций были Е.~Е.~Арманд (начиная с 2021~года), В.~А.~Плунгян, Е.~В.~Рахилина, Р.~В.~Ронько; в числе студентов, активно участвовавших в разные годы в работе с шугнанским материалом, помимо двух авторов публикуемых ниже статей, были также В.~Д.~Бутолин, Е.~В.~Востокова, В.~Д.~Гребнева, Ф.~М.~Даниэль, В.~В.~Иванова, Ю.~Ю.~Макаров, М.~Г.~Меленченко, С.~К.~Михайлов, Д.~А.~Новокшанов, А.~А.~Сергиенко\fn{За время, прошедшее с выхода этой статьи, в нашей научной группе появились новые участники: ряды руководителей пополнили Д.~А.~Рыжова и О.~И.~Беляев, а в число студентов ВШЭ-участников проекта вошли С.~С.~Главатских, Н.~И.~Киреев, Е.~О.~Коробова, Т.~М.~Луговской, А.~В.~Овчинникова, П.~В.~Падалка, С.~В.~Седунова, В.~Д.~Тимофеева, А.~О.~Шаврина и Б.~М.~Якубсон — \i{прим.~переиздания}.}. Здесь представлены самые первые результаты нашей работы, но мы, безусловно, надеемся на продолжение этих исследований~— несмотря на многочисленные организационные и финансовые трудности, которые неизбежны в такого рода деятельности и часто кажутся (но, к счастью, не всегда оказываются) непреодолимыми.

Следует сказать несколько слов и о, так сказать, идеологии и общем формате наших экспедиций. Памирские языки представляют собой живую и чрезвычайно сложную реальность; существенное продвижение в её понимании требует сочетания разных методов и подходов. В более детальном изучении нуждаются все аспекты этой реальности: от фонетического до социолингвистического, включая, разумеется, морфологию, синтаксис, грамматические категории и лексику. На наш взгляд, эффективным первым шагом в такой работе было бы создание современных компьютерных инструментов для памирских языков, в первую очередь электронного корпуса текстов. Основы этой работы были заложены в ходе первых экспедиций; для корпуса использовались как специально записанные нами устные рассказы носителей языка, так и доступные записи фольклорных текстов (как опубликованные, начиная с ранних текстов из собрания Зарубина [\cite*{zarubin1960}] и более поздних изданий фольклора памирских народов, см.~\parencite{shakarmamadov2005}, так и хранящиеся в архиве Института гуманитарных наук в Хороге), а также немногочисленные литературные тексты (поэтические и прозаические). Отдельным направлением наших исследований стала работа со словарём Карамшоева: как уже было сказано, этот словарь уникален в том отношении, что по сути представляет собой богатейшее собрание естественных примеров высокого качества, отражающих не только современный шугнанский язык, но и образцы речи носителей первой половины XX~века. Мы предприняли усилия по созданию электронной базы шугнанской лексики на основе этого словаря; в настоящее время этот ресурс доступен для всех желающих по адресу \i{\href{https://pamiri.online}{pamiri.online}} и продолжает развиваться (там же можно найти и ряд других полезных инструментов, например орфографический конвертер, позволяющий автоматически преобразовывать шугнанские тексты в разные системы записи, и шугнанский морфологический анализатор).

Публикуемые ниже статьи в полной мере используют созданные участниками памирского проекта инструменты: основным источником примеров являются корпус шугнанского языка (где фигурируют как ранние записи, так и полевые материалы, непосредственно собранные нами) и дополняющая его лексическая база словаря Карамшоева. Одним из результатов такого подхода оказалась возможность микродиахронических наблюдений: первые записи шугнанской речи были сделаны около ста лет назад, и как оказалось, за это время в языке произошли достаточно ощутимые изменения на всех уровнях. Из публикуемых ниже статей наиболее систематические наблюдения такого рода присутствуют в статье Д.~Г.~Чистяковой [\hyperref[chapter-chist-clitic]{2022b}], обнаружившей, что поведение клитик в «раннем» и «позднем» современном шугнанском подчиняется разным типам правил. Другим важным результатом является возможность производить количественные наблюдения над шугнанскими данными (касающиеся частотности и дистрибуции отдельных форм, лексем, конструкций и~т.~п.); в публикуемых статьях это также находит отражение.

В заключение хотелось бы отметить, что наши исследования не только не принесли бы сколько-нибудь значимых результатов, но даже не могли бы быть начаты без постоянной, многообразной и бескорыстной помощи памирцев, которая сопровождала нас на всех этапах нашей деятельности. Все наши исследования проводились совместно со специалистами Института гуманитарных наук Хорога, щедро делившимися с нами своими знаниями памирских языков, тонкой интуицией и эрудицией. Мы хотели бы особо отметить помощь и содействие нынешнего директора института Х.~С.~Каландарова и заместителя директора Ш.~С.~Некушоевой, а также наших постоянных консультантов из числа сотрудников института~— М.~Ардабаевой, Ч.~Назаршоевой, Н.~Ризвоншоевой. Неоценимой на всех этапах нашей работы была организационная помощь и моральная поддержка проректора Российско-таджикского славянского университета Х.~Д.~Шамбезоды и сотрудников кафедры теоретической и прикладной лингвистики РТСУ Д.~М.~Искандаровой и М.~Б.~Давлатмировой. С благодарностью вспоминаем мы и безвременно ушедших от нас Шодихона Юсуфбекова (первого директора Института гуманитарных наук в Хороге, щедрого, гостеприимного и бесконечно преданного делу изучения памирских языков) и Гульнисо Ризвоншоеву (собирательницу и несравненного знатока памирского фольклора, чьи записи составляют основу сегодняшнего шугнанского корпуса). \i{Тама-рд қуллуғ-и бисйор!}\fn{[{\sc pron.2pl-dat} благодарность-{\sc ez} большой]; отметим падежный формант при местоимении (в современном шугнанском у личных местоимений фактически начинает развиваться система вторичных падежей позднего образования «поверх» двухпадежной), а также заимствованную из таджикского изафетную конструкцию (памирским языкам не свойственную) с исконно шугнанской вершиной ‘благодарность’, таджикским формантом изафета и таджикским же по происхождению зависимым ‘большой, много’~— подобная лексико-грамматическая интерференция очень характерна для современного шугнанского языка (другие примеры такого рода приводятся в \parencite{dodykhudoeva2020_lexical}).}
