\chapter*{Предисловие к~сборнику}
\setcounter{page}{2}
\addcontentsline{toc}{chapter}{Предисловие к~сборнику}
\setcounter{section}{0}
\chaptermark{Предисловие к~сборнику}
\label{chapter-intro}

Настоящая монография включает статьи, опубликованные участниками проекта по описанию и документации памирских языков Национального исследовательского университета «Высшая школа экономики» (ВШЭ). Проект выполнялся силами Школы лингвистики и Института классического Востока и античности Факультета гуманитарных наук ВШЭ (Москва) совместно с сотрудниками Института гуманитарных наук Национальной академии наук Таджикистана (Хорог) и при поддержке Университета Центральной Азии.

Проект начал работу в 2018~году, и уже на следующий год была организована первая поездка сотрудников и студентов университета в город Хорог, центр распространения шугнанского языка в Таджикистане. С тех пор участники проекта приезжают на Памир почти каждый год, а его география постепенно расширилась, и нам удалось поработать также с другими памирскими языками: бартангским, рушанским и ишкашимским. За это время было создано несколько электронных лингвистических ресурсов: оцифрованные онлайн-словари памирских языков, морфологический анализатор и конвертер орфографий, корпус текстов шугнанского языка — все эти ресурсы доступны онлайн на сайте \i{\href{https://pamiri.online}{pamiri.online}}. Таким образом за время существования проекта нам удалось собрать и опубликовать материалы по огромному количеству научных тем. В настоящий сборник вошла только их небольшая часть — а именно статьи, вышедшие в 2020–2023~годах. Далее мы приводим более полный список работ, представляющих результаты нашего проекта.

Значительную часть нашей работы составляют исследования лексики шугнанского языка, в которых мы сотрудничали со специалистами из Института гуманитарных наук в Хороге. В сборник вошли сразу четыре статьи, посвящённые разным глаголам шугнанского языка:
глаголам движения вниз [\b{\hyperref[chapter-rakh-down]{Рахилина, Некушоева 2020}}],
глаголам прятания [\b{\hyperref[chapter-rakh-hide]{Рахилина и др. 2021}}],
глаголам со значением ‘литься’ и ‘сыпаться’ [\b{\hyperref[chapter-armand-pour]{Арманд, Некушоева 2022}}]
и глаголам со значением поиска [\b{\hyperref[chapter-rakh-search]{Рахилина и др. 2023}}].
Продолжает этот ряд недавняя теоретическая работа \parencite{rakhilina_ryzhova2025}, посвящённая семантическому полю \fakesc{ИСЧЕЗАТЬ}, в которой анализируется в том числе шугнанская лексика.

Участники проекта также изучили ряд вопросов, связанных с общим синтаксисом шугнанского языка. Помимо статей о
дифференцированном маркировании объекта [\b{\hyperref[chapter-chist-dom]{Чистякова 2022a}}],
правилах употребления субъектной энклитики 3-го~лица единственного числа =\i{и} [\b{\hyperref[chapter-chist-clitic]{Чистякова 2022b}}]
и посессивных конструкциях с местоимениями [\b{\hyperref[chapter-ronko-poss]{Ронько 2022}}],
которые входят в эту книгу, мы также опубликовали работы по следующим темам:
структура простого предложения \parencite{chistiakova2022_word_order},
структура именной группы \parencite{sarkisov2018},
лично-числовые клитики в целом \parencite{chistiakova2023_wackernagel},
расщеплённая непереходность \parencite{chistiakova2023},
неканонические подлежащие \parencite{sergienko2023},
валентности предикатов \parencite{chistiakova_ryzhova2023},
сочинительные союзы \parencite{padalka_melenchenko2024} и
ответные частицы ‘да’ и ‘нет’ \parencite{padalka_ryzhova2024}.
Кроме того, участники проекта, работавшие в долине реки Бартанг, исследовали некоторые синтаксические особенности современного бартангского языка, а именно:
структуру простого предложения \parencite{belyaev2024_clause},
стратегии маркирования актантов \parencite{sergienko2025_alignment},
подчинительные союзы \parencites{belyaev2025_subd}{belyaev2025_conj}{belyaev2025_constr},
а также дифференцированное маркирование объекта \parencite{belyaev2024_dom}
и расщеплённую непереходность (в сравнении с шугнанским) \parencite{chistiakova2025}.

Большое число наших исследований посвящено глагольной системе шугнанского языка. В сборник вошла только статья о семантике шугнанского Перфекта [\b{\hyperref[chapter-melen-evid]{Меленченко 2023b}}]; в других же работах обсуждаются:
категории будущего времени и проспектива \parencite{grebneva2023_thesis},
категория Плюсквамперфекта \parencite{melenchenko2025_pluperfect},
употребления глагольных форм в нарративах \parencite{melenchenko2025_diploma},
особенности глагола \i{жӣwҷ} ‘любить’ \parencite{melenchenko2024_love},
а также синкретизм форм женского рода и множественного числа в прошедших временах \parencite{sergienko_kasenov2023}
и древние глагольные приставки \parencite{armand2024}.

Не менее интересными для лингвистов являются и грамматические особенности имён существительных и местоимений шугнанского языка. Помимо работы [\b{\hyperref[chapter-badeev-demon]{Бадеев 2022}}], вошедшей в настоящий сборник, мы исследовали:
неопределённые местоимения \parencite{badeev_sergienko2023},
категорию множественного числа \parencites{butolin2022}{timofeeva2023},
грамматикализацию дательного послелога =\textit{ард} \parencite{padalka_etal2026} и локативных послелогов \parencite{novokshanov2025},
«вертикальный» пространственный дейксис \parencite{yakubson2023},
распределение существительных по роду, причём как собственно в шугнанском языке \parencite{badeev2022_gender},
так и в бартангском [\cite{badeev2023_bartangi}, \cite*{badeev2025_gender}]
и в памиро-гиндукушском ареале в целом \parencite{badeev2023_hindukush}.

Наши работы, посвящённые фонетике и фонологии, также не ограничиваются статьёй [\b{\hyperref[chapter-makplun-morphon]{Макаров, Плунгян 2023}}] о морфонологии шугнанского языка, которую вы найдёте в книге. Некоторые другие важные работы посвящены
оглушению согласных [\cite{makarov2022_blowing}, \cite*{makarov2023_aspiration}, \cite*{makarov2024_essential}]
и произношению кратких гласных на конце слова \parencite{makarov2023_lowering},
различным особенностям согласных [\cite{makarov2023_obstruents}, \cite*{makarov2023_uvular}]
и устройству слога \parencite{makarov2023_phonotactics},
а также фразовой просодии шугнанского языка \parencite{sedunova2024}.

В сборник вошли и некоторые публикации более общего характера. Это
вводная статья [\b{\hyperref[chapter-plun-retro]{Плунгян 2022}}] об истории «памирского проекта» ВШЭ
и статья [\b{\hyperref[chapter-melen-ortho]{Меленченко 2023a}}] о проблемах стандартизации шугнанского алфавита. Среди других публикаций такого рода отметим также
статью \parencite{makarov_etal2022}, посвящённую результатам разработки сайта \i{\href{https://pamiri.online}{pamiri.online}},
и русско-англо-шугнанский разговорник \parencite{sergienko2021}, вышедший в издательстве ВШЭ.
Полный список наших работ можно найти на странице \i{\href{https://pamiri.online/blog/publications}{pamiri.online/blog/publications}}.

\section*{Особенности переиздания}

Работы, вошедшие в сборник, в целом публикуются в том виде, в котором они были изданы первоначально, однако при объединении их в монографию они были дополнительно отредактированы и в ряде отношений унифицированы: исправлены редкие орфографические и пунктуационные ошибки, обновлена нумерация и оформление примеров и ссылок на источники. Все примеры на шугнанском и других языках шугнано-рушанской группы приведены к кириллической орфографии (см.~[\hyperref[chapter-melen-ortho]{Меленченко 2023a}]) и отглоссированы по единому стандарту. В случаях, когда, на наш взгляд, текст статьи, изданной несколько лет назад, требует уточнений или исправлений с нынешней точки зрения, такие комментарии приводятся в сносках с пометой «\i{прим.~переиздания}».

Языковые примеры в статьях сборника помечаются \i{курсивом}, а их значения — ‘одинарными кавычками’ (например: \i{зив}~‘язык’). Те примеры, которые носители языка при опросе посчитли некорректными, помечены астериском (\b{*}). Если пример корректен, но не в упомянутом значении или контексте, используется знак решётки (\b{\#}). В случаях, когда мнения носителей по поводу примера расходятся (одни считают его нормальным, а другие — неправильным), используется вопросительный знак (\b{?}).

Все примеры-предложения снабжены \b{глоссированием} — специальной разметкой, отражающей морфологическую структуру слов. Морфемы, выражающие грамматические значения, обозначаются глоссами — условными ярлыками, записанными латинскими буквами (например, глосса [{\sc sg}] значит «единственное число»). При глоссировании мы следуем основным положениям Лейпцигских правил глоссирования \parencite{leipzig}. Одна глосса соответствует одной морфеме, даже если у морфемы есть несколько разных значений. Корни и аффиксы отделяются друг от друга дефисами (\b{-}), клитики отделяются знаком равенства (\b{=}). Если одна морфема выражает несколько грамматических или лексических значений, они записываются через точку, например: \i{х̌ойд} [читать.{\sc prs.3sg}] ‘читает’. Список глосс и их расшифровка приведены ниже:

{\small
\begin{itemize}
  \item 1, 2, 3 — 1-е, 2-е, 3-е лицо
  \item {\sc add1}, {\sc add2} — аддитивы, ‘тоже’ (=\i{га}, \i{мис})
  \item {\sc adv} — адвербиализатор, суффикс наречий (-\i{аθ})
  \item {\sc agn} — суффикс со значением «деятеля» (-\i{ӣҷ})
  \item {\sc all} — аллатив, направление (\i{ба}-)
  \item {\sc and1}, {\sc and2}, {\sc and3} — сочинительные союзы (\i{ху}, \i{ат}, \i{атā})
  \item {\sc apud} — апудэссив, ‘около, у X-а’ (\i{хез})
  \item {\sc cause} — маркер причины (\i{ҷāт}, \i{авен})
  \item {\sc com} — комитатив, совместность (\i{қати})
  \item {\sc comp} — сравнительная степень (-\i{ди})
  \item {\sc compl} — комплементайзер, подчинительный союз (\i{иди})
  \item {\sc cont1}, {\sc cont2} — локатив с частями тела (\i{чи}-, \i{ми}-)
  \item {\sc cop} — копула, связка
  \item {\sc d1}, {\sc d2}, {\sc d3} — I, II, III серии указательных местоимений (\i{ми}, \i{ди}, \i{wи})
  \item {\sc dat} — датив, послелог со значением дательного падежа (=\i{ард})
  \item {\sc dim} — диминутив, уменьшительно-ласкательный аффикс
  \item {\sc down} — латив ‘ниже’, перемещение к ориентиру ниже (\i{ар})
  \item {\sc el} — элатив, ‘из’ (\i{аз})
  \item {\sc emph} — эмфатическая, усилительная частица (\i{ик}=)
  \item {\sc eq} — латив-экватив, перемещение к ориентиру на том же уровне (\i{тар})
  \item {\sc ez} — изафет
  \item {\sc f} — женский род
  \item {\sc fut} — маркер будущего времени и хабитуалиса (=\i{та})
  \item {\sc goal} — цель, назначение (\i{пис})
  \item {\sc hb} — хабитив, суффикс со значением обладания, наличия (-\i{дор})
  \item {\sc indef} — неопределённый артикль (\i{йи})
  \item {\sc inf} — суффикс Инфинитива
  \item {\sc int} — интенсификатор (=\i{аθ})
  \item {\sc lim1}, {\sc lim2} — лимитативы, ‘до’ (\i{то}, \i{=ец})
  \item {\sc loc} — локатив, послелог со значением местного падежа (=\i{анд})
  \item {\sc m} — мужской род
  \item {\sc neg} — отрицание
  \item {\sc nom} — номинатив (именительный падеж)
  \item {\sc o} — \i{oblique}, косвенный падеж у местоимений
  \item {\sc or} — союз ‘или’ (\i{йо})
  \item {\sc place} — суффикс места (-\i{зор})
  \item {\sc p.loc} — посессивный локатив, ‘у X-а’ (=\i{ҷа})
  \item {\sc pf} — Перфект (непрямое прошедшее время)
  \item {\sc pfv} — перфектив
  \item {\sc pl} — множественное число
  \item {\sc pqp} — Плюсквамперфект
  \item {\sc proh} — прохибитив (\i{мā}-)
  \item {\sc pron} — личные местоимения
  \item {\sc prs} — Презенс (настояще-будущее время)
  \item {\sc pst} — Претерит (прямое прошедшее время)
  \item {\sc ptcl} — дискурсивная частица (\i{аҷаб}, \i{ку})
  \item {\sc ptcp1} — Результативное причастие (-\i{ин})
  \item {\sc ptcp2} — Пассивное причастие (-\i{ак})
  \item {\sc purp} — полный Инфинитив (на -\i{оw})
  \item {\sc q} — вопросительная частица (=\i{о})
  \item {\sc recp} — реципрокальное местоимение ‘друг друга’
  \item {\sc redup} — редупликация
  \item {\sc refl} — рефлексив, возвратное местоимение (\i{ху})
  \item {\sc sg} — единственное число
  \item {\sc sub} — субэссив, ‘под’ (\i{бӣр})
  \item {\sc subd} — подчинительный союз (\i{ца})
  \item {\sc subst} — субстантивирующий суффикс
  \item {\sc sup} — суперэссив, ‘на, над’ (\i{тӣр})
  \item {\sc up} — латив ‘выше’, перемещение к ориентиру выше (\i{пи})
  \item {\sc voc} — звательная частица
\end{itemize}}

\section*{Сведения об авторах}

В этом разделе приведены актуальные (на 2025-й год) сведения об участниках проекта, чьи статьи перепечатаны в настоящем сборнике:

{\small
\begin{itemize}
  \item \i{\b{Арманд}, Елена Евгеньевна} (\i{armandlena@yandex.ru}) — специалист по истории персидского языка и лексике памирских языков; кандидат филологических наук, доцент Института классического Востока и античности НИУ~ВШЭ
  \item \i{\b{Бадеев}, Артём Олегович} (\i{badeev.artem@live.com}) — исследователь демонстративов и систем рода памирских языков; выпускник ВШЭ, сотрудник сектора иранских языков Института языкознания РАН, стажёр-исследователь в Институте сирийского языка Бет Мардуто
  \item \i{\b{Макаров}, Юрий Юрьевич} (\i{ym@pamiri.online}) — исследователь фонетики и фонологии, разработчик и администратор сайта \i{\href{https://pamiri.online}{pamiri.online}}; аспирант Кембриджского университета (Великобритания), выпускник ВШЭ, младший научный сотрудник Сектора типологии Института языкознания РАН
  \item \i{\b{Меленченко}, Максим Глебович} (\i{maksmerben@gmail.com}) — исследователь видовременных систем памирских языков; выпускник ВШЭ, стажёр-исследователь Международной лаборатории языковой конвергенции ВШЭ
  \item \i{\b{Плунгян}, Владимир Александрович} (\i{plungian@iling-ran.ru}) — специалист по морфологии и типологии грамматики; доктор филологических наук, профессор Отделения теоретической и прикладной лингвистики МГУ им.~М.~В.~Ломоносова, академик РАН, заведующий сектором типологии Института языкознания РАН
  \item \i{\b{Рахилина}, Екатерина Владимировна} (\i{rakhilina@gmail.com}) — специалист по лексической семантике и типологии; доктор филологических наук, профессор и руководитель Школы лингвистики НИУ~ВШЭ
  \item \i{\b{Ронько}, Роман Витальевич} (\i{romanronko@gmail.com}) — специалист по диалектологии и грамматическим категориям имени; кандидат филологических наук, доцент Школы лингвистики НИУ~ВШЭ
  \item \i{\b{Рыжова}, Дарья Александровна} (\i{daria.ryzhova@mail.ru}) — специалист по лексической типологии; кандидат филологических наук, доцент Школы лингвистики НИУ~ВШЭ, научный сотрудник Международной лаборатории языковой конвергенции ВШЭ
  \item \i{\b{Чистякова}, Дарья Георгиевна} (\i{dchistiakova@uliege.be}) — исследовательница глагольного морфосинтаксиса памирских языков; аспирантка Льежского и Лёвенского университетов (Бельгия), выпускница ВШЭ
\end{itemize}}

В число авторов также входят наши дорогие коллеги из Института гуманитарных наук в Хороге:

{\small
\begin{itemize}
  \item \i{\b{Ардабаева}, Мадина Шодиевна} (\i{madina.ardabaeva@mail.ru}) — специалист по антропологии и культуре Памира, научный сотрудник Отдела истории, этнографии и археологии Института гуманитарных наук им.~Б.~Искандарова Национальной академии наук Таджикистана
  \item \i{\b{Некушоева}, Шаҳло Саиднуриддиновна} (\i{nekushoeva@bk.ru}) — специалист по лексике памирских языков, кандидат филологических наук, заместитель директора по научной и учебной части Института гуманитарных наук им.~Б.~Искандарова Национальной академии наук Таджикистана
\end{itemize}}

\section*{Благодарности}

Мы сердечно благодарим Институт гуманитарных наук им.~Б.~Искандарова, Университет Центральной Азии (Хорог) и Российско-таджикский славянский университет (Душанбе) за плодотворное сотрудничество и содействие в исследованиях в течение многих лет, а также Факультет гуманитарных наук ВШЭ за финансовое и техническое обеспечение работы проекта.

Выражаем отдельную благодарность Е.~Е.~Арманд, М.~О.~Бажукову, А.~О.~Бузанову, Е.~В.~Рахилиной и А.~А.~Сергиенко за советы и техническую помощь при подготовке сборника.
