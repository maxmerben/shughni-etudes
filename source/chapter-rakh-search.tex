\chapter*{Поле \fakesc{ИСКАТЬ}: шугнанские данные в~свете лексической типологии}
\addcontentsline{toc}{chapter}{\textit{Е.~Рахилина, Д.~Рыжова, М.~Ардабаева}. \textbf{Поле \fakesc{ИСКАТЬ}: шугнанские данные в~свете лексической типологии}}
\setcounter{section}{0}
\chaptermark{Поле \fakesc{ИСКАТЬ}: шугнанские данные в~свете лексической типологии}
\label{chapter-rakh-search}

\begin{customauthorname}
Екатерина Рахилина, Дарья Рыжова, Мадина~Ардабаева
\end{customauthorname}

\begin{englishtitle}
\i{Semantic field {\sc seek}: Shughni data in the light of lexical typology\\{\small Ekaterina Rakhilina, Daria Ryzhova, Madina Ardabaeva}}
\end{englishtitle}

\begin{abstract}
В статье рассматриваются глаголы и глагольные конструкции с семантикой поиска в шугнанском языке на типологическом фоне. На материале словарных данных и результатов элицитационных сессий с носителями языка выявляются особенности лексикализации поля \fakesc{ИСКАТЬ} в современном шугнанском языке, а также делаются предположения относительно динамики развития этой системы с середины ХХ~века до сегодняшнего дня. Мы показываем, что в этой зоне выделяется лексическая доминанта — глагол \i{х̌икӣдоw} — с очень широкой сферой употребления, включающей поиск объектов разных типов (референтных и нереферентных) разными субъектами (людьми или животными) и разными способами (зрительно, тактильно, с помощью речевой деятельности и~т.~п.). В отдельных зонах он конкурирует с другими глаголами и конструкциями: в зоне обыскивания пространства употребляется конструкция \i{тӣр чи бӣр чӣдоw} ‘переворачивать вверх дном’, а также глаголы \i{чӣх̌тоw} ‘смотреть’ и \i{қилāптоw} ‘рыться’, обозначающие поиск глазами и руками соответственно; конструкция \i{сироқ чӣдоw} (буквально ‘поиск делать’) может описывать поиск, осуществляемый посредством расспросов окружающих. Несколько глаголов, обозначавших, согласно словарным данным, поиск в темноте (\i{wарθāптоw}, \i{зи(р)ғāптоw} и \i{wарwарθтоw}), вышли из узуса, еще два — \i{тилāптоw} ‘просить’ и \i{х̌иқāптоw} ‘болтать, выдумывать’ — утратили значение поиска. Описание системы шугнанских глаголов поля \fakesc{ИСКАТЬ} дополняется анализом их полисемии: выделяются случаи употребления этих глаголов вне зоны поиска и объясняется природа таких семантических связей.
\end{abstract}

\vfill

\begin{keywords}
глаголы поиска, иранские языки, колексификация, лексика, лексическая типология, памирские языки, полисемия, шугнанский язык.
\end{keywords}

\begin{eng-abstract}
The paper deals with verbs and verbal constructions with the meaning ‘to seek, to search, to look for’ (the semantic field {\sc seek}) in the Shughni language against a typological background. On the basis of dictionary data and the results of elicitation sessions with native speakers, the peculiarities of the lexicalization of the {\sc seek} field in modern Shughni are revealed, and assumptions are made regarding the dynamics of the development of this system from the middle of the XXth century to the present day. We show that, in this semantic field, there is a clear lexical dominant — the verb \i{x̌ikīdow} — with a very wide scope of use, including search for objects of different types (specific and non-specific) by different subjects (humans or animals) and in different ways (visually, tactually, with the help of speech activity, etc.). In some semantic domains within the field of seeking, it competes with other verbs and constructions: in the domain of searching a space, the construction \i{tīr či bīr čīdow} ‘to turn upside down’ is used, as well as the verbs \i{čīx̌tow} ‘to look’ and \i{qilāptow} ‘to rummage’, denoting visual and tactual search, respectively; the construction \i{siroq čīdow} (lit.~‘search do’) can describe a search carried out by asking others around. Several verbs denoting, according to the dictionary data, a search in the dark (\i{warθāptow}, \i{zi(r)ɣāptow}, and \i{warwarθtow}), went out of use, two more — \i{tilāptow} ‘to ask’ and \i{x̌iqāptow} ‘to chatter, make things up’ — lost the meaning of seeking. The description of the system of the Shughni verbs of seeking is supplemented by an analysis of their polysemy: cases of using these verbs outside the {\sc seek} field are highlighted, and the nature of such colexifications is discussed.
\end{eng-abstract}

\begin{eng-keywords}
colexification, Iranian, lexical typology, lexicon, Pamir area, polysemy, Shughni, verbs of search.
\end{eng-keywords}

\begin{acknowledgements}
Статья подготовлена в ходе проведения исследования №~23-00-012 «Смежность семантических полей в типологической перспективе» в рамках Программы «Научный фонд Национального исследовательского университета «Высшая школа экономики» (НИУ~ВШЭ).
\end{acknowledgements}

\begin{initialprint}
\fullcite{rakhilina_etal2023}\end{initialprint}

\section{Введение} \label{search-intro}

Настоящая работа посвящена полю \fakesc{ИСКАТЬ} в шугнанском языке — в типологическом контексте. Контекстом служат системные исследования поля {\sc искать–находить} на материале более 30~языков из индоевропейской, абхазо-адыгской, нахско-дагестанской, чукотско-камчатской, уральской, алтайской и сино-тибетской семей в \parencite{eureka2018}, выполненные в рамках подхода к анализу лексики, разработанного Московской лексико-типологической группой. В ходе этих исследований, как обычно в лексической типологии определённой семантической зоны, ср.~\parencites{maisak_rakhilina2007}{rakhilina_etal2020}, были выявлены такие семантические противопоставления, которые в языках мира могут выражаться лексически — а значит порождать квазисинонимы со значением ‘искать’. Как неоднократно было показано на разном лексическом материале, в глагольной лексике источником противопоставлений такого рода является аргументная структура глаголов, включая типичные способы заполнения аргументных слотов. Поэтому в наших работах мы, следуя филморовской фреймовой семантике \parencite{fillmore1976}, говорим о \b{фреймовой типологии} глагольной лексики \parencite{rakhilina_reznikova2013}, имея в виду ситуации, обозначаемые глаголом, и их типовых участников.

Для семантического поля \fakesc{ИСКАТЬ} тоже характерны лексические противопоставления, связанные с меной типов аргументов: субъекта, объекта и места поисков. Помимо этого, искать можно разными способами — и глаголы с семантикой поиска отличаются друг от друга в том числе тем, специфицируют ли они тип действий, совершаемых субъектом с целью найти искомый объект. Часто оказывается, что на синхронном уровне способ поиска не закреплён в семантике глагола (такие слова мы называем «недоспецифицированными», см.~\hyperref[search-unspec]{Раздел~2}), но внутренняя форма или диахронические данные показывают, что семантическим источником для этого глагола были вполне определённые действия, ср.~англ.~\i{look for}, очевидно восходящее к значению ‘смотреть’. Поэтому, говоря о глаголах поля \fakesc{ИСКАТЬ}, мы принимаем во внимание не только их сегодняшнее значение, но и по возможности их этимологию.

Шугнанский язык, с которым мы будем работать, принадлежит к восточноиранским языкам, на нем говорит порядка 100~000 человек в Горно-Бадахшанской автономной области в Таджикистане и в провинции Бадахшан в Афганистане. Ареально шугнанский принадлежит к памирским языкам. В нашем исследовании мы опираемся на словарные и собственные полевые данные. В словаре шугнанского языка Д.~Карамшоева [\cite*{karamshoev1988}; \cite*{karamshoev1991}; \cite*{karamshoev1999}] имеется порядка десяти лексических единиц, способных выражать значение поиска. Мы уточнили этот материал, а также дополнили его с помощью элицитации по типологической анкете в ходе экспедиции НИУ~ВШЭ в г.~Хорог (Таджикистан) при поддержке Университета Центральной Азии в 2022~году. Каждый контекст обсуждался не менее чем с четырьмя носителями шугнанского языка. В результате мы получили представление о синхронном состоянии лексической системы в этой зоне и благодаря более ранним словарным данным можем строить предположения о динамике развития этой системы примерно с середины XX~века до сегодняшнего дня.

В статье, помимо введения, четыре раздела. В следующем, \hyperref[search-unspec]{втором разделе} мы обсудим понятие недоспецифицированности в применении к лексике. Далее, в \hyperref[search-frames]{Разделе~3}, мы обозначим типологически релевантные противопоставления и фреймы \fakesc{ИСКАТЬ}, представленные в шугнанском, а в \hyperref[search-sources]{четвёртом разделе} обсудим возможные источники для лексификации зоны поиска в этом языке. Итоги разбора данных подводятся в \hyperref[search-results]{Разделе~5 «Результаты и обсуждение»}. Там же мы предложим наше понимание значимости шугнанских данных как для методики работы с недоспецифицировнными глаголами (в частности, глаголами поиска), так и для лексической типологии в целом.

\section{Недоспецифицированная лексика} \label{search-unspec}

Типичная для глагольной семантической зоны ситуация (или фрейм), как правило, определяется набором аргументов-слотов и их свойствами (тоже типичными): разные объекты по-разному прыгают, падают, плавают, качаются, вращаются, звучат и пр. Одно дело, когда качается маятник (и другие похожие на него жестко закреплённые сверху объекты, как люстра), а другое дело — мягкая вертикальная поверхность, как занавеска, и совсем иначе — жёсткий объект, закреплённый снизу, как дерево \parencites{rakhilina_prokofieva2005}{shapiro2015}. Мена свойств аргумента естественным образом трансформирует, так сказать, облик ситуации в целом, семантистам остается только исчислить возможный круг таких аргументов.

Тем более очевиден переход от одного фрейма к другому, если меняется или перепрофилируется \b{набор} аргументов. В теоретической семантике эти эффекты обычно связываются с явлением метонимии, которая порождает полисемию глагола в конкретном языке \parencites{langacker1987}{paducheva2004}, — но типологический взгляд на полисемию, вызванную меной аргументной структуры, заставляет ожидать не только расхождения значений, но и расхождений лексем. Именно так устроено поле вращения. Для вращения вокруг собственной оси ось как бы инкорпорирована в семантику глагола и не соответствует никакому аргументу — такие фреймы, как, например, ‘поворачиваться’ или ‘вращаться (о двери)’ одноместны. Но если ориентир вращения внешний, как в (\gethref{exsearch1}–\gethref{exsearch2}), глагол становится двухместным — и может, как видно уже на этих примерах, лексически отличаться от одноместных \parencite{krugliakova2010}.

\ex<exsearch1>
\i{…\underline{лодка} стала \b{описывать круги} вокруг плавающего в море хозяина.} \trailingcitation{[«Столица», 13.10.1997 (НКРЯ)]}
\xe

\ex<exsearch2>
\i{…\underline{птицы} \b{кружат над} кронами гигантских тополей…} \trailingcitation{[«Дорога в снегопад», А.~Уткин, 2008–2010 (НКРЯ)]}
\xe

Существенно, что добавление и упразднение одного аргумента сразу же затрагивает остальные и поэтому кардинально меняет то, что мы только что назвали «обликом» ситуации, а могли бы назвать «гештальтом»: птицы, как правило, кружат над, а не вертятся или вращаются, тогда как двери не могут двигаться вокруг внешних объектов, и так далее. Гештальтность в этом смысле — как взаимообусловленность свойств аргументов — в принципе свойственна многоместным предикатам, ср.~представление о том, что звери иначе едят жидкое, чем люди \parencite{croft2009}. Она усиливает противопоставления между фреймами и приводит к возникновению лексических противопоставлений, ср.~русск. \i{лакать} vs \i{пить} (подробнее о глаголах еды и питья в языках мира см.~\parencite{newman2009}). Как исследователям нам могут быть удивительны отдельные лексикализации, которые встречаются в языках мира. Так, для глаголов плавания локативный аргумент кажется незначимым для всех фреймов — плавания человека, плавания судов, плавания по течению воды или на поверхности жидкости \parencite{maisak_rakhilina2007} и практически фиксированным как водная субстанция (плавание в супе или в крови глубоко периферийно на общем фоне). Однако в индонезийском \parencite{lander_kramarova2007} обнаруживается возможность продуктивных лексикализаций плавания по типу водоёма (буквально:~‘озерить’ как ‘плавать по озеру’ или ‘проливить’ как ‘переплывать пролив’).

Все сказанное касается общих методов лексической типологии — и общего устройства лексически выраженных противопоставлений глаголов в языках мира, как мы их себе представляем — и в полной мере относится и к глаголам с семантикой ‘искать’ (соответствующие противопоставления в применении к шугнанскому материалу мы рассмотрим подробнее в \hyperref[search-frames]{Разделе~3}). Между тем поле \fakesc{ИСКАТЬ} обладает и некоторыми особенными свойствами, которые усложняют соответствующие ему лексические системы по сравнению с обычными. Парадокс \fakesc{ИСКАТЬ} состоит в следующем. Очевидно, что глаголы с семантикой ‘искать’ обозначают последовательность физических действий, всегда имеющую определённую цель, и во многих случаях предполагают и заранее заданный конкретно-референтный объект (как в русск. \i{искал свои очки}). Такие глаголы должны были бы быть похожи на другие глаголы с физическим значением — как ‘пилить’, ‘солить’ или ‘поливать’ — и получить похожее описание, в котором последовательно «прописаны» все действия субъекта (и возможных инструментов) над объектом, ср.~(1)~взять пилу, (2)~привести её в контакт с объектом (обычно деревом), (3)~двигать пилу определённым образом — чтобы с её помощью разделить объект на части. Однако такого описания глаголу \i{искать} мы дать не можем, потому что не знаем, из каких именно действий состоит свойственная ему последовательность. Толкование \fakesc{ИСКАТЬ}, таким образом, остаётся так сказать «недоспецифицированным»: мы знаем, что субъект хочет получить себе некоторый объект, которого у него нет (но, возможно, был некоторое время назад), и предпринимает ряд действий, в каждом случае свой. В зависимости от обстоятельств он заглядывает под диван, осматривает местность, шарит в темноте и прочее — эти действия вполне конкретные, но закрепить их в толковании глагола \i{искать} (как мы это делаем для \i{пилить} или \i{солить}) мы не можем.

Поле \fakesc{ИСКАТЬ} не вполне уникально в этом отношении. Полностью недоспецифицированных предикатов, у которых не определено всё пространство от презумпции до результата, действительно не так много, но глаголы со значением ‘прятать’, ‘исчезать’, ‘готовиться’ или, например, ‘мстить’ устроены так же (специальному исследованию глаголов прятания как недоспецифицированных в типологическом аспекте посвящена статья \parencite{reznikova2022}, данные адыгейского в том же ключе освещены в \parencite{bagirokova_ryzhova2022}, ср.~также близкий, но не тождественный недоспецифицированным глаголам класс «интерпретационных» глаголов, выделяемых в \parencite{apresjan2004}, и понятие \i{underspecification} в \parencite{geeraerts2016}). С теоретической точки зрения недоспецифицированные предикаты в определённом смысле похожи на абстрактные значения — например, эмоции. У абстрактных значений основная ситуация часто не может быть названа и тоже остается «невидимой», но только по другой причине: потому что не принадлежит физическому миру. Абстрактные ситуации «находят выход» в том, что лексифицируются за счет метафорических переносов предикатов с физическим значением, ср.~\i{кипеть} или \i{расстраиваться}. Эти предикаты служат для них семантическими источниками \parencite{koevecses2000}, своими для каждой отдельной семантической зоны. Они составляют интереснейший объект типологического исследования (см.,~например, \parencite{apresjan1997}). Лексификация недоспецифицированных предикатов тоже имеет свои источники. Другое дело, что поскольку это предикаты физического, а не абстрактного плана, здесь нет метафоризации, и действуют иные общие для них механизмы. Однако семантические источники тоже каждый раз свои и тоже представляют объект для типологизации. Мы поговорим о них в \hyperref[search-frames]{Разделе~3} и предложим их описание для шугнанского.

\section{Фреймы \fakesc{ИСКАТЬ} и их лексификация в шугнанском} \label{search-frames}

\subsection{Типологический экскурс}

Как уже было сказано, основные противопоставления внутри семантической зоны обычно связаны с аргументной структурой. Предикаты со значением ‘искать’ можно считать трёхместными: Х ищет Y в Z, где X — прототипически одушевлённый (обычно человек), Y — объект поиска, единичный или множественный, Z — место (пространство, контейнер, а также, метонимически, вещество, занимающее такое место). Такая трактовка аргументной структуры строится на таком представлении о семантике \fakesc{ИСКАТЬ}, согласно которому Х не знает, где находится Y; хочет это узнать (и, возможно, получить Y в собственность или пользование); предполагает, что Y находится в месте или местах Z и производит в этих местах некоторые действия (ср.~толкование для первого значения русского глагола \i{искать} в \parencite{uryson2023}). Косвенным свидетельством в пользу такой трактовки и такой аргументной структуры служат и противопоставления в зоне \fakesc{ИСКАТЬ}, которые, как показывают типологические исследования в \parencite{eureka2018}, связаны либо с типом субъекта Х, либо с типом объекта Y, либо с местом Z.

Для субъекта X речь идет о таксономической оппозиции человек / животное: в языках встречаются глаголы поиска, относящиеся исключительно к животным и маркирующие в этом случае совершенно особый фрейм поиска животным еды или добычи (но не, например, потерянного детеныша, убежища, собственного хвоста и прочего), ср.~именно в этом исходном значении русск. \i{рыскать} (преимущественно о волках\fn{В современном русском это исходное значение не так употребительно. Чаще рыскать используется метафорически — и в этом качестве применимо к людям: \i{…перепуганные фанаты принялись \b{рыскать} по интернету в поисках какой-либо информации} [Lenta.ru, 2017.11].}). Применительно к объекту поиска Y таксономия оказывается не важна, зато важна референтность искомого объекта: представляет ли он конкретную вещь, конкретного человека / животного, ср.~‘искать потерянные ключи / очки’ — или задано только множество свойств, которым должен обладать объект, ср.~пример Е.~В.~Падучевой \i{Он ищет новую секретаршу} \parencite[94]{paducheva1985}. В русском языке референтный и нереферентный объект поиска противопоставлены в приставочных глаголах \i{разыскивать} (только о референтном, конкретном и утраченном) vs. \i{подыскивать} (о нереферентном объекте, носителе определённых интересующих субъекта свойств), ср.~(\gethref{exsearch3}–\gethref{exsearch4}). Наконец, возможны такие лексикализации ситуации поиска, которые переносят фокус с объекта поиска на зону поиска, ср.~англ.~\i{search} в противовес \i{look for}, или узко специализированный русский глагол \i{обыскивать}, при котором объект поиска, по выражению Е.~В.~Падучевой, «уходит в тень» настолько, что вообще не может быть поверхностно выражен непосредственно при \fakesc{ИСКАТЬ}, ср.~(\gethref{exsearch5}).

\ex<exsearch3>
\i{И долго потом пришлось \b{разыскивать} таксиста, чтобы вернуть ему девяносто девять рублей.} \trailingcitation{[«Иллюзии без иллюзий», И.~Кио, 1995–1999 (НКРЯ)]}
\xe

\ex<exsearch4>
\i{Я вспомнил рассказы охотников о встречах с медведями и начал даже \b{подыскивать} дерево, куда можно будет залезть, если вдруг покажется зверь.} \trailingcitation{[«Первое дело», Ф.~Искандер, 1956 (НКРЯ)]}
\xe

\pex<exsearch5>
\a<a> \i{\b{Искали} пропажу по всему городу.}
\a<b> \i{\b{Обыскали} весь город (*от / *для пропажи) в поисках пропажи.}
\xe

\subsection{Доминантный глагол \i{х̌икӣдоw}}

Обсуждение современного шугнанского материала мы начнём со стоящего особняком глагола \i{х̌икӣдоw}, который можно назвать доминантным. В словаре он иллюстрируется несколькими однотипными примерами, в которых в роли объекта поиска всегда выступает одушевлённое существо. Обычно этот объект референтный, ср.~(\gethref{exsearch6})\fn{Поиск по словарю, транслитерация и (частично) глоссирование примеров произведены с помощью платформы \i{\href{https://pamiri.online}{pamiri.online}} \parencite{makarov_etal2022}.}, хотя без более широкого контекста это не всегда очевидно, ср.~(\gethref{exsearch7}–\gethref{exsearch8}). Как видно по примерам, глагол \i{х̌икӣдоw} переходный, то есть участник с ролью объекта поиска занимает при нём позицию синтаксического прямого объекта (прямой объект выделен в примерах подчёркиванием)\fn{В современном шугнанском языке субъект и объект в переходной клаузе в большинстве случаев никак не маркируются, и определение синтаксических ролей производится на основе порядка слов, который в шугнанском обычно SOV, и/или глагольного согласования с субъектом. Исключение составляют местоимения, у которых есть прямая и косвенная формы, а также некоторые особые контексты, в которых прямой объект может маркироваться элативным предлогом \i{аз}. Подробнее о шугнанских переходных клаузах и дифференцированном объектном маркировании см.~[\hyperref[chapter-chist-dom]{Чистякова 2022a}].}.

\ex<exsearch6>
\begingl
\gla Ту=та мāш \b{х̌икар-и}=йо?//
\glc {\sc pron.2sg=fut} {\sc pron.1pl} искать-{\sc prs.2sg=q}//
\glft ‘Ты нас \b{ищешь}, что ли?’ \trailingcitation{\parencite[264]{karamshoev1999}}//
\endgl \xe

\ex<exsearch7>
\begingl
\gla Одам-ен=ен дузд \b{х̌икӯд}.//
\glc человек-{\sc pl=3pl} вор искать.{\sc pst}//
\glft ‘Люди \b{разыскивали} вора.’ \trailingcitation{\parencite[264]{karamshoev1999}}//
\endgl \xe

\ex<exsearch8>
\begingl
\gla Ага wам вирод йак сол=ец чӣр ~~~~~~~~~~~~~~~~~~~~~~~~~~~~~~~~~ ца \b{х̌икӣрт}, на-виред.//
\glc если {\sc d3.f.sg.o} брат один год={\sc lim2} горный\_козёл ~ {\sc subd} искать.{\sc prs.3sg} {\sc neg}-найти.{\sc prs.3sg}//
\glft ‘Если её брат даже целый год \b{будет искать} горного козла — не найдёт.’ \trailingcitation{\parencite[264]{karamshoev1999}}//
\endgl \xe

Общение с носителями языка показывает, что этот глагол используется значительно шире. Так, в позиции субъекта при нём может быть не только человек, но и животное (\gethref{exsearch9})\fn{Примеры, при которых не указан источник, получены авторами в ходе элицитации.}; объект может быть и неодушевлённым (\gethref{exsearch10}), и нереферентным (\gethref{exsearch11}).

\ex<exsearch9>
\begingl
\gla Пиш тар дарго ху-рд аwқот \b{х̌икӣрт}.//
\glc кошка {\sc eq} двор {\sc refl-dat} пища искать.{\sc prs.3sg}//
\glft ‘Кошка во дворе \b{ищет} себе еду.’//
\endgl \xe

\ex<exsearch10>
\begingl
\gla Уз=ум бийор рӯз=и дароз ху кашилйок \b{х̌икӯд}.//
\glc {\sc pron.1sg=1sg} вчера день={\sc ez} длинный {\sc refl} кошелек искать.{\sc pst}//
\glft ‘Я вчера весь день \b{искал} свой кошелёк.’//
\endgl \xe

\ex<exsearch11>
\begingl
\gla Уз англиси муалим \b{х̌икар-ум}.//
\glc {\sc pron.1sg} английский учитель искать-{\sc prs.1sg}//
\glft ‘Я \b{ищу} учителя по английскому языку [в новую школу].’//
\endgl \xe

Поиск нереферентного объекта предполагает перебор различных элементов некоторого множества с целью подобрать такой, который будет соответствовать изначально заданным свойствам, однако множество этих элементов заранее не определено и может даже полностью отсутствовать. Семантически близкая ситуация — перебор конкретных, определённых элементов с целью выбора одного или нескольких из ограниченного множества — обычно получает своё собственное лексическое оформление, ср.~русск. \i{выбирать} или английское \i{choose}. В шугнанском, однако, и на эту область распространяется базовый глагол поиска \i{х̌икӣдоw}, см.~(\gethref{exsearch12}).

\ex<exsearch12>
\begingl
\gla Уз=ум wи кӯдак=ард мӯн дāк-т=ат, ~~~~~~~~~ йу=йи дер=ец \b{х̌икӯд}, чиду̊м зêз-д.//
\glc {\sc pron.1sg=1sg} {\sc d3.m.sg.o} ребёнок={\sc dat} яблоко дать-{\sc pst=and2} ~ {\sc d3.m.sg=3sg} долго={\sc lim2} искать.{\sc pst} какой брать-{\sc prs.3sg}//
\glft ‘Я дала ребенку яблоки, и он долго \b{выбирал}, какое [из них] взять.’//
\endgl \xe

Более того, некоторые носители допускают помещение локативного участника~Z (пространство поиска) в позицию прямого объекта~Y, ср.~(\gethref{exsearch13}–\gethref{exsearch14})\fn{Примеры, которые не всем носителям кажутся допустимыми, мы помечаем знаком «?».}. Однако чаще участник, обозначающий пространство поиска при глаголе \i{х̌икӣдоw}, вводится тем или иным локативным показателем: предлогом \i{ар} с базовой семантикой перемещения вниз, лативным послелогом =\i{(а)рд} (\gethref{exsearch15})\fn{Отметим, что вариант \i{ху сӯмкā=рд} в примере (\gethref{exsearch15}) некоторыми носителями интерпретируется как поиск не внутри сумки, а на её поверхности.} или посессивно-локативным послелогом =\i{ҷа}, который присоединяется к обозначениям людей (\gethref{exsearch16}).

\ex<exsearch13>
\begingl
\gla \ljudge{\b{?}}Уз=ум фук=аθ ҷо \b{х̌икӯд}.//
\glc {\sc pron.1sg=1sg} весь={\sc int} место искать.{\sc pst}//
\glft ‘Я всё \b{обыскал}’. (букв. ‘всё пространство’)//
\endgl \xe

\ex<exsearch14>
\begingl
\gla \ljudge{\b{?}}Wи дузд=ен \b{х̌икӯд}=ат, пӯл-ен ~~~~~~~~~~~~~~~~ wи=ҷа на-вирӯд.//
\glc {\sc d3.m.sg.o} вор={\sc 3pl} искать.{\sc pst=and2} деньги-{\sc pl} ~ {\sc d3.m.sg.o=p.loc} {\sc neg}-найти.{\sc pst}//
\glft ‘Вора \b{обыскали}, но денег у него не нашли.’//
\endgl \xe

\ex<exsearch15>
\begingl
\gla Уз=ум ар ху сӯмкā / ху сӯмкā=рд ~~~~~~~~~~~~~~~~~~~~~ чӣд wих̌ӣӡ-ен \b{х̌икӯд}.//
\glc {\sc pron.1sg=1sg} {\sc down} {\sc refl} сумка ~~~~~~ {\sc refl} сумка={\sc dat} ~ дом ключ-{\sc pl} искать.{\sc pst}//
\glft ‘Я \b{искал(а)} в сумке ключи от дома.’//
\endgl \xe

\ex<exsearch16>
\begingl
\gla Ку ди=ҷа \b{х̌икар}.//
\glc {\sc ptcl} {\sc d2.m.sg.o=p.loc} искать[{\sc imp}]//
\glft ‘\b{Обыщи}-ка его. ’ \trailingcitation{\parencite[264]{karamshoev1999}}//
\endgl \xe

\subsection{Обыск пространства}

Чтобы подчеркнуть интенсивность поиска в ограниченном пространстве, используют конструкцию \i{тӣр чи бӣр чӣдоw} (верх {\sc cont1} низ делать) ‘переворачивать вверх дном’, буквально ‘верх на низ делать’, при которой обозначение пространства поиска занимает позицию прямого дополнения. Это выражение лучше всего подходит для описания поиска (обычно безрезультатного) неодушевлённого объекта в небольшом контейнере, как в примере (\gethref{exsearch17}). Примеры типа (\gethref{exsearch18}), где объект одушевлённый, а пространство поиска довольно большое, допускаются не всеми носителями. Для описания поиска одушевлённого объекта может использоваться глагол \i{чӣх̌тоw} ‘смотреть’ (\gethref{exsearch19}). Этот глагол переходный, но в позицию прямого объекта при нём помещается объект, а не пространство поиска.

\ex<exsearch17>
\begingl
\gla Уз=ум ху сӯмкā \b{тӣр} \b{чи} \b{бӣр} \b{чӯд} ~~~~~~~~~~~~~~~ ас.рӯйи ху wих̌ӣӡ-ен.//
\glc {\sc pron.1sg=1sg} {\sc refl} сумка верх {\sc cont1} низ делать.{\sc pst} ~ из\_за {\sc refl} ключ-{\sc pl}//
\glft ‘Я \b{перерыл(а)} всю сумку в поисках ключей.’//
\endgl \xe

\ex<exsearch18>
\begingl
\gla \ljudge{\b{?}}Одам-ен=ен фук х̌āр \b{тӣр} \b{чи} \b{бӣр} \b{чӯд}=ат, wи дузд=ен на-вирӯд.//
\glc человек-{\sc pl=3pl} весь город верх {\sc cont1} низ делать.{\sc pst=and2} {\sc d3.m.sg.o} вор={\sc 3pl} {\sc neg}-найти.{\sc pst}//
\glft ‘Люди \b{обыскали} (=~перевернули) весь город, но вора не нашли.’//
\endgl \xe

\ex<exsearch19>
\begingl
\gla Мāш=ам фук=аθ ҷо ху жоw \b{чӯх̌-т}.//
\glc {\sc pron.1pl=1pl} весь={\sc int} место {\sc refl} корова смотреть-{\sc pst}//
\glft ‘Мы \b{искали} свою корову повсюду.’ (букв. ‘по всему пространству’)//
\endgl \xe

Еще более узкой сферой действия в этой зоне, чем конструкция \i{тӣр чи бӣр чӣдоw} ‘переворачивать вверх дном’, характеризуется глагол \i{қилāптоw}, который по словарю \parencite{karamshoev1999} имеет два значения: ‘искать’ и ‘возиться, работать’. Значение ‘работать’ не подтверждается носителями, однако этот глагол известен всем опрошенным в значении ‘рыться, возиться (о животных)’, в том числе в поисках еды, ср.~(\gethref{exsearch20}–\gethref{exsearch21}). Аналогично и про человека можно сказать, что он «роется» в своих вещах в поисках чего-либо (\gethref{exsearch22}). Этот глагол в значении обыска пространства используется только в том случае, если человек действительно перебирает что-то руками. Глагол \i{қилāптоw} непереходный: участник с ролью пространства поиска вводится пространственными предлогами (как правило, предлогом \i{ар} ‘вниз’, ср.~(\gethref{exsearch20})), а объект поиска маркируется послелогом =\i{ҷāт} — показателем причины или цели, ср.~(\gethref{exsearch21}–\gethref{exsearch22}).

\ex<exsearch20>
\begingl
\gla Йу куд ар зимāδ \b{қилāп-т}.//
\glc {\sc d3.m.sg} собака {\sc down} земля рыться-{\sc pst}//
\glft ‘Собака \b{рылась} в земле.’//
\endgl \xe

\ex<exsearch21>
\begingl
\gla Пиш дарго=ра аwқот=ҷāт \b{қилāп-т}.//
\glc кошка двор={\sc dat} пища={\sc cause} рыться-{\sc pst}//
\glft ‘Кошка \b{рылась} во дворе в поисках еды.’//
\endgl \xe

\ex<exsearch22>
\begingl
\gla Йā ɣ̌иник=и фук бӯӣн-ен=ард \b{қилап-т} мӯн=ҷāт.//
\glc {\sc d3.f.sg} женщина={\sc 3sg} весь мешок-{\sc pl=dat} рыться-{\sc pst} яблоко={\sc cause}//
\glft ‘Женщина \b{перерыла} все мешки в поисках яблока.’//
\endgl \xe

К последним двум глаголам — \i{чӣх̌тоw} и \i{қилāптоw} — мы вернёмся в следующем разделе в связи с проблемой семантических источников.

\section{Семантические источники \fakesc{ИСКАТЬ} и утраченные шугнанские глаголы} \label{search-sources}

Источниками недоспецифицированных глаголов во многих случаях становятся предикаты, которые уточняют «пропущенную» в их семантике зону — в нашем случае, способ поиска. Значит, нам нужно представить себе, какого рода усилия предпринимает человек, когда он ищет, и какие инструменты использует. Оказывается, что часто с точки зрения языков мира человек ищет глазами, то есть смотрит (ср.~русск.~\i{высматривать}, англ.~\i{look for} или чешск. \i{hledat}), но может пользоваться и другими каналами восприятия, например, слушать (ср.~\i{подслушивать}), — а звери обычно полагаются на нюх (ср.~здесь метафорическое \i{вынюхивать}). Ещё один значимый инструмент поиска — это руки (тактильный поиск): именно руками \i{шарят} в темноте, \i{переворачивают} (\i{всё вверх дном}), а также, с помощью дополнительных инструментов и исключительно метафорически, \i{шерстят}, \i{прочёсывают}, \i{перерывают}, \i{копают}, \i{перебирают} и так далее. В основном это можно делать только в поисках потерянного, а в поисках неопределённого предмета с нужными свойствами подходят глаголы речи, ср.~\i{выведывать}, \i{(вы/рас)спрашивать}, \i{допрос}\fn{Понятно, что жёсткого распределения функций между тактильными глаголами и глаголами речи быть не может. Например, в группу поиска предмета по свойствам попадает и русск.~\i{подбирать} — как в \i{подбирать персонал}.}. И наконец, человек, как охотник, может \i{выслеживать кого-то}, а также просто \i{ходить в поисках} (ср.~русск.~\i{находить} / \i{найти}), и тогда главным инструментом оказываются ноги\fn{О том, что, в частности, русское \i{следить} восходит именно к идее движения за кем-то следом, а не наблюдения, см.~\parencite{rakhilina_bychkova2022}.}. Некоторые из этих источников описаны С.~М.~Толстой на славянском материале \parencite{tolstaya2013}, более широкая типологическая картина дана в \parencite{eureka2018}.

Имея в виду эту картину, посмотрим теперь на шугнанский материал. Во-первых, заметим, что наш доминантный глагол \i{х̌икӣдоw}, будучи недоспецифицированным, не различает способы поиска и охватывает почти все типологически выделенные возможности, ср.~(\gethref{exsearch23}) для поиска глазами, (\gethref{exsearch24}) для поиска руками или поиска, предполагающего разного рода речевую деятельность (ср.~пример (\gethref{exsearch11}), который подразумевает собеседование с претендентом на должность учителя).

\ex<exsearch23>
\begingl
\gla Йу=йи ху ну̊м ар wи рӯйихāт \b{х̌икӯд}.//
\glc {\sc d3.m.sg=3sg} {\sc refl} имя {\sc down} {\sc d3.m.sg.o} список искать.{\sc pst}//
\glft ‘Он \b{искал} своё имя в списке.’//
\endgl \xe

\ex<exsearch24>
\begingl
\gla Уз=ум ху δуст қати ар х̌ац ху тин \b{х̌икӯд}.//
\glc {\sc pron.1sg=1sg} {\sc refl} рука {\sc com} {\sc down} вода {\sc refl} монета искать.{\sc pst}//
\glft ‘Я \b{искал} рукой в воде свою монету.’//
\endgl \xe

Однако в шугнанской системе есть и специализированные глаголы со своими источниками, подпадающие под семантический переход ‘конкретный способ поиска’ $\rightarrow$ ‘общий поиск’. Д.~Карамшоев выделяет целую серию лексических единиц, демонстрирующих такие модели.

Первыми назовем глагол \i{тилāптоw} и конструкцию \i{сироқ чӣдоw} — их объединяет значение поиска как расспроса, причем обе единицы заимствованы из таджикского. Глагол \i{тилāптоw}, согласно \parencite[138]{dodykhudoeva1988}, восходит к таджикскому \i{талабидан} ‘требовать, просить, желать, приглашать’. В таджикско-русском словаре \parencite{saymiddinov_etal2006} есть также сложный глагол \i{талаб кардан} (букв. ‘требование делать’), который выражает значения ‘требовать; просить; искать’, — они же представлены в словаре Д.~Карамшоева для шугнанского \i{тилāптоw}. Однако в современном языке, по нашим полевым данным, у \i{тилāптоw} осталось только значение ‘просить’, а метафора поиска полностью утрачена. Существительное \i{сироқ} связано с таджикским \i{суроғ} ‘осведомление, расспрашивание, разыскивание, розыск, поиск’ (в дарвазских говорах — \i{сироҳ}, см.~\parencite[269]{rozenfeld1956}) и, согласно словарю Д.~Карамшоева, в шугнанском языке тоже имеет значение ‘расспросы, поиски’ (а также ‘визит, поход в гости’ и ‘признание, уважение’). Шугнанский сложный глагол \i{сироқ чӣдоw} (буквально ‘поиск делать’) может обозначать поиск одушевлённого объекта посредством долгих расспросов окружающих, причем объект поиска при этом сложном глаголе занимает позицию прямого дополнения (\gethref{exsearch25}). В словаре Д.~Карамшоева выделяется также фразеологизованное предложное сочетание с аналогичным значением \i{дар-сироқ δêдоw}, где \i{дар} — таджикский локативный предлог ‘в’, а \i{δêдоw} — глагол ‘падать, ударяться’. Опрошенные нами носители эту конструкцию не употребляют.

\ex<exsearch25>
\begingl
\gla Ку wи му нархар ар Пастев \b{сироқ} \b{кин}.//
\glc {\sc ptcl} {\sc d3.m.sg.o} {\sc pron.1sg.o} осёл {\sc down} Пастев поиск делать[{\sc imp}]//
\glft ‘\b{Расспроси}-ка в Пастеве про моего осла.’ \trailingcitation{\parencite[579]{karamshoev1991}}//
\endgl \xe

Два других глагола, соответствующих типологическим ожиданиям, мы уже упоминали: это \i{чӣх̌тоw} ‘смотреть’ и \i{қилāптоw} ‘рыться’. Глагол \i{чӣх̌тоw} ‘смотреть’ ожидаемо обозначает «зрительный поиск» (\gethref{exsearch26}), но может употребляться и шире. Заметим здесь, что в более широких и общих, семантически производных употреблениях \i{чӣх̌тоw} часто выступает в форме Императива, см.~(\gethref{exsearch27}).

\ex<exsearch26>
\begingl
\gla Ақоби фук=аθ ҷо риwух̌-т=ат, аwқот \b{чӯх̌-т}.//
\glc орёл весь={\sc int} место летать.{\sc m-pst=and2} пища смотреть-{\sc pst}//
\glft ‘Орёл повсюду летает, \b{высматривает} добычу.’//
\endgl \xe

\ex<exsearch27>
\begingl
\gla Ку ар му нāн ебак дӯсик пӯл \b{чис}=ху, му-рд дāк.//
\glc {\sc ptcl} {\sc down} {\sc pron.1sg.o} мать карман немного деньги смотреть[{\sc imp}]={\sc and1} {\sc pron.1sg.o-dat} дать[{\sc imp}]//
\glft ‘\b{Поищи} в кармане моей матери немного денег и дай мне.’ \trailingcitation{\parencite[367]{karamshoev1999}}//
\endgl \xe

Второй из уже упомянутых глаголов, \i{қилāптоw}, тоже развивался в полном соответствии с типологическим каноном: как мы помним, основное его значение — ‘рыться (о животных)’ — развило зооморфную метафору ‘перебирать что-то руками в поисках чего-то (о человеке)’, см.~примеры (\gethref{exsearch20}–\gethref{exsearch22}) выше\fn{Здесь нужно заметить, что словарная статья для этого глагола устроена иначе. Д.~Карамшоев первым его значением называет ‘искать’, а вторым — ‘возиться, работать’, однако, по всей видимости, оба они производны от более конкретного образа — поведения животного, — «пропущенного» в словарном толковании, но хорошо известного всем опрошенным нами носителям.}. Однако он может употребляться и шире: обозначать поиск ощупью (\gethref{exsearch28}) или даже просто без зрительной поддержки, ср.~(\gethref{exsearch29})\fn{Правда, последнее расширение допускается не всеми носителями.}. При этом объект поиска должен быть обязательно неодушевлённым, даже если поиск не зрительный (\gethref{exsearch30}) — это сочетаемостное свойство наследуется из исходных контекстов, но синтаксис новых употреблений меняется, сближаясь с синтаксической моделью ‘искать’: в новых употреблениях \i{қилāптоw} может принимать прямой объект, см.~(\gethref{exsearch28}–\gethref{exsearch29}).

\ex<exsearch28>
\begingl
\gla Ту писен=āм ар х̌ац ~~~~~~~~~~~~~~~~~~~~~~~~~~~~~~~~~~~~~~~~~~~ цу̊нд \b{қилāп-т}, на-вӯд=āм.//
\glc {\sc pron.2sg} точильный\_камень={\sc 1pl} {\sc down} вода ~ сколько рыться-{\sc pst} {\sc neg}-нести.{\sc pst=1pl}//
\glft ‘Твой точильный камень [мы] сколько ни \b{искали} в воде, не нашли.’ \trailingcitation{\parencite[508]{karamshoev1999}}//
\endgl \xe

\ex<exsearch29>
\begingl
\gla \ljudge{\b{?}}Уз=ум торик-и=нди дер=ец ху чӣд \b{қилāп-т}.//
\glc {\sc pron.1sg=1sg} темно-{\sc subst=loc} долго={\sc lim2} {\sc refl} дом искать-{\sc pst}//
\glft ‘В темноте я долго \b{искал} свой дом.’ \trailingcitation{\parencite[508]{karamshoev1999}}//
\endgl \xe

\ex<exsearch30>
\begingl
\gla Ужи торик сут=ат, уз=ат му вирод=āм дер=ец йакдигар \b{х̌икӯд} / *қилāп-т.//
\glc уже темно стать.{\sc pst.m.sg=and2} {\sc pron.1sg=and2} {\sc pron.1sg.o} брат={\sc 1pl} долго={\sc lim2} {\sc recp} искать.{\sc pst} ~~~~~~ рыться-{\sc pst}//
\glft ‘Уже стемнело, и мы с братом долго \b{искали} друг друга.’//
\endgl \xe

Заметим, что это типичный сценарий формирования глагола поиска на основе лексемы из другого семантического поля: исходный глагол семантически связан с обработкой или преодолением пространства и управляет локативом или предложной группой, но по мере формирования семантики общего поиска приобретает возможность употребляться переходно, ср.~сербск. \i{тражити за неким} ‘идти по следу \b{за} кем-то’ $\rightarrow$ \i{тражити књигу} ‘искать книгу’ \parencites{tolstaya2013}{ryzhova_stankovich2018}.

Согласно данным Д.~Карамшоева, поиск ощупью может передавать также глагол \i{х̌иқāптоw}, совмещающий значения ‘искать, разыскивать, шарить, нащупывать’ и ‘болтать, говорить впустую, разглагольствовать’, см.~следующий словарный пример:

\ex<exsearch31>
\begingl
\gla Цу̊нд=и тар торик-и \b{х̌иқāп-т}, на-вӯд=и.//
\glc сколько={\sc 3sg} {\sc eq} темно-{\sc subst} шарить-{\sc pst} {\sc neg}-нести.{\sc pst=3sg}//
\glft ‘Сколько он ни \b{шарил} в темноте, не нашёл (ничего).’ \trailingcitation{\parencite[276]{karamshoev1999}}//
\endgl \xe

Современные носители такое употребление не подтверждают, но этот глагол известен им в несколько ином значении: ‘выдумывать, нести чушь’ (\gethref{exsearch32}).

\ex<exsearch32>
\begingl
\gla Йā=йи бийор дарс=анд дис \b{х̌иқāп-т}=ху, ~~~~~~~~~~~~~~ wам-ард=ен ду нêδ-д.//
\glc {\sc d3.f.sg=3sg} вчера урок={\sc loc} так выдумывать-{\sc pst=and1} ~ {\sc d3.f.sg.o-dat=3pl} два сажать-{\sc pst}//
\glft ‘Вчера на уроке она \b{выдумывала} / говорила глупости, и ей поставили двойку.’//
\endgl \xe

Связь с семантикой поиска кажется здесь сомнительной — прежде всего ввиду уникальности такого перехода: типологически он не подтверждается нашими данными \parencite{eureka2018} (ср.~также многоязычные базы данных колексификаций и семантических переходов \parencite{clics2020} и \parencite{datsemshift}, где такая связь тоже не засвидетельствована), и в целом трудно установить сколько-нибудь объяснимые непосредственные отношения между таким значением-источником и таким значением-целью. Однако и ставить под сомнение данные Д.~Карамшоева не хотелось бы: известно, что ошибку лексикографа стоит подозревать в последнюю очередь. Нам кажется, что некоторую гипотезу по поводу этого перехода можно сформулировать, если рассмотреть \i{х̌иқāптоw} в ряду оставшихся трёх словарных глаголов поиска, — к сегодняшнему дню, как показывают наши полевые данные, полностью утраченных.

Два глагола — \i{wарθāптоw} и \i{зи(р)ғāптоw}, — которые ещё употреблялись в 1960–70-е годы, когда собирали словник, отражены в словаре очень похоже на \i{х̌иқāптоw}, как имеющие два значения: ‘искать’ и ‘болтать’. Разница в том, что у первого исходным значится ‘болтать’, а у второго (как у \i{х̌иқāптоw}) — ‘искать’. Сама возможность разного порядка значений, когда лексикографу — носителю языка — всё равно, как их расположить, свидетельствует о том, что ни одно из них не исходно. Это \b{первый вывод}, который даёт нам этот словарный материал. Условия поиска, которые заложены во всех трёх глаголах (по словарным толкованиям), тоже очень схожи: это нащупывание рукой, обычно необходимое для поиска небольшого предмета \b{в темноте} — как в \i{wарθāптоw} ‘разыскивать, шарить, нащупывать’, ср.~(\gethref{exsearch33}), или \i{х̌иқāптоw} ‘искать, разыскивать, шарить, нащупывать’, ср.~(\gethref{exsearch31}) выше, а также менее конкретная в отношении способа (глазами? ногами?) ситуация поиска больших предметов — но тоже в темноте: например, нужного дома, как в \i{зи(р)ғāптоw} ‘искать, разыскивать (в темноте)’, ср.~(\gethref{exsearch34}). Таким образом, для шугнанского поиск в темноте выделен и имел разные способы выражения. Важно, что каждый раз он оказывался сопряжён со значением ‘болтать, говорить чепуху, выдумывать, фантазировать’. Такая повторяемость, пусть и в одном и том же языке, подтверждает правдоподобие такого семантического соседства — и это \b{второй вывод}, который дает анализ этих словарных данных.

\ex<exsearch33>
\begingl
\gla Йу тар х̌āб-и wих̌ӣӡ=и \b{wарθāп-т}, на-вӯд=и.//
\glc {\sc d3.m.sg} {\sc eq} ночь-{\sc subst} ключ={\sc 3sg} шарить-{\sc pst} {\sc neg}-нести.{\sc pst=3sg}//
\glft ‘Он \b{нащупывал} ключ в темноте, но не нашёл.’ \trailingcitation{\parencite[319]{karamshoev1988}}//
\endgl \xe

\ex<exsearch34>
\begingl
\gla Х̌āб-и=ндê=м ту чӣд \b{зиғāп-т} \b{зиғāп-т}, ~~~~~~~~~~~~~~~~~~~~ байелā=м wи вӯд.//
\glc ночь-{\sc subst=loc=1sg} {\sc pron.2sg} дом искать-{\sc pst} искать-{\sc pst} ~ насилу={\sc 1sg} {\sc d3.m.sg.o} нести.{\sc pst}//
\glft ‘[Я] \b{искал-искал} ночью твой дом, насилу нашёл.’ \trailingcitation{\parencite[565]{karamshoev1988}}//
\endgl \xe

Между тем у Д.~Карамшоева есть ещё глагол \i{wарwарθтоw}. Он тоже уже ушёл из узуса, он тоже поисковый во втором значении, и он тоже обозначал тактильный поиск в темноте, ср.~(\gethref{exsearch35}). Однако в отличие от предыдущих, его первое значение очень прозрачно: это мелкие физические движения, без направления и определённой цели: ‘возиться, двигаться; шевелиться; барахтаться’, ср.~(\gethref{exsearch36}). В паре ‘возиться’ $\rightarrow$ ‘шарить’ понятна и связь, и направление связи. Действительно, ‘шарить’ предполагает мелкие движения руками, беспорядочные при поиске в темноте, у которых, тем не менее, есть цель, тогда как ‘возиться / барахтаться’ тоже предполагает мелкие движения — правда, не только рук, а всего человеческого тела, — но бесцельно. При семантическом переходе ‘возиться’ $\rightarrow$ ‘шарить’ исходная затронутость движением всего тела сужается до рук, а к беспорядочности добавляется цель. Эти попутные семантические изменения не случайны и не исключительны. Действительно, при переходе глагола движения в сферу человеческой деятельности, которая всегда целенаправленна, мы обычно наблюдаем возникновение цели, ср.~этот эффект для \i{крутиться}: \i{он крутится как умеет} ‘много делает, чтобы все успеть’. Естественно и сужение, ср.~для \i{возиться}:

\begin{itemize}
  \item (исходное) \i{возится в луже} как движение всего \b{тела} $\rightarrow$
  \item (производное) \i{она возится на кухне} как ‘готовит’ (прежде всего, движение \b{рук})
\end{itemize}

\ex<exsearch35>
\begingl
\gla Цу̊нд=ум тар х̌āб-и \b{wарwарθ-т}, нā-вӯд=ум.//
\glc сколько={\sc 1sg} {\sc eq} ночь-{\sc subst} возиться-{\sc pst} {\sc neg}-нести.{\sc pst=1sg}//
\glft ‘Сколько я ни \b{шарил} в темноте, ничего не нашёл.’ \trailingcitation{\parencite[317]{karamshoev1988}}//
\endgl \xe

\ex<exsearch36>
\begingl
\gla Ар жиниҷ=ум δод=ху, \b{wарwарθ-т}=ум.//
\glc {\sc down} снег={\sc 1sg} упасть.{\sc pst=and1} возиться-{\sc pst=1sg}//
\glft ‘Я увяз в снегу и \b{барахтался} в нём.’ \trailingcitation{\parencite[317]{karamshoev1988}}//
\endgl \xe

Сужение исходного значения беспорядочного движения может дать в результате не только движение рук, но и движение языка, то есть процесс говорения как беспорядочного болтания (болтовни) языком, ср.~русск.~\i{болтать}, а также \i{трепаться}. Возникающая при этом стилистическая сниженность может затронуть и содержание разговора / рассказа собеседника: он говорит не дельное, а чепуху и/или неправду, выдумывает, фантазирует, врёт и тому подобное. Именно таков, как нам представляется, источник неожиданного, на первый взгляд, значения ‘болтать’ в шугнанских данных.

Таким образом, исходным для всего этого кластера глаголов, по-видимому, как и у \i{wарwарθтоw}, является значение мелких бесцельных беспорядочных движений (утраченное у многих глаголов к моменту сбора словника), которое преобразуется, с одной стороны, в ‘беспорядочно двигать [~= шарить] руками’ (а затем в более общее ‘искать в темноте’), а с другой стороны, той же операцией сужения, оно же (а не ‘шарить руками’) преобразуется в ‘беспорядочно двигать [~= болтать / трепать] языком’, то есть говорить чепуху и сочинять бредни.

\section{Результаты и обсуждение} \label{search-results}

Задача этой статьи была поставлена максимально комфортно для исследователя: поле \fakesc{ИСКАТЬ}, которое мы выбрали, уже в определённой степени структурировано и описано в \parencite{eureka2018}, причем его предварительные исследования были проведены на немалом языковом материале. Правда, семантика \fakesc{ИСКАТЬ}, как мы показали, устроена не совсем канонически с точки зрения лингвистической теории: она может быть определена как «недоспецифицированная», то есть лишённая конкретики в отношении физических действий, которые предпринимает субъект Х для достижения цели ‘найти Y’. Для таких предикатов нужна не только фреймовая типология, апеллирующая к аргументной структуре (субъект X ищет объект Y в месте Z), но еще и типология источников лексификации. Другими словами, для них нужен обоснованный список / классификация семантических областей, из которых заимствуются глаголы, моделирующие возможную физическую активность человека при поиске.

Мы показали, что шугнанская система \fakesc{ИСКАТЬ} доминантная: по нашим данным, она имеет глагол \i{х̌икӣдоw}, способный покрывать любую ситуацию поиска. Поэтому фактически в ней нет аргументных противопоставлений за счёт субъекта (Х), объекта (Y) или места (Z): нет ни специального глагола для поиска особых Х-ов — например, для поиска животными (типа русск.~i{рыскать}), — ни лексически противопоставленных предикатов поиска референтных или нереферентных Y, ни особого глагола «прочёсывания территории» Z типа русского \i{обыскать}. Однако на периферии поля \fakesc{ИСКАТЬ} у доминантного \i{х̌икӣдоw} есть несколько конкурентов: конструкция \i{тӣр чи бӣр чидоw} ‘переворачивать вверх дном’ в зоне обыска пространства, а также глаголы \i{чӣх̌тоw} ‘смотреть’ и \i{қилāптоw} ‘рыться’, первый из которых обозначает зрительный или не очень старательный, «поверхностный» поиск, а второй — наоборот, тщательный обыск небольшой области или просто поиск руками. Кроме того, оказалось, что шугнанская система разнообразна в отношении источников, особенно если учитывать не только современный шугнанский, но и шугнанский 50–60-летней давности, сохранённый в словаре Д.~Карамшоева. В этой расширенной системе есть беспорядочное движение и ощупывание, зрительное восприятие и — в заимствованных словах и конструкциях — просьбы и расспросы. Ещё один типологически распространённый источник — движение следом — в шугнанском не зафиксирован, но зато в шугнанской системе на некотором этапе её развития, вероятно, лексически выделялся поиск больших объектов в темноте, который технически (но, видимо, не этимологически) предполагает перемещение, то есть «поиск ногами».

Всё это источники, с которыми мы уже сталкивались в других языках. В этой части шугнанский материал подтверждает и уточняет уже имеющиеся у нас типологические данные. В то же время, встраивая шугнанские данные в общую типологию поля недоспецифицированных глаголов поиска, мы получили новый опыт того, как с такими полями работать. Это позволяет нам сделать целый ряд выводов, в том числе методологического характера.

Во-первых, правильно было разбить работу на два этапа. Сначала, на первом этапе — обследовать фреймы, которые задает аргументная структура, чтобы оценить, в какой степени новый язык способен лексифицировать уже известные нам противопоставления, и нет ли новых. На втором этапе — классифицировать семантические источники, «восполняющие» неопределённость в значении глаголов.

Во-вторых, на примере шугнанского мы увидели важность исторического материала: без привлечения к анализу уже утраченных лексем поле выглядело бы гораздо беднее. С точки зрения типолога (а не лексикографа) не играет роли, устарели словарные данные или нет, главное — их достоверность (хотя в далёкой перспективе, когда лексическая типология накопит большой материал, было бы интересно посмотреть, какие противопоставления сохраняются лучше, а какие уходят раньше).

В-третьих, шугнанский дал нам хороший пример нестандартной и необъяснимой колексификации ‘искать’–‘болтать’ и опыт работы с такого рода парами. Понятно, что ровно эта колексификация может и не встретиться в другом языке, но похожая проблема может возникнуть на другом материале. Мы убедились в неслучайности такой пары, показали, что её члены связаны не напрямую и нашли для них общий семантический источник, который и был причиной их колексификации.

\pagebreak[4]

Наконец, четвёртое, что показал шугнанский: семантическая зона может быть покрыта лексикой неравномерно — в конкретном языке отдельные её фрагменты могут оказаться лексифицированы особенно плотно. В нашем случае это поиск руками, ср.~\i{қилāптоw} ‘рыться’, а также \i{х̌иқāптоw}, \i{wарθāптоw}, \i{зи(р)ғāптоw} и \i{wарwарθтоw}, которые, согласно словарным данным, некогда выражали значение ‘шарить, нащупывать, искать в темноте’. Пока у нас нет статистики по поводу того, насколько значим этот тип поиска по сравнению с другими для языков мира как семантический источник. Однако несомненно, что в шугнанском именно поиск руками (а также связанный с ним поиск в темноте) имеет своего рода привилегии перед остальными — особенно в исторической перспективе. Даже доминантный глагол поиска \i{х̌икӣдоw} оказывается семантически сопряжён с семантикой выбора, производной от перебора объектов руками.

Как видим, частная задача описания системы шугнанского как очередного языка в зоне \fakesc{ИСКАТЬ} существенно продвигает нас в понимании принципов организации лексики в целом.