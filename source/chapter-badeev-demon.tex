\chapter*{Вариативность в~употреблении указательных местоимений\\в~шугнанском языке}
\addcontentsline{toc}{chapter}{\textit{А.~Бадеев}. \textbf{Вариативность в~употреблении указательных местоимений в~шугнанском языке}}
\setcounter{section}{0}
\chaptermark{Вариативность в~употреблении указательных местоимений…}
\label{chapter-badeev-demon}

\begin{customauthorname}
Артём Бадеев
\end{customauthorname}

\begin{englishtitle}
\i{Variability in the usage of demonstratives in Shughni\\{\small Artyom Badeev}}
\end{englishtitle}

\begin{abstract}
Данная статья представляет собой обзор основных особенностей системы указательных местоимений и их дейктических функций в шугнанском языке. В основе исследования лежат данные полевой работы, а именно эксперимента, в ходе которого информанты озвучивали предложения с шугнанскими демонстративами. Полученные данные изложены в виде таблиц с последующим обсуждением. В исследовании используются различные теоретические подходы к изучению дейксиса: с одной стороны, бицентрический подход, опирающийся на анализ отношений участников речевого акта, с другой — представления об оппозиции лично- и дистантно-ориентированных систем в свете представления о дейктическом как удалённом от дейктического центра, то есть говорящего. В этом ключе данная статья стремится совместить достижения обоих подходов. В результате предлагается обновлённое представление о вариативности указательных местоимений в шугнанском.
\end{abstract}

\begin{keywords}
восточноиранские языки, шугнанский язык, демонстративы, дейксис
\end{keywords}

\begin{eng-abstract}
This article provides an overview of the main features of the Shughni system of demonstratives and their deictic functions. The study is based on the fieldwork data, namely an experiment during which the informants responded with Shughni demonstratives. The data obtained are presented in the form of tables with subsequent discussion. The study uses diverse theoretical approaches: on the one hand, a bicentric approach to deixis, examining it through the relations of the participants in a speech act, on the other hand, the idea of the opposition between person- and distance-oriented systems following the mainstream concept of deixis as remoteness from a deictic center, i.~e.~the speaker. In this vein, the article seeks to combine the achievements of both approaches. As a result, an updated understanding of the variability of Shughni demonstratives is provided.
\end{eng-abstract}

\begin{eng-keywords}
Eastern Iranian languages, Shughni, demonstratives, deixis
\end{eng-keywords}

\begin{acknowledgements}
Публикация подготовлена в ходе проведения работы (проекта №~22-00-034) в рамках Программы «Научный фонд Национального исследовательского университета «Высшая школа экономики» (НИУ~ВШЭ)» в 2022~году. Автор хотел бы выразить благодарность всем участникам Шугнанской экспедиции НИУ~ВШЭ 2019~года, в первую очередь Плунгяну В.~А., Рахилиной Е.~В., Арманд Е.~Е., Ронько Р.~В., Сергиенко А.~А., Бутолину В.~Д., а также Арманд Е.~Е. за помощь в написании этой статьи, работникам Института памирских языков г.~Хорога и РОО~«НУР» г.~Москвы и всем информантам, принявшим участие в экспериментах, лёгших в основу этой работы.
\end{acknowledgements}

\begin{initialprint}
\fullcite{badeev2022}\end{initialprint}

\section{Введение} \label{dem-intro}

Указательные местоимения в языках мира обладают дейктической функцией. В работе \parencite[237]{plungian2011} указывается: «под дейксисом (греч. ‘указание’) в общем случае понимается “шифтерная” ориентация объекта или ситуации, т.~е.~указание на их положение в пространстве или времени относительно “дейктического центра”, связанного с речевым актом». В свою очередь традиционно \parencite[117–122]{buhler2011} различается дейксис лица, пространства и времени. Указательные местоимения (иначе: демонстративы) отвечают за пространственный дейксис: общую удалённость, положение референта относительно непосредственных участников дискурса (локуторов) и дейктического центра \parencite[35–36]{diessel1999}. Так, последний может быть понят как область концептуализации дейктических значений, отправная точка для указания. Обычно дейктический центр рассматривается как область, где находится говорящий, центр речевого акта \parencites[254]{plungian2011}[35–36]{diessel1999}. Демонстративы формируют оппозицию в отношении удалённости от дейктического центра. В русском языке различаются указательные местоимения \i{этот} и \i{тот}: \i{этот} является указанием на дейктический центр, тогда как \i{тот} на всё, что находится за его пределами. Эта типология могла бы объяснить системы демонстративов, основанные сугубо на отношениях удалённости, но неадекватна для тех систем, где употребление демонстративов независимо (или же зависимо не только) от их положения относительно дейктического центра. Х.~Диссель описал разногласия лингвистов по этому вопросу: одна точка зрения основана на том, что демонстративы в языках мира неразрывно связаны с указанием на дистантный контраст; исходя из другой позиции, употребление демонстративов зависит от расположения референтов относительно локуторов, сами демонстративы могут не иметь при этом дейктической функции \parencite[37–39]{diessel1999}. Последнее предположение дало современной лингвистике представление об оппозиции личного и дистантного контрастов в дейксисе \parencite[282–286]{anderson_keenan1985}. Сам Х.~Диссель сообщает, что дейктический центр может находиться не только в области говорящего (исключая адресата), но и в общей области обоих локуторов \parencite[41]{diessel1999}.

Интересно иметь в виду также эксперимент с дейксисом телефонного разговора Ю.~Д.~Апресяна, согласно которому «ключевой фигурой» дейксиса является не говорящий, а наблюдатель, который обозревает всю ситуацию со стороны. 

\begin{small}\begin{quote}
«\i{Пусть некто ищет авторучку, которая, по сведениям говорящего, лежит на телефонном столике сбоку от телефонного аппарата. Говорящий может сидеть спиной к столику. Однако, направляя поиск, он будет ориентировать ручку относительно телефона не со своей точки зрения, а с точки зрения ищущего}» \parencite[278]{apresian1986}.
\end{quote}\end{small}

Так, можно сказать, говорящий создаёт свой образ, который и становится дейктическим центром. Позиции наблюдателя и говорящего могут совпадать, однако в отдельных случаях, например, когда говорящий не видит референт и адресата, но направляет последнего так, как будто наблюдает референт с его позиции, дейктический центр и позиция говорящего могут не совпадать. Соответственно, помимо удалённости от дейктического центра, видимость объекта, а также его расположение и перемещение по вертикальной оси (\i{elevation}), ориентиры на местности (в гору или вниз по реке) могут играть решающую роль в употреблении демонстративов, хотя дейктическая природа этих факторов оспаривается \parencite[41]{diessel1999}.

Предметом изучения данной работы является употребление указательных местоимений в шугнанском языке (иранская группа, восточно-иранская подгруппа, памирские языки) в пространственном контексте (временной контекст в данной работе не рассматривается)\fn{Более подробную классификацию родства см.~\parencite[787–788]{edelman_dodykhudoeva2009_shughni}.}. Демонстративы шугнанского языка были описаны рядом исследователей-памироведов как в широком рассмотрении класса местоимений, так и в рамках исследований дейксиса. Приведём некоторые из работ \parencites{karamshoev1963}{edelman1976}{belikov1972}{alamshoev1994}{yusufbekov1998}{muller2015}. Из них дейксису в шугнанском посвящены монография \parencite{yusufbekov1998} с крупной базой примеров и концептуализацией дейктических понятий, а также сравнительно недавняя работа \parencite{muller2015}, которая не учитывает данные \parencite{yusufbekov1998}, однако рассматривает дейксис в шугнанском языке с позиции грамматики ролей и референции (Role and Reference Grammar — RRG). Цель нашей работы заключается в том, чтобы с современных типологических позиций и с новым языковым материалом описать вариативность в употреблении указательных местоимений в шугнанском языке.

\section{Методология работы} \label{dem-method}

В данной работе анализируются примеры употребления демонстративов в пространственных контекстах, полученные в ходе интервьюирования на основе метода, применённого и описанного в статье А.~А.~Ростовцева-Попеля [\cite*{popiel2009}]. В основе проведённого нами исследования лежит эксперимент, производимый в пространстве комнаты или же вне помещения, где информантам предлагалось поучаствовать в диалоге в роли говорящего и озвучить просьбу адресату (в этой роли выступал автор данной работы) подать ему какой-либо предмет, а именно: «подай мне (э)тот~X», употребив один из демонстративов. Предметы располагались на одной горизонтальной оси на одинаковом расстоянии в 1–2~метрах друг от друга. Их количество, а также положение участников дискурса менялось от одного эксперимента к другому. Ответы информантов, собранные в таблицах, будут приведены ниже. Предпочтение отдавалось первым полученным ответам, однако, если информант ссылался на альтернативную возможность, эти данные также фиксировались.

Всего было проведено четыре эксперимента: три основных (участвовали 12 информантов, эксперименты проводились подряд один за другим) и один дополнительный (участвовали два информанта, не принимавших участия в других экспериментах). Были опрошены носители шугнанского языка, проживающие в городе Хорог, кишлаке Дашт (джамоат~Мирсаид Миршакар, ГБАО, Таджикистан), в Душанбе, в Москве и Московской области. Среди опрошенных были как люди, окончившие только среднюю школу, так и люди с высшим образованием. Были представлены как мужчины, так и женщины младшего и среднего возраста от 17 до 60~лет. По роду занятий информанты были школьниками, студентами, преподавателями в институте или национальных группах, домохозяйками.

В следующих разделах будет дан обзор системы указательных местоимений (далее УМ) шугнанского языка, затем по очереди будут анализироваться данные каждого из проведённых экспериментов, сопровождаемые иллюстрациями, примерами употребления демонстративов и таблицами с ответами информантов там, где это необходимо. На основе как имеющихся описаний шугнанской системы УМ, так и работ на материале других языков будут рассмотрены полученные в ходе каждого из экспериментов ответы информантов. Так, в \hyperref[dem-exp1]{четвёртом разделе} будут приведены данные первого эксперимента, направленного на определение немаркированного компонента в шугнанской системе УМ, в \hyperref[dem-exp2]{пятом} — второго, где ответы информантов будут поделены на две стратегии употребления УМ на основании лично-дистантной оппозиции дейктических систем, в \hyperref[dem-exp3]{шестом} — третьего эксперимента, теста на невидимость с дополнительным сбором данных для уточнения возможности «общего указания» в шугнанском языке. \hyperref[dem-conclusion]{Седьмой раздел} представляет собой заключение.

\section{Обзор системы указательных местоимений шугнанского языка} \label{dem-overview}

В Таблице \ref{tab:dem1} ниже представлены указательные местоимения шугнанского языка. В шугнанском языке различают прямую и косвенную падежные формы, при этом местоимения в косвенном падеже могут выражать в том числе значение посессивности \parencite[31]{alamshoev1994}. Все местоимения в косвенной форме, а также местоимения третьей серии в прямом падеже имеют грамматический показатель рода. Также все местоимения различают формы единственного и множественного числа. Помимо дейктической функции, указательные местоимения всех степеней удаления обоих падежей могут употребляться в качестве определённого артикля. М.~М.~Аламшоев [\cite*[34]{alamshoev1994}] отмечает, что это происходит «при повествовании в тех случаях, когда предмет, лицо или понятие собеседнику ясны, а референт лишь напоминает о них», ср.~пример (\gethref{exdem1}).

\ex<exdem1>
\begingl
\gla \b{Дам} чӣнгāл=анд йи ғулā аɣ̌дал бар-ҷой виц=ху, даδ одам=ат айwу̊н наɣ̌ҷӣд-оw на-лāких̌т. \b{Йā} аɣ̌дал анҷӣвд=ат ар чи=йаθ абêзд.//
\glc {\sc d2.f.sg.o} лес={\sc loc} один большой дракон на-место быть.{\sc pf.f/pl=and1} тогда человек={\sc and2} животное проходить.{\sc inf-purp} {\sc neg}-разрешать.{\sc prs.3sg} {\sc d3.f.sg} дракон держать.{\sc prs.3sg=and2} каждый кто.{\sc o=int} глотать.{\sc prs.3sg}//
\glft ‘В \b{этом} лесу жил один большой дракон и не давал пройти людям и животным. (\b{Тот}) дракон хватал любого и проглатывал.’ \trailingcitation{\parencite[34]{alamshoev1994}}//
\endgl \xe

Вслед за работами \parencites[]{belikov1972}[]{yusufbekov1998}[]{edelman1976} мы обращаемся к типологии К.~Бругмана \parencite{brugmann1904}, который выделяет три серии указательных местоимений: I, II и III, где I серия — это указание на сферу говорящего (\i{Ich-Deixis}), II серия — указание на сферу адресата (\i{Du-Deixis}), она же берёт на себя эмфатическое и анафорическое указание (\i{Der-Deixis}), а III серия — указание на сферу другого, третьего лица (\i{Jener-Deixis}). Эта типология, построенная на материале индоевропейских языков, позволяет рассмотреть шугнанский и шире индоевропейский дейксис в диахронии.

\begin{table}
 \centering
 \caption{Система демонстративов в шугнанском языке \parencite[12]{yusufbekov1998}}
 \smallskip
 \label{tab:dem1}
 \begin{tabular}{c|c|ccc} \toprule
 серия & падеж & {\sc m.sg} & {\sc f.sg} & {\sc pl} \\ \midrule
 \multirow{2}{*}{I (\i{Ich-Deixis})} & прям. & \multicolumn{2}{c}{\i{йам}} & \i{мāδ} \\
 & косв. & \i{ми} & \i{мам} & \i{мев} \\ \midrule
 \multirow{2}{*}{II (\i{Du-Deixis} /~\i{Der-Deixis})} & прям. & \multicolumn{2}{c}{\i{йид}} & \i{дāδ} \\
 & косв. & \i{ди} & \i{дам} & \i{дев} \\ \midrule
 \multirow{2}{*}{III (\i{Jener-Deixis})} & прям. & \i{йу} & \i{йā} & \i{wāδ} \\
 & косв. & \i{wи} & \i{wам} & \i{wев} \\ \bottomrule
 \end{tabular}
\end{table}

Принятие типологии К.~Бругмана в нашей работе связано с особенностями системы демонстративов шугнанского языка, где референция происходит как относительно дейктического центра, так и позиций обоих участников дискурса, что нередко для трёхчастных систем \parencite[38–41]{diessel1999}. В свою очередь Ш.~П.~Юсуфбеков описал шугнанскую систему демонстративов как бицентрическую (в противоположность моноцентрической), где определяющую роль в употреблении указательных местоимений играет положение участников дискурса относительно референта и друг друга \parencite[138–139]{yusufbekov1998}. В свою очередь, типологическая оппозиция «ближний–средний–дальний», отражающая отдалённость референта от дейктического центра \parencite[39]{diessel1999}, видится нам неподходящей для характеристики системы демонстративов шугнанского языка. Принять её означало бы игнорировать доводы предшественников.

\vfill

\section{Эксперимент первый. Определение немаркированного компонента} \label{dem-exp1}

\begin{figure}[h]
 \centering
 \caption{Первый эксперимент \parencite[28]{popiel2009}}
 \smallskip
 \label{fig:dem1}
 \includegraphics[width=0.8\textwidth]{img/dem1.jpg}
\end{figure}

В каждом эксперименте представлены разные ситуации. Это можно увидеть на Рисунке \ref{fig:dem1}, где по очереди изображены два опыта.

В ходе первого опыта между локуторами — говорящим (S) и адресатом (A) — на равном расстоянии находились референты под номерами №1 и №2 (Рисунок \ref{fig:dem1}). Во втором опыте объект под номером №1 отсутствовал. В этом эксперименте и дальнейших это могли быть любые подручные предметы (чашка, яблоко, книга и~т.~п.). Для нашей работы не было принципиально, чтобы объекты были одинаковыми (например, две чашки, три книги и~т.~п.). В обоих опытах говорящему предлагалось обратиться к адресату с просьбой подать ему объект под номером №2. Во всех случаях информанты прибегали к употреблению II серии демонстративов:

\ex<exdem2>
\begingl
\gla Ку \b{дам} чашкā дāк.//
\glc {\sc ptcl} {\sc d2.f.sg.o} чашка дать[{\sc imp}]//
\glft ‘Пожалуйста, подай мне \b{(э)ту} чашку’ \trailingcitation{[источник: эксперимент]}//
\endgl \xe

Здесь следует рассмотреть УМ II серии. В шугнанском языке они отвечают за область, широко определяемую по шкале удалённости от дейктического центра, но находящуюся возле адресата или, иначе говоря, соотносимую с его сферой\fn{Под сферой здесь подразумевается пространство, которое говорящий определяет как область одного из локуторов. Такие области, как правило, подвижны и могут меняться в зависимости от разных факторов (ср.,~к примеру, кардиоидную область говорящего в тех языках, где невидимость объекта запрещает употребление ближнего указательного местоимения при референции к нему \parencite[20]{levinson2018}).}. Помимо идеальных примеров, когда референт видим для участников речевого акта, находится ближе к адресату, чем к говорящему, II серия употребляется в том числе в случаях, когда адресат находится вне поля зрения говорящего, а сам говорящий может только предполагать присутствие адресата, когда к нему обращается \parencite[20–25]{yusufbekov1998}. Несомненно, определение этой сферы есть задача говорящего. Как можно видеть из эксперимента, наличие или отсутствие референта возле говорящего не вносит изменений в употребление демонстративов. Это и требовалось проверить на опыте. Выбор информантом местоимения в этом эксперименте, согласно \parencite[27]{popiel2009}, позволяет определить немаркированный компонент в дейктической системе. Та же особенность была замечена Ш.~П.~Юсуфбековым: демонстративы II серии могут утрачивать пространственное значение и обозначать указание как таковое, скорее всего анафорического характера \parencite[22]{yusufbekov1998}. Как можно заключить, этот эксперимент обозначает позицию демонстратива II серии как немаркированного компонента в шугнанском дейксисе.

\section{Эксперимент второй. Внутриязыковая лично-дистантная оппозиция} \label{dem-exp2}

В ходе следующего эксперимента в первом опыте между локуторами на равной удалённости находились три объекта, см.~Рисунок \ref{fig:dem2}. Во втором опыте объекты занимали пространство в ряд перед говорящим и адресатом. В обоих опытах референты были видимы для локуторов. Информантам предлагалось указать на каждый из объектов под номерами №1 (ближайший к говорящему), №2 (средний) или №3 (дальний) при наличии в дискурсивном пространстве двух других.

\begin{figure}
 \centering
 \caption{Второй эксперимент \parencite[28]{popiel2009}}
 \smallskip
 \label{fig:dem2}
 \includegraphics[width=0.8\textwidth]{img/dem2.jpg}
 \end{figure}

Результаты, полученные в ходе этого и следующего экспериментов, представлены в виде таблиц с ответами информантов, где в двух столбцах, представляющих два опыта эксперимента, содержатся данные — демонстративы, употребленные информантами при указании на объекты по пути от ближнего к дальнему. I обозначает I серию, II — II серию, III — III серию демонстративов. По второму эксперименту данные разделены на две таблицы, которые мы будем рассматривать как разные стратегии употребления указательных местоимений. Выделяя основные тенденции и рассматривая каждую парадигму употребления информантом демонстративов как систему, мы перейдем к частным случаям.

\begin{sidewaystable}
 \centering
 \caption{Эксперимент 2}
 \smallskip
 \label{tab:dem2}
 \begin{tabular}{c|c|ccc|ccc} \toprule
 \multirow{2}{*}{{\small стратегия}} &\multirow{2}{*}{{\small \makecell[c]{инфор-\\мант}}} & \multicolumn{3}{c|}{опыт 1} & \multicolumn{3}{c}{опыт 2} \\
 & & {\small объект №1} & {\small объект №2} & {\small объект 3} & {\small объект №1} & {\small объект №2} & {\small объект №3} \\ \midrule
 \multirow{8}{*}{A} & 1 & I & II & II & I & III & III \\
 & 2 & I & II & II & I & II & III \\
 & 3 & I & II & II & I & II & III \\
 & 4 & I & I & II & I & I & III \\
 & 5 & I & II & —* & I & III & III \\
 & 6 & I & II & II & I & III & III \\
 & 7 & I & II & II & I & III & III \\
 & 8 & II & II & II/III & II & II & III \\ \midrule
 \multirow{4}{*}{B} & 9 & I & II & III & I & I & III \\
 & 10 & I & III & III & I & III & III \\
 & 11 & I & II & III & I & I/II & III \\
 & 12 & I & I & III & I & II & III \\ \bottomrule
 \end{tabular}
 \\
 \medskip
 \hspace*{\fill}{\small *Ответ информанта зафиксировать не удалось.}
\end{sidewaystable}

Каждый эксперимент можно представить как поступательное изменение относительно предыдущего. Так, в первом опыте мы имеем картину предыдущего эксперимента, изменённую таким образом, что в расширенное дискурсивное пространство был добавлен ещё один референт. Вместе с тем во втором эксперименте информанты называют все референты, а не только референт №2, как в первом. Таким образом, сначала наша задача состоит в том, чтобы проследить, как меняется употребление УМ носителями при увеличении количества референтов до трёх.

Рассмотрим опыт 1 в рамках стратегии A (Таблица \ref{tab:dem2}). Мы видим доминирующую стратегию маркирования, в которой объект №2 во всех случаях, кроме одного, маркируется демонстративом II серии. Так же употребляется он и в отношении объекта №3 во всех случаях, кроме одного, где ответ информанта не удалось зафиксировать. Применительно к аналогичной ситуации в португальском языке, А.~А.~Ростовцев-Попель пишет о необходимости показать контраст между ближним и сравнительно удалённым объектом \parencite[27]{popiel2009}. Касательно референта, находящегося на наибольшем удалении от говорящего, такой же необходимости различать в указании его и средний референт не возникает. В той же работе рассматривается ситуация, в которой невозможно употребить УМ III серии и при указании на объект, «находящийся во внутреннем пространстве коммуникации», т.~е.~находящийся между локуторами: А.~А.~Ростовцев-Попель признаёт, что языки, которые не допускают такого указания, более склонны к дистантному ориентированию, нежели те, в которых оно возможно \parencite[27]{popiel2009}. Примеров такого употребления в шугнанском нами не выявлено. К той же части таблицы, где употребление демонстративов II серии не встречается вовсе, мы вернёмся позднее.

Теперь обратимся к стратегии B (Таблица \ref{tab:dem2}) для первого опыта. Наглядно в сравнении с данными, рассмотренными выше, употребление информантами демонстративов III серии при указании на референт №3, то есть объект, ближайший к адресату. Так, можно наблюдать, как в первом случае одна часть информантов чётко определяет референт №3 в область адресата (II серия), в то время как другая употребляет в его отношении демонстративы III серии. Употребление последних в шугнанском языке зафиксировано в отношении пространственно удалённых объектов, а в дискурсивном пространстве ассоциируется со сферой 3-го лица, то есть не соотносимой с локуторами \parencite[26]{yusufbekov1998}. Таким образом, информанты, употребившие в данном контексте УМ III серии, склонны игнорировать позицию адресата, отдавая предпочтение параметру удалённости референта №3.

Теперь посмотрим, как отвечали информанты во втором опыте. Если для стратегии A изменение позиции адресата повлекло изменения в указании на референты (принципиален переход от II серии к III в отношении референта №3), то в стратегии B этого не происходит (в двух случаях возможно несколько вариантов УМ, их мы рассмотрим позднее). В таком случае можно сказать, что стратегия B решает проблему появления референта №3 тем, что игнорирует позицию адресата. Ведь всё те же информанты употребляли в ходе первого эксперимента демонстративы II серии, значит, увеличение числа референтов изменило подход к указанию в сторону более дистантно-ориентированного. Исходя из анализа Ростовцева-Попеля, положение адресата в дистантно-ориентированной системе не является опорной точкой при наличии трёх и более референтов \parencite[29]{popiel2009}. Во втором опыте позиции локуторов совпадают, а использование указательных местоимений становится чуть более упорядоченным: абсолютно доминируют УМ I и III серии в указании соответственно на референты №1 и №3.

В указании на референт №2 как в стратегии A, так и в стратегии B используются все три типа маркирования демонстративов (I, II, III), однако принципы их употребления в разных стратегиях разные. Так, в рамках стратегии B, по-видимому, доминирует дистантная ориентация и УМ II серии используется для указания на объект средней удалённости от дейктического центра. В иных случаях для построения дистантной оппозиции употребляются демонстративы только I и III серий — тогда все референты маркируются либо как I–III–III, либо как I–I–III. В ответах этих информантов при появлении третьего референта II серия выходит из употребления, и система работает как двухчастная. Важно подчеркнуть тот факт, что II серия УМ не полностью отсутствует в идиолектах этих информантов, поскольку в первом эксперименте II серию употребляли 100\% участников эксперимента.

За исключением одного случая (информант 4), в стратегии A используется УМ II или III серии для указания на референт №2. Исходя из сугубо дистантного определения шугнанских УМ, III серия употребляется в отношении далёкого, II серия — относительно близкого референта (видимого в обоих случаях) \parencite[40]{yusufbekov1998}. Эта дистантная неопределённость их разграничения, зависимость его от личного отношения говорящего были отмечены в работе \parencite{belikov1972}. Употребить как II, так и III серию можно в одной и той же ситуации:

\pex<exdem3>
\a<a> \begingl
\gla Йам сит-дор жиниҷ, \b{йид} тозā.//
\glc {\sc d1.sg} земля-{\sc hb} снег {\sc d2.sg} чистый//
\endgl
\a<b> \begingl
\gla Йам сит-дор жиниҷ, \b{йу} тозā.//
\glc {\sc d1.sg} земля-{\sc hb} снег {\sc d3.m.sg} чистый//
\glft ‘Этот снег грязный, \b{тот} — чистый.’ \trailingcitation{\parencite[74–75]{belikov1972}}//
\endgl \xe

Объяснить применение разных стратегий в употреблении демонстративов можно через признание идиолектной оппозиции лично- и дистантно-ориентированных дейктических систем. Из рассмотренных А.~А.~Ростовцевым-Попелем языков мы вновь берём португальский как образец: когда во втором опыте теряет значение позиция адресата, в этом языке выходят из употребления УМ II серии, и система становится двухчастной \parencite[28–29]{popiel2009}. Такой же вариант маркирования предлагает и стратегия A во втором опыте (большинство случаев). В иных случаях, где система сохраняет употребление демонстративов II серии, доминирует определение относительно дейктического центра, и последние отсылают не к сфере адресата, а, как это происходит и в стратегии B, указывают на объект, находящийся на среднем удалении от дейктического центра. Таким образом, полученные в результате эксперимента данные подтверждают первоначальное разделение на две группы.

Рассмотрим тот случай в рамках стратегии A, где для указания на ближайшие к говорящему два объекта используются демонстративы I серии (обозначены в двух опытах в Таблице \ref{tab:dem2} как I–I–II и I–I–III — информант 4). Так, использование дейктика II серии в первом опыте (III серии во втором) для обозначения референта №3 уже не позволяет назвать эту систему дистантно-ориентированной: информант учитывает позицию адресата. Возможно, так выглядит несколько нетипичная для шугнанского модель лично-ориентированной системы, где из-за вытеснения II серии I серия частично заняла её «нишу» в дискурсе. Также пограничными можно назвать два случая в Таблице \ref{tab:dem2}. Первый из них (информант 11) обозначен как I–II–III в первом опыте и I–I/II–III во втором. В первом опыте соблюдается выбор демонстративов, типичный для стратегии B, однако, когда положение адресата меняется во втором опыте, появляется вариативность, где при выборе УМ II серии говорящий придерживается стратегии B, но при выборе УМ I серии реагирует на перемещение адресата. Систему, где выбор был совершён в пользу I серии, справедливо считать лично-ориентированной.

Последним нерассмотренным отдельным случаем является информант 8 в Таблице \ref{tab:dem2}: II–II–II/III и II–II–III. Здесь информант допускает повтор в маркировании референта №3 из первого во второй опыт (III~$\rightarrow$~III). Этот повтор совершенно не свойственен стратегии A. В том же месте допускается и типичный для лично-ориентированной стратегии переход II~$\rightarrow$~III. Мы же относим этого информанта к стратегии A из-за выбора демонстративов в маркировании референта №1. Употребление II серии УМ в его отношении не типично ни для одной из стратегий (и более того, ни для одного из информантов в этом эксперименте). Однако выбор II серии говорит больше в пользу личной ориентации, чем дистантной. В дистантно-ориентированной системе употребление II серии объясняется выражением средней степени удалённости от дейктического центра. В данных стратегии B мы видим такие системы, индифферентные к позиции адресата, в них УМ существуют в парадигме «ближний–средний–дальний» или «ближний–дальний», когда система редуцируется до двухчастной. Парадигма «средний–дальний» невозможна для двухчастных систем, ведь двухчастная система предполагает бинарную оппозицию «близко–далеко». В пользу дистантной ориентации здесь можно говорить лишь в том случае, если информант 8 во всех ответах употребляет II серию УМ в значении ближней степени удалённости. В действительности, в речи информанта 8 отсутствуют УМ I серии во всех трёх проведённых экспериментах. Примеров такого замещения в существующих описаниях шугнанского нами не обнаружено, однако полностью исключать этот вариант нельзя.

Иначе объяснить такое проявление II серии можно через вариативность в употреблении демонстративов I и II серий. Существующие описания позволяют предположить определённую свободу в выборе между этими сериями. В действительности, «если говорящий оценивает расстояние между собой и адресатом как “близкое” и наблюдаемый объект находится в поле их зрения, может быть употреблена как I, так и II серия» \parencite[35]{yusufbekov1998}.

\pex<exdem4>
\a<a> \begingl
\gla \b{Дāδ} ту коргар-ен=ен фук=аθ йаст=о?//
\glc {\sc d2.pl} {\sc pron.2sg} рабочий-{\sc pl=3pl} все={\sc int} есть={\sc q}//
\endgl
\a<b> \begingl
\gla \b{Мāδ} ту коргар-ен=ен фук=аθ йаст=о?//
\glc {\sc d1.pl} {\sc pron.2sg} рабочий-{\sc pl=3pl} все={\sc int} есть={\sc q}//
\glft ‘\b{Эти} твои работники все присутствуют?’ \trailingcitation{\parencite[35]{yusufbekov1998}}//
\endgl \xe

Приведённые выше примеры мы взяли у Ш.~П.~Юсуфбекова. Описанный им опыт очень схож с тем, что можно наблюдать во втором опыте второго эксперимента, а также в отношении референта №2 для первого опыта эксперимента. Бригадир, обращаясь к звеньевому, находится вместе с ним в одном дискурсивном пространстве, а работники, о которых первый спрашивает второго, — в другом. Вышерассмотренные примеры говорят нам о существовании пограничных областей коммуникативного пространства, где допустимо употребление как I, так и II серии, как II, так и III серии демонстративов.

Теперь выдвинем предположение о двух стратегиях построения дейктической системы, наблюдаемых в этом эксперименте. Нельзя сказать, что их данные представляют собой безупречную оппозицию: можно наблюдать колебания от более дистантного к более личному пониманию пространства носителем. Однако между ними можно провести чёткую границу, где одна система будет принадлежать стратегии A, а другая — стратегии B (как в примере с тремя УМ II серии, рассмотренном выше — стратегия определяется в зависимости от употребления одного ключевого дейктика). При этом выпадение из употребления УМ II серии и переход системы к двухчастной свойственны обеим стратегиям. В случае стратегии A это происходит из-за исчезновения точки опоры в лице адресата во втором опыте, в случае стратегии B — из-за изначального непринятия во внимание положения адресата. Разделение данных эксперимента на стратегии A и B даёт возможность постулировать лично-дистантную оппозицию в рамках дейксиса шугнанского языка. Таким образом, к стратегии A отнесены те идиолекты, которым свойственна личная ориентация, а к стратегии B те, которым свойственна дистантная. Стратегия A представляется нам более традиционной для шугнанского языка, исходя из вышеупомянутых памироведческих описаний.

Иначе можно было бы представить природу подобного разделения как диалектную, т.~е.~связать её с областью распространения тех или иных языковых явлений. Так, например, М.~Жиц Фукс в своём исследовании дейксиса в хорватском языке находит различия в употреблении демонстративов в зависимости от того, проживает носитель в столичном Загребе или в сельской местности \parencite[55]{zic_fuchs1996}. Такое разделение не представляется нам возможным, поскольку носителей шугнанского, предпочитающих ту или иную стратегию, не удаётся распределить тем же образом в зависимости от происхождения, «образа жизни» или «физического окружения», как то обнаружила для хорватского М.~Жиц Фукс \parencite[60]{zic_fuchs1996}. В Таблице \ref{tab:dem4} представлено распределение ответов информантов и местности их происхождения и проживания. Курсивом отмечены ответы информантов в рамках стратегии B. Не претендуя на статистический анализ, отметим, что данные не позволяют представить какой-либо зависимости выбора стратегии от места. Информанты, происходящие из одного города или джамоата (сельской общины в Таджикистане) или проживающие в одном городе, прибегают к разным стратегиям употребления УМ.

\begin{sidewaystable}
 \centering
 \caption{Результаты эксперимента 2 и места происхождения / проживания информанта}
 \smallskip
 \label{tab:dem4}
 \begin{tabular}{c|ccc|ccc|cc} \toprule
 информант & \multicolumn{3}{c|}{опыт 1} & \multicolumn{3}{c|}{опыт 2} & \makecell{место\\ происхождения} & \makecell{место\\ проживания} \\ \midrule
 3 & I & II & II & I & II & III & Кушк (дж.~Поршинев)* & Хорог \\
 11 & I & II & III & I & I/II & III & Поршинев & Москва и МО* \\
 4 & I & I & II & I & I & III & Вуж (дж.~Вер) & Москва и МО \\
 5 & I & II & — & I & III & III & — & Душанбе \\
 6 & I & II & II & I & III & III & Душанбе & Душанбе \\
 2 & I & II & II & I & II & III & Хорог & Хорог \\
 7 & I & II & II & I & III & III & Хорог & Душанбе \\
 9 & I & I & III & I & I & III & Хорог & Хорог \\
 8 & II & II & II/III & II & II & III & дж.~Дарморахт & Душанбе \\
 10 & I & III & III & I & III & III & Парзудж & Хорог \\
 1 & I & II & II & I & III & III & \makecell{Дашт (дж. Мирсаид\\ Миршакар)} & Дашт \\
 12 & I & II & III & I & II & III & Рошткалинский~дж. & Душанбе\\ \bottomrule
 \end{tabular}
 \\
 \medskip
 \hspace*{\fill}{\small *\i{дж.} — джамоат, МО — Московская область.}
\end{sidewaystable}

\section{Эксперимент третий. Тест на невидимость} \label{dem-exp3}

В основе третьего эксперимента были позиции адресата и говорящего, отличные от опыта 2 второго эксперимента только тем, что референты оказывались у них за спиной, см.~Рисунок \ref{fig:dem3}. Референты были уже «известны» участникам коммуникации из предыдущего эксперимента и становились невидимы для них лишь на этом этапе, что позволяет говорить об анафорическом контексте при их упоминании, как в мнемонических детских играх, где участники определённое время запоминают предметы, а после этого перечисляют их по памяти.

\begin{figure}[H]
 \centering
 \caption{Третий эксперимент \parencite[30]{popiel2009}}
 \smallskip
 \label{fig:dem3}
 \includegraphics[width=0.8\textwidth]{img/dem3.jpg}
\end{figure}

В Таблице \ref{tab:dem5} приведены результаты эксперимента. Для сравнения также указаны ответы информантов в последнем опыте второго эксперимента, когда локуторы стоят лицом к ряду объектов (он же на второй половине рисунка \ref{fig:dem3}). \b{Жирным} выделены те случаи, где выбор местоимения не зависит от видимости~/~невидимости референта.

\begin{table}[h]
 \centering
 \caption{Результаты эксперимента 3}
 \smallskip
 \label{tab:dem5}
 \begin{tabular}{c|ccc|ccc} \toprule
 информант & \multicolumn{3}{c|}{эксперимент 3} & \multicolumn{3}{c}{\makecell{эксперимент 2\\ (опыт 2)}} \\ \midrule
 1 & III & III & III & I & III & III \\
 2 & \b{I} & \b{II} & \b{III} & \b{I} & \b{II} & \b{III} \\
 3 & \b{I} & \b{II} & \b{III} & \b{I} & \b{II} & \b{III} \\
 4 & I & III & III & I & I & III \\
 5 & II & III & III & I & III & III \\
 6 & II/III & III & III & I & III & III \\
 7 & II & III & III & I & III & III \\
 8 & II & III & III & II & II & III \\
 9 & \b{I} & \b{I} & \b{III} & \b{I} & \b{I} & \b{III} \\
 10 & \b{I} & \b{III} & \b{III} & \b{I} & \b{III} & \b{III} \\
 11 & I & II/III & III & I & I/II & III \\
 12 & II & II & II/III & I & II & III\\ \bottomrule
 \end{tabular}
 \\
 \medskip
\end{table}

\pagebreak[4]

Сразу следует отметить, что мы не разделяем данные третьего эксперимента на отдельные стратегии употребления демонстративов. В том числе то разделение, что имело место в рамках предыдущего эксперимента, здесь не сохраняется. Однако представляется возможным выделить несколько групп УМ, имеющих общие подходы к реализации дискурса.

Первой такой группой можно назвать ту, в рамках которой употребление демонстративов не зависит от видимости/невидимости референтов (4 случая). В их число входят системы, тяготеющие как к личной, так и к дистантной ориентации.

Однако системы идиолектов, наиболее устремлённые к личной ориентации, в эту группу не входят. Критерием выделения таковых стало исчезновение из употребления II серии УМ при совпадении позиций говорящего и адресата во втором эксперименте (информанты 1, 4, 5, 6, 7). Мы знаем, что УМ III серии употребляются как исключения как из сферы адресата, так и из сферы говорящего и имеют семантику удалённости \parencite[32]{yusufbekov1998}. Таким же образом, по-видимому, исключаются референты из пространства речевого акта, оказываясь невидимыми. Для двух информантов из этой группы (информанты 1 и 6) существует способ маркирования, заключающийся в употреблении УМ III серии относительно всех референтов. Этот способ маркирования (III–III–III), засвидетельствованный в речи двух носителей, имеет типологическое сходство с дейктической системой португальского языка, отмечаемой А.~А.~Ростовцевым-Попелем в том же контексте \parencite[30]{popiel2009}. Речи информантов 4, 8 и 11 также свойственен сдвиг I~$\rightarrow$~III или II~$\rightarrow$~III — с той разницей, что он не приводит к маркированию всех референтов III серией УМ. На основании ответов этих пяти информантов выделим вторую группу, в которой стимул невидимости вызывает маркирование референтов при помощи III серии УМ.

Третью группу мы выделим из тех примеров, где реакцией на невидимость референтов стал сдвиг в сторону II серии (I~$\rightarrow$~II и III~$\rightarrow$~II). К ней можно отнести ответы информантов 5, 6, 7, 11, 12. Заметим, что идиолекты одних и тех же информантов (6, 11) допускают отнесённость как ко второй, так и к третьей группе. В свою очередь, у Ш.~П.~Юсуфбекова есть пример употребления УМ I и II серии в схожем контексте. Говорящий сидит спиной к окну и задаёт вопрос (\getfullhref{exdem5.a}), его собеседник, напротив, наблюдает за происходящим снаружи (\getfullhref{exdem5.b}):

\pex<exdem5>
\a<a> \begingl
\gla — Ар \b{дам} мāш кийоска=йен ку газӣт вӯɣ̌ҷ=о, ~~~~~~ йид ғал чуст?//
\glc ~~~~~~ {\sc down} {\sc d2.f.sg.o} {\sc pron.1pl} киоск={\sc 3pl} {\sc ptcl} газета нести.{\sc pf=or} ~ {\sc d2.sg} ещё закрытый//
\glft ‘Привезли ли в \b{(э)тот} (II~серия) наш киоск газеты, или он ещё закрыт?’//
\endgl
\a<b> \begingl
\gla — \b{Йам}=и йет чӯɣ̌ҷ=ат, газӣт=ен ~~~~~~~~~~~~~~~~~~~ ғал гуму̊н на-вӯɣ̌ҷ.//
\glc ~~~~~~ {\sc d1.sg=3sg} открытый делать.{\sc pf=and2} газета={\sc 3pl} ~ ещё видимо {\sc neg}-нести.{\sc pf}//
\glft ‘\b{Этот} (I~серия) [киоск] уже открыт, но газеты, видимо, ещё не привезли.’ \trailingcitation{\parencite[39]{yusufbekov1998}}//
\endgl \xe

Здесь существуют два критерия разграничения I и II серий: во-первых, это «видимость/невидимость определяемого объекта для говорящего и собеседника (т.~е.~видимое выступает как близкое, невидимое воспринимается как далёкое)» \parencite[39]{yusufbekov1998}, во-вторых, упоминается субъективная точка зрения говорящего, вызванная его отличным от адресата местоположением. Поскольку же в третьем эксперименте позиции говорящего и адресата совпадают, внимание здесь следует сосредоточить на первом критерии.

В свою очередь переход III~$\rightarrow$~II, наблюдаемый в идиолекте информанта 12 в контексте третьего эксперимента, с трудом поддаётся объяснению через дейктическое противопоставление, ведь II серия УМ в оппозиции II~\textasciitilde~III серий отвечает за условно более близкий объект, находящийся в пространстве, соотносимом с собеседником. Эти данные заставляют нас обратиться к определению II серии как общеуказательной. Тенденцию к утрате у II серии УМ значения указания на сферу собеседника и возникновение оппозиции II~\textasciitilde~III серий по наличию у II серии «значения подчёркнутого указания» и отсутствия такового у III серии отмечает Ш.~П.~Юсуфбеков [\cite*[41]{yusufbekov1998}]. Наша уверенность в употреблении II серии УМ здесь в значении общеуказательной основывается на ряде причин. Во-первых, в эксперименте 3 данные информанта 12 дают нам систему, где все референты маркируются II серией. Это подтверждает отсутствие противопоставления из-за удалённости референтов друг от друга и от дейктического центра. Во-вторых, мы определили дейктик II серии как немаркированный компонент в шугнанской системе. Из этого следует, что употребление II серии информантом в таком контексте (II–II–II) недейктическое, оно не маркирует удалённость от дейктического центра, однако указывает на значение дейктика II серии как дейктика «по умолчанию».

\pagebreak[2]

С целью исключить, возможно, слишком навязчивую интерпретацию пространственного определения референтов в парадигме «ближний–средний–дальний», мы включили в систему четвёртый референт и провели дополнительный эксперимент. В него были включены два опыта, необходимые для решения нашей задачи: первый — опыт 2 второго эксперимента, но с четырьмя объектами вместо трёх, второй — опыт третьего эксперимента с четырьмя объектами вместо трёх (см.~Рисунок \ref{fig:dem3}). В Таблице \ref{tab:dem6} представлены ответы информантов: в них абсолютно преобладает употребление II серии УМ. Только в одном случае информант 2 не указал на второй по удалённости референт с помощью II серии, однако допустил возможность такого указания. Этот эксперимент даёт нам большее представление о II серии как общеуказательной. Возможно, когда количество референтов, расположенных на приблизительно одинаковом удалении, становится больше трёх, необходимость в дейктическом противопоставлении этих объектов резко отпадает.

\begin{table}
 \centering
 \caption{Результаты дополнительного эксперимента}
 \smallskip
 \label{tab:dem6}
 \begin{tabular}{c|cccc|cccc} \toprule
 информант & \multicolumn{4}{c|}{опыт 1} & \multicolumn{4}{c}{опыт 2} \\ \midrule
 1 & II & II & II & II & I/II & II & II & II \\
 2 & II & II & II & II & II & II & III(II) & II\\ \bottomrule
 \end{tabular}
 \\
 \medskip
\end{table}

Сделаем выводы по третьему эксперименту. Основное разделение по третьему эксперименту — наличие или отсутствие реакции на стимул невидимости. Противопоставление лично- и дистантно-ориентированных систем в этом контексте не даёт нам такого чёткого разделения, как в предыдущем эксперименте, однако членение на отдельные меньшие группы в данных сохраняется. В первую группу мы выделили системы, где фактор невидимости информантами игнорировался. Во вторую и третью группы вошли идиолекты, в которых невидимость референтов стала стимулом для большего употребления III и II серии УМ соответственно. Мы видим, что шугнанский дейксис допускает определённые способы маркирования невидимых референтов, но нельзя сказать, что какой-то один из них доминирует.

Теперь, опираясь на вышеперечисленные примеры, обратим внимание на данные К.~Мюллер. В работе \parencite{muller2015} шугнанские УМ группируются по степени удалённости от говорящего: ближней, средней и дальней соответственно; это местоимения \i{ми}/\i{мам}/\i{мāδ}, \i{ди}/\i{дам}/\i{дāδ} и \i{wи}/\i{wам}/\i{wāδ} \parencite[57]{muller2015}. УМ \i{йам} и \i{йид} указаны как детерминативы с дейктическими функциями: \i{йам} указывает на ближнюю область в рядом с говорящим, \i{йид} на ближнюю область в пространстве непосредственно перед говорящим \parencite[59]{muller2015}. Однако ещё по примерам (\gethref{exdem6})–(\gethref{exdem7}) \parencites[74–77]{belikov1972}[22]{yusufbekov1998} можно установить, что \i{йам} и \i{йид} употребляются как демонстративы и не требуют после себя именной группы, как это делают детерминативы. К демонстративам \i{йам} и \i{йид} относит в своей классификации и М.~Аламшоев [\cite*[31]{alamshoev1994}].

\ex<exdem6>
\begingl
\gla Йу=йи бāд дараw бирêх̌т су̊д, ~~~~~~~~~~~~~~~~~~~~~~~~~~~~~~~~~~~~ ху лу̊в-ҷ=и: «Аҷаб ба-маза на-вад? ~~~~~~~~~~~~ Ду̊нд ик=га рӯған ца вед, \b{йам} соф-аθ ба-маза су̊д».//
\glc {\sc d3.m.sg=3sg} потом {\sc inc} пить.{\sc inf} стать.{\sc prs.3sg} ~ {\sc and1} говорить-{\sc pf=3sg} {\sc ptcl} {\sc all}-вкус {\sc neg}-быть.{\sc pst.f/pl} ~ столько {\sc emph=add} масло {\sc subd} быть.{\sc prs.3sg} {\sc d1.sg} совсем-{\sc adv} {\sc all}-вкус стать.{\sc prs.3sg}//
\glft ‘И он потом начал пить и сказал: «[Разве] не удивительно вкусно? Ещё немного масла было бы, \b{это} стало бы совсем вкусно».’ \trailingcitation{\parencite[75]{belikov1972}}//
\endgl \xe

\ex<exdem7>
\begingl
\gla — Бийор=ам цу̊нд ту х̌икӯд, ~~~~~~~~~~~~~~~~~~~ на-вӯд=āм ту. ~~~~~~~~~~~~~~~~~~~~~~~~~~~~~~~~~~~~~~~~~~~~~~~~~~~~~~~~~~~~~~~~~~~~~~~~~~~~ — Цу̊нд=анд вуд \b{йид}?//
\glc ~~~~~~ вчера={\sc 1pl} сколько {\sc pron.2sg} искать.{\sc pst} ~ {\sc neg}-найти.{\sc pst=1pl} {\sc pron.2sg} ~ ~~~~~~ сколько={\sc loc} быть.{\sc pst.m.sg} {\sc d2.sg}//
\glft ‘— Вчера столько тебя искали, но не нашли. — Во сколько было \b{это}?’ \trailingcitation{\parencite[22]{yusufbekov1998}}//
\endgl \xe

На рассмотренном выше примере (\gethref{exdem5}), а также на основании обозначенных выше контекстов употребления II серии УМ мы видим, что соотнесение \i{йид} с сугубо ближней областью в пространстве перед говорящим у Мюллер [\cite*[59]{muller2015}] не раскрывает всю вариативность употребления этого местоимения в шугнанском языке. Из выводов Мюллер [\cite*[60]{muller2015}] также следует, что употребление местоимений \i{ми}/\i{мам}/\i{мāδ}, \i{ди}/\i{дам}/\i{дāδ} — I и II серии соответственно — невозможно в контекстах невидимости. Из полученных нами данных эксперимента 3 можно выделить систему, которая действительно работает таким образом (идиолект информанта 1). Несомненно, способ маркирования невидимых референтов при помощи III серии существует. Однако, исходя из того же примера (\gethref{exdem5}), а также совокупности данных, полученных нами в ходе эксперимента, мы можем опровергнуть эти выводы К.~Мюллер.

\section{Заключение} \label{dem-conclusion}

В ходе нашего исследования была установлена большая вариативность в построении дейктических систем в шугнанском языке. Это связано с разнонаправленными стратегиями, применявшимися носителями в ходе экспериментов. Первый эксперимент позволил нам установить немаркированный компонент в шугнанской системе — II серию УМ, при этом его данные были лишены даже единичных отклонений. Такое определение II серии УМ коррелирует с её значением подчёркнутого указания. Результаты второго эксперимента подтверждают существование личной и дистантной оппозиции в языке. Мы установили две стратегии употребления демонстративов в данном контексте, где принципом разделения является наличие или отсутствие реакции носителя на смену положения адресата в пространстве. Третий эксперимент позволил выявить способы маркирования невидимых объектов. Его данные представляются наиболее вариативными, вместе с этим разделение на группы в третьем эксперименте не соответствует разделению на две стратегии во втором эксперименте. Также мы склонны считать, что природу вариативности систем нельзя объяснить диалектным территориальным членением, так как приведённая нами география распространения явлений не даёт нам для этого никаких оснований. Дистантная ориентация в контексте трансформации системы в двухчастную, засвидетельствованная в проведённых нами экспериментах, не была отмечена ранее в описаниях шугнанского языка. В свою очередь, определение УМ только через парадигму удаленности «ближний–средний–дальний» в не в полной мере отражает семантику трёхчастных систем. Также и в шугнанском, факторы видимости, соотнесённости со сферами говорящего и адресата, а также количества объектов референции влияют на употребление того или иного УМ и создают основания для внутриязыковой вариативности.
