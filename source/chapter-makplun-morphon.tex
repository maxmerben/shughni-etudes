\chapter*{К~описанию шугнанской морфонологии: предварительные замечания}
\addcontentsline{toc}{chapter}{\textit{Ю.~Макаров, В.~Плунгян}. \textbf{К~описанию шугнанской морфонологии: предварительные замечания}}
\setcounter{section}{0}
\chaptermark{К~описанию шугнанской морфонологии: предварительные…}
\label{chapter-makplun-morphon}

\begin{customauthorname}
Юрий Макаров, Владимир Плунгян
\end{customauthorname}

\begin{englishtitle}
\i{Towards the description of Shughni morphophonology: Preliminary notes\\{\small Yury Makarov, Vladimir Plungian}}
\end{englishtitle}

\begin{abstract}
В настоящих заметках мы хотели бы привлечь внимание к чрезвычайно богатой и разнообразной системе вокалических и консонантных чередований, засвидетельствованных в шугнанском языке. Мы ограничимся только шугнанским материалом (как наилучшим образом документированным на сегодняшний день), хотя рассматриваемые явления в большинстве случаев характерны и для других памирских языков и могут считаться их общей чертой. Своеобразные системы чередований памирских языков, безусловно, представляют значительный интерес не только для иранистики, но и для ареальной типологии и для той части общей теории языка, которая занимается взаимодействием морфологии и фонологии (или, как сейчас модно говорить, проблематикой интерфейса).
\end{abstract}

\begin{keywords}
морфонология, шугнанский язык, памирские языки, иранские языки, чередования.
\end{keywords}

\begin{eng-abstract}
In these notes, we would like to draw attention to the extremely rich and diverse system of vocalic and consonantal alternations attested in the Shughni language. We will consider only the Shughni material (as best documented to date), although the phenomena in question are in most cases characteristic of other Pamir languages and can be considered their common feature. The peculiar systems of alternations of Pamir languages, of course, are of considerable interest not only for Iranian studies, but also for areal typology and for that part of general language theory that deals with morphology/phonology interface.
\end{eng-abstract}

\begin{eng-keywords}
morphophonology, Shughni, Pamir languages, Iranian languages, alternations.
\end{eng-keywords}

\begin{initialprint}
\fullcite{makarov_plungian2023}\end{initialprint}

В настоящих заметках мы хотели бы привлечь внимание к чрезвычайно богатой и разнообразной системе вокалических и консонантных чередований, засвидетельствованных в шугнанском\fn{Шугнанский язык, насчитывающий около~100~тысяч носителей, распространён в нескольких районах таджикского и афганского Бадахшана (в том числе в центре Горно-Бадахшанской автономной области городе Хорог) и является наиболее крупным и социолингвистически значимым языком так называемой памирской подгруппы восточноиранской ветви иранской группы. Он входит в состав шугнано-рушанского кластера, объединяющего около десятка близкородственных памирских языков и диалектов, но с гораздо меньшим числом говорящих.} языке. Мы ограничимся только шугнанским материалом (как наилучшим образом документированным на сегодняшний день), хотя рассматриваемые явления в большинстве случаев характерны и для других памирских языков и могут считаться их общей чертой. Своеобразные системы чередований памирских языков, безусловно, представляют значительный интерес не только для иранистики, но и для ареальной типологии и для той части общей теории языка, которая занимается взаимодействием морфологии и фонологии (или, как сейчас модно говорить, проблематикой интерфейса).

При описании чередований шугнанского языка мы используем как (немногочисленные) данные существующих грамматических описаний (прежде всего \parencite{karamshoev1963}, а также \parencite{edelman_yusufbekov1999_shughni}) и словаря \parencite{karamshoev1988}\fn{Электронная версия этого словаря с рядом исправлений и дополнений размещена на сайте \i{\href{https://pamiri.online}{pamiri.online}}, см.~также \parencite{makarov_etal2022}.}, так и собственные полевые материалы, полученные в ходе работы с носителями шугнанского языка в 2017–2022~годах. Особое значение для нашего исследования имеет статья \parencite{muravieva1975}, являющаяся единственной известной нам публикацией, специально посвящённой данной тематике. Работа И.~А.~Муравьёвой была написана до появления обширного словаря Д.~Карамшоева, но в ней, помимо полевых материалов автора, учтены данные словаря в издании шугнанских текстов И.~И.~Зарубина [1960], а также базовые сведения о шугнанской фонетике, приведённые в исследовании \parencite{sokolova1953}. Важную роль для решения нашей задачи играют также данные сравнительно-исторических исследований (ср.,~например, \parencites{pakhalina1983}{edelman1986}).

В шугнанском языке можно выделить два различных класса чередований (между которыми, однако, нет абсолютно жёсткой границы). С одной стороны, это чередования, которые сами по себе являются носителями различных грамматических значений — как правило, единственными, без поддержки сегментных средств. С другой стороны, это чередования, которые можно считать обусловленными фонологическим контекстом: последовательность фонем A заменяется на последовательность фонем B в определённом фонетическом окружении (либо всегда, либо только в определённых условиях, характер которых также важен для классификации чередований).

Данное различие хорошо известно в теоретической лингвистике и фигурирует под разными названиями. Чередования первого типа могут называться, например, «непозиционными» (как у И.~А.~Муравьёвой) или «апофониями» (как это предлагал И.~А.~Мельчук, ср.~\parencite{melchuk2000} и, более специально, \parencite{melchuk1996}). Чередования второго типа — соответственно «позиционные» или «альтернации» — далее подразделяются по степени обязательности появления чередования в соответствующем контексте и по типу ограничений на реализацию чередования: так, известно противопоставление «автоматических» и «исторических» чередований \parencite{maslov1979}, а также различение внутри «исторических» чередований тех, реализация которых связана с лексическими или с грамматическими признаками, и тому подобных; подробнее см.~обзор этой проблематики в \parencite{plungian2009}.

В шугнанском языке представлены все эти типы чередований; попробуем ниже дать предварительную характеристику основным явлениям в этой обширной области. Поскольку фонетические различия между вариантами языка могут быть достаточно значительны, заранее условимся, что в отсутствие специальных замечаний ниже описывается та разновидность шугнанского языка, на которой говорят в Хороге. Перед этим кратко охарактеризуем шугнанский сегментный инвентарь в сопоставлении с латинской иранистической орфографией (приводится в скобках, когда отличается от символа МФА)\fn{Вообще для фонологического описания шугнанского вокализма не обязательно использовать понятие долготы. Несмотря на наличие фонетических свидетельств существования различий по длительности между фонемами типа /i/ и /ɪ/, вполне достаточно качественного противопоставления между этими сегментами (по ряду и подъёму, по напряжённости). В первой части нашей статьи, однако, мы придерживаемся более традиционных взглядов на шугнанский вокализм, чтобы не создавать слишком сильного разрыва с предшествующими описаниями.}\fn{Таблицы~1 и 2 и транскрипция шугнанских примеров в статье были отредактированы в соответствии с нынешними представлениями авторов о шугнанской фонетике. В таблицах в угловых скобках <> указаны буквы, соответствующие этим фонемам в шугнанской кириллице, — \i{прим.~переиздания}.}.

\begin{sidewaystable}
 \centering
 \caption{Консонантизм шугнанского языка}
 \smallskip
 \label{tab:morphon2}
 \begin{tabular}{r|ccccccccc} \toprule
 {\small \makecell[r]{\\\\\\способ~$\downarrow$ и место~$\rightarrow$\\образования}} & \rotatebox[origin=l]{270}{\hspace{-20pt}{\small билабиальные~~~}} & \rotatebox[origin=l]{270}{\hspace{-20pt}{\small \makecell[l]{лабио-\\дентальные}}} & \rotatebox[origin=l]{270}{\hspace{-20pt}{\small дентальные}} & \rotatebox[origin=l]{270}{\hspace{-20pt}{\small альвеолярные}} & \rotatebox[origin=l]{270}{\hspace{-20pt}{\small \makecell[l]{палато-\\альвеолярные}}} & \rotatebox[origin=l]{270}{\hspace{-20pt}{\small палатальные}} & \rotatebox[origin=l]{270}{\hspace{-20pt}{\small велярные}} & \rotatebox[origin=l]{270}{\hspace{-20pt}{\small увулярные}} \\ \midrule
 \multirow{2}{*}{{\small взрывные}} & p <п> & & t <т> & & & & k <к> & q <қ> \\
 & b <б> & & d <д> & & & & ɡ <г> & \\
 \multirow{2}{*}{{\small фрикативные}} & & f <ф> & θ <θ> & s <с> & ʃ <ш> & & x <х̌> & χ <х> \\
 & & v <в> & ð <δ> & z <з> & ʒ <ж> & & ɣ <ɣ̌> & ʁ <ғ> \\
 \multirow{2}{*}{{\small аффрикаты}} & & & ʦ <ц> & ʧ <ч> & & & & \\
 & & & ʣ <ӡ> & ʤ <ҷ> & & & & \\
 {\small носовые} & m <м> & & n <н> & & & & & \\
 {\small дрожащие} & & & & & r <р> & & & \\
 {\small аппроксиманты} & w <w> & & & & & j <й> & & \\
 {\small боковые аппроксиманты} & & & l <л> & & & & & \\ \bottomrule
 \end{tabular}
\end{sidewaystable}

\begin{table}
 \centering
 \caption{Вокализм шугнанского языка}
 \smallskip
 \label{tab:morphon1}
 \begin{tabular}{r|cccccc} \toprule
 подъём~$\downarrow$ и ряд~$\rightarrow$ & \multicolumn{2}{c}{передний} & \multicolumn{3}{c}{средний} & задний \\ \midrule
 верхний & \multicolumn{2}{c}{i <ӣ>} & & & & u <ӯ>
 \\
 средне-верхний & e <е> & ø <у̊> & ɪ <и> & & ʊ <у> & \\
 средне-верхний & & & ɛ <ê> & & ɔ <о> & \\
 нижний & & & & a <а> & & ɑ <ā> \\ \bottomrule
 \end{tabular}
\end{table}

\section{Непозиционные («грамматические») чередования} \label{morphon-nepozic}

Яркая особенность памирских языков, как уже было сказано, — это обилие несегментных морфологических средств, используемых для выражения разнообразных грамматических значений. В области именной морфологии чередованиями выражаются граммемы рода у указательных местоимений и некоторых прилагательных\fn{А также у группы наречий образа действия (типа \i{θуппаст} [м.] $\sim$ \i{θаппаст} [ж.] ‘быстро; сразу’) и производных от прилагательных существительных, обозначающих свойство (типа \i{ӡулики} ‘детство (мужчины)’ $\sim$ \i{ӡалики} ‘детство (женщины)’ от \i{ӡулик} $\sim$ \i{ӡалик} ‘маленькая’).} (и у тех существительных, которые обозначают пары гендерно противопоставленных живых существ типа \i{ворҷ} ‘жеребец’ $\sim$ \i{вêрӡ} ‘кобыла’). У существительных чередование возможно также в основе множественного числа некоторых лексем (ср.~\i{пуц} ‘сын’ $\sim$ \i{пацен} ‘сыновья’), однако в современном языке такие формы утрачиваются.

В области глагольной морфологии чередования имеют существенно бо́льшую функциональную нагрузку и выражают граммемы практически всех базовых глагольных категорий: Претерита, Перфекта, Инфинитива, а также лица, числа и рода подлежащего; аффиксальная морфология в выражении этих значений либо вовсе не участвует, либо участвует минимально.

Важно отметить также, что дистрибуция гласных в глагольных словоформах в большинстве случаев подчиняется дополнительным (мор)фонологическим правилам, применение которых накладывается на непозиционные чередования гласных и усложняет общую картину. К важнейшим из этих правил относятся:

\begin{enumerate}[(1)]
	\item \label{exmorphon1} \b{Морфонологическое удлинение} гласного глагольного корня перед группой согласных, последний из которых — показатель 3-го~лица ед.~ч. Презенса -\i{д} / -\i{т} (ср. \i{wинум} ‘вижу’ $\sim$ \i{wӣнт} ‘видит’)\fn{Существенно, что показатель 3-го~лица ед.~ч. — единственный чисто консонантный аффиксальный показатель лица/числа подлежащего; все остальные показатели начинаются с гласной.}; в случае гласных /a/ или /ɑ/ в аналогичных контекстах происходит чередование с /ɔ/ (ср.~\i{вираɣ̌ен} ‘ломаются’ $\sim$ \i{вироɣ̌д} ‘ломается’, \i{wāфум} ‘плету’ $\sim$ \i{wофт} ‘плетёт’).
	\item \label{exmorphon2} \b{Морфонологическая монофтонгизация}, предписывающая чередование дифтонга /ɑw/ в глагольном корне с гласной /ø/ перед тем же показателем 3-го~лица ед.~ч., ср.: \i{нāwен} ‘плачут’ $\sim$ \i{ну̊д} ‘плачет’, \i{сирāwен} ‘отделяются’ $\sim$ \i{сиру̊д} ‘отделяется’. Характерное исключение — \i{мāwен} ‘мяукают’ $\sim$ \i{мāwт} ‘мяукает’; ср.~также отсутствие подобного чередования у имён, например, \i{кāwҷ} ‘щенок (оскорбительное)’\fn{В статье И.~А.~Муравьёвой указывается и на монофтонгизацию /ɑj/ в /e/, однако, по данным \i{\href{https://pamiri.online}{pamiri.online}}, это актуально не для собственно шугнанского (где финаль \i{ай} реализуется как \i{и} перед вокалическими и как e перед консонантными аффиксами), но для его диалектных вариантов (например, баджувского и барвазского).}.
	\item \label{exmorphon3} \b{Преназальное повышение}, при котором /ɔ/ чередуется с /ø/, а /ɛ/ с /e/ перед носовыми, см.~также (\hyperref[exmorphon7]{7})–(\hyperref[exmorphon8]{8}) ниже.
\end{enumerate}

Несмотря на позиционный характер, само применение указанных правил происходит далеко не во всех случаях, когда контекст это в принципе позволяет; существуют многочисленные исключения, связанные с индивидуальными особенностями глагольных лексем. Чередования также нередко не действуют в недавних заимствованиях и ономатопоэтических словах.

Собственно же непозиционных чередований гласных в шугнанском языке имеется два типа: в традиционных терминах их можно назвать \i{i}- и \i{a}-перегласовкой. При \i{i}-перегласовке производной ступенью оказывается либо /ɛ/ (в случае исходных /ø/ или /ɔ/; перед носовыми выступает в виде алломорфа с /e/), либо /ɪ/ (при исходных /a/, /ɪ/, /ʊ/) или /i/ (при исходных /ɑ/, /i/, /u/). При \i{a}-перегласовке производной ступенью оказывается либо /ɔ/ (в случае исходной гласной /u/), либо /a/ или /ɑ/ (длительность наследуется от исходной гласной). В работе И.~А.~Муравьёвой эти чередования обозначаются как «\i{i}-ступень» и «\i{a}-ступень» соответственно (диахронически по крайней мере некоторые из них действительно восходят к «умлауту», то~есть регрессивной ассимиляции гласной корня под влиянием древней \i{i}-образной или \i{a}-образной гласной суффикса, впоследствии исчезнувшей)\fn{Ср. последовательное использование той же терминологии у Дж.~И.~Эдельман: «<…> в языках шугнано-рушанской группы <…> в результате действия \i{i}- и \i{a}-умлаута с последующим отпадением заударных слогов <…>, обусловивших умлаутные перегласовки в корне, вырабатываются определённые чередования гласных и согласных <…>, играющие морфонологическую роль» \parencite[205]{edelman1986}.}.

Каждое из указанных чередований является носителем большого количества разнообразных грамматических значений как в области именной, так и в области глагольной морфологии. 

Начнём с именной морфологии. Категория рода (состоящая из двух граммем, мужского и женского рода) является словоклассифицирующей для существительных и словоизменительной для прилагательных, указательных местоимений и непереходных глаголов в формах прошедших времён. Противопоставление по роду возможно только в формах единственного числа; при этом регулярность в морфологическом выражении словоизменительных различий по роду свойственна только указательным местоимениям. В шугнанском представлены три дейктических серии этих местоимений (противопоставленные, на первый взгляд, по степени близости к дейктическому центру); дейктические местоимения всех серий различают также формы прямого и косвенного падежа. Из возможных таким образом 12~форм единственного числа (3~степени дальности × 2~рода × 2~падежа) род различают формы косвенных падежей во всех сериях\fn{Ниже в примерах, если специально не оговорено иное, первой всегда приводится форма мужского рода, второй — форма женского.} (I~\i{м.и} $\sim$ \i{м.ам}, II~\i{д.и} $\sim$ \i{д.ам}, III~\i{w.и} $\sim$ \i{w.ам}) и формы прямых падежей в III~серии (\i{й.у} $\sim$ \i{й.ā}); формы прямых падежей I~(\i{йа.м}) и II~(\i{йи.д}) серий рода не различают. Как можно видеть, в целом различие по роду у местоимений выражается скорее супплетивно, однако элементы морфемного (или субморфемного) членения в их составе прослеживаются (выше они показаны точками внутри словоформ): так, показателями дейктической серии являются консонантные элементы \i{м}- / \i{д}- / \i{w}- соответственно, показателями косвенного падежа являются суффиксы -\i{и} для мужского и -\i{ам} для женского рода, а показателем прямого падежа — префикс с начальным \i{й}-. Наиболее затруднено морфемное членение как раз у форм прямого падежа III~серии, различающих род (\i{йу} $\sim$ \i{йā}), но и в их составе выделяется гласный /ʊ/, связанный с выражением мужского рода, и гласные /a ɑ/, связанные с выражением женского рода (подробнее см.~ниже). 

Итак, указательные местоимения в целом хорошо вписываются в одну из моделей выражения рода: все формы женского рода характеризуются наличием гласных /a ɑ/, а все формы мужского рода — наличием гласных /ɪ/ или /ʊ/.

Что касается прилагательных (и других согласуемых классов слов), то они для выражения рода могут использовать как чередование гласных, так и чередование финальных согласных корня\fn{Уже в работе \parencite{muravieva1975} справедливо отмечалось, что чередованиям в шугнанских словоформах могут подвергаться только две последние фонемы основы: в первую очередь это гласная, но в непозиционные чередования могут быть вовлечены и финальные шипящие аффрикаты в основе Перфекта, чередующиеся в этом случае со свистящими. Кроме того, особый класс чередований согласных корня возможен перед альвеолярным показателем основы Претерита в глаголе: в первую очередь это чередования \i{б}/\i{в}, \i{с}/\i{х̌} и \i{з}/\i{х̌}. Их можно описывать как позиционные, но при этом перед омонимичным показателем 3-го~лица ед.~ч. Презенса подобных чередований не происходит, ср.~(первой приводится форма 3-го~лица ед.~ч. Презенса, второй — основа Претерита): \i{х̌ебт}:\i{х̌ӣвд} ‘колотить’, \i{дивест}:\i{дивих̌т} ‘показывать’, \i{абêзд}:\i{абох̌т} ‘глотать’; в то же время имеются глаголы типа \i{ниwозд}:\i{ниwêзд} ‘играть (на струнном инструменте)’, где конечная согласная такому чередованию не подвергается.}. Наиболее распространенным способом выражения граммемы женского рода является \i{a}-перегласовка: она представлена в прилагательных и в формах Претерита непереходных глаголов (переходные глаголы род не различают). Для выражения женского рода при образовании феминитивов от существительных мужского рода используется \i{i}-перегласовка (\i{a}-перегласовка в этой функции тоже встречается, но значительно реже), а также в формах женского рода Перфекта непереходных глаголов (таким образом, претерит и Перфект женского рода почти всегда имеют в шугнанском разные огласовки). Огласовка форм мужского рода при этом может быть различной, но чаще всего встречаются гласные /ʊ/, /u/ или /ɔ/; у прилагательных и глаголов (а также местоимений) в мужском роде иногда встречается огласовка /i/. 

При \i{i}-перегласовке гласных финальные согласные /ʧ/ и /ʤ/ также чередуются с /ʦ/ и /ʣ/ соответственно.

Примеры на \i{а}-перегласовку:

\begin{itemize}
	\item{у прилагательных: \i{шут} $\sim$ \i{шат} ‘хромой/ая’, \i{журн} $\sim$ \i{жарн} ‘круглый/ая’, \i{кут} $\sim$ \i{кат} ‘короткий/ая’, \i{рӯшт} $\sim$ \i{рошт} ‘красный/ая’; \i{цӣх̌} $\sim$ \i{цāх̌} ‘горький/ая’;}
	\item{у существительных: \i{чух̌} $\sim$ \i{чах̌} ‘петух $\sim$ курица’, \i{вӯйд} $\sim$ \i{войд} ‘чёрт $\sim$ ведьма’;}
	\item{у глаголов в Претерите: \i{зибуд} $\sim$ \i{зибад} ‘прыгнул(а)’, \i{тӯйд} $\sim$ \i{тойд} ‘уехал(а)’; \i{сифӣд} $\sim$ \i{сифāд} ‘поднялся $\sim$ поднялась’.}
\end{itemize}

Примеры на \i{i}-перегласовку:

\begin{itemize}
	\item{у прилагательных: \i{маɣ̌ӡу̊нҷ} $\sim$ \i{маɣ̌ӡенӡ} ‘голодный/ая’ (едва ли не единственный пример; следует иметь в виду, что в этих формах представлены «глубинные» \i{о} $\sim$ \i{ê}, подвергшиеся преназальному повышению, см.~выше);}
	\item{у существительных: \i{буц} $\sim$ \i{биц} ‘детёныш, дитя’, \i{куд} $\sim$ \i{кид} ‘кобель $\sim$ сука’, \i{пуш} $\sim$ \i{пиш} ‘кот $\sim$ кошка’, \i{гуҷ} $\sim$ \i{гиҷ} ‘козлёнок $\sim$ козочка’; \i{нибос} $\sim$ \i{нибêс} ‘внук $\sim$ внучка’, \i{ворҷ} $\sim$ \i{вêрӡ} ‘конь $\sim$ кобыла’;}
	\item{у глаголов в Перфекте: \i{зибуδҷ} $\sim$ \i{зибиц} ‘прыгнул(а)’; \i{тӯйҷ} $\sim$ \i{тиц} (< *\i{тӣйц}) ‘уехал(а)’; \i{сифӣδҷ} $\sim$ \i{сифӣц} ‘поднялся $\sim$ поднялась’ (следует обратить внимание также на сложные консонантные преобразования в исходе форм женского рода). У существительных и прилагательных данные чередования имеют весьма ограниченное применение: субстантивные феминитивы в целом немногочисленны и непродуктивны, а прилагательные, чередующиеся по роду, составляют явное меньшинство (так, в грамматике \parencite{karamshoev1963} таких пар перечислено менее сорока). Не подвергаются чередованию такие частотные прилагательные, как, например, \i{ғулā} ‘большой, старший’, \i{дах̌т} ‘ровный, широкий’, \i{тунд} ‘острый, едкий’, \i{х̌ӣн} ‘серый, голубой’ и многие другие. Большинство адъективных лексем с родовой перегласовкой либо теряют чередование в речи молодых носителей, либо вовсе неизвестны младшему поколению. В то же время в глагольных формах Претерита и Перфекта чередование по роду сохраняется; представляет особый интерес тот факт, что показатель женского рода в формах Претерита и Перфекта формально разный: в первом случае это \i{a}-перегласовка, во втором случае — \i{i}-перегласовка.}
\end{itemize}

Обратимся теперь к глагольной морфологии. В шугнанском глаголе морфологически выражаются следующие словоизменительные граммемы: Презенс, Претерит, Перфект, Инфинитив; в формах Презенса морфологически выражаются лицо и число подлежащего, в формах Претерита и Перфекта непереходных глаголов морфологически выражается число и (в формах единственного числа) род подлежащего. В выражении всех этих значений участвуют непозиционные чередования, формируя так называемые основы глагола: в общем случае могут различаться 1)~общая основа Презенса ({\sc npst}), 2)~основа 3-го~лица ед.~ч. Презенса, 3)~основа Претерита ед.~ч. мужен.~р., 4)~основа Претерита ед.~ч. жен.~р. и множ.~числа, 5)~основа Перфекта ед.~ч. мужен.~р., 6)~основа Перфекта ед.~ч. жен.~р., 7)~основа Перфекта множ.~числа, 8)~основа Инфинитива\fn{Формально может выделяться также особая основа Императива, однако для рассматриваемой здесь проблематики она нерелевантна: в общем случае Императив образуется от основы Презенса, и особая основа Императива всегда нерегулярна, тогда как перечисленные выше основы образуются путём чередований. Правила образования основ глагола наиболее подробно рассмотрены в \parencite{karamshoev1963} и в статье \parencite{muravieva1975}, где предпринимается попытка установить общие закономерности образования глагольных основ и выделить основные словоизменительные типы шугнанского глагола; к сожалению, в статье И.~А.~Муравьёвой из рассмотрения исключены основы Перфекта, с образованием которых в шугнанском связаны дополнительные сложности.}. При этом основа Презенса не имеет сегментных показателей (выражаясь только чередованием). Показателем 3-го~лица ед.~ч. Презенса, основы Претерита и основы Инфинитива замечательным образом является один и тот же омонимичный суффикс -\i{д} / -\i{т} (в первом приближении, алломорф -\i{т} употребляется после глухих согласных, а также после сонорных, что необычно, ср.~англ.~\i{merged} /məːʤd/, но также и \i{wined} /wɑjnd/ или \i{killed} /kɪld/. Сегментными показателями основы Перфекта являются аффрикаты (со сложным распределением глухих, звонких, шипящих и свистящих алломорфов), присоединяемые к основе Претерита, однако часто с нерегулярными преобразованиями последней, на которых мы не будем подробно останавливаться: ограничимся общим указанием на то, что дентальный взрывной показатель основы Претерита в основах на согласный исчезает (как в глаголе ‘плести’: Претерит \i{wӣфт} $\sim$ Перфект \i{wӣфч}), а в основах на гласный может в основе Перфекта (а)~подвергаться спирантизации, переходя в дентальный фрикативный, как в \i{зинод}~$\rightarrow$ \i{зиноδҷ} ‘мыть’, (б)~заменяться на велярный фрикативный, как в \i{мӯд}~$\rightarrow$ \i{мӯɣ̌ҷ} ‘умирать’, \i{чӯд}~$\rightarrow$ \i{чӯɣ̌ҷ} ‘делать’, (в)~исчезать, как в \i{дӣт}~$\rightarrow$ \i{дӣч} ‘бить’.

Таким образом, во всех случаях, кроме основы Презенса, чередования сопровождаются некоторой минимальной суффиксацией, однако роль чередований в образовании всех восьми возможных основ глагола является ведущей. Вместе с тем, следует подчеркнуть, что в современном шугнанском языке чередования не являются обязательными: существуют глаголы, часть или даже все основы которых полностью совпадают, причём даже одинаковые или сходные в фонетическом отношении основы могут вести себя по-разному. Так, глагол \i{палойс}- ‘работать’ имеет неизменяемую основу (с общей основой Презенса \i{палойс}-, Презенсом 3-го~лица ед.~ч., основой Претерита и Инфинитива \i{палойс.т}, основой Перфекта \i{палойс.ч} и~т.~п.), а глагол \i{риwойс}- ‘голодать’ противопоставляет обе основы Презенса (\i{риwойс}-) всем остальным основам (\i{риwêй}- с основой Претерита и Инфинитива \i{риwêй.д}, Перфекта \i{риwêй-ҷ} и~т.~п.; в баджувском диалекте имеется также другой корень, общий в основах Претерита и Перфекта \i{риwӯй}-). Аналогичным образом глагол \i{тāр}- ‘чистить, убирать грязь’ имеет неизменяемую основу во всех формах, тогда как глагол \i{вāр}- ‘приносить’ обладает максимальным разнообразием основ: общая През.~\i{вāр}-, През. 3-го~лица ед.~ч. \i{вӣр.т}, Прет.~\i{вӯ.д}, Перф.~\i{вӯɣ̌.ҷ}, Инф.~\i{вӣ-д}.

Разумеется, полное описание шугнанской глагольной морфологии выходит далеко за рамки настоящей статьи, поэтому ограничимся констатацией следующих фактов.

\begin{itemize}
	\item{У каждой глагольной граммемы имеется выражающий её тип непозиционного чередования; в некоторых случаях таких типов может быть несколько; возможно также отсутствие чередования.}
	\item{Выбор из возможных для каждой граммемы типов чередований (включая отсутствие чередований) является индивидуальным словарным свойством глагольной лексемы.}
\end{itemize}

Имея это в виду, заметим, что (i)~в общей основе Презенса возможна \i{a}-перегласовка, (ii)~в основе 3-го~лица ед.~ч. Презенса — \i{i}-перегласовка, (iii)~в основе Претерита ед.~ч. жен.~рода и множ.~числа возможна \i{a}-перегласовка (как в формах прилагательных и маргинальных формах множ.~числа существительных типа \i{куд} $\sim$ \i{кад.ен} ‘собаки’), (iv)~в основе Перфекта ед.~ч. жен.~рода возможна \i{i}-перегласовка, (v)~в основе Перфекта множ.~числа возможна \i{a}-перегласовка (аналогично Претериту и именному множ.~числу) и, наконец, (vi)~в основе Инфинитива возможна \i{i}-перегласовка. Таким образом, по типу огласовок объединяются, с одной стороны, 3-е~лицо ед.~ч. Презенса, Перфект ед.~ч. жен.~рода и Инфинитив (\i{i}-образные модели) и, с другой стороны, общий Презенс и Претерит ед.~ч. жен.~рода / множ.~числа (\i{a}-образные модели). Всё это — основы с преобразованной огласовкой (точнее, те, где преобразование огласовки возможно); нетрудно заметить, что оставшаяся вне этого перечня основа претерита мужского рода сохраняет (точнее, может сохранять) «исходную» огласовку (как правило, это /ʊ/, /u/ или /i/, аналогично прилагательным мужского рода).

Дальнейшие преобразования, наблюдаемые в различных глагольных формах, объясняются уже (нерегулярными и неавтоматическими) позиционными чередованиями фонем; часть из них мы упоминали выше.

\section{Позиционные и автоматические чередования} %\label{morphon-pozic}

На другом конце шкалы находятся такие чередования, реализация которых полностью или почти полностью обусловлена фонетическим контекстом; они происходят наиболее «автоматически» и меньше поддаются контролю говорящего.

Одним из наиболее заметных таких чередований является аллофоническое варьирование фонем /ɪ/ и /ʊ/ в конечной позиции. Тогда как внутри слова в закрытом слоге они реализуются закрытыми аллофонами, оказываясь в конце фонетического слова, /ɪ/ реализуется как [ɛ], а /ʊ/ — как [o] или даже [ɔ]: /dɪl/ [dɪl] ‘сердце’, но /dɪ/ [dɛ] ‘бей!’; /kʊt/ [kʊt] ‘короткий’, но /kʊ/ [kɔ] ‘ну-ка’. Такие реализации /ɪ/ и /ʊ/ отмечались уже в середине XX~века (ср.~\parencite{sokolova1953}, где этот процесс описан как собственно фонетический). Итак, соответствующее фонологическое правило формулируется просто\fn{Здесь и ниже долгота в транскрипции обозначается только у /ɑ/ {[}+tense{]} (ср.~/a/ {[}–tense{]}); длительность частично соответствует признаку {[}±tense{]}.}:

\ex[exno=4] \begingl[everygla=] \label{exmorphon4}
\gla \phonr{\phonfeat[l]{–low \\ –tense}}{\phonfeat[l]{+tense}}{\#},//
\glft {\small где \# обозначает границу фонетического слова.}//
\endgl \xe

Однако в действительности ситуация сложнее. Если указать границу фонетического слова в качестве контекста, отсекаются случаи присоединения энклитик, хотя, например, вопросительная частица /ɔ/ не препятствует действию правила (contra \parencite{sokolova1953}): \i{Йид пари.} /ˈjɪd paˈrɪ/ [ˈjɪt paˈɾɛ] ‘Это пери (волшебное существо)’, \i{Йид пари=о?} /ˈjɪd paˈrɪɔ/ [ˈjɪt paˈɾɛjɔ] ‘Это пери?’. Подходящим решением на данном этапе было бы отнести это правило к области морфонологии и использовать следующую формулировку\fn{Стоит отметить, что [–low –tense]~$\rightarrow$~[+tense] / \_\_]\textsubscript{σ}, где ]\textsubscript{σ} — слоговая граница, просто неверно: \i{Йид дил=о?} [ˈjɪd‿ ˈdɪ.lɔ], *[ˈjɪd‿ ˈdɛ.lɔ] ‘Это сердце?’.}:

\ex[exno=5] \begingl[everygla=] \label{exmorphon5}
\gla \phonr{\phonfeat[l]{–low \\ –tense}}{\phonfeat[l]{+tense}}{{]}\textsubscript{корень}},//
\glft {\small где {]}\textsubscript{корень} обозначает границу корня слова.}//
\endgl \xe

Фонема /ʊ/ также подчиняется правилу (\hyperref[exmorphon5]{5}): \i{ду=йи х̌āб=анд} /ˈdʊjɪ~ˈxɑband/ [ˈdɔjɛ~ˈxɑbant] ‘в два (часа) ночи’.

При определении домена действия правила нужно учесть не только клитики, но и морфемы. Формулировка (\hyperref[exmorphon5]{5}) предсказывает, что суффиксы, присоединяющиеся к основе, не будут препятствием для процесса. В действительности это верно лишь для некоторых из них. Так, суффикс множественного числа -\i{ен}, присоединяясь к корню на /ɪ/\fn{Проверить, что будет происходить с основами на /ʊ/ в аналогичной ситуации, весьма трудно: в основном это слова незнаменательных частей речи (частица /kʊ/ ‘ну-ка’), или местоимения (/mʊ/ ‘{\sc 1sg.o}’), или заимствования, которые потенциально могут быть переосмыслены как имеющие /u/. Более того, по данным \i{\href{https://pamiri.online}{pamiri.online}}, отношение словарных входов, оканчивающихся на /ʊ/, ко входам, оканчивающимся на /ɪ/, — 1 к 55.}, не оказывает никакого влияния на (\hyperref[exmorphon5]{5}): \i{пари-йен} /parɪˈen/ [paɾɛˈjen] ‘пери-{\sc pl}’; аналогично — с корнями на /ʊ/: \i{ҷоду} /ʤɔˈdʊ/ [ʤɔˈdɔ] ‘колдун’, \i{ҷоду-йен} /ʤɔˈdʊen/ [ʤɔdɔˈjen] ‘колдун-{\sc pl}’. С глагольными суффиксами всё иначе: \i{си.ц} [sɪʦ], *[sɛʦ] ‘идти.{\sc pf.f.sg}’; \i{ву.д} [vʊt], *[vot] и *[vɔt] ‘быть.{\sc pst.m.sg}’. Видимо, формулировка должна содержать информацию о типе суффикса:

\ex[exno=6] \begingl[everygla=] \label{exmorphon6}
\gla \phonr{\phonfeat[l]{–low \\ –tense}}{\phonfeat[l]{+tense}}{{]}\textsubscript{корень}}   {\oneof{\textsubscript{клитики~или неглагольные суффиксы} {[} \\ \#}},//
\glft {\small где фигурные скобки обозначают вариативную часть правила («или»).}//
\endgl \xe

Наконец, возможно ещё одно решение. По крайней мере для фонемы /ɪ/ ясно, что в современном языке она не встречается в своём основном виде в конечной позиции. Это же с некоторыми оговорками верно и для /ʊ/. Если в середине XX~века, по описанию \parencite{sokolova1953}, ещё были ситуации, когда конечная /ɪ/ могла реализоваться как [ɪ], то для современных носителей произнесения типа [paˈɾɪ] ‘пери’ представляются неверными. Это же отображается в практической орфографии, которой шугнанцы пользуются в повседневной жизни; в ней данное слово могло бы быть записано как \i{паре}, но никак не *\i{пари}. Правило наподобие (\hyperref[exmorphon6]{6}) можно исключить из синхронного описания фонологии, если принять, что все словоформы, кончающиеся на /ɪ/, на самом деле уже содержат в исходе фонему /ɛ/, то есть (\hyperref[exmorphon6]{6}) — часть диахронии\fn{Тем не менее для описания заимствований это правило синхронно необходимо.}.

Более фонологичным, на первый взгляд, является правило преназального повышения, ср.~также (\hyperref[exmorphon3]{3}) выше:

\ex[exno=7] \begingl[everygla=] \label{exmorphon7}
\gla \phonr{ɔ}{ø}{\phonfeat[l]{+nasal}},//
\endgl \xe

Несмотря на обилие примеров с сочетаниями \i{у̊н} и \i{у̊м} в базе \i{\href{https://pamiri.online}{pamiri.online}} (их около 1200), есть и словарные входы с \i{он} и \i{он} (около 150). Часть из них объясняется тем, что в других вариантах шугнанского, например в барвазском, правило (\hyperref[exmorphon7]{7}) не действует, ср.~барвазское слово \i{фағон} ‘постель’. Другая часть, однако, содержит лексику, использующуюся и в собственно шугнанском. Так, в слове \i{қонӯн} ‘закон, обычай’ преназального повышения не наблюдается.

Проблема кроется в том, что формулировка (\hyperref[exmorphon7]{7}), соответствующая известным нам описаниям \parencites{sokolova1953}{edelman_yusufbekov1999_shughni}{edelman_dodykhudoeva2009_shughni}{olson2017}, не учитывает слоговые границы, имеющие в преназальном повышении существенный вес. Слова \i{қонӯн} /qɔ.ˈnun/ ‘закон, обычай’, \i{фонӣ} /fɔ.ˈni/ ‘тленный’, \i{даромад} /da.rɔ.ˈmad/ ‘доход’, \i{гому} /ɡɔ.ˈmʊ/ ‘тулуп, овчина (крытая материей)’ и другие имеют слоговую границу между /ɔ/ и носовым, и правило не применяется. Уточнённая формулировка такова:

\ex[exno=8] \begingl[everygla=] \label{exmorphon8}
\gla \phonb{ɔ}{ø}{{[}C\textsubscript{1}}{\phonfeat[l]{+nasal}(C\textsubscript{1}){]}\textsubscript{σ}},//
\glft {\small где C\textsubscript{1} значит «по крайней мере один согласный», а скобки обозначают факультативную часть.}//
\endgl \xe

И всё же как объяснить то, что в словах типа \i{арзу̊ни} /a.rzø.ˈnɪ/ ‘дешевизна, доступность’ (<~тадж.~\i{арзонӣ}) преназальное повышение имеет место? Оказывается, что значимо ещё и то, от чего словоформа производна (\i{арзу̊ни} $\leftarrow$ \i{арзу̊н} /aˈ.rzøn/ ‘дешёвый’). Если производящая основа удовлетворяет условиям правила (\hyperref[exmorphon8]{8}), то результат его применения сохраняется и в производных лексемах. Итак, снова оказывается, что «чисто фонетическое правило» всё-таки требует обращения к морфологии.

В (\hyperref[exmorphon3]{3}) упоминается действие этого правила и для /ɛ/, которое «повышается» до /e/. Действительно, есть ряд причин полагать, что это так. Например, по данным \i{\href{https://pamiri.online}{pamiri.online}}, /ɛ/ встречается перед носовыми лишь в 20~словоформах. Бо́льшая их часть на самом деле не является исключением из-за слоговых границ, ср.~\i{мêнат} /mɛ.ˈnat/ ‘труд’. Тем не менее ряд факторов говорит о меньшей категоричности этого чередования по сравнению с (\hyperref[exmorphon8]{8}). Так, в стяжённых формах типа \i{зêм} ‘беру’ (ср.~полную форму \i{зêзум}) /ɛ/ перед носовым встречается в закрытом слоге. Более того, нередко возможна вариативность внутри шугнанских форм типа \i{мему̊н}~$\sim$ \i{мêму̊н} ‘гость’, что может говорить либо о разнообразных способах передачи таджикского сочетания \i{еҳ} (ср.~тадж.~\i{меҳмон} ‘гость’), либо о том, что чередование /ɛ/ с /e/ менее последовательно и/или регулируется другими правилами.

В заключение обратимся к консонантным процессам. Шугнанские шумные согласные могут быть подвержены конечному оглушению: /kʊd/ [kʊt] ‘собака’. В определённых ситуациях (например, при нарочито отчётливом чтении) глубинная звонкость может сохраняться путём добавления гласного призвука после конечного звонкого; более того, на сохранение звонкости (в том числе и без гласного призвука) может влиять и глубина просодического шва (см.~подробнее об этом понятии в \parencite{krivnova2015}). Суффиксы и энклитики, как правило, препятствуют оглушению: /kʊden/ [kʊˈden] ‘собаки’\fn{Существует вариант /kaden/ ‘собаки [ж.]’, содержащий корневое чередование. У современных носителей, по нашему опыту, этот вариант с перегласовкой встречается реже; см.~также замечание о корневых перегласовках в именных частях речи выше.}, \i{Йид куд=о?} /ˈjɪd ˈkʊdɔ/ [ˈjɪt ˈkʊdɔ] ‘Это собака?’. Можно постулировать следующее правило конечного оглушения:

\ex[exno=9] \begingl[everygla=] \label{exmorphon9}
\gla \phonr{\phonfeat[l]{+consonantal \\ +voiced}}{\phonfeat[l]{–voiced}}{\#},//
\endgl \xe

Несмотря на (\hyperref[exmorphon9]{9}), глубинная глухость/звонкость конечного сегмента всё равно выражается фонетически. Во-первых, как было показано в \parencite{makarov2022_plosives}, гласный перед глубинно звонким конечным сегментом реализуется более длительно, чем перед глубинно глухим. Во-вторых, в полном произношении конечные взрывные, оглушаясь, различаются количеством придыхания и интенсивностью взрыва: у глубинно звонких оно в целом короче, чем у глухих (см.~также \parencite{makarov2023_aspiration}).

Если принять, что эти акустические параметры значимы перцептивно, встаёт вопрос о том, как это должно моделироваться фонологически. Должен ли вводиться специальный признак [±long]\fn{Традиционные описания шугнанского вокализма оперируют терминами «долгий» и «краткий» гласный. В правилах (\hyperref[exmorphon9]{4}–\hyperref[exmorphon9]{9}) им частично соответствует признак [±tense], см.~также примечания на этот счет выше.} для гласных? Чем различаются придыхательные аллофоны на месте конечных взрывных? Адекватно ли такое простое правило конечного оглушения, как (\hyperref[exmorphon9]{9})?

Оставляя поиск ответов на эти и другие вопросы будущим исследованиям, отметим, что автоматические чередования лишь на первый взгляд просты и поверхностны. С одной стороны, за кажущейся «чистой фонетикой» может скрываться морфонология, а с другой, даже за привычным (например, русскоязычному читателю) автоматическим чередованием может скрываться нечто более сложное.