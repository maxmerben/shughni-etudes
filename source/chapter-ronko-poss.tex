\chapter*{Посессивные конструкции с~местоимениями в~шугнанском языке}
\addcontentsline{toc}{chapter}{\textit{Р.~Ронько}. \textbf{Посессивные конструкции с~местоимениями в~шугнанском языке}}
\setcounter{section}{0}
\chaptermark{Посессивные конструкции с~местоимениями в~шугнанском языке}
\label{chapter-ronko-poss}

\begin{customauthorname}
Роман Ронько
\end{customauthorname}

\begin{englishtitle}
\i{Possessive constructions with pronouns in Shughni\\{\small Roman Ron’ko}}
\end{englishtitle}

\begin{abstract}
В статье рассматриваются правила выбора посессивных конструкций с местоимениями в шугнанском языке. В шугнанском языке посессивность может выражаться с помощью личных и указательных местоимений, местоимений с локативным послелогом =\i{анд} и конструкций, которые включают в себя и стандартное местоимение, и местоимение, маркированное локативным показателем. В работе постулируется два синтаксических типа посессора, а также исследуются функциональные различия между этими типами конструкций, анализируются типы семантических отношений между посессором и обладаемым в исследуемых конструкциях.
\end{abstract}

\begin{keywords}
посессивность, внешний посессор, предикативная посессивность, личный локатив, неотчуждаемость, шугнанский язык
\end{keywords}

\begin{eng-abstract}
The paper describes possessive constructions with pronouns in Shughni. In this language, the possessor can be expressed by personal and demonstrative pronouns, pronouns with the locative postposition =\i{and}, and constructions including both a standard pronoun and a pronoun marked with a locative marker. I compare these construction types and analyze types of semantic relations between the possessor and the possessee. Besides that, two syntactic types of possessive constructions are distinguished and their syntactic features are described.
\end{eng-abstract}

\begin{eng-keywords}
possession, external possessor, predicative possession, personal locative, Shughni
\end{eng-keywords}

\begin{acknowledgements}
Публикация подготовлена в результате проведения исследования (проекта №~22-00-034) в рамках Программы «Научный фонд Национального исследовательского университета “Высшая школа экономики” (НИУ~ВШЭ)» в 2022~году. Я благодарю коллектив шугнанской экспедиции, с которым неоднократно обсуждал материал, приведённый в работе, Юрия Александровича Ландера и Марию Александровну Холодилову, а также анонимных рецензентов, чьи соображения очень помогли в работе над текстом статьи.
\end{acknowledgements}

\begin{initialprint}
\fullcite{ronko2022}\end{initialprint}

\section{Введение}

Наша работа посвящена выбору конструкции с разными типами посессоров в шугнанском языке. Вариативность конструкций, выражающих эту грамматическую категорию, может быть обусловлена разными семантическими и синтаксическими факторами \parencites{kibrik_etal2006}{aikhenvald2019}{dahl_koptjevskaja_tamm2001}{lander2004}{koenig_haspelmath1997}. Так, для выражения посессивности важно, содержится ли рассматриваемое значение в предикате (английская конструкция с глаголом \i{have} или русская \i{у}+{\sc gen} \i{есть}) или в именной группе, в одной ли составляющей находятся посессор и обладаемое или в разных. На выбор конструкции может влиять семантический тип отношения между посессором и обладаемым: родственные отношения, отношение часть–целое, отношение легального обладания и так далее. На выбор конструкции также способны влиять информационная структура предложения и порядок слов.

В фокусе внимания нашей работы находятся конструкции с притяжательными местоимениями. Существует три конструкции с местоимениями со значением принадлежности: конструкция с личным местоимением в косвенном падеже (\getfullhref{exposs1.a}), конструкция с притяжательным местоимением, маркированным локативным послелогом =\i{анд}\fn{В данной работе мы вслед за иранистической традицией будем называть =\i{анд} локативным послелогом, однако морфологический статус этой единицы не совсем ясен. Подробное обсуждение этой проблемы представлено в работе \parencite{sarkisov2018}.} (\getfullhref{exposs1.b}) и конструкция с двумя местоимениями, одно из которых маркировано локативным послелогом (\getfullhref{exposs1.c}).

\pex<exposs1>
\a<a> \begingl
\gla \b{му} стол//
\glc {\sc pron.1sg.o} стол//
\endgl
\a<b> \begingl
\gla \b{му-нд} стол//
\glc {\sc pron.1sg.o-loc} стол//
\endgl
\a<c> \begingl
\gla \b{му-нд} \b{му} стол//
\glc {\sc pron.1sg.o-loc} {\sc pron.1sg.o} стол//
\glft ‘\b{мой} стол’ \trailingcitation{[сконструированные примеры]}//
\endgl \xe

В нашей статье мы постараемся установить механизм выбора той или иной конструкции в зависимости от разных факторов.

Материалы для данной работы были получены несколькими способами. Во-первых, был использован словарь шугнанского языка \parencite{karamshoev1988}. Данный словарь содержит 30~000 слов, которые в качестве иллюстративного материала снабжены фрагментами текстов, записанных автором словаря во время полевой работы. Работая с этим словарём, мы использовали корпусный подход. Были извлечены контексты с местоимениями в посессивном значении, которые впоследствии подверглись количественной обработке. Не извлекались примеры, которые содержат пометы, обозначающие диалектную принадлежность\fn{Имеются в виду пометы, обозначающие принадлежность к «баджувскому диалекту» и «шахдаринскому говору».}. Кроме того, мы использовали материал текстов, записанных в экспедициях НИУ~ВШЭ на Памир 2018–2021~годов в городе Хорог и в ближайших кишлаках, а также данные, собранные методом элицитации.

Статья устроена следующим образом: в \hyperref[poss-morph]{разделе~2} представлены парадигмы местоимений шугнанского языка и описаны некоторые их морфологические особенности. В \hyperref[poss-syntax]{разделе~3} рассматриваются синтаксические типы посессивных конструкций. В \hyperref[poss-distrib]{разделе~4} анализируется влияние на выбор конструкции семантического типа отношения между посессором и обладаемым.

\section{Морфология личных местоимений} \label{poss-morph}

В шугнанском языке в роли личного посессора выступают личные и указательные местоимения в косвенном падеже. Сведения о системе местоимений шугнанского языка содержатся в значительном количестве работ [см.~\cites{karamshoev1963}{alamshoev1994}{yusufbekov1998}{edelman1999_shugrush}{edelman_yusufbekov1999_shughni}{edelman_dodykhudoeva2009_shughni}]. Парадигма личных местоимений 1-го и 2-го лица представлена в Таблице~\ref{tab:poss1}.

\begin{table}[h]
 \centering
 \caption{Личные местоимения шугнанского языка (1-е и 2-е лица)}
 \smallskip
 \label{tab:poss1}
 \begin{tabular}{c|cccc} \toprule
 падеж & {\sc 1sg} & {\sc 2sg} & {\sc 1pl} & {\sc 2pl} \\ \midrule
 {\sc nom} & \i{(w)уз} & \multirow{2}{*}{\i{ту}} & \multirow{2}{*}{\i{мāш}} & \multirow{2}{*}{\i{тама}} \\
 {\sc obl} & \i{му} & & & \\ \bottomrule
 \end{tabular}
\end{table}

Формы именительного ({\sc nom}) и косвенного ({\sc obl}) падежей местоимений различаются только у местоимения ‘я’ и у местоимений 3-го лица, в роли которых выступают указательные местоимения, представленные в Таблице~\ref{tab:poss2}.

\begin{table}[h]
 \centering
 \caption{Указательные местоимения}
 \smallskip
 \label{tab:poss2}
 \begin{tabular}{c|c|ccc} \toprule
 серия & падеж & {\sc m.sg} & {\sc f.sg} & {\sc pl} \\ \midrule
 \multirow{2}{*}{I {\small («у говорящего»)}} & {\sc nom} & \multicolumn{2}{c}{\i{йам}} & \i{мāδ} \\
 & {\sc obl} & \i{ми} & \i{мам} & \i{мев} \\ \midrule
 \multirow{2}{*}{II {\small («у слушающего»)}} & {\sc nom} & \multicolumn{2}{c}{\i{йид}} & \i{дāδ} \\
 & {\sc obl} & \i{ди} & \i{дам} & \i{дев} \\ \midrule
 \multirow{2}{*}{\makecell{III {\small («дальняя дистанция»)}}} & {\sc nom} & \i{йу} & \i{йā} & \i{wāδ} \\
 & {\sc obl} & \i{wи} & \i{wам} / \i{wем} & \i{wев} \\ \bottomrule
 \end{tabular}
\end{table}

\pagebreak[4]

Указательные местоимения имеют три ряда, различающих степень удалённости указываемого объекта\fn{Подробное описание системы указательных местоимений шугнанского языка см.~в статье [\hyperref[chapter-badeev-demon]{Бадеев 2022}]; см.~также \parencite{alamshoev1994}.}. В каждом ряду различаются формы числа (единственное и множественное) и падежа (прямой и косвенный). Указательные местоимения в косвенном падеже различаются по родам.

Кроме личных и указательных местоимений, в роли посессора может выступать также рефлексивное местоимение \i{ху}.

Местоимения, выражающие посессивность, способны не только выражать ее самостоятельно, но и присоединять к форме косвенного падежа локативный послелог =\i{анд}. В Таблице~\ref{tab:poss3} представлены все встретившиеся в нашем материале посессивные формы, включающие в свой состав местоимения. Формы, заканчивающиеся на гласную, присоединяют показатель =\i{нд}, а формы, которые заканчиваются на согласную, способны присоединять два варианта данного показателя — =\i{анд} и =\i{инд}. Согласно нашим данным, показатель =\i{инд} характерен для шахдаринского говора шугнанского языка, однако некоторые материалы словаря \parencite{karamshoev1988} противоречат этому утверждению.

\begin{table}[h]
 \centering
 \caption{Посессивные формы местоимений}
 \smallskip
 \label{tab:poss3}
 \begin{tabular}{c|cc} \toprule
 глосса & \makecell{местоимение\\({\sc obl})} & \makecell{местоимение\\+ -\i{анд} [{\sc loc}]} \\ \midrule
 {\sc pron.1sg.o} & \i{му} & \i{му-нд} \\
 {\sc pron.2sg} & \i{ту} & \i{ту-нд} \\
 {\sc pron.1pl} & \i{мāш} & \i{мāш-анд} / \i{мāш-инд} \\
 {\sc pron.2pl} & \i{тама} & \i{тама-нд} \\
 {\sc d3.m.sg.o} & \i{wи} & \i{wи-нд} \\
 {\sc d2.m.sg.o} & \i{ди} & \i{ди-нд} \\
 {\sc d3.f.sg.o} & \i{wем} & \i{wем-анд} / \i{wем-инд} \\
 {\sc d3.pl.o} & \i{wев} & \i{wев-анд} / \i{wев-инд} \\
 {\sc refl} & \i{ху} & \i{ху-нд} \\ \bottomrule
 \end{tabular}
\end{table}

Формы местоимений \i{wāδ-анд/инд} ({\sc d3.pl.o-loc}), \i{мев-анд/инд} ({\sc d1.pl.o-loc}), \i{дев-анд/инд} ({\sc d2.pl.o-loc}), \i{мам-анд/инд} ({\sc d1.f.sg.o-loc}), \i{дам-анд/инд} ({\sc d2.f.sg.o-loc}) в посессивном значении используются редко и не встречаются в материале словаря \parencite{karamshoev1988}, в работах по шугнанскому языку \parencites{alamshoev1994}{karamshoev1963}{edelman1999_shugrush}{edelman_yusufbekov1999_shughni}{yusufbekov1998}{edelman_dodykhudoeva2009_shughni}, в текстах \parencite{zarubin1960}, а также в текстах памирских экспедиций НИУ~ВШЭ 2018–2021~годов, однако являются возможными и были получены путем элицитации.

Кроме указанных способов выражения посессивности, в шугнанском языке есть конструкция, в которой участвуют два местоимения: притяжательное местоимение (совпадающее с местоимением в форме косвенного падежа) и местоимение с показателем =\i{анд}, как в примере (\gethref{exposs2}):

\ex<exposs2>
\begingl
\gla \b{Му-нд} \b{му} толи лап гандā.//
\glc {\sc pron.1sg.o-loc} {\sc pron.1sg.o} плохой очень судьба//
\glft ‘\b{У меня} плохая судьба.’ \trailingcitation{\parencite[125]{karamshoev1988}}//
\endgl \xe

В этом примере после местоимения с показателем локатива-посессива сразу идет местоимение без этого показателя. Здесь необходимо пояснить, что первую позицию в этой конструкции могут занимать не только местоимения с локативно-посессивным показателем, но и существительные с этим показателем.

\section{Синтаксические типы посессивных конструкций} \label{poss-syntax}

\subsection{Конструкции с внешним посессором}

В языках мира существуют два разных синтаксических типа посессоров — внутренний и внешний. Внутренним посессором называется конструкция, в которой посессор находится в той же фразовой составляющей (именной группе), что и вершина. Внешний посессор отличается от внутреннего тем, что он находится в отдельной от обладаемого составляющей. Конструкции с внешним посессором, в отличие от конструкций с внутренним посессором, есть не во всех языках мира \parencite[591]{koenig_haspelmath1997}, так что вопрос о наличии внешнего посессора в том или ином языке является отдельной задачей. Примером конструкции со внешним посессором в русском языке является предложение (\gethref{exposs3}).

\ex<exposs3>
\i{Вася оторвал \b{ему} ногу.}
\xe

В шугнанском языке в примерах с притяжательными местоимениями, маркированными локативными показателями, обладаемое может быть отделено от посессора с помощью наречий (\getfullhref{exposs4.a}). Предложения с местоимением без локативного показателя, которое отделено от обладаемого наречием, запрещаются информантами при элицитации (\getfullhref{exposs4.b}). Вместо таких предложений информантами предлагаются предложения типа (\getfullhref{exposs4.c}) с контактным расположением посессора и обладаемого.

\pex<exposs4>
\a<a> \begingl
\gla \b{Wи-нд} бийор \b{ризӣн} подаwу̊н сат.//
\glc {\sc d3.m.sg.o-loc} вчера дочь ходящий стать.{\sc pst.f/pl}//
\glft ‘\b{У него} вчера \b{дочь} пошла [научилась ходить].’//
\endgl
\a<b> \begingl
\gla \ljudge{*}\b{Wи} бийор \b{ризӣн} подаwу̊н сат.//
\glc {\sc d3.m.sg.o} вчера дочь ходящий стать.{\sc pst.f/pl}//
\glft {\small ожидаемое значение:} ‘\b{У него} вчера \b{дочь} пошла [научилась ходить].’//
\endgl
\a<c> \begingl
\gla \b{Wи} \b{ризӣн} бийор подаwу̊н сат.//
\glc {\sc d3.m.sg.o} дочь вчера ходящий стать.{\sc pst.f/pl}//
\glft ‘\b{Его дочь} вчера пошла [научилась ходить].’ \trailingcitation{[элицитация]}//
\endgl \xe

В примере (\gethref{exposs7}) тем же наречием \i{бийор} ‘вчера’ местоимение с локативным показателем отделяется от цепочки, состоящей из местоимения без локативного показателя и обладаемого. Мы объясняем приведённые факты тем, что притяжательное местоимение без локативного показателя составляет одну именную группу с вершиной (является внутренним посессором), в отличие от местоимения с локативным показателем.

\ex<exposs7>
\begingl
\gla \b{Wи-нд} бийор \b{wи} \b{ризӣн} подаwу̊н сат.//
\glc {\sc d3.m.sg.o-loc} вчера {\sc d3.m.sg.o} дочь ходящий стать.{\sc pst.f/pl}//
\glft ‘\b{У него} вчера \b{(его) дочь} пошла [научилась ходить].’ \trailingcitation{[элицитация]}//
\endgl \xe

\subsection{Конструкции с личным локативом}

Некоторые конструкции с внешним посессором имеют ограниченную функциональность. Так, в шугнанском языке именная группа с местоимением в локативе может быть глагольным зависимым, как в примере (\getfullhref{exposs8.a}). Использование местоимений без локативного показателя в данных конструкциях невозможно (\getfullhref{exposs8.b}).

\pex<exposs8>
\a<a> \begingl
\gla \b{Wев-анд} δуд су̊д.//
\glc {\sc d3.pl.o-loc} дым идти.{\sc prs.3sg}//
\glft ‘\b{У них} дымится.’ \trailingcitation{\parencite[507]{karamshoev1988}}//
\endgl
\a<b> \begingl
\gla \ljudge{*}\b{Wев} δуд су̊д.//
\glc {\sc d3.pl.o} дым идти.{\sc prs.3sg}//
\glft {\small ожидаемое значение:} ‘\b{У них} дымится.’ \trailingcitation{[элицитация]}//
\endgl \xe

В примерах такого типа выражено значение «личного локатива» в терминах \parencite{daniel2003}. Личным локативом называются грамматические конструкции, которые используются для обозначения места, ассоциируемого с местом проживания человека. Выражения данного значения посредством посессивных конструкций распространены в языках мира \parencite{zhigulskaya2015}.

\subsection{Предикативная посессивность}

Другой тип конструкций с внешним посессором может выражать так называемую предикативную посессивность. Предикативная посессивность — тип посессивности, в котором данные отношения составляют предикативное ядро высказывания \parencites{stassen2001}{stassen2013}, см.~примеры из русского и английского языков:

\pagebreak[4]

\ex<exposs10>
\i{I \b{have} a cat.}
\xe

\ex<exposs11>
\i{У меня \b{есть} кошка.}
\xe

В шугнанском языке предикативная посессивность выражается конструкцией с бытийным глаголом \i{йаст}. Отсутствие обладания выражается с помощью отрицательного слова \i{нист}. Основная местоименная конструкция, которая используется при выражении предикативной посессивности, — это конструкция с местоимением, маркированным локативно-посессивным показателем (\getfullhref{exposs12.a}). Конструкции с местоимениями без локативно-посессивного маркера в данном случае использоваться не могут (\getfullhref{exposs12.b}), а конструкции с локативным показателем и дополнительным местоимением, выражающие предикативную посессивность, некоторые носители разрешили употребить в контекстах с особой информационной структурой (\getfullhref{exposs12.c}).

\pex<exposs12>
\a<a> \begingl
\gla \b{Му-нд} стол йаст.//
\glc {\sc pron.1sg.o-loc} стол есть//
\glft ‘\b{У меня} есть стол.’//
\endgl
\a<b> \begingl
\gla \ljudge{*}\b{Му} стол йаст.//
\glc {\sc pron.1sg.o} стол есть//
\glft {\small ожидаемое значение:} ‘\b{У меня} есть стол’.//
\endgl
\a<c> \begingl
\gla \ljudge{\b{?}}\b{Му-нд} \b{му} стол йаст.//
\glc {\sc pron.1sg.o-loc} {\sc pron.1sg.o} стол есть//
\glft <Контекст: Вам нужен стол? Нет у меня есть свой.\\
Я подарю тебе стол, а твой возьму. Нет,> ‘\b{у меня} \i{есть} стол.’ \trailingcitation{[элицитация]}//
\endgl \xe

То же самое мы можем наблюдать при выражении отрицания обладания:

\pex<exposs13>
\a<a> \begingl
\gla \b{Му-нд} стол нист.//
\glc {\sc pron.1sg.o-loc} стол есть.{\sc neg}//
\endgl
\a<b> \begingl
\gla \ljudge{*}\b{Му} стол нист.//
\glc {\sc pron.1sg.o} стол есть.{\sc neg}//
\endgl
\a<c> \begingl
\gla \ljudge{\b{?}}\b{Му-нд} \b{му} стол нист.//
\glc {\sc pron.1sg.o-loc} {\sc pron.1sg.o} стол есть.{\sc neg}//
\glft ‘\b{У меня} нет стола.’ \trailingcitation{[элицитация]}//
\endgl \xe

В словаре Карамшоева встретилось четыре примера (примеры \gethref{exposs14}–\gethref{exposs17}), в которых конструкция, включающая в себя и местоимение с локативным показателем, и местоимение без него, выражает предикативную посессивность.

\ex<exposs14>
\begingl
\gla \b{Му-нд} \b{му} ворҷ=анд башāнд урамā вад.//
\glc {\sc pron.1sg.o-loc} {\sc pron.1sg.o} конь={\sc loc} хороший попона быть.{\sc pst.f/pl}//
\glft ‘\b{У моего} коня была красивая попона’ \trailingcitation{\parencite[120]{karamshoev1988}}//
\endgl \xe

\ex<exposs15>
\begingl
\gla \b{Му-нд} \b{му} қāд пāцт вуд.//
\glc {\sc pron.1sg.o-loc} {\sc pron.1sg.o} рост низкий быть.{\sc pst.m.sg}//
\glft ‘\b{У меня} был низкий рост’ \trailingcitation{\parencite[131]{karamshoev1991}}//
\endgl \xe

\ex<exposs16>
\begingl
\gla \b{Ту-нд} \b{ту} ганда-ги ик=ид вуд диди, му хез=ат на-йат.//
\glc {\sc pron.2sg-loc} {\sc pron.2sg} плохой-{\sc subst} {\sc emph=d2.sg} быть.{\sc pst.m.sg} {\sc compl} {\sc pron.1sg.o} {\sc apud=2sg} {\sc neg}-прийти.{\sc pst}//
\glft ‘Твоё упущение было (букв. \b{у тебя} было упущение) в том, что ты не приходил ко мне’ \trailingcitation{\parencite[125]{karamshoev1988}}//
\endgl \xe

\ex<exposs17>
\begingl
\gla \b{Wи-нд} \b{wи} доних̌ башāнд вуд.//
\glc {\sc d3.m.sg.o-loc} {\sc d3.m.sg.o} знание хороший быть.{\sc pst.m.sg}//
\glft ‘\b{У него} были хорошие знания’ \trailingcitation{\parencite[492]{karamshoev1988}}//
\endgl \xe

Как видно, конструкции предикативной посессивности выражаются тем же средством, что и конструкции с внешним посессором, то есть местоимениями с маркером локатива.

Таким образом, местоимения (и, соответственно, имена вообще, см.~пример \gethref{exposs18}) с локативным показателем участвуют в образовании конструкций с предикативной посессивностью и конструкций с личным локативом, которые мы будем считать подтипом внешнего посессора.

\ex<exposs18>
\begingl
\gla Аҳмед \b{Парвиз=анд} \b{wи} пуц.//
\glc Ахмед Парвиз={\sc loc} {\sc d3.m.sg.o} сын//
\glft ‘Ахмед — сын Парвиза’ (буквально ‘Ахмед — \b{у Парвиза его} сын’) \trailingcitation{[элицитация]}//
\endgl \xe

\section{Функциональное распределение посессивных конструкций с местоимениями} \label{poss-distrib}

\subsection{Отчуждаемая vs. неотчуждаемая принадлежность}

В данном разделе мы рассмотрим разные типы посессивных отношений и попытаемся понять, как они влияют на выбор конструкции. Среди посессивных отношений в типологической литературе принято различать отчуждаемые и неотчуждаемые отношения между посессором и обладаемым. Отчуждаемым типом отношений считается, например, «законное обладание» (“legal ownership”, согласно определению \parencite{aikhenvald2019}). Неотчуждаемыми считаются такие классы отношений, как часть–целое (к примеру, части тела или части растения: \i{цветок одуванчика}), родственные отношения (сын, брат) и неотъемлемые человеческие свойства (например, характер). Среди отношений часть–целое явным образом выделяются типы отношений, которые (предположительно) ведут себя типологически по-разному. Так, имеет смысл различать отношение между частями тела и человеком, которому они принадлежат, частями дома и самим домом. Различия между данными типами отношений подробно обсуждаются в разных работах, см.,~например \parencites{aikhenvald2019}{dahl_koptjevskaja_tamm2001}. Некоторые из обсуждаемых типов отношений, а именно части тела и термины родства, очевидно, являются гораздо более частотными, чем части дома или части растений. Частотность является важным фактором грамматикализации тех или иных конструкций, и гипотеза о более высокой степени грамматикализации посессивных отношений «часть тела–человек» или родственных отношений в связи с частотностью контекстов, где эти отношения фигурируют, выглядит правдоподобной (см.,~например, \parencite{haspelmath2008}).

Типы отношений и отчуждаемость также являются важным параметром для определения синтаксического типа посессора. В исследовании \parencite{koenig_haspelmath1997} приводится иерархия типов посессивных отношений, в которой для конструкции с внешним посессором наиболее характерны неотчуждаемые отношения, а именно части тела:

\ex<exposs19>
часть тела $<$ одежда $<$ другие контекстно уникальные предметы
\xe

В нашем исследовании мы рассматриваем следующие типы отношений: родственные отношения, отношения часть–целое, а именно части тела, человек и его неотъемлемые свойства (характер, сила, харизма, удача, выносливость…), непосредственное обладание или законное владение (собственность). Рассматривая примеры с законным владением, мы ограничились исключительно обладанием предметами. Выбор типов отношений основан на классификации, представленной в \parencite{aikhenvald2019}, а также продиктован нашим материалом.

Конструкции с локативно-посессивным суффиксом на местоимении и конструкции с дублированием местоимения были получены путём сплошной выборки из словаря. Среди конструкций с местоимениями мы рассматриваем местоимение \i{му} ({\sc pron.1sg.o}) в посессивном значении (примеры извлечены из первого тома словаря). Далее нам не удалось проанализировать примеры из группы часть–целое, не относящиеся к человеку, по причине малого их количества (6~примеров) и отношения между человеком и его неотъемлемыми свойствами (19~примеров), а также некоторые другие типы отношений, в том числе те, которые мы не смогли классифицировать.

В Таблице~\ref{tab:poss4} приводятся данные, которые будут использованы для дальнейшего анализа.

\begin{table}[h]
 \centering
 \caption{Типы конструкций и типы отношений}
 \smallskip
 \label{tab:poss4}
 \begin{tabular}{c|ccc|c} \toprule
 тип конструкции & родственники & предметы & части тела & всего \\ \midrule
 \makecell[c]{местоимение\\с локативом} & 31 (50\%) & 27 (43\%) & 4 (7\%) & 62 \\
 \makecell[c]{местоимение\\+ местоимение\\с локативом} & 7 (20\%) & 4 (12\%) &  23 (68\%) & 34 \\
 \makecell[c]{только\\местоимение} & 173 (52\%) & 92 (28\%) & 68 (20\%) & 333 \\ \midrule
 всего & 211 & 123 & 95 & 429 \\ \bottomrule
 \end{tabular}
\end{table}

\pagebreak[4]

Необходимо отметить, что конструкции с локативным показателем и местоимением (\i{мунд му} ‘у меня моё’), хотя и встречаются реже всего, являются тем не менее относительно частотными.

Далее мы попарно сравним разные группы объектов и попытаемся выявить статистическую значимость различий с помощью двустороннего варианта точного критерия Фишера. В связи с тем, что в Таблице~\ref{tab:poss4} у нас есть девять попарных сравнений, вероятность ошибки увеличивается в девять раз, поэтому мы будем вынуждены применить поправку Бонферрони, которая предлагает умножить полученное значение вероятности (\i{p}-value) на количество сравнений, то есть в нашем случае на девять \parencite[350]{lehmann_romano2005}.

\subsection{Конструкции с локативным суффиксом vs. конструкции со стандартным личным посессором}

Конструкции со внутренним личным посессором вообще являются наиболее частотными в нашем материале. Статистически значимые различия между маркированием отношений законного обладания (над предметами) и родственных отношений в исследуемых типах конструкций не обнаружены, что следует из Таблицы~\ref{tab:poss5}.

\begin{table}[h]
 \centering
 \caption{Родственные отношения и обладание предметами в конструкциях с местоимениями и конструкциях с местоимениями, маркированными локативными послелогами}
 \smallskip
 \label{tab:poss5}
 \begin{tabular}{c|cc} \toprule
 тип конструкции & родственники & предметы \\ \midrule
 местоимение с локативом & 31 (15\%) & 27 (22\%) \\
 местоимение & 173 (85\%) & 92 (78\%) \\ \bottomrule
 \end{tabular}
\end{table}

Части тела маркируются конструкцией, в которой местоимение с локативом сочетается с внутренним личным посессором исключительно редко. Владение предметами и родственные отношения маркируются данными конструкциями чаще, однако эти различия являются статистически незначимыми (двусторонний вариант точного критерия Фишера с поправкой Бонферрони $p = 0,8964$. Результат статистически незначимый $p > 0.05$). Статистически значимыми являются только различия между маркированием частей тела и обладания предметами (см.~Таблицы~\ref{tab:poss6}–\ref{tab:poss7}\fn{Двусторонний вариант точного критерия Фишера с поправкой Бонферрони. В Таблице~\ref{tab:poss6}: $p = 0,018$; результат статистически значимый ($p < 0,05$). В Таблице~\ref{tab:poss7}: $p = 0,3493$; результат статистически незначимый ($p > 0,05$).}).

\begin{table}[h]
 \centering
 \caption{Обладание частями тела и обладание предметами в конструкциях с местоимениями и конструкциях с местоимениями, маркированными локативными послелогами}
 \smallskip
 \label{tab:poss6}
 \begin{tabular}{c|cc} \toprule
 тип конструкции & части тела & предметы \\ \midrule
 местоимение с локативом & 4 (5\%) & 27 (22\%) \\
 местоимение & 68 (95\%) & 92 (78\%) \\ \bottomrule
 \end{tabular}
\end{table}

\begin{table}[h]
 \centering
 \caption{Обладание частями тела и родственные отношения в конструкциях с местоимениями и конструкциях с местоимениями, маркированными локативными послелогами}
 \smallskip
 \label{tab:poss7}
 \begin{tabular}{c|cc} \toprule
 тип конструкции & части тела & родственники \\ \midrule
 местоимение с локативом & 4 (5\%) & 31 (15\%) \\
 местоимение & 68 (95\%) & 173 (85\%) \\ \bottomrule
 \end{tabular}
\end{table}

Для всех типов отношений конструкция со стандартным личным посессором — самый частотный способ выражения.

\subsection{Конструкции с локативным суффиксом vs. конструкции с локативным суффиксом и местоимением}

В Таблице~\ref{tab:poss8} мы пытаемся сравнить распределение данных конструкций с двумя типами отношений: владение предметами и родственные отношения.

\begin{table}[h]
 \centering
 \caption{Родственные отношения и обладание предметами в конструкциях с местоимениями, маркированными локативными послелогами, и конструкциях с двумя местоимениями}
 \smallskip
 \label{tab:poss8}
 \begin{tabular}{c|cc} \toprule
 тип конструкции & части тела & родственники \\ \midrule
 местоимение с локативом & 27 (87\%) & 31(82\%) \\
 \makecell{местоимение\\+ местоимение с локативом} & 4 (13\%) & 7 (18\%) \\ \bottomrule
 \end{tabular}
\end{table}

Интересно, что статистически значимых различий между употреблениями посессивных конструкций для маркирования законного обладания и родственных отношений также не наблюдается (двусторонний вариант точного критерия Фишера с поправкой Бонферрони, $p = 3,0393$; результат статистически незначимый, $p > 0,05$), несмотря на то, что это безусловно разные типы посессивных отношений, и родственные отношения в некоторых языках могут квалифицироваться как неотчуждаемые, так что мы могли бы ожидать здесь различий. Таким образом, в шугнанском языке посессивные конструкции, которые выражают обладание неодушевлёнными предметами, и родственные отношения чаще маркируются конструкцией с локативно-посессивным показателем. В Таблице~\ref{tab:poss9} мы сравниваем распределение данных конструкций между родственными отношениями и частями тела.

\begin{table}[h]
 \centering
 \caption{Родственные отношения и обладание частями тела в конструкциях с местоимениями, маркированными локативными послелогами, и в конструкциях с двумя местоимениями}
 \smallskip
 \label{tab:poss9}
 \begin{tabular}{c|cc} \toprule
 тип конструкции & части тела & родственники \\ \midrule
 местоимение с локативом & 4 (15\%) & 31 (82\%) \\
 \makecell{местоимение\\+ местоимение с локативом} & 23 (85\%) & 7 (18\%) \\ \bottomrule
 \end{tabular}
\end{table}

В отличие от конструкций, выражающих родственные связи, обладание частями тела чаще маркируется конструкциями с локативом и местоимением. Различия здесь статистически значимые (двусторонний вариант точного критерия Фишера с поправкой Бонферрони, $p = 0,00009$; результат статистически значимый, $p < 0,05$).

Конструкция с локативом и местоимением (\i{мунд му}) в первую очередь предпочитает кодировать части тела, но иногда может кодировать уникальные родственные отношения и совсем изредка владение какими-то уникальными предметами или предметами одежды:

\ex<exposs20>
\begingl
\gla \b{Wим-инд}=ен wим сифц-ен парδод.//
\glc {\sc d3.f.sg.o-loc=3pl} {\sc d3.f.sg.o} бусы-{\sc pl} продать.{\sc pst}//
\glft ‘\b{Её} бусы продали’ \trailingcitation{(баджувский диалект) \parencite[99]{karamshoev1988}}//
\endgl \xe

Таким образом, из двух конструкций с внешним посессором конструкция \i{мунд му} в большей степени соответствует иерархии, приведенной в \parencite{koenig_haspelmath1997}. Кажется, что конструкции типа \i{мунд му} агрегируют в себе «центральные» свойства внешнего посессора. Неотчуждаемые объекты, плотно прилегающие объекты (например, одежда) или уникальные, специально выделенные объекты, которые являются обладаемым, выражаются конструкцией с внешним посессором \parencite{kibrik_etal2006}.

\section{Заключение}

Среди рассмотренных конструкций принадлежности в шугнанском языке выделяются два синтаксических типа: конструкции с внутренним и с внешним посессором. В образовании конструкций с внешним посессором участвует локативный послелог =\i{анд}, маркирующий обладателя. У конструкций с внешним посессором существуют отдельные функциональные подтипы, выражающие предикативную посессивность и личный локатив. Конструкции с местоимениями без локативного послелога являются внутренним посессором.

Нами были рассмотрены разные типы посессивных отношений и выявлены некоторые статистические закономерности распределения этих типов по разным синтаксическим конструкциям. Родственные отношения кодируются теми же конструкциями, что и отношения законного обладания предметами, в отличие от отношений между человеком и частями его тела. Самые частотные конструкции — конструкции с внутренним личным посессором — могут маркировать все типы отношений. Конструкции с показателем =\i{анд} и местоимением чаще всего маркируют обладание частями тела, реже родственные отношения и законное обладание предметами. Конструкции с локативным показателем почти не используются при именах, обозначающих части тела.