\chapter*{Глаголы ʽлить(ся)ʼ и~ʽсыпать(ся)ʼ в~шугнанском языке}
\addcontentsline{toc}{chapter}{\textit{Е.~Арманд, Ш.~Некушоева}. \textbf{Глаголы ʽлить(ся)ʼ и~ʽсыпать(ся)ʼ в~шугнанском языке}}
\setcounter{section}{0}
\chaptermark{Глаголы ʽлить(ся)ʼ и~ʽсыпать(ся)ʼ в~шугнанском языке}
\label{chapter-armand-pour}

\begin{customauthorname}
Елена Арманд, Шаҳло Некушоева
\end{customauthorname}

\begin{englishtitle}
\i{The verbs meaning ‘to pour’ in Shughni\\{\small Elena Armand, Shahlo Nekushoeva}}
\end{englishtitle}

\begin{abstract}
Статья посвящена исследованию лексической семантики глаголов со значением ʽлить(ся)ʼ и ʽсыпать(ся)ʼ в шугнанском языке, одном из памирских языков, на котором говорят в Горно-Бадахшанской АО Таджикистана. Материалом для статьи послужили словарные данные шугнанского языка [\cite{karamshoev1988}; \cite*{karamshoev1991}; \cite*{karamshoev1999}], а также полевые материалы 2021~года, собранные во время экспедиции в г.~Хорог. В работе последовательно рассматриваются шугнанские лексемы со значением ʽлить(ся)ʼ и ʽсыпать(ся)ʼ и проводится анализ примеров их употреблений. Работа завершается построением типологической карты, основанной на шугнанском материале.
\end{abstract}

\begin{keywords}
лексическая типология, \fakesc{ЛИТЬ(СЯ)}–\fakesc{СЫПАТЬ(СЯ)}, шугнанский язык, памирские языки.
\end{keywords}

\begin{eng-abstract}
In this paper, we undertake a study of the lexical semantics of verbs with the meaning ‘to pourʼ in one of the Eastern Iranian languages of the Pamirs — Shughni, spoken in the Gorno-Badakhshan Autonomous Province of Tajikistan. Data for the paper were extracted from the published dictionary of the Shughni language [Karamshoev \cite*{karamshoev1988}–\cite*{karamshoev1999}], as well as field materials collected by the authors in 2021 (Khorog). The paper consistently examines all Shughni lexemes with the meaning ‘pour’ and analyzes examples of their use. The work ends with the construction of a typological map based on the Shughni material.
\end{eng-abstract}

\begin{eng-keywords}
lexical typology, \i{{\sc to~pour}}, Shughni, Pamir languages.
\end{eng-keywords}

\begin{acknowledgements}
Публикация подготовлена в ходе проведения исследования (проект №~22-00-034) в рамках Программы «Научный фонд Национального исследовательского университета “Высшая школа экономики” (НИУ ВШЭ)» в 2022~г.
\label{pour-acknow}
\end{acknowledgements}

\begin{initialprint}
\fullcite{armand_nekushoeva2022}\end{initialprint}

\section{Введение} \label{pour-intro}

Статья посвящена описанию ситуаций перемещения жидкостей и сыпучих веществ, то есть глаголов со значением ʽлить(ся)ʼ и ʽсыпать(ся)ʼ в шугнанском языке\fn{Шугнанский язык относится к шугнано-рушанской группе, принадлежащей северно-памирской подгруппе восточноиранской группы иранских языков. Внутри восточноиранских языков принято выделять группу памирских языков, к которым, кроме шугнанского, относятся также рушанский, язгулямский, ваханский, ишкашимский, бартангский и другие. Эти бесписьменные языки распространены на территории современной Горно-Бадахшанской автономной области Республики Таджикистан, а также на территории Афганистана, Пакистана и КНР. Шугнанский язык распространён в Рошткалинском и Шугнанском районах Таджикистана и в Афганском Бадахшане. Шугнан находится в долине реки Пяндж и в долине её притоков — Гунт, Шахдара (Рошткалинский район) и Баджув. Число говорящих на шугнанском — примерно 100~тысяч человек \parencite{edelman_yusufbekov1999_shughni}.}.

Из известных нам работ по описанию данного семантического поля отметим статьи «Глаголы перемещения веществ в некоторых финно-угорских языках» \parencite{kashkin2020} и «Глаголы перемещения веществ в типологической перспективе» \parencite{dzedzic2016} (на материале 11~языков), а также дипломную работу «Глаголы движения и перемещения веществ: семантика и типология» \parencite{dzedzic2017}.

Обе эти статьи и дипломная работа написаны в рамках \i{MLexT} / фреймового подхода \parencite{rakhilina_reznikova2013} и опираются на семь основных фреймов, значимых для семантического поля перемещения веществ, выделенных в \parencite{dzedzic2016}: дождь; снег; вода в реке; вода из крана; вещество из контейнера (дыра в контейнере); жидкость из контейнера (опрокинутый контейнер); вещество из контейнера (чрезмерное количество вещества в контейнере). В той же статье отмечается, что для лексикализации идеи перемещения вещества релевантными оказываются следующие параметры: 1)~тип вещества (жидкость — сыпучее вещество); 2)~количество вещества (перемещение отдельными квантами — перемещение сплошным потоком); 3)~свойства исходного контейнера (полное освобождение от содержимого — частичное освобождение от содержимого) \parencite[29–30]{dzedzic2016}.

В дипломной работе Е.~А.~Дзедзич используются данные 13~языков. В этой работе добавлен четвертый параметр «характер движения»: плавное или бурным потоком. Для каузированного перемещения веществ, по мнению автора, имеет значение: 1)~контролируемость — неконтролируемость действия; 2)~направление перемещения: в контейнер — из контейнера; 3)~тип конечной точки: контейнер или поверхность. В одной из подглав анализируется шугнанский материал, взятый в основном из шугнанско-русского словаря Д.~Карамшоева. К сожалению, в данной работе учтены не все шугнанские глаголы, попадающие в семантическое поле \fakesc{ЛИТЬ(СЯ)}–\fakesc{СЫПАТЬ(СЯ)}.

В целом развивая эти идеи, мы рассматриваем как непереходные глаголы и ситуации (семантически декаузативные), так и переходные (семантические каузативные)\fn{В иранистике и индоевропеистике термин «каузативные глаголы» используется для описания определённой глагольной словообразовательной модели — ср.~шугнанские глаголы с суффиксом -\i{ен}-, например, \i{разентоw} ‘заставлять падать; высыпать’. В связи с этим в настоящей работе для обозначения соответствующей семантики, а не суффикса каузативности мы употребляем термин «семантически (де)каузативный»~— \i{прим.~переиздания}.}, а также отдельно рассматриваем ситуации перемещения жидких и сыпучих веществ, связанных с физиологией человеческого тела, такие как выпадение волос, течение крови, пота, слюны, слёз и тому подобные.

Языковые данные изначально были собраны по шугнанско-русскому словарю Д.~Карамшоева [\cite*{karamshoev1988}; \cite*{karamshoev1991}; \cite*{karamshoev1999}], затем эти данные были проанализированы и скорректированы во время экспедиции, которая проходила в июне 2021~года в городе Хорог Горно-Бадахшанской автономной области Республики Таджикистан. После опроса информантов нам пришлось отказаться от анализа нескольких глаголов, отмеченных в словаре и попадающих в исследуемое семантическое поле, поскольку носители языка эти глаголы не знают и не употребляют. К сравнению мы будем иногда привлекать данные таджикского как доминирующего языка, опираясь на современный таджикско-русский словарь \parencite{mirboboev2006}.

В статье последовательно рассмотрены следующие ситуации, входящие в семантическую зону перемещения жидких и сыпучих веществ:

\pagebreak[2]

\begin{enumerate}
	\setcounter{enumi}{1}
	\item Переходные глаголы. Действия человека с жидким и сыпучим веществом
	\begin{enumerate}
		\item Ситуации намеренного высыпания или выливания (контролируемое действие)
		\begin{enumerate}
			\item переливание / пересыпание из сосуда в сосуд, зачерпывание
			\item переливание / пересыпание из сосуда в открытое пространство
		\end{enumerate}
		\item Ситуации неаккуратного обращения с жидким или сыпучим объектом (неконтролируемое действие)
	\end{enumerate}
	
	\item Непереходные глаголы
	\begin{enumerate}
		\item Самопроизвольное перемещение вещества при нарушении целостности сосуда
		\item Природные явления, в которых задействованы жидкие и сыпучие объекты (явления окружающего мира), к ним относятся
		\begin{enumerate}
			\item выпадение осадков
			\item сезонные природные явления или природное нарушение целостности (падение снега с веток, листьев с деревьев, лепестков с цветов, камней с горы и~т.~п.)
			\item течение рек и разных потоков жидкостей
		\end{enumerate}
		\item Движение жидкостей и сыпучих веществ, связанных с физиологией человеческого тела
		\begin{enumerate}
			\item течение слёз, слюны, пота
			\item течение крови
			\item выпадение зубов и волос
		\end{enumerate}
	\end{enumerate}
	
\end{enumerate}

\section{Переходные глаголы. Действия человека с~жидким или сыпучим веществом} \label{pour-2}

\subsection{Ситуации намеренного высыпания или выливания} \label{pour-21}

\subsubsection{2.1.a. Переливание / пересыпание из сосуда в сосуд, зачерпывание} \label{pour-21a}

Процесс активного перемещения жидкого или сыпучего объекта подразумевает определенное действие с сосудом, содержащим в себе вещество: например, его надо наклонить над другим, чтобы вещество переместилось в другой сосуд, или же вещество можно зачерпнуть из большого сосуда, погрузив в него сосуд меньшего объема. В шугнанском языке переходные глаголы со значением ʽналитьʼ и ʽнасыпатьʼ не различают жидкий или сыпучий тип вещества-объекта, поскольку для обоих процессов используются одни и те же глаголы — они перечислены далее.

Сложный глагол \i{холи чӣдоw}\fn{Как и в других иранских языках, в шугнанском довольно употребительны так называемые сложноимённые глаголы, представляющие собой устойчивое сочетание именной (существительное, прилагательное или наречие) и глагольной части, которая и несёт всю грамматическую информацию (глаголов, входящих в такие сочетания, немного).} (\gethref{expour1})–(\gethref{expour2}) состоит из именной части \i{холи}, которая является  заимствованным из таджикского языка прилагательным ʽпустой, порожнийʼ, и глагольной части \i{чӣдоw} ʽделатьʼ. В результате получается переходный глагол со значением ‘опорожнять, высыпать, переливать, освобождатьʼ (про жидкое и сыпучее) \parencite[218–219]{karamshoev1999}. Он используется в большинстве бытовых ситуаций обращения человека с пищевыми продуктами: налить / вылить / перелить чай / суп, насыпать / высыпать / пересыпать сахар / муку. По всей видимости, этот глагол передает идею освобождения, опустошения сосуда, содержащего вещество, от объекта, жидкого или сыпучего, и не важно, перемещается ли объект из большего по объёму сосуда в меньший, или из меньшего в больший, важно, что ориентиром является исходный сосуд, а цель действия — его освобождение от объекта.

\pagebreak[4]

В тех же ситуациях употребляется простой многозначный глагол \i{чӣдоw} ‘делать, изготовлятьʼ (\gethref{expour1})–(\gethref{expour4})\fn{Ширчой — традиционный напиток памирцев, состоящий из молока (\i{шӣр}) и чая (\i{чой}) с добавлением соли, перца и топлёного жира.}\fn{Примеры, приведённые в статье, получены методом элицитации, если не указано иное.}: одним из его значений, отмеченных в словаре, зафиксировано ʽкласть, помещать, сыпать, литьʼ \parencite[123–124]{karamshoev1991}. Однако \i{чӣдоw} попадает в семантическую зону \fakesc{ЛИТЬ(СЯ)–СЫПАТЬ(СЯ)} лишь частично, поскольку обслуживает ситуацию ʽпереместить объектʼ, при этом качество объекта для этого глагола не имеет значения — он может быть жидким (‘налить супа’), сыпучим (‘насыпать муки’) или цельным, не членимым на кванты (‘положить мясо в мешок’).

\ex<expour1>
\begingl
\gla Сāраки=йи му мӯм мāш-ард чой (\b{холи}) \b{чӯд}.//
\glc утро={\sc 3sg} {\sc pron.1sg.o} бабушка {\sc pron.1pl-dat} чай пустой делать.{\sc pst}//
\glft ‘Утром бабушка \textbf{налила} нам чай.’//
\endgl \xe

\ex<expour2>
\begingl
\gla Му нāн=и қанд ар банка (\b{холи}) \b{чӯд}.//
\glc {\sc pron.1sg.o} мать={\sc 3sg} сахар {\sc down} банка пустой делать.{\sc pst}//
\glft ‘Мама \textbf{насыпала} сахар в банку.’//
\endgl \xe

\ex<expour3>
\begingl
\gla Шӣрчой му-рд дӯс-аθ (\textbf{холи}) \textbf{кин}.//
\glc ширчай {\sc pron.1sg.o-dat} немного-{\sc adv} пустой делать[{\sc imp}]//
\glft ‘\textbf{Налей} мне самую малость ширчая.’//
\endgl \xe

\ex<expour4>
\begingl
\gla Ар му пуц мут дӯс йоɣ̌ҷ=и \textbf{чӯд}.//
\glc {\sc down} {\sc pron.1sg.o} сын горсть немного мука={\sc 3sg} делать.{\sc pst}//
\glft ‘Он(а) \textbf{насыпал(а)} в ладонь моему сыну немного муки.’//
\endgl \xe

При погружении меньшего по объёму сосуда в больший — для жидких веществ, или перемещения сыпучих, состоящих из относительно крупных элементов («картошка»), а также для пересыпания зерна лопатой из большой кучи в мешок или другую ёмкость используется глагол \i{бих̌чӣдоw} ʽчерпать, зачерпывать, разливать, насыпать (жидкости и сыпучие вещества)ʼ (\gethref{expour5})–(\gethref{expour6}). Этот глагол используется для обозначения ситуации разделения вещества большого объема на отдельные порции. Например, в соответствующей словарной статье приводятся примеры разлить в миски или разлить по цистернам для жидких веществ и насыпать зерно в мешок для сыпучего объекта \parencite[257–258]{karamshoev1988}.

\ex<expour5>
\begingl
\gla Дам х̌ац йāх ди=ху, wуз \textbf{бих̌чāм}.//
\glc {\sc d2.f.sg.o} вода лёд бить[{\sc imp}]={\sc and1} {\sc pron.1sg} черпать.{\sc prs.1sg}//
\glft ‘Расколи лёд на воде, и я \textbf{зачерпну} [воды].’ \trailingcitation{\parencite[257]{karamshoev1988}}//
\endgl \xe

\ex<expour6>
\begingl
\gla Дам сêр ар бӯҷӣн \textbf{бих̌ча}.//
\glc {\sc d2.f.sg.o} зерно {\sc down} мешок сыпать[{\sc imp}]//
\glft ‘\textbf{Насыпь} то зерно в мешок.’ \trailingcitation{\parencite[258]{karamshoev1988}}//
\endgl \xe

\subsubsection{2.1.b. Переливание / пересыпание из сосуда в открытое пространство} \label{pour-21b}

Эта группа ситуаций отличается от (\hyperref[pour-21a]{2.1.a}) направлением перемещения и типом конечной точки — из контейнера в открытое пространство. В ситуации ʽвылить, высыпатьʼ используются глаголы, различающие способ рассыпания / разливания, но не различающие тип объекта по признаку «жидкое — сыпучее». Здесь важным признаком оказывается движение руки и интенсивность процесса.

При равномерном, спокойном высыпании, например, зерна из сосуда на землю, или выливании воды из ведра используется тот же, что и в ситуации намеренного высыпания из сосуда в сосуд, глагол \i{холи чӣдоw} (буквально ʽосвобождать, делать пустымʼ), при этом целью действия является освобождение исходного контейнера, а ориентиром перемещения вещества может быть как сосуд (примеры \gethref{expour1}–\gethref{expour3}), так и открытое пространство.

\ex<expour7>
\begingl
\gla Ках̌т ситол=ти \textbf{холи} \textbf{кин}.//
\glc зерно стол={\sc sup} пустой делать[{\sc imp}]//
\glft ‘\textbf{Высыпь} (из сосуда) зерно на стол.’//
\endgl \xe

Если для рассыпания или выливания производится резкое движение руками при переворачивании сосуда, то есть для обозначения интенсивного перемещения объекта, а также при ненамеренном, неаккуратном действии используется глагол \i{тис чӣдоw} ʽвыливать, проливать; рассыпать, просыпатьʼ \parencite[82]{karamshoev1999}. Этот глагол также попадает в группу (\hyperref[pour-22]{2.2}).

\ex<expour8>
\begingl
\gla Дам х̌ац \textbf{тис} \textbf{кин}.//
\glc {\sc d2.f.sg.o} вода разлитый делать[{\sc imp}]//
\glft ‘\textbf{Вылей} эту воду.’//
\endgl \xe

Когда же акцент делается на способе рассыпания вещества, жидкого или сыпучего, а именно — равномерно по поверхности широким горизонтальным движением руки от себя (то есть когда объект намеренно рассыпают или разбрасывают), используется глагол \i{ɣ̌ӣбтоw} / \i{ɣ̌ӣптоw} ʽбрызгать, разбрызгивать; разбрасывать; рассыпать; разливать; выплескиватьʼ \parencite[483–484]{karamshoev1999}.

\ex<expour9>
\begingl
\gla Ках̌т зимāδ=ард \textbf{ɣ̌ӣп}, чах̌-ен wи хен.//
\glc зерно земля={\sc dat} рассыпать[{\sc imp}] курица-{\sc pl} {\sc d3.m.sg.o} кушать.{\sc prs.3pl}//
\glft ‘\textbf{Насыпь} (по большой поверхности) это зерно на землю, куры его склюют.’//
\endgl \xe

В эту же группу попадает глагол \i{пирех̌тоw} ‘сыпать, рассыпать, высыпатьʼ \parencite[416]{karamshoev1991}, отличающийся способом движения руки — здесь акцентируется движение пальцами, при этом объект движется строго по вертикали, этот глагол правильно было бы перевести ʽсыпать щепотьюʼ; он употребляется для перемещения сыпучих веществ, состоящих из мелких частиц: например, соли, молотого перца, различных специй, сахарного песка (\gethref{expour10}). Согласно этимологическому словарю иранских языков Д.~И.~Эдельман, это глагол состоит из глагольного корня, к которому восходит современный шугнанский глагол \i{рих̌тоw} ʽлить(ся), сыпать(ся)ʼ, и древнего преверба *\i{пати}- ʽпротив, перед, к, уʼ \parencite[243]{edelman2020_dict} или *\i{пари}-. Глагольный префикс указывает на усиление действия, на завершение, на действие, направленное сквозь, через что-либо \parencite[177]{edelman2020_dict}.

\ex<expour10>
\begingl
\gla Бāд ху йингах̌т wирех̌-т=ху, ~~~~~~~~~~~~~~~~~~~~~~~~~~~~~~~~~~~~~~~~~~~ wим-ирд намак \textbf{пирех̌-т}.//
\glc потом {\sc refl} палец разрезать-{\sc pst=and1} ~ {\sc d3.f.sg.o-dat} соль сыпать-{\sc pst}//
\glft ‘Тогда он разрезал свой палец и \textbf{посыпал} на него соль.’ \trailingcitation{\parencite[416]{karamshoev1991}}//
\endgl \xe

Лишь отчасти в эту группу попадает сложный глагол \i{тихӣрм чӣдоw} ʽразбрасывать, раскидыватьʼ (\i{тихӣрм} ʽразбросанныйʼ) \parencite[83]{karamshoev1999}. Судя по примерам, приведённым в словаре (‘разбросать солому, золу, навоз, подстилки под коровами’), объект действия может быть сыпучим или членимым на отдельные кванты, но не может быть жидким. В отличие от первых четырёх глаголов этой группы, для \i{тихӣрм чӣдоw} не важен способ движения руки или орудия.

\ex<expour11>
\begingl
\gla Wох̌=ат δӣд=ен \textbf{тихӣрм} \textbf{чӯɣ̌ҷ}.//
\glc трава={\sc and2} навоз={\sc 3pl} разбросанный делать.{\sc pf}//
\glft ‘\textbf{Разбросали} солому и навоз.’//
\endgl \xe

В ситуации намеренного перемещения сыпучего объекта с одной открытой поверхности на другую употребляется простой переходный глагол \i{питêwдоw} ʽбросать, кидатьʼ \parencite[387–388]{karamshoev1991} (\gethref{expour12}). Этот глагол используется также в ситуациях намеренного перемещения отдельных предметов, не членимых на кванты (‘бросать яблоко’, ‘закладывать табак’), поэтому он лишь частично попадает в исследуемую семантическую зону.

\ex<expour12>
\begingl
\gla Ахмед сит=и бел қати \textbf{питêw-д}.//
\glc Ахмед песок={\sc 3sg} лопата {\sc com} кидать-{\sc pst}//
\glft ‘Ахмед \textbf{пересыпал} песок лопатой.’//
\endgl \xe

Также в шугнанском языке есть специализированный глагол \i{wиӡêртоw} ʽразбрасывать (навоз), удобрять навозомʼ \parencite[353]{karamshoev1988} (\gethref{expour13}).

\ex<expour13>
\begingl
\gla Δӣд=ен йод тар замӣн=ху, ~~~~~~~~~~~~~~~~~~~~~~~~~~~~~~~~~~~~~~~~~~~~~~~~~~~~~ даδ=ен \textbf{wиӡêр-т} wам.//
\glc навоз={\sc 3pl} нести.{\sc pst} {\sc eq} поле={\sc and1} ~ потом={\sc 3pl} разбрасывать-{\sc pst} {\sc d3.f.sg.o}//
\glft ‘Привозили навоз в поле и потом \textbf{разбрасывали} его.’ \trailingcitation{\parencite[353]{karamshoev1988}}//
\endgl \xe

В ситуации намеренного перемещения жидкости из сосуда в открытое пространство используется сложный глагол \i{х̌ац δêдоw} ʽполивать, орошатьʼ, состоящий из именной части \i{х̌ац} ʽводаʼ и простого многозначного глагола \i{δêдоw} в своем прямом значении ʽдавать, отдаватьʼ \parencite{karamshoev1999}.

\ex<expour14>
\begingl
\gla Зимāδ \textbf{х̌ац} \textbf{δа}, йу̊д-ард лап қоқ.//
\glc земля вода дать[{\sc imp}] {\sc d1-dat} очень сухой//
\glft ‘\textbf{Полей} землю, тут очень сухо.’//
\endgl \xe

Если речь идет о поливе большой поверхности (сельскохозяйственных угодий), используется простой глагол \i{видêӡдоw} ʽорошать, поливатьʼ \parencite[303]{karamshoev1988} (\gethref{expour15}) или синонимичный ему сложный глагол с тем же корнем в именной части \i{видоҷ чӣдоw} ʽорошать, поливать поляʼ, состоящий из существительного \i{видоҷ} ʽорошение, поливʼ и глагольной части \i{чӣдоw} ʽделатьʼ (\gethref{expour16}).

\ex<expour15>
\begingl
\gla Ху жиндам=ат \textbf{видӯйд}=о?//
\glc {\sc refl} пшеница={\sc 2sg} поливать.{\sc pst=q}//
\glft ‘Ты \textbf{полил} свою пшеницу?’ \trailingcitation{\parencite[303]{karamshoev1988}}//
\endgl \xe

\ex<expour16>
\begingl
\gla Асӣд=та wуз пӣнӡ гектāр замӣн \textbf{видоҷ} \textbf{кин-ум}.//
\glc этот\_год={\sc fut} {\sc pron.1sg} пять гектар земля полив делать-{\sc prs.1sg}//
\glft ‘В этом году я буду \textbf{орошать} пять гектаров земли.’ \trailingcitation{\parencite[304]{karamshoev1988}}//
\endgl \xe

\subsection{Ситуации неаккуратного обращения с жидким или сыпучим объектом} \label{pour-22}

Для описания ситуации неаккуратного обращения с жидкими или сыпучими веществами (твёрдыми, членимыми на отдельные кванторы, например, ягоды, спички и тому подобное) используются сложные глаголы с именной частью \i{тис}: семантически каузативный \i{тис чӣдоw} ‘выливать, проливать; рассыпать, просыпатьʼ, при котором жидкое или сыпучее вещество становится прямым объектом, и его декаузативная пара \i{тис ситтоw} ‘выливаться; рассыпаться’. Значение \i{тис} в словаре не определено, поскольку как самостоятельное слово эта единица не употребляется.

\ex<expour17>
\begingl
\gla — Йу̊д-анд чӣз \textbf{тис} \textbf{суδҷ}? ~~~~~~~~~~~~~~~~~~~~~~~~~~~~~~~~~~~~~~~~~~~~ — Йу̊д-анд(ӣр) Фариза рӯған \textbf{тис} \textbf{чӯд}.//
\glc ~~~~~ {\sc d1-loc} вещь разлитый стать.{\sc pf.m.sg} ~ ~~~~~~ {\sc d1-loc} Фариза масло разлитый делать.{\sc pst}//
\glft ‘— Что здесь \textbf{разлилось}? — Здесь Фариза \textbf{разлила} масло.’//
\endgl \xe

Залоговой парой к нему становится глагол \i{тис ситтоw}\fn{В шугнанском языке, как и во многих других иранских языках, залоговые пары сложноимённых глаголов образуются путём замены глагольной части \i{чӣдоw} ‘делать’ (активный залог) на глагольную часть \i{ситтоw} ‘делаться, становиться’ (пассивный залог).} ‘выливаться, проливаться, разливаться; высыпаться, рассыпаться, просыпатьсяʼ \parencite[82]{karamshoev1999}. Это пассивный глагол, в котором сыпучее или жидкое вещество становится логическим субъектом действия. Важным элементом значения обоих глаголов является процесс переливания или пересыпания через край сосуда или любого другого вместилища — например, мешка для сыпучих веществ (\gethref{expour17}–\gethref{expour20}).

\ex<expour18>
\begingl
\gla Ар wам чайнак х̌ац лап вад=ху, ~~~~~~~~~~~~ йā чой \textbf{тис} \textbf{сат}.//
\glc {\sc down} {\sc d3.f.sg.o} чайник вода много быть.{\sc pst.f/pl=and1} ~ {\sc d3.f.sg} чай разлитый стать.{\sc pst.f/pl}//
\glft ‘В том чайнике было много воды, и чай \textbf{перелился}.’//
\endgl \xe

\ex<expour19>
\begingl
\gla Х̌увд ҷу̊х̌ δод=ху, ~~~~~~~~~~~~~~~~~~~~~~~~~~~~~~~~~~~~~~~~~~~~~~~~~~~~~~~~ плитка-йард \textbf{тис} \textbf{сут}.//
\glc молоко кипение упасть.{\sc pst=and1} ~ плита-{\sc dat} разлитый стать.{\sc pst.m.sg}//
\glft ‘Молоко закипело и \textbf{разлилось} на плите.’//
\endgl \xe

\ex<expour20>
\begingl
\gla Wи-нд қāп биринҷ қати аз wи сӣвд=ти wех̌-т=ху, биринҷ \textbf{тис} \textbf{сут}.//
\glc {\sc d3.m.sg.o-loc} мешок рис {\sc com} {\sc el} {\sc d3.m.sg.o} плечо={\sc sup} падать-{\sc pst=and1} рис разлитый стать.{\sc pst.m.sg}//
\glft ‘Мешок с рисом свалился у него с плеча, и рис \textbf{рассыпался}.’//
\endgl \xe

Только для сыпучих веществ используется каузативный глагол \i{разентоw} ‘осыпать, рассыпать, высыпатьʼ \parencite[474]{karamshoev1991} — он образован от основы настоящего времени глагола \i{рих̌тоw} ʽсыпатьсяʼ, который используется только по отношению к сыпучим веществам\fn{Например, ‘не пролей!’ — \i{тис мак!}}.

\ex<expour21>
\begingl
\gla Дим биринҷ \textbf{мā-разен}!//
\glc {\sc d2.f.sg.o} рис {\sc proh}-сыпать[{\sc imp}]//
\glft ‘Не \textbf{просыпь} этот рис!’//
\endgl \xe

\section{Непереходные глаголы} \label{pour-3}
\subsection{Самопроизвольное перемещение вещества при нарушении целостности сосуда или любого другого вместилища} \label{pour-31}

В шугнанском языке глаголы, обозначающие движение жидкости или сыпучих веществ при потере функциональности сосуда или любого другого вместилища, различают жидкое или сыпучее вещество и траекторию движения; важным элементом значения оказывается также интенсивность движения вещества. При перемещении сыпучих веществ сверху вниз используется глагол \i{рих̌тоw} ʽсыпатьсяʼ, который употребляется при строго вертикальном движении объекта (\gethref{expour22}–\gethref{expour23}). Глагол с этим же корнем \fn{Перс.~\i{рихтан}, тадж.~\i{рехтан} 1)~лить, наливать; 2)~литься, течь; проливаться; 3)~впадать, втекать, вливаться (о реке, ручье); 4)~сыпать, осыпать; 5)~сыпаться, просыпаться; 6)~осыпаться, опадать (о листьях); 7)~падать, выпадать (о зубах, волосах) \parencite[695]{mirboboev2006}.} в персидском и таджикском языках имеет более широкое применение, является доминирующим в семантическом поле \fakesc{ЛИТЬ(СЯ)–СЫПАТЬ(СЯ)} и употребляется по отношению к жидким и сыпучим веществам. В обоих языках этот глагол двузалоговый, то есть обозначает активное и пассивное действие.

\ex<expour22>
\begingl
\gla Аз дишид сит \textbf{рих̌-т}.//
\glc {\sc el} крыша почва сыпаться-{\sc pst}//
\glft ‘С крыши \textbf{сыпался} песок.’//
\endgl \xe

\ex<expour23>
\begingl
\gla Му сифц-ен=ен ас му мāк тӣр \textbf{рих̌-т}.//
\glc {\sc pron.1sg.o} бусы-{\sc pl=3pl} {\sc el} {\sc pron.1sg.o} шея {\sc sup} сыпаться-{\sc pst}//
\glft ‘Мои бусы \textbf{рассыпались} [упав] с моей шеи.’//
\endgl \xe

Если же траектория движения не важна, а важно, что в результате перемещения вещество, разделившись на кванторы, стало занимать большую поверхность чего-либо, употребляется (а)~сложный глагол \i{тихӣрм ситтоw}\fn{См.~комментарий к \i{тихӣрм} в \hyperref[pour-21b]{разделе~2.1.b}.} ʽстановиться разбросаннымʼ \parencite[83]{karamshoev1999}, где \i{тихӣрм} — ʽразбросанныйʼ, а глагольная часть — \i{ситтоw} ʽстановиться, делаться, превращатьсяʼ, или (б)~сложный глагол \i{тис ситтоw} ‘выливаться, проливаться, разливаться; высыпаться, рассыпаться, просыпатьсяʼ, для которого важным элементом значения является ʽперемещение через край контейнераʼ (\gethref{expour24}).

\ex<expour24>
\begingl
\gla Пӯрг=и бӯҷӣн δод=ху, биринҷ ~~~~~~~~~~~~~~~~~ \textbf{тихӣрм} / \textbf{тис} \textbf{сат}.//
\glc мышь={\sc 3sg} мешок ударить.{\sc pst=and1} рис ~ рассыпанный ~~~~~~ рассыпанный стать.{\sc pst.f/pl}//
\glft ‘Мышь прогрызла мешок, и рис \textbf{просыпался}.’//
\endgl \xe

При нарушении целостности сосуда, вмещающего в себе жидкое вещество, в зависимости от интенсивности течения используются глаголы \i{чиктоw} / \i{чактоw} ʽкапать, течьʼ для слабого течения (\gethref{expour26}) или глагол \i{тӣдоw} ʽидти, течь, струитьсяʼ для больших потоков жидкости (\gethref{expour27}).

\ex<expour26>
\begingl
\gla Тар кухни кран \textbf{чак-т}, соз wи чӣд-оw.//
\glc {\sc eq} кухня кран капать-{\sc prs.3sg} целый {\sc d3.m.sg.o} делать.{\sc inf-purp}//
\glft ‘В кухне \textbf{течёт} кран, его нужно починить.’//
\endgl \xe

\ex<expour27>
\begingl
\gla Чалак ку̊ɣ̌ӡ виц=ху, х̌ац аз wам-анд \textbf{тӣзд}.//
\glc ведро дыра быть.{\sc pf.f/pl=and1} вода {\sc el} {\sc d3.f.sg.o-loc} идти.{\sc prs.3sg}//
\glft ‘Ведро оказалось дырявым, и из него \textbf{течёт} вода.’//
\endgl \xe

\subsection{Природные явления, в которых задействованы жидкие и сыпучие объекты (явления окружающего мира)} \label{pour-32}

\subsubsection{3.2.a. Выпадение осадков} \label{pour-32a}

Ситуации природных явлений, связанных с выпадением осадков, пересекаются с семантической зоной ʽпадатьʼ; подробнее о глаголах падения в шугнанском языке см.~[\hyperref[chapter-rakh-down]{Рахилина, Некушоева 2020}].

В отличие от таджикского и персидского, где есть специализированный глагол перс.~\i{бāридан} ʽидти (о дожде, снеге)ʼ \parencite{rubinchik1983} (подробнее о глаголах, использующихся для описания процесса выпадения осадков в персидском языке, см.~\parencite{armand_nikitenko2020}), тадж.~\i{боридан} 1)~ʽидти (об осадках)ʼ; 2)~пер.~ʽсыпаться; литьсяʼ \parencite[157]{mirboboev2006}, в шугнанском языке специализированного глагола нет.

Чаще всего для природных явлений, связанных с выпадением осадков, дождя или снега, используется глагол \i{δêдоw}\fn{О поле падения см.~статью [\hyperref[chapter-rakh-down]{Рахилина, Некушоева 2020}] в настоящем сборнике — \i{прим.~переиздания}.}, одним из значений которого является ‘падать, выпадать, идти (об осадках)ʼ \parencite[501–502]{karamshoev1988} (\gethref{expour28}), а также глагол \i{анҷафцтоw} ‘начинаться, идти (об осадках)ʼ \parencite[107]{karamshoev1988}, при этом глаголе само слово ‘снегʼ или ‘дождьʼ употребляется факультативно (\gethref{expour29}).

\ex<expour28>
\begingl
\gla Ката-рӯз бору̊н \textbf{δод}=ху, сел=и фук йод.//
\glc много-день дождь упасть.{\sc pst=and1} сель={\sc 3sg} всё нести.{\sc pst}//
\glft ‘Весь день \textbf{шёл} дождь, и сель все унёс.’//
\endgl \xe

\ex<expour29>
\begingl
\gla Динйо=та нур \textbf{анҷафц-т}.//
\glc мир={\sc fut} сегодня идти\_осадки-{\sc prs.3sg}//
\glft ‘Сегодня \textbf{пойдёт дождь} (снег).’//
\endgl \xe

Для обозначения слабого дождя используют специализированный глагол \i{цирактоw} ʽморосить, капать, накрапыватьʼ \parencite[305]{karamshoev1999} (\gethref{expour30}), также для слабых осадков, дождя и снега, используется звукоподражательный глагол \i{цуртоw} ʽкапать, моросить, шуршать (например, об осадках)ʼ \parencite[315]{karamshoev1999} с основным значением ʽшуршать, шелестетьʼ (\gethref{expour31}).

\ex<expour30>
\begingl
\gla Бору̊н \textbf{цирак-т}.//
\glc дождь моросить-{\sc prs.3sg}//
\glft ‘Дождь \textbf{моросит}.’//
\endgl \xe

\ex<expour31>
\begingl
\gla Жиниҷ ғал ас қāст-аθ \textbf{цар-т}.//
\glc снег({\sc f}) ещё {\sc el} чуть-{\sc adv} шуршать.{\sc f-prs.3sg}//
\glft ‘Cнег ещё \textbf{идёт} (букв. шуршит).’//
\endgl \xe

\subsubsection{3.2.b. Природное нарушение целостности или сезонные природные явления} \label{pour-32b}

Глагол \i{рих̌тоw} ʽсыпаться, осыпаться, опадать, высыпаться, выпадатьʼ \parencite[474]{karamshoev1991} используется для обозначения условно вертикального падения сыпучих (не жидких) объектов, например, падения снега с веток, опадания листьев или семян с деверьев и кустарников или лепестков с цветов (\gethref{expour32}), (\gethref{expour33}). Существительное, образованное от того же глагольного корня, \i{рӣх̌т}, означает ʽснежный обвал, снежная лавинаʼ. Возможно, в таких ситуациях проявляется исходное значение этого общеиранского корня — ʽпокидать, освобождатьʼ. Так, в этимологическом словаре иранских языков для праиранского корня *\i{raik}-:\i{rik}- отмечается: «В древнеиранских диалектах от глаголов со значением ʽпокидать, отпускать, выпускать, освобождатьʼ отделилась группа глаголов с развившейся семантикой ʽлить(ся), сыпать(ся), выливать(ся), осыпать(ся)ʼ, образовав затем в некоторых языках омонимичные пары» \parencite[342]{edelman2020_dict}. Как и в персидском языке, этот глагольный корень обозначает потерю целостности некоторого объекта путём разделения его на более мелкие составляющие части и их последующего падения. Эту идею подтверждают однокоренные глаголы, например, глагол \i{вирих̌тоw} ʽломать(ся), разбивать(ся); раскалывать(ся)ʼ. То же относится и к падению листьев с дерева осенью — так, шугнанское слово \i{пāркрез} ʽлистопадʼ состоит из существительного \i{пāрк} ʽлист (растения)ʼ и глагольной части \i{рез} ʽпадениеʼ. В примере (\gethref{expour32}) снежный сугроб, лежащий на вершине горы, сначала разделился на части, а затем одна или несколько его частей упали вниз.

\ex<expour32>
\begingl
\gla Жиниҷ ас кӯ ну̊л=ти \textbf{рих̌-т}.//
\glc снег {\sc el} гора пик={\sc sup} сыпаться-{\sc pst}//
\glft ‘Снег \textbf{осыпался} с вершины горы.’//
\endgl \xe

\ex<expour33>
\begingl
\gla Тӣрамо=та парк-ен \textbf{раз-ен}.//
\glc осень={\sc fut} лист-{\sc pl} сыпаться-{\sc prs.3pl}//
\glft ‘Осенью листья \textbf{осыпаются}.’//
\endgl \xe

Для падения большого количества снега с горы также используется глагол \i{хато ситтоw} ʽпадать, сваливаться; скользить, выскальзыватьʼ, где \i{хато} — ʽошибка; заблуждение, оплошностьʼ, однако этот глагол, видимо, следует отнести к семантическому полю \fakesc{ПАДАТЬ}, а не \fakesc{СЫПАТЬСЯ}, поскольку он обозначает падение большого количества снега как нечленимого на отдельные кванты целого.

\ex<expour34>
\begingl
\gla Жиниҷ ас кӯ=ти \textbf{хато} \textbf{сут}.//
\glc снег {\sc el} гора={\sc sup} соскользнувший стать.{\sc pst.m.sg}//
\glft ‘Снег \textbf{сорвался} с горы.’//
\endgl \xe

Для описания процесса осыпания мелких, не связанных единством, предметов, то есть падения не вертикально вниз, а соприкасаясь с поверхностью, используется сложный глагол \i{оле ситтоw} ʽкатиться, валитьсяʼ (\gethref{expour35}), состоящий из слова \i{оле} ʽкувырком, катясьʼ (судя по словарной статье, оно употребляется только в составе сложных глаголов) и глагол \i{ситтоw} ‘стать’.

\ex<expour35>
\begingl
\gla Бāд=и бору̊н ас кӯ жӣр-ен=ен \textbf{оле} \textbf{сат}.//
\glc после={\sc ez} дождь {\sc el} гора камень-{\sc pl=3pl} кувырком стать.{\sc pst.f/pl}//
\glft ‘После дождя с горы \textbf{посыпались} мелкие камни.’//
\endgl \xe

\subsubsection{3.2.c. Течение рек и разных потоков жидкости (природные явления, не связанные с деятельностью человека)} \label{pour-32c}

Для больших потоков жидкости в большинстве случаев используется глагол \i{тӣдоw} ‘идти; течьʼ \parencite[71–72]{karamshoev1999}, при этом не различается горизонтальное или вертикальное перемещение, а ориентиры движения ʽоткудаʼ, ʽкудаʼ, ʽвнизʼ задаются пространственными предлогами и послелогами (\gethref{expour36}–\gethref{expour38}). Этот же глагол употребляется в случае движения большого потока воды сверху вниз по поверхности стены (\gethref{expour39}) или движения селевых потоков, свойственных горной местности (\gethref{expour40}). По всей видимости, глагол \i{тӣдоw} обозначает перемещение жидкости, которая движется, соприкасаясь с поверхностью чего-либо (земли/стены/горы и~т.~п.).

\ex<expour36>
\begingl
\gla Кӯ бӣрӣн х̌ац \textbf{тӣзд}.//
\glc гора {\sc sub} вода идти.{\sc prs.3sg}//
\glft ‘Под горой \textbf{течёт} река.’//
\endgl \xe

\ex<expour37>
\begingl
\gla Аз мāш қишлоқ ар тагов-ди дарйо \textbf{тӣзд}.//
\glc {\sc el} {\sc pron.1pl} кишлак {\sc down} внизу-{\sc comp} река идти.{\sc prs.3sg}//
\glft ‘От нашего кишлака вниз \textbf{течёт} река.’//
\endgl \xe

\ex<expour38>
\begingl
\gla Ар wи кӯ-йаθ шар.шара \textbf{тӣзд}.//
\glc {\sc down} {\sc d3.m.sg.o} гора-{\sc adv} водопад идти.{\sc prs.3sg}//
\glft ‘По той горе \textbf{идёт} водопад.’//
\endgl \xe

\ex<expour39>
\begingl
\gla Ар ди деwол-аθ х̌ац \textbf{тӣзд}.//
\glc {\sc down} {\sc d2.m.sg.o} стена-{\sc adv} вода идти.{\sc prs.3sg}//
\glft ‘По этой стене \textbf{течёт} вода.’//
\endgl \xe

\ex<expour40>
\begingl
\gla Wӯвд х̌абу̊на-рӯз ми Мӯн=анд сел \textbf{тӯйд}.//
\glc семь ночь-день {\sc d1.m.sg.o} Мун={\sc loc} сель идти.{\sc pst.m.sg}//
\glft ‘Семь суток в Муне \textbf{шёл} сель.’ \trailingcitation{\parencite[72]{karamshoev1999}}//
\endgl \xe

Для обозначения движения сильного потока воды используется глагол \i{рахнā чӣдоw}, для которого в словаре даётся значение ʽпробивать отверстие, проламыватьʼ, где именная часть \i{рахнā} — ʽпролом, пробоина, брешь; щель, отверстиеʼ \parencite[479]{karamshoev1991}, однако этот глагол употребляется в значении ʽхлынуть, течь потокомʼ в тех ситуациях, когда вода пробивает отверстие в чём-либо.

\ex<expour41>
\begingl
\gla Х̌ац=и ас тӣр-аθ \textbf{рахнā} \textbf{чӯд}.//
\glc вода={\sc 3sg} {\sc el} {\sc sup-adv} поток делать.{\sc pst}//
\glft ‘Вода сверху потоком \textbf{текла}.’//
\endgl \xe

Изредка для обозначения движения селевых потоков используется сложный глагол \i{раwу̊н ситтоw} ʽидти, отправляться в путь, трогаться, начинать движениеʼ \parencite[472–473]{karamshoev1991}, состоящий из именной части \i{раwу̊н} ʽидущий, отправляющийся, намеревающийся идтиʼ и глагольной части \i{ситтоw} ʽстатьʼ. \i{Раwу̊н} — это заимствованное из таджикского языка причастие настоящего времени глагола \i{рафтан} ʽидти, уходитьʼ, которое в шугнанском языке немного изменило исходное значение ʽидущий, движущийсяʼ и во многих контекстах приобрело проспективное значение ʽсобираться идтиʼ.

\ex<expour42>
\begingl
\gla Тарма.тари δед δед=ху, ~~~~~~~~~~~~~~~~~~~~~~~~~~~~~~~~~ сел=ат санг \textbf{раwу̊н} \textbf{су̊д}.//
\glc распутица упасть.{\sc prs.3sg} упасть.{\sc prs.3sg=and1} ~ сель={\sc and2} камень идущий стать.{\sc prs.3sg}//
\glft ‘Если идёт дождь и снег, \textbf{случится} горный обвал.’ \trailingcitation{\parencite[554–555]{karamshoev1991}}//
\endgl \xe

Для обозначения слабого течения жидкости используется глагол \i{нах̌тӣдоw} ʽвыходить; возникать, появлятьсяʼ, так можно сказать о смоле дерева, текущей по стволу, однако этот глагол с трудом можно отнести к семантическому полю ʽлитьсяʼ.

\ex<expour43>
\begingl
\gla Буор=анд дарахт шӣрā \textbf{нах̌тӣзд}.//
\glc весна={\sc loc} дерево сок выйти.{\sc prs.3sg}//
\glft ‘Весной сок \textbf{сочится} из дерева (смола).’//
\endgl \xe

\subsection{Движение жидкостей и сыпучих веществ, связанных с физиологией человеческого тела} \label{pour-33}

\subsubsection{3.3.a. Течение слёз, пота, слюны} \label{pour-33a}

Глаголы, обозначающие перемещение жидких и сыпучих веществ, употребляются и по отношению к процессам, связанным с физиологией человеческого тела (например, у человека текут слёзы, слюна, кровь, пот, выпадают зубы, и сыпятся волосы).

В ситуации ʽтечьʼ о слезах употребляются четыре разных глагола. Если слеза воспринимается как отдельная капля, а не как текущая жидкость употребляется глагол \i{рих̌тоw} ʽсыпатьсяʼ, который, как правило, обслуживает сыпучие, а не жидкие вещества; при этом глагол стоит в форме 3~лица мн.~ч. и согласуется со словом ‘слёзы’ по числу.

\ex<expour44>
\begingl
\gla Потх̌о=нд=ен мис wи йӯх̌к-ен \textbf{рих̌-т}.//
\glc царь={\sc loc=3pl} тоже {\sc d3.m.sg.o} слеза-{\sc pl} литься-{\sc pst}//
\glft ‘И у царя \textbf{полились} слёзы.’ \trailingcitation{\parencite[474]{karamshoev1991}}//
\endgl \xe

Если же слёзы льются обильно и воспринимаются как большой поток жидкости, используется тот же глагол, что и для обозначения течения рек и больших потоков жидкости: \i{тӣдоw} ʽидти, течь, струитьсяʼ, при этом глагол \i{тӣдоw} стоит в форме 3~лица ед.~ч., а слово ʽслёзыʼ — во множественном числе (\gethref{expour45}). Глагол \i{тӣдоw} также используется в ситуациях, когда течёт пот (\i{арāқ}) или слюни (\i{шāф}), как в случае перемещения больших потоков жидкости, поскольку слёзы, пот и слюни текут, соприкасаясь с какой-нибудь поверхностью, в данном случае — с поверхностью человеческого тела (\gethref{expour46}).

\ex<expour45>
\begingl
\gla Йӯх̌к-ен ас цем=анд \textbf{ти-йен}.//
\glc слеза-{\sc pl} {\sc el} глаз={\sc loc} идти-{\sc prs.3pl}//
\glft ‘Слёзы \textbf{льются} из глаз.’ (‘слеза’ как капля $\rightarrow$ \i{рих̌тоw}, а если поток $\rightarrow$ \i{тӣдоw})//
\endgl \xe

\ex<expour46>
\begingl
\gla Ту шāф \textbf{тӣзд}, ху ғêв ҷāм ки.//
\glc {\sc pron.2sg} слюна идти.{\sc prs.3sg} {\sc refl} рот закрытый делать[{\sc imp}]//
\glft ‘У тебя \textbf{текут} слюни, закрой рот.’ \trailingcitation{\parencite[414]{karamshoev1999}}//
\endgl \xe

Также со слезами может употребляться глагол \i{зибидоw} ʽпрыгать, отваливаться, отскакивать; падать, валитьсяʼ.

\ex<expour47>
\begingl
\gla Wи нāн wи ба-йоδ δод=ат, ~~~~~~~~~~~~~~~~~~~~~~~~ wи йу̊х̌к мис \textbf{зибуд}.//
\glc {\sc d3.m.sg.o} мама {\sc d3.m.sg.o} {\sc all}-память упасть.{\sc pst=and2} ~ {\sc d3.m.sg.o} слеза тоже прыгать.{\sc pst.m.sg}//
\glft ‘Он вспомнил свою маму, и у него слёзы \textbf{покатились} (из глаз).’//
\endgl \xe

Для обозначения начала действия используется сложный глагол \i{раwу̊н ситтоw}, как в примере (\gethref{expour48}).

\ex<expour48>
\begingl
\gla Wи ғиδā=нд=ен wи йӯх̌к-ен \textbf{раwу̊н} \textbf{сат}.//
\glc {\sc d3.m.sg.o} мальчик={\sc loc=3pl} {\sc d3.m.sg.o} слеза-{\sc pl} идущий стать.{\sc pst.f/pl}//
\glft ‘У юноши \textbf{потекли} слёзы.’ \trailingcitation{\parencite[65]{karamshoev1991}}//
\endgl \xe

\subsubsection{3.3.b. Течение крови} \label{pour-33b}

Глаголы, обозначающие перемещение крови из тела, различаются по интенсивности движения. Если течение крови слабое, используется глагол \i{чиктоw} / \i{чактоw} ʽкапать, течьʼ, который обозначает перемещение жидкости отдельными квантами, каплями, или \i{тӣдоw} ʽидти, течь, струитьсяʼ, то~есть течь по поверхности тела или повязки (\getfullhref{expour49.a}). Если кровь сочится, появляется на поверхности тела или, например, на повязке, употребляется глагол \i{нах̌тӣдоw} ʽвыходить, возникать, появлятьсяʼ (\getfullhref{expour49.b}). При очень сильном кровотечении используется сложный глагол \i{тармезак δêдоw} ‘струиться с напором, бить фонтаномʼ \parencite[41]{karamshoev1999}, где именная часть \i{тармезак} — ʽмочаʼ, а глагольная часть — многозначный глагол \i{δêдоw} (\gethref{expour52}). Также для сильного течения крови используется глагол \i{ташрā ситтоw}, который не зафиксирован в словаре Карамшоева (\gethref{expour53}).

\pex<expour49>
\a<a> \begingl
\gla Хӯн пӣс wи бӣнт-аθ \textbf{чак-т} ~~~~~~~~~~~~~~~~~~~~~~~~ / \textbf{тӣзд}.//
\glc кровь {\sc goal} {\sc d3.m.sg.o} повязка-{\sc adv} капать-{\sc prs.3sg} ~ ~~~~~~ идти.{\sc prs.3sg}//
\glft ‘Кровь \textbf{сочится} сквозь повязку.’//
\endgl
\a<b> \begingl
\gla Хӯн тар wи бӣнт мис \textbf{нах̌тӯйд}.//
\glc кровь {\sc eq} {\sc d3.m.sg.o} повязка {\sc add2} выйти.{\sc pst.m.sg}//
\glft ‘Кровь \textbf{сочилась} сквозь повязку.’//
\endgl \xe

\ex<expour52>
\begingl
\gla Дис хах̌ захм=и вуд, ~~~~~~~~~~~~~~~~~~~~~~~~~~~~~~~~~~~~~~~~~~~~~~~~~~~~~ йу хӯн=и \textbf{тармезак} \textbf{δод}.//
\glc такой сильный рана={\sc 3sg} быть.{\sc pst.m.sg} ~ {\sc d3.m.sg} кровь={\sc 3sg} моча дать.{\sc pst}//
\glft ‘Рана была такой глубокой, что кровь \textbf{хлестала фонтаном}.’ \trailingcitation{\parencite[41]{karamshoev1999}}//
\endgl \xe

\ex<expour53>
\begingl
\gla Уз=ум δод чи-пӣц, му нêӡ хӯн ~~~~~~~~~~~~~ \textbf{ташрā} \textbf{сут}.//
\glc {\sc pron.1sg=1sg} упасть.{\sc pst} {\sc cont1}-лицо {\sc pron.1sg.o} нос кровь ~ хлынувший стать.{\sc pst.m.sg}//
\glft ‘Я упала вниз лицом, и у меня кровь \textbf{хлынула} из носа.’//
\endgl \xe

\subsubsection{3.3.c. Выпадение зубов и волос} \label{pour-33c}

Процесс выпадения зубов и волос различает интенсивность процесса, точнее — единичность или множественность ситуации. Поскольку интенсивное выпадение зубов и волос приводит к нарушению целостности объекта, в таких случаях употребляется глагол \i{рих̌тоw} ʽсыпатьсяʼ (\gethref{expour54})–(\gethref{expour55}), здесь также можно восстановить этимологическое значение ʽпокидать [привычное место]ʼ. Если же выпадает один зуб или один волос, то используется глагол \i{wêх̌тоw} ʽпадать, валиться; рушитьсяʼ (\gethref{expour56}).

\ex<expour54>
\begingl
\gla Wи мӯсафед=анд wи δинду̊н-ен=ен \textbf{рих̌-т}.//
\glc {\sc d3.m.sg.o} старик={\sc loc} {\sc d3.m.sg.o} зуб-{\sc pl=3pl} сыпаться-{\sc pst}//
\glft ‘У старика \textbf{выпали} все зубы.’//
\endgl \xe

\ex<expour55>
\begingl
\gla Wи-нд wи ғӯнҷ ач.га \textbf{на-раз-д}.//
\glc {\sc d3.m.sg.o-loc} {\sc d3.m.sg.o} волосы больше {\sc neg}-сыпаться-{\sc prs.3sg}//
\glft ‘У него перестали выпадать волосы (букв. больше \textbf{не выпадают}).’//
\endgl \xe

\ex<expour56>
\begingl
\gla Wи кудак=анд wи δинду̊н \textbf{wêх̌-т}.//
\glc {\sc d3.m.sg.o} ребёнок={\sc loc} {\sc d3.m.sg.o} зуб падать-{\sc pst}//
\glft ‘У мальчика \textbf{выпал} молочный зуб.’//
\endgl \xe

\section{Выводы} \label{pour-conclusions}

В семантическое поле перемещения жидких и сыпучих веществ \fakesc{ЛИТЬ(СЯ)–СЫПАТЬ(СЯ)} попадает примерно 30~глаголов. Их можно классифицировать, во-первых, по признаку переходности–непереходности (в группе переходных глаголов есть также разделение на контролируемое–неконтролируемое действие), во-вторых, по тому, различают ли они тип вещества (жидкое или сыпучее) или нет, в-третьих, по признаку интенсивности перемещения вещества (слабо–нейтрально–сильно), а также по признаку широты охвата поверхности. Для переходных глаголов в качестве признака также выделяется способ движения рукой или наличие инструмента для переливания / пересыпания.

Из 30 глаголов два встречаются в наших примерах чаще остальных. Это непереходный глагол \i{рих̌тоw} ʽсыпатьсяʼ, который используется, во-первых, в ситуациях самопроизвольного перемещения вещества при нарушении целостности сосуда, во-вторых, при природном нарушении целостности, то~есть падении снега с веток, листьев с деревьев, лепестков с цветов и~т.~п., в том числе при выпадении зубов и волос. Этот глагол обозначает вертикальное перемещение сыпучих веществ или объектов, членимых на отдельные кванты, при нарушении целостности или разрушении объекта.

Непереходный глагол \i{тӣдоw} ʽидти; течьʼ используется, во-первых, в ситуации самопроизвольного перемещения жидкости при нарушении целостности сосуда, во-вторых, течения рек и разных потоков жидкостей, в-третьих, течения слез, пота, слюны, крови, то~есть для обозначения перемещения относительно больших объемов жидкости сплошным потоком по поверхности.

По признаку интенсивности действия глаголу \i{тӣдоw} противопоставлен глагол \i{чиктоw} / \i{чактоw} ʽкапатьʼ, который используется в ситуациях самопроизвольного перемещения вещества при нарушении целостности сосуда и течения крови у человека или животного.

Для контролируемого действия при перемещении объекта из контейнера в контейнер не различается тип объекта, однако важным признаком оказывается цель действия. Для глагола \i{холи чӣдоw} ʽналивать, насыпатьʼ цель действия — освобождение сосуда от объекта. Для описания процесса разделения объекта на части / порции с помощью другого сосуда или инструмента используется глагол \i{бих̌чӣдоw} ʽчерпать, зачерпывать, разливать, насыпать (жидкости и сыпучие вещества)ʼ. Для обозначения перемещения объекта, в том числе жидкого или сыпучего, из одного места в другое используется глагол \i{чӣдоw} в значениях ʽкласть; помещать; сыпать; литьʼ.

Для ситуаций контролируемого перемещения жидкого или сыпучего вещества в открытое пространство важными параметрами оказываются интенсивность перемещения и способ движения рукой: \i{тис чӣдоw} ʽвыливать, рассыпатьʼ — резким движением, \i{пирех̌тоw} ʽсыпатьʼ — пальцами, \i{ɣ̌ӣбтоw} / \i{ɣ̌ӣптоw} ʽразбрасывать; разливатьʼ — широким движением руки по большой поверхности, \i{питêwдоw} ʽбросать, разбрасыватьʼ — с помощью орудия, например, лопатой. Специализированные глаголы для перемещения жидкости различают площадь поверхности: при небольшой площади поверхности употребляется глагол \i{х̌ац δêдоw} ʽполиватьʼ, а если площадь поверхности большая, используются однокоренные глаголы \i{видêӡдоw} и \i{видоҷ чӣдоw} ʽорошатьʼ.

Непереходные глаголы различают естественное (природное) перемещение жидких или сыпучих объектов и перемещения, являющиеся результатом человеческой деятельности. К первой группе относятся глаголы, различающие интенсивность движения. Это глаголы, обозначающие выпадение осадков: нейтральный глагол \i{δêдоw} ʽидти (об осадках)ʼ, \i{цирактоw} и \i{цуртоw} ʽморосить, капать, накрапыватьʼ о слабых осадках, а также \i{анҷафцтоw} ʽначинаться, идти (об осадках)ʼ.

Глаголы, обозначающие самопроизвольное перемещение жидкостей, различаются интенсивностью: \i{чиктоw} / \i{чактоw} ʽкапать, течьʼ — слабое движение, \i{нах̌тӣдоw} ʽструитьсяʼ — среднее, \i{тармезак δêдоw} ʽструиться с напором, бить фонтаномʼ — сильное движение. Сюда же относится стилистически маркированный \i{раwу̊н ситтоw} ‘начинать движение, идти (о слезах)ʼ — этот калькированный по таджикской модели глагол с заимствованной именной частью употребляется, как правило, в сказках.

Только два глагола из нашего списка образуют залоговые пары, это \i{тис чӣдоw} ʽвыливать, высыпать, рассыпатьʼ — \i{тис ситтоw} ʽвыливаться, рассыпатьсяʼ (через край сосуда, в котором находится объект, о жидких и сыпучих объектах) и \i{тихӣрм чӣдоw} ʽразбрасывать, раскидыватьʼ — \i{тихӣрм ситтоw} ʽбыть разбросаннымʼ (не только о сыпучих, но и о нечленимых объектах, в результате действия объект занимает большое пространство).

Непереходный глагол \i{оле ситтоw} ʽкатиться, валитьсяʼ обычно обозначает движение мелких предметов, например, камней, определённым способом — перекатываясь. Глагол \i{wêх̌тоw} ʽпадатьʼ не входит в семантическое поле \fakesc{ЛИТЬ(СЯ)–СЫПАТЬ(СЯ)}.

По всей видимости, для шугнанского языка мы не должны выделять в отдельную группу глаголы, обозначающие движение жидкостей и сыпучих веществ, связанных с физиологией человеческого тела. В таком случае подгруппу ситуаций \hyperref[pour-33a]{3.3.а} «течение слёз, слюны, пота» мы можем объединить с подгруппой \hyperref[pour-32c]{3.2.c} «течение рек и разных потоков жидкостей»; ситуации течения крови выделить в отдельную подгруппу \hyperref[pour-33b]{3.3.b}, а подгруппу \hyperref[pour-33c]{3.3.c} «выпадение волос и зубов» — отнести к семантическому полю \fakesc{ПАДАТЬ}.

Таким образом, лексическая карта семантического поля \fakesc{ЛИТЬ(СЯ)–СЫПАТЬ(СЯ)} для шугнанского языка будет выглядеть так:

\begin{sidewaysfigure}
\def\arraystretch{1.2}
\centering
\caption{Семантическое поле \fakesc{ЛИТЬ(СЯ)–СЫПАТЬ(СЯ)} в шугнанском языке (переходные глаголы)}
\medskip
\begin{forest}
  for tree={
    child anchor=north,
    parent anchor=south,
    grow'=south, text centered,
    draw, fit=tight,
    anchor=south,
    inner sep=1mm,
    font=\scriptsize}
    [переходные глаголы
      [ненамеренное\\действие
        [сыпучее\\(\i{разентоw}), tier=2]
        [жидкое\\и сыпучее\\(\i{тис чӣдоw}), tier=2]
      ]
      [намеренное действие
        [переливание /\\пересыпание\\из сосуда в сосуд, tier=2
          [перемещение объекта\\из одной точки\\в другую\\(\i{чӣдоw})]
          [освобождение\\от объекта\\исходного сосуда\\(\i{холи чӣдоw})]
          [деление объекта\\на части / порции\\(\i{бих̌тоw})]
        ]
        [переливание /\\пересыпание\\из сосуда\\на поверхность, tier=2
          [рассыпать /\\разливать\\равномерно\\по поверхности, calign=child, calign child=-2
            [жидкое\\по большой\\поверхности\\(\i{видêӡдоw} /\\\i{видоҷ чӣдоw})]
            [разбрасывать\\навоз\\(\i{wиӡêртоw})]
            [жидкое\\(\i{х̌ац δêдоw})]
            [сыпучее\\(\i{ɣ̌ӣптоw})]
            [разбрасывать\\разделяя\\на части\\(\i{тихӣрм чӣдоw})]
		  ]
          [рассыпать\\пальцами\\(\i{пирех̌тоw})]
          [пересыпать\\лопатой\\(\i{питêwдоw})]
		  [освобождать\\сосуд\\наклоняя
            [нейтрально\\(\i{холи чӣдоw})]
            [интенсивно\\(\i{тис чӣдоw})]
          ]
	]
      ]
    ]
\end{forest}

\end{sidewaysfigure}

\begin{sidewaysfigure}
\def\arraystretch{1.2}
\centering
\caption{Семантическое поле \fakesc{ЛИТЬ(СЯ)–СЫПАТЬ(СЯ)} в шугнанском языке (непереходные глаголы)}
\medskip
\begin{forest}
  for tree={
    child anchor=north,
    parent anchor=south,
    grow'=south, text centered,
    draw, fit=tight,
    anchor=south,
    inner sep=1mm,
    font=\scriptsize}
    [непереходные глаголы, tier=0
        [природные явления, tier=1
		[природное\\нарушение\\целостности, tier=2
		      [катясь\\по поверх-\\ности\\(\i{оле ситтоw}), tier=3]
		      [верти-\\кально\\(\i{рих̌тоw}), tier=3]
		]
		[течение рек\\и других\\потоков\\жидкости, calign=last, tier=3
		      [сильно;\\пробивая\\брешь\\(\i{рахнā чӣдоw}), tier=4]
		      [начало\\действия\\(\i{раwу̊н}\\i{ситтоw}), tier=4]
		      [по поверх-\\ности\\(\i{тӣдоw}), tier=4]
		]
	    [выпадение\\осадков, tier=4
		      [шурша\\(\i{цуртоw}), tier=5]
		      [слабо\\(\i{цирактоw}), tier=5]
		      [начинаться\\(\i{анҷафцтоw}), tier=5]
		      [нейтрально\\(\i{δêдоw}), tier=5]
		]
		[течение крови, tier=2
		      [слабо\\(\i{чиктоw} /\\\i{чактоw}), tier=3]
		      [сильно, tier=3
		          [фонтаном\\(\i{тармезак}\\\i{δêдоw}), tier=4]
    		      [струёй\\(\i{ташрā}\\\i{ситтоw}), tier=4]
		      ]
		      [нейтрально\\(\i{тӣдоw}), tier=3]
		]
	  ]
        [самопроизвольное\\перемещение\\вещества\\при нарушении\\целостности, tier=3
	    [сыпучее, tier=4
		      [по большой\\поверхности\\(\i{тихӣрм ситтоw}), tier=5]
    		  [вертикально\\(\i{рих̌тоw}), tier=5]
		]
		[жидкое, tier=4
		      [слабо\\(\i{чиктоw} /\\\i{чактоw}), tier=5]
		      [сильно\\(\i{тӣдоw}), tier=5]
		]
	  ]
    ]
\end{forest}

\end{sidewaysfigure}