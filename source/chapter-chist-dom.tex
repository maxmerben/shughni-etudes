\chapter*{Дифференцированное объектное маркирование в~шугнанском языке}
\addcontentsline{toc}{chapter}{\textit{Д.~Чистякова}. \textbf{Дифференцированное объектное маркирование в~шугнанском языке}}
\setcounter{section}{0}
\chaptermark{Дифференцированное объектное маркирование в~шугнанском языке}
\label{chapter-chist-dom}

\begin{customauthorname}
Дарья Чистякова
\end{customauthorname}

\begin{englishtitle}
\i{Differential object marking in~Shughni\\{\small Daria Chistiakova}}
\end{englishtitle}

\begin{abstract}
Во многих иранских языках встречается дифференцированное объектное маркирование (ДОМ). Вследствие того, что во многих иранских языках в ходе развития были утрачены падежи, для маркирования прямого объекта стали использоваться специальные показатели, и в разных языках на их использование могут влиять разные факторы: время и аспект, иерархия определённости и одушевлённости, информационная структура \parencites[33]{windfuhr2009}{bossong1985}. В шугнанском языке маркирование объекта возникает нерегулярно, кроме того, вследствие отсутствия письменной нормы в самом языке среди носителей наблюдается высокая вариативность. Цель данной работы — описать факторы, влияющие на маркирование объекта в шугнанском языке.
\end{abstract}

\begin{keywords}
дифференцированное объектное маркирование, шугнанский язык, переходность, иерархия определённости и одушевлённости, референтность.
\end{keywords}

\begin{eng-abstract}
Many Iranian languages have developed differential object marking (DOM). Since many languages have lost case inflection, special markers were used to mark direct objects. In different Iranian languages, DOM may depend on different factors: specificity hierarchy, tense and aspect or information structure \parencites[33]{windfuhr2009}{bossong1985}. In Shughni special marking of the object occurs irregularly; in addition, due to the lack of a written tradition, there is high variability among native speakers. The purpose of this study is to describe the factors influencing the special marking of an object in Shughni.
\end{eng-abstract}

\begin{eng-keywords}
differential object marking, Shughni, transitivity, specificity hierarchy, animacy.
\end{eng-keywords}

\begin{acknowledgements}
Публикация подготовлена по результатам исследования (проект~№~22-00-034) в рамках Программы «Научный фонд Национального исследовательского университета “Высшая школа экономики” (НИУ~ВШЭ)» в~2022~году.
\end{acknowledgements}

\begin{initialprint}
\fullcite{chistiakova2022_dom}\end{initialprint}

\section{Введение} \label{dom-intro}

Во многих иранских языках встречается дифференцированное маркирование аргументов \parencite[63]{korn2016_dialectology} и, в частности, дифференцированное объектное маркирование (ДОМ). Вследствие того, что во многих иранских языках в ходе развития была утрачена или радикально редуцирована падежная система, для маркирования прямого объекта стали использоваться специальные показатели, и в разных языках на их использование могут влиять разные факторы: время и аспект, иерархия определённости и одушевлённости, информационная структура \parencites[33]{windfuhr2009}{bossong1985}.

В шугнанском языке дифференцированное маркирование объекта встречается нерегулярно; кроме того, вследствие отсутствия письменной нормы в самом языке среди носителей наблюдается высокая вариативность. Цель данной работы — описать факторы, влияющие на маркирование объекта в шугнанском языке. В \hyperref[dom-shughni]{первой части} статьи я приведу необходимые для дальнейшего рассуждения факты из грамматики шугнанского языка, а также существующие уже описания маркирования объекта в других памирских языках. Во \hyperref[dom-factors]{второй части} я описываю непосредственно факторы, которые оказываются значимыми и для шугнанского языка: в \hyperref[dom-corpus]{Разделе~2.1} приведены данные, полученные путём корпусного исследования \parencite{makarov_etal2022}, в \hyperref[dom-elicit]{Разделе~2.2} — полевые данные, собранные путём элицитации в июне 2022~года в городе Хорог при поддержке Университета Центральной Азии.

\subsection{Шугнанский язык} \label{dom-shughni}

Шугнанский язык входит в шугнано-рушанскую группу памирских языков, которые относятся к восточноиранской группе иранских языков. В шугнано-рушанскую группу, помимо шугнанского языка, входят также рушанский, хуфский, сарыкольский, рошорвский и бартангский языки и их диалекты. Стандартный, но не строгий порядок слов в предложении — SOV, порядок составляющих в именной группе — {\sc Dem} {\sc Adj} N, есть предлоги и послелоги. В шугнанском языке есть прямой и косвенный падежи, которые поверхностно выражаются на местоимениях; имена остаются немаркированными, поэтому синтаксические роли участников обычно выводятся из порядка слов и/или из глагольного согласования.

Ещё один способ определить синтаксические роли связан с особенным употреблением демонстративов. В шугнанском языке есть три серии указательных местоимений, все они могут использоваться как демонстративы, согласующиеся по роду, числу и падежу, как притяжательные местоимения или анафорически. Кроме того, демонстративы третьей серии в современном шугнанском языке используются как определённый артикль. Таким образом, по прямой или косвенной форме указательного местоимения можно определить падеж именной группы, определённой этим местоимением.

\ex<exdom1>
\begingl
\gla {[}Wи карасӣн{]}=ен йоδҷ-ат=о?//
\glc {\sc d3.m.sg.o} керосин={\sc 3pl} нести.{\sc pf-pqp=q}//
\glft ‘[Они] отвезли керосин?’ \trailingcitation{\parencite[91]{karamshoev1988}}//
\endgl \xe

В примере (\gethref{exdom1}) лицо и число подлежащего выражены при помощи отделяемой энклитики =\i{ен} [{\sc 3pl}], занимающей ваккернагелевскую позицию после первой полной составляющей. Косвенная форма прямого объекта видна из косвенной формы указательного местоимения, отображаемой в глоссах как [{\sc o}] (англ.~\i{oblique case}).

\subsection{ДОМ в шугнано-рушанской группе} \label{dom-dom}

В языках шугнано-рушанской группы с разной частотностью может встречаться дифференцированное объектное маркирование: прямой объект иногда может маркироваться при помощи предлога \i{ас} [{\sc el}], имеющего аблативное значение. Для шугнанского языка (и других языков шугнано-рушанской группы) это было отмечено, например, в словаре Карамшоева [\cite*[137]{karamshoev1988}], в работах \parencites[43]{pakhalina1969_pamir}[804]{edelman_dodykhudoeva2009_shughni} и в работе Г.~Боссонга, посвященной дифференцированному объектному маркированию в новоиранских языках \parencite[98–103]{bossong1985}.

В примере (\gethref{exdom2}) прямой объект ‘её муж’ маркирован предлогом \i{ас}, в примере (\gethref{exdom3}) прямой объект никак не маркирован.

\ex<exdom2>
\begingl
\gla Чāй \b{ас} дам чор wӣн-ч?//
\glc кто {\sc el} {\sc d2.f.sg.o} муж видеть-{\sc pf}//
\glft ‘Кто видел её мужа?’ \trailingcitation{\parencite[219]{karamshoev1991}}//
\endgl \xe

\ex<exdom3>
\begingl
\gla Йида тама=йāм wӣн-т.//
\glc {\sc d2.loc} {\sc pron.2pl=1pl} видеть-{\sc pst}//
\glft ‘Вот мы вас увидели.’ \trailingcitation{\parencite[166]{karamshoev1991}}//
\endgl \xe

Предлог \i{ас} происходит от древнеиранского аблативного предлога \*\i{hača}. Видимо, он был заимствован из таджикского языка, но стал использоваться шире, чем в таджикском. В обычном своём значении в шугнанском языке \i{ас} маркирует источник, стимул, объект сравнения, причину и используется как показатель партитива.

Согласно \parencites[101]{bossong1985}[163–167]{payne1980} специальное маркирование объекта в той или иной степени встречается во всех языках шугнано-рушанской группы, во всех языках показатель прямого объекта происходит от предлога \i{ас}. Наиболее часто такое маркирование встречается в сарыкольском, рошорвском и бартангском языках с топикальными одушевлёнными определёнными прямыми объектами. Особое внимание уделяется сарыкольскому языку: в сарыкольском языке предлог \i{ас} в функции показателя прямого объекта превратился в проклитику \i{а}= (а в качестве предлога сохранил форму \i{ас}) \parencite[101]{bossong1985}. Боссонг считает, что частотность дифференцированного объектного маркирования в сарыкольском сопоставима с частотностью \i{rā} в персидском, то есть это полноценный регулярный ДОМ. Для шугнанского и рушанского отмечается, что предлог \i{ас} как маркер прямого объекта используется значительно реже, но предполагается, что на его употребление также влияют определённость и одушевлённость объекта \parencite[101]{bossong1985}.

Боссонг также отмечает для сарыкольского языка, что в именной группе проклитика \i{а}= занимает место между определением и определяемым \parencite[101]{bossong1985}. Стоит уточнить, что проклитика присоединяется непосредственно к определяемому слову, если указательное местоимение употреблено как притяжательное (\gethref{exdom4}), но слева от всей именной группы, если указательное местоимение используется как демонстратив (\gethref{exdom5}). В приведенных ниже примерах в первом случае проклитика \i{а}= присоединяется непосредственно к существительному-прямому объекту, пропустив притяжательное местоимение \i{wи} ‘его’, во втором, напротив, проклитика присоединяется к демонстративу, оформляя целиком группу \i{ди х̌ытыр} (‘этого верблюда’):

\ex<exdom4>
\begingl
\gla …wаз=ам {[}wи \b{а}=мырδо{]} параδу-д…//
\glc {\sc pron.1sg=1sg} {\sc d3.m.sg.o} {\sc acc}=труп продать-{\sc pst}//
\glft ‘…я его труп продал…’ \trailingcitation{\parencite[54]{pakhalina1969_pamir}}//
\endgl \xe

\ex<exdom5>
\begingl
\gla {[}\b{А}=ди х̌ытыр{]}=ат ас ко выг?//
\glc {\sc acc=d2.m.sg.o} верблюд={\sc 2sg} {\sc el} где привести.{\sc pst}//
\glft ‘Откуда ты достал этого верблюда?’ \trailingcitation{\parencite[55]{pakhalina1969_pamir}}//
\endgl \xe

Для шугнанского языка такое противопоставление притяжательной и указательной функций местоимений нехарактерно.

\section{Выявленные факторы} \label{dom-factors}

В этом разделе рассматриваются факторы, которые оказываются значимыми для маркирования объекта в шугнанском языке.

\subsection{Корпусные данные} \label{dom-corpus}

В ходе корпусного исследования я разобрала фольклорный текст «Белая горная коза», записанный в 1958~году И.~И.~Зарубиным (цит.~по~\parencite[85–91]{shakarmamadov2005}). В нём были размечены все переходные клаузы с выраженным прямым объектом: я отмечала наличие/отсутствие \i{ас}; время клаузы; свойства объекта (определённость, одушевлённость); порядок слов; какой использован глагол (сложный или простой); чем выражен объект (местоимение или именная группа).

В 13 из 130~клауз прямой объект маркирован при помощи предлога \i{ас}. Все объекты одушевлённые и определённые, по остальным параметрам не наблюдается никаких ограничений. Тем не менее, клауз с определёнными и одушевлёнными прямыми объектами, не маркированными при помощи \i{ас}, гораздо больше (44~случая). Таким образом, мы не можем считать одушевлённость и определённость прямого объекта непосредственно влияющими факторами.

Если посмотреть на глаголы, с которыми используется \i{ас}, то можно предположить, что на маркирование объекта влияет тип глагола. Употребление \i{ас} частотнее с не прототипически переходными глаголами, семантическая роль прямого объекта которых ближе к стимулу, объекту сравнения или источнику (например, \i{пêхцтоw} ‘спросить’, \i{қӣwдоw} ‘звать’, \i{ринӣх̌тоw} ‘забывать’, \i{зер чӣдоw} ‘победить’, буквально ‘вниз сделать’).

Я предполагаю, что в этой группе менее прототипически переходных глаголов наблюдается не дифференцированное маркирование объекта, а вариативная модель управления, что объясняет такую частотность примеров с употреблением \i{ас}. Однако с другими (более прототипически переходными) глаголами (\i{wӣнтоw} ‘видеть’, \i{зӣдоw} ‘убивать’, \i{δêдоw} ‘дать, ударить’) наблюдается именно дифференцированное объектное маркирование.

Глаголы, которые я называю здесь прототипически переходными, сопоставлены с глаголами из работы С.~С.~Сая [\cite*[560–563]{sai2018}], где в результате типологического исследования глаголам присваивается «индекс переходности», варьирующийся от $0$ до $1$, в зависимости от того, в каком проценте языков для кодирования объекта данного глагола используется канонически переходная конструкция. Так, индекс переходности глаголов «убивать» и «ломать» равен $1$, так как во всех рассмотренных языках для кодирования объекта этих глаголов используется переходная конструкция. Индексы глаголов «ударить» ($0,79$) и «видеть» ($0,91$) также достаточно высоки, чтобы рассматривать их как глаголы, для которых характерна каноническая переходная конструкция. Напротив, индекс глагола «забывать» ($0,37$) скорее низкий, то есть во многих языках объект этого глагола оформляется косвенным падежом. Так как в корпусных и элицитированных примерах с этим глаголом в шугнанском языке существенно чаще встречаются примеры с \i{ас}, я предполагаю, что частотность косвенного маркирования связана именно с вариативной моделью управления, а не с ДОМ.

\begin{sidewaystable}
 \centering
 \caption{Маркирование объекта в тексте «Белая горная коза»}
 \smallskip
 \label{tab:dom1}
 \begin{tabular}{c|c|cc} \toprule
 & глагол & \makecell{объект,\\маркированный \i{ас}} & \makecell{объект,\\не маркированный\\дополнительно} \\ \midrule
 \multirow{5}{*}{{\small \makecell{менее\\прототипически\\переходные\\глаголы $\rightarrow$\\вариативная\\модель\\управления}}} & \i{лу̊вдоw} ‘сказать’ & 2 & 3 \\
  & \i{қӣwдоw} ‘звать’ & 2 & 3 \\
  & \i{пêхцтоw} ‘спрашивать’ & 3 & 4 \\
  & \i{ринӣх̌тоw} ‘забывать’ & 1 & 0 \\
  & \makecell{\i{зер чӣдоw} ‘победить’,\\букв.~‘вниз сделать’} & 1 & 0 \\
 \multirow{3}{*}{{\small \makecell{более\\прототипически\\переходные\\глаголы}}} & \i{wӣнтоw} ‘видеть’ & 1 & 3 \\
  & \i{зӣдоw} ‘убивать’ & 2 & 10 \\
  & \i{δêдоw} ‘дать, ударить’ & 1 & 5 \\ \bottomrule
 \end{tabular}
\end{sidewaystable}

Я также разобрала «Сказку про трёх братьев», записанную В.~С.~Соколовой в~1948–1949~годов на шугнанском, рушанском, бартангском и сарыкольском языках (цит.~по~\parencite[50–57]{pakhalina1969_pamir}). Из сравнения примеров с маркированным прямым объектом в разных языках видно, что в шугнанском и рушанском языках маркирование объекта встречается реже всего: в обоих языках есть только один пример с \i{ас}, в то время как в сарыкольском их~23. Бартангский вариант оказался посередине: в тексте встретилось 10~примеров с маркированием объекта.

\subsection{Элицитация} \label{dom-elicit}

В шугнанском языке отсутствует письменная норма, вследствие чего наблюдается очень высокая вариативность, которая распространяется в том числе на частоту и регулярность маркирования объекта в речи носителей: контексты, в которых у одних информантов \i{ас} возникал спонтанно, другие носители могли запрещать; если после анкеты я спрашивала у информантов, как им кажется, в каких ситуациях так говорят, ответы тоже отличались разительно. В ходе элицитации было опрошено 9~носителей в возрасте от~15 до~43~лет.

Один из немногих строгих запретов был получен на употребление \i{ас} с именной частью сложного глагола.

\ex<exdom6>
\begingl
\gla \ljudge{*}Му боб бийор \b{ас} нақли чу.//
\glc {\sc pron.1sg.o} отец вчера {\sc el} история делать.{\sc pst}//
\glft ‘Мой отец вчера рассказал историю.’ \trailingcitation{\parencite[7]{sergienko2022}}//
\endgl \xe

Это кажется важным замечанием, учитывая частотность в шугнанском языке сложных глаголов разной степени композициональности и, зачастую, высокий уровень синтаксической свободы у именной части — настолько высокий, что А.~А.~Сергиенко предлагает считать её псевдоинкорпорированным прямым объектом: именная часть может свободно передвигаться внутри предложения, может модифицироваться прилагательными и местоимениями, но не может быть маркирована при помощи \i{ас} \parencite[7]{sergienko2022}.

Также в процессе элицитации мне встретилось одно употребление \i{ас}, которое, кажется, нигде раньше не описывалось: если я пыталась поместить \i{ас} в контекст, где он был не очень органичен, многие информанты могли разрешить \i{ас} в значении ‘такой же, как’. Пример (\gethref{exdom7}) без \i{ас} значил бы ‘Мой дядя привёз мне этот мяч’, но появление предлога сразу меняет смысл предложения. Указательные местоимения третьей серии (например, \i{wам} [{\sc d3.f.sg.o}]) могут использоваться как определённые артикли (согласующиеся по роду, числу и падежу), как притяжательные местоимения или анафорически.

\ex<exdom7>
\begingl
\gla Му холак=и му-рд \b{ас} wам пу̊т вӯд.//
\glc {\sc pron.1sg.o} дядя={\sc 3sg} {\sc pron.1sg.o-dat} {\sc el} {\sc d3.f.sg.o} мяч принести.{\sc pst}//
\glft ‘Мой дядя привёз мне такой же мяч, как у неё.’ \trailingcitation{[элицитация, 2022]}//
\endgl \xe

Вероятнее всего, такое употребление возникло из стандартного употребления \i{ас} при объекте сравнения: \i{Ас Барсем тӣр-ди Шорв} [{\sc el} Барсем верх-{\sc comp} Шорв] ‘Шорв выше Барсема’ \parencite[140]{karamshoev1988}.

В остальном элицитированные данные часто выглядели противоречивыми. С уверенностью можно сказать, что ни время, ни аспект в шугнанском языке не влияют на маркирование прямого объекта.

Я попробовала обобщить разные контексты, в которых носители использовали \i{ас}, и получила три основных фактора: определённость и референтность прямого объекта, эмфатическое выделение и одушевлённость прямого объекта (точнее, высокое положение пациенса относительно агенса на шкале одушевлённости-определённости Сильверстайна \parencite[176]{silverstein1976}). У одной из носительниц в спонтанном употреблении встретился также партитивный контекст, но в связи с тем, что все остальные носители этот пример запретили, я не рассматриваю партитивность как возможный фактор. Несмотря на то, что я стараюсь описывать факторы по отдельности, чаще всего в примерах можно наблюдать комбинацию факторов.

Одна из моих гипотез состояла в том, что специальное маркирование объекта в первую очередь закрепилось за закрытым классом непрототипически переходных глаголов, прямой объект которых с точки зрения семантической роли близок к стимулу, объекту сравнения или источнику (‘спрашивать’, ‘прощать’, ‘побеждать’ и~так~далее). В процессе элицитации я пришла к выводу, что с менее прототипически переходными глаголами ДОМ закрепился как альтернативная модель управления, а в более прототипически переходных контекстах (‘убить’, ‘съесть’, ‘ударить’) на выбор оформления прямого объекта влияют свойства самого объекта.

Важно сказать, что дифференцированное маркирование объекта так и не стало правилом в шугнанском языке: по крайней мере в современном языке нет контекста, в котором \i{ас} в качестве маркера прямого объекта был бы обязательным.

\subsubsection{Определённость и референтность} \label{dom-definite}

В большинстве примеров прямой объект, маркированный при помощи \i{ас}, должен быть также маркирован демонстративом третьей серии, выступающим в роли определённого артикля. Тем не менее, далеко не каждый определённый прямой объект получает дополнительное маркирование: видимо, это один из факторов, увеличивающих вероятность появления ДОМ, но не определяющих его.

Интересно внимательнее посмотреть на референциальный статус прямого объекта, который допускает маркирование при помощи \i{ас}. В примере (\getfullhref{exdom8.a}) запрещено использование \i{ас} с неопределёнными прямыми объектами, но разрешено с определёнными (\getfullhref{exdom8.b}). Один из определённых объектов в (\getfullhref{exdom8.b}) дополнительно маркирован эмфатической частицей \i{ик}=\fn{Можно обратить внимание, что \i{ик}= оформляет сразу всю конструкцию \i{ас wи аwқот} ‘этот ужин’, ср.~\i{ик=[wам дирахт бун=анд]} [{\sc emph=d3.f.sg.o} дерево низ={\sc loc}) ‘под этим деревом’ \parencite[280]{karamshoev1988}.}:

\pex<exdom8>
\a<a> \begingl
\gla Му нāн=и (*ас) куртā анцӯв-д=атā, ~~~~~~~~~~~~~~~~~~~~~~~ му йах=и (*ас) аwқот пêх-т.//
\glc {\sc pron.1sg.o} мама={\sc 3sg} {\sc el} платье шить-{\sc pst=and3} ~~~~~~~~~~~ {\sc pron.1sg.o} сестра={\sc 3sg} {\sc el} еда готовить-{\sc pst}//
\glft ‘Мама сшила платье, а сестра приготовила ужин.’ \trailingcitation{[элицитация, 2022]}//
\endgl
\a<b> \begingl
\gla Му нāн=и \b{ас} wам куртā анцӯв-д=атā, ~~~~~~~~~~~ му йах=и ик=\b{ас} wи аwқот пêх-т.//
\glc {\sc pron.1sg.o} мама={\sc 3sg} {\sc el} {\sc d3.f.sg.o} платье шить-{\sc pst=and3} ~~~~~~~~~~~ {\sc pron.1sg.o} сестра={\sc 3sg} {\sc emph=el} {\sc d3.m.sg.o} еда готовить-{\sc pst}//
\glft ‘Мама сшила это платье, а сестра приготовила вот этот ужин.’ \trailingcitation{[элицитация, 2022]}//
\endgl \xe

Интересно, что контексты с генерическим употреблением именной группы (\gethref{exdom9}) удачнее, чем контексты с референтной неопределённой именной группой (\gethref{exdom10}–\gethref{exdom11}). Это совпадает с наблюдениями Т.~Гивона, который объединял определённые и генерические именные группы, противопоставляя их референтным неопределённым и нереферетным неопределённым. Гивон объединял определённые и генерические именные группы как наиболее топикальные ИГ, при помощи которых наиболее часто выражено подлежащее \parencite[295]{givon1979}. Таким образом, тот факт, что в шугнанском языке именно определённые и генерические именные группы в роли объекта могут получать дополнительное маркирование, говорит о том, что маркируется нарушение иерархии: специальное оформление получает прямой объект, обладающий свойствами, характерными скорее для подлежащего.

\ex<exdom9>
\begingl
\gla Уз=ум ху зиндаги=нди ачаθ \b{ас} wурҷ на-wӣн-ч.//
\glc {\sc pron.1sg=1sg} {\sc refl} жизнь={\sc loc} совсем {\sc el} волк {\sc neg}-видеть-{\sc pf}//
\glft ‘Я никогда в жизни не видел волка.’ \trailingcitation{[элицитация, 2022]}//
\endgl \xe

\ex<exdom10>
\begingl
\gla Бийор=ум уз йак-ум бор (?\b{ас}) wурҷ wӣн-т.//
\glc вчера={\sc 1sg} {\sc pron.1sg} один-{\sc ord} раз {\sc el} волк видеть-{\sc pst}//
\glft ‘Вчера я впервые увидел волка.’ \trailingcitation{[элицитация, 2022]}//
\endgl \xe

\ex<exdom11>
\begingl
\gla Ғāц-биц-ак=и (?\b{ас}) тӯδ=ат ~~~~~~~~~~~~~~~~~~~~~~~~ сӣзд нêδ-д.//
\glc девочка-детёныш.{\sc f-dim=3sg} {\sc el} тутовник={\sc add2} ~ джида сажать-{\sc pst}//
\glft ‘Девочка посадила тутовник и джиду.’ \trailingcitation{[элицитация, 2022]}//
\endgl \xe

\subsubsection{Эмфатическое выделение ИГ} \label{dom-emphatic}

Возможно, именно в связи со своей «необязательностью» маркирование объекта закрепилось в эмфатических контекстах, где прямой объект получает дополнительное выделение. Информанты часто спонтанно порождали \i{ас} в контекстах с отрицанием (\gethref{exdom9}), либо в контекстах с эмфатической частицей \i{ик}= и демонстративом (\gethref{exdom12}).

\ex<exdom12>
\begingl
\gla Wам вирод=и ик=\b{ас} wам тӯδ вирух̌-т.//
\glc {\sc d3.f.sg.o} брат={\sc 3sg} {\sc emph=el} {\sc d3.f.sg.o} тутовник ломать-{\sc pst}//
\glft ‘Её брат сломал вот этот самый тутовник.’ \trailingcitation{[элицитация, 2022]}//
\endgl \xe

Кроме контекстов, где непосредственно ИГ маркирована при помощи специального показателя, есть примеры, где вся клауза оказывается наделена эмфазой: например, это может быть неожиданная информация, выделенная интонационно.

\ex<exdom13>
\begingl
\gla Йи.лāв=и ниғух̌-т диди, wи нāн=ат ~~~~~~~~~~~~~~~~~~~~~~~~ йу жиндӯрв-ак=та ик=дис лу̊в-ен диди: «Аррāнг вид, нур х̌āб \b{ас} wи ху калтā-ди пуц зӣн-āм».//
\glc немного={\sc 3sg} слышать-{\sc pst} {\sc compl} {\sc d3.m.sg.o} мать={\sc add2} ~ {\sc d3.m.sg} оборотень-{\sc dim=fut} {\sc emph}=так говорить-{\sc prs.3pl} {\sc compl} как быть.{\sc prs.3sg} сегодня ночь {\sc el} {\sc d3.m.sg.o} {\sc refl} большой-{\sc comp} сын убить-{\sc prs.1pl}//
\glft ‘[Он] услышал, что его мать и оборотень говорят: «Во что бы то ни было, сегодня ночью убьём старшего сына».’ \trailingcitation{[сказка «Белая горная коза»]}//
\endgl \xe

Можно сказать, что эмфатическое выделение придает бо́льшую значимость прямому объекту, вследствие чего он получает дополнительное маркирование. Это не уникальное явление: например, в бенгали специальный показатель обычно не присоединяется к неодушевлённому и неопределённому объекту, но может появиться, если объект оказывается в фокусе\fn{“If the theme is under focus or contrastive stress, the marker \i{ke} occurs” \parencite[461]{subbarao2016}.} \parencite[461]{subbarao2016}. В шугнанском языке контрастный фокус на объекте может провоцировать маркирование объекта, но если объект в контрастном фокусе не обладает описанными выше референциальными свойствами (определённая и генерическая именная группа), \i{ас} не будет употребляться.

\subsubsection{Одушевлённость} \label{dom-animacy}

Другой тип контекстов, в которых появляется объектный показатель, связан одновременно с одушевлённостью и с описанной выше проблемой неразличения синтаксических ролей. В контекстах, где агенс и пациенс никак не маркированы, единственный способ отличить подлежащее от прямого объекта — порядок слов. Тем не менее, в ситуации, где агенс значительно выше пациенса на шкале одушевлённости Сильверстайна, распределение ролей очевидно. Сложности возникают в контекстах, где пациенс занимает непрототипически высокую позицию в иерархии Сильверстайна, и при этом ни на агенсе, ни на пациенсе падеж не выражается.

\ex<exdom14>
\begingl
\gla Низорā=йи \b{ас} Сафӣнā δод.//
\glc Низора={\sc 3sg} {\sc el} Сафина ударить.{\sc pst}//
\glft ‘Низора ударила Сафину.’ \trailingcitation{[элицитация, 2022]}//
\endgl \xe

Если порядок слов нарушается, и объект выносится вперёд, маркирование объекта может быть использовано для различения участников, котороые в противном случае могут быть перепутаны (\gethref{exdom15}). Подобный порядок слов с тем же смыслом без предлога \i{ас} невозможен.

\ex<exdom15>
\begingl
\gla \b{Ас} Саӣдā=йи Аҳмед жӣwҷ.//
\glc {\sc el} Саида={\sc 3sg} Ахмед любить//
\glft ‘Ахмед любит Саиду.’ \trailingcitation{[элицитация, 2022]}//
\endgl \xe

Пример (\gethref{exdom16}) был одним из наиболее удачных — его часто выдавали спонтанно, и никто ни разу его не запретил, несмотря на то, что и агенс, и пациенс находятся не очень высоко на шкале одушевлённости — то есть важно в первую очередь их положение друг относительно друга, а не просто одушевлённость.

\ex<exdom16>
\begingl
\gla Куд \b{ас} пӯрг на-хӣрт.//
\glc собака {\sc el} мышь {\sc neg}-кушать.{\sc prs.3sg}//
\glft ‘Собаки мышей не едят.’ \trailingcitation{[элицитация, 2022]}//
\endgl \xe

Подобное употребление \i{ас} частично совпадает с обобщениями Боссонга, который считал, что в шугнано-рушанской группе регулярно маркируются имена собственные и личные местоимения \parencite[102]{bossong1985}. Я полагаю, это происходит потому, что они вероятнее всего будут располагаться высоко на иерархии одушевлённости Сильверстайна, но не менее важным фактором (безусловно связанным с позицией пациенса на иерархии) становится необходимость различать синтаксические роли участников в контекстах, где они явно не выражены.

\subsubsection{Случаи партитивного употребления \i{ас}} \label{dom-as}

Самый редкий из встретившихся в спонтанном употреблении примеров с \i{ас} — партитивный. В своем обычном предложном употреблении (не в качестве маркера прямого объекта) \i{ас} является основным показателем партитива в шугнанском:

\ex<exdom17>
\begingl
\gla \b{Ас} ху ворҷ-ен йӣw му-рд дāк.//
\glc {\sc el} {\sc refl} лошадь-{\sc pl} один {\sc pron.1sg.o-dat} дать[{\sc imp}]//
\glft ‘Дай мне одну из своих лошадей.’ \trailingcitation{\parencite[140]{karamshoev1988}}//
\endgl \xe

Теоретические многие контексты с \i{ас} можно было бы анализировать как партитивные, но единственный спонтанно полученный пример с партитивным употреблением \i{ас} запретила большая часть информантов.

\ex<exdom18>
\begingl
\gla \ljudge{\b{?}}Йу=йи дӯс \b{ас} палоw хӯд.//
\glc {\sc d3.m.sg=3sg} немного {\sc el} плов кушать.{\sc pst}//
\glft ‘Он поел немного плова.’ \trailingcitation{[элицитация, 2022]}//
\endgl \xe

В любых контекстах, претендующих на партитивность, при прямом объекте требуется также употребление демонстратива и квантификатора (\getfullhref{exdom19.a}). Вероятнее всего, партитивность здесь лишь случайный контекст, а основным фактором остается определённость прямого объекта.

\pex<exdom19>
\a<a> \begingl
\gla Йу=йи дӯс \b{ас} wи х̌ӯвд бирох̌-т.//
\glc {\sc d3.m.sg=3sg} немного {\sc el} {\sc d3.m.sg.o} молоко пить-{\sc pst}//
\endgl
\a<b> \begingl
\gla \ljudge{*}Йу=йи дӯс \b{ас} х̌ӯвд бирох̌-т.//
\glc {\sc d3.m.sg=3sg} немного {\sc el} молоко пить-{\sc pst}//
\endgl
\a<c> \begingl
\gla \ljudge{*}Йу=йи \b{ас} wи х̌ӯвд бирох̌-т.//
\glc {\sc d3.m.sg=3sg} {\sc el} {\sc d3.m.sg.o} молоко пить-{\sc pst}//
\glft ‘Он выпил (немного) молока.’ \trailingcitation{[элицитация, 2022]}//
\endgl \xe

\section{Заключение}

Среди языков шугнано-рушанской группы дифференцированное маркирование объекта в шугнанском (и в рушанском) встречается наименее регулярно: не существует контекстов, где маркирование прямого объекта при помощи \i{ас} было бы строго обязательным. Кроме того, среди носителей наблюдается высокая вариативность в употреблении объектного показателя. Стоит отдельно выделить контекст, в котором употребление \i{ас} напоминает ДОМ, но, как кажется, не является им: это употребление \i{ас} как бы при объекте сравнения, но само сравнение оказывается «свёрнуто» до определения при прямом объекте (\gethref{exdom6}).

С некоторым классом менее прототипически переходных глаголов, прямой объект которых с точки зрения семантической роли близок к стимулу, объекту сравнения или источнику (‘спрашивать’, ‘прощать’, ‘побеждать’, ‘забывать’ и~так~далее) ДОМ закрепился в качестве альтернативной модели управления. Маркирование объектов при этих глаголах более частотно и не ограничено свойствами именной группы.

Возвращаясь непосредственно к маркированию прямого объекта, можно выделить следующие факторы, влияющие на появление ДОМ:

\begin{enumerate}
  \item Свойства именной группы: с большей вероятностью маркируются определённые именные группы и именные группы с генерическим референциальным статусом; также влияет одушевлённость: маркируется пациенс, который на шкале одушевлённости Сильверстайна находится так же высоко, как и агенс.
  \item Эмфатическое выделение прямого объекта: в контекстах с эмфатической частицей \i{ик}=, а также в контекстах с отрицанием маркирование прямого объекта при помощи \i{ас} более вероятно.
  \item Неразличение падежей участников (например, при именах собственных), при том, что участники занимают близкие позиции на шкале одушевлённости и определённости.
\end{enumerate}

Чаще всего встречаются комбинации сразу нескольких факторов, которые можно попробовать обобщить следующим образом: пациенс оказывается наделён свойствами, типичными скорее для агенса, а синтаксические роли участников сложно понять из контекста. Таким образом дополнительное маркирование при помощи предлога \i{ас} отражает непрототипически высокий статус пациенса, а также позволяет внести ясность в синтаксическую структуру.

Стоит отметить, что для многих индоарийских языков (бенгальский, малаялам, тамильский, марвари, урду) сочетание таких факторов, как одушевлённость, определённость и (в некоторых случаях) контрастный фокус часто оказывается ключевым фактором для появления объектного показателя \parencite[464]{subbarao2016}, так что шугнанский язык соответствует типологическим ожиданиям, однако факультативность и вариативность в выборе оформления прямого объекта сильно затемняет общую картину.

Из-за малого количества данных и большой вариативности по говорящим влияние факторов, описанных в работе, часто не является строгим, а лишь отражает тенденцию. Эта работа является первой попыткой изучить устройство дифференцированного маркирования объекта в шугнанском языке; в будущем я планирую более детально изучить взаимодействие факторов между собой за счёт увеличения объема данных.
