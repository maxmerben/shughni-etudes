\chapter*{Эвиденциальный Перфект в~шугнанском языке}
\addcontentsline{toc}{chapter}{\textit{М.~Меленченко}. \textbf{Эвиденциальный Перфект в~шугнанском языке}}
\setcounter{section}{0}
\chaptermark{Эвиденциальный Перфект в~шугнанском языке}
\label{chapter-melen-evid}

\begin{customauthorname}
Максим Меленченко
\end{customauthorname}

\begin{englishtitle}
\i{Evidential Perfect in Shughni\\{\small Maksim Melenchenko}}
\end{englishtitle}

\begin{abstract}
В статье с семантической точки зрения рассматривается одна из видовременных форм шугнанского языка, традиционно именуемая Перфектом. Эта форма имеет много ареально-типологических сходств с перфектами в других языках Западной и Центральной Азии. В частности, шугнанский развил эвиденциальное противопоставление между Претеритом и Перфектом, в котором Перфект используется для незасвидетельствованных событий. Он также выражает экспериенциальную и ирреальную семантику. Выделяется небольшой класс «стативно-перфектных» глаголов, у которых форма Перфекта выражает состояние в настоящем. Кроме того, в статье анализируются «миративные» употребления глагола ‘быть’ в форме Перфекта, а также употребление перфектных форм в нарративах.
\end{abstract}

\begin{keywords}
время, иранские языки, миративность, памирские языки, перфект, шугнанский язык, эвиденциальность.
\end{keywords}

\begin{eng-abstract}
This paper describes the semantics of one of the verb forms in Shughni, traditionally called Perfect. This form displays many areally-motivated typological similarities with perfects in other languages of Western and Central Asia. In particular, Shughni has developed an evidential opposition between the Preterite and the Perfect, in which the Perfect is used for non-witnessed events. It is also used to express experiential and irreal semantics. A small class of “stative-perfect” verbs is distinguished, for which the Perfect form expresses a state in the present. In addition, the article analyzes so-called “mirative” uses of the Perfect of the copula, as well as the use of Perfect forms in narratives.
\end{eng-abstract}

\begin{eng-keywords}
evidentiality, Iranian, mirativity, Pamir area, perfect, Shughni, tense.
\end{eng-keywords}

\begin{acknowledgements}
Автор выражает глубокую благодарность Сильвии Лураги, С.~К.~Михайлову, В.~А.~Плунгяну и К.~В.~Филатову за ценные советы, комментарии и обсуждения, а также всем шугнанским консультантам, в особенности Мадине Ардабаевой и Амригул Шоназаровой. Публикация подготовлена в ходе проведения исследования (проект №~22-00-034) в рамках Программы «Научный фонд Национального исследовательского университета “Высшая школа экономики” (НИУ~ВШЭ)» в 2023~году.
\end{acknowledgements}

\begin{initialprint}
\fullcite{melenchenko2023_evidential}\end{initialprint}

\section{Введение} \label{evid-intro}

Перфект~— глагольная категория, с трудом поддающаяся единообразной типологической характеристике. Для европейских языков его семантика традиционно описывается в терминах понятий «\textbf{результат}» или «\textbf{текущая релевантность}» \parencite[52–65]{comrie1976}. Кроме того, одним из наиболее частотных значений, выражаемых перфектами, является экспериенциальность~— указание на опыт в прошлом без привязки к конкретному событию \parencite[67]{bybee_dahl1989}. Так, в посвящённой перфектам главе в атласе WALS перфектами считаются только формы, которые имеют и значение результата, и экспериенциальное значение \parencite{dahl_velupillai2013}.

У языков Западной и Центральной Азии, включая многие иранские, тюркские и уральские языки, а также языки Кавказа, одним из наиболее частотных значений перфекта зачастую является \textbf{непрямая эвиденциальность} \parencites[93]{lazard1999}[375]{lindstedt2000}. Этот ареал также известен как Большой эвиденциальный пояс \parencite[14–15]{plungian2016}. Известно, что перфекты часто развиваются из результативных причастий \parencites[68–69]{bybee_etal1994}[367–373]{lindstedt2000}, однако они характеризуются диахронической нестабильностью и поэтому сами легко подвергаются семантическому сдвигу. Они могут заменить обычное прошедшее время~— этот известный феномен называют «аористическим сдвигом»~— или же приобрести те самые эвиденциальные значения и постепенно грамматикализоваться в специализированный показатель эвиденциальности \parencite[95–97]{bybee_etal1994}. По-видимому, в первую очередь перфект начинает использоваться для событий, о которых говорящий узнаёт путём логического вывода («инферентив»), а уже потом может расшириться до событий, о которых говорящий узнаёт от другого человека («репортатив»). По словам С.~Ферхеес, семантический сдвиг в сторону эвиденциальности вызван сменой направления каузальной импликатуры, присущей результативу: если имеется видимый результат, то ему должно соответствовать событие в прошлом \parencite[78]{verhees2019}. Так по результату человек достраивает картину прошлого, которое он не наблюдал,~— это и есть инферентивное значение.

Настоящая статья представляет собой описание семантики Перфекта в шугнанском языке, который принадлежит к шугнано-рушанской подгруппе внутри восточноиранских языков. Шугнанский также включается в памирские языки~— ареальное объединение внутри восточноиранской группы. На нём говорит около 90~000 человек в Таджикистане и Афганистане \parencite[787–788]{edelman_dodykhudoeva2009_shughni}. Как и многие другие иранские языки, шугнанский развил способ выражения эвиденциального статуса событий в прошлом с помощью видовременных форм глагола. В основном форма Перфекта используется для выражения незасвидетельствованных событий, в то время как для выражения засвидетельствованных событий используется Претерит. В статье описываются особенности употребления шугнанского Перфекта. Используемые данные получены преимущественно от носителей шугнанского в г.~Хорог (Таджикистан) и в Москве в 2022–2023~годах, также использовались данные из корпуса шугнанского языка\fn{Корпус доступен по ссылке: \i{\href{https://linghub.ru/shughni_corpus/search}{linghub.ru/shughni\_corpus/search}}. Подробнее о корпусе см.~в работе \parencite{makarov_etal2022}. На момент проведения исследования в корпусе содержалось 72 текста общим объёмом более 55~000 словоформ. Для анализа нарративов использовались 18~нарративов из корпуса~— это сказки и истории из письменных источников (13), устные тексты, записанные в экспедициях, (4) и фрагменты перевода Евангелия от Луки.} и электронной версии \parencite{makarov_etal2022} словаря Д.~К.~Карамшоева [\cite*{karamshoev1988}].

В \hyperref[evid-verb]{разделе~2} описывается глагольная система шугнанского языка. В \hyperref[evid-sem]{разделе 3} детально описана семантика шугнанского Перфекта. Собранный материал показывает, что основным значением Перфекта является эвиденциальность (\hyperref[evid-evid]{подраздел~3.1}). При этом он также имеет экспериенциальные (\hyperref[evid-exper]{подраздел~3.2}) и контрфактические употребления (\hyperref[evid-modal]{подраздел~3.5}), а для небольшого числа глаголов имеет только результативно-стативное значение (\hyperref[evid-result]{подраздел~3.3}). В \hyperref[evid-be]{подразделе~3.4} обсуждаются особые употребления глагола бытия в форме Перфекта, а в \hyperref[evid-narr]{подразделе~3.6}~— использование Перфекта в нарративах. В \hyperref[evid-conclusion]{разделе~4} делаются выводы и обобщения.

\section{Глагольная система шугнанского языка} \label{evid-verb}

Шугнанский язык обладает сравнительно простой видовременной системой, в которой выделяются три основных формы: Презенс (также непрошедшее или настояще-будущее), Претерит (или прошедшее) и Перфект. Есть также нефинитные, неиндикативные, а также образованные от основных формы (Плюсквамперфект, Императив, Инфинитивы и различные причастия). Все эти формы синтетические; аналитические глагольные формы в шугнанском обычно не выделяются\fn{Имеется ряд аналитических конструкций~— например, две конструкции с инцептивным значением \parencite[352]{parker2023}~— но традиционно они не включаются в глагольную парадигму.}. Глаголы имеют три основы, которые различают разные формы~— Презенс, Претерит и Перфект. Основы Претерита и Перфекта образуются от основы Презенса с помощью специальных суффиксов; при этом у многих глаголов все три основы также различаются нерегулярными чередованиями в корне. У многих глаголов в Претерите и Перфекте различаются две основы: для мужского рода единственного числа и для женского рода или множественного числа. Основа Претерита всегда маркируется суффиксом -\i{т}/-\i{д}, а Перфектная основа маркируется суффиксом -\i{ц}/-\i{ӡ} для формы женского рода или множественного числа, а во всех остальных случаях (в мужском роде или когда род и число не различаются) используется суффикс -\i{ч}/-\i{ҷ}. Кроме того, у многих глаголов имеется особая форма для 3 лица единственного числа Презенса. См.~Таблицу \ref{tab:evid1}\fn{Примеры из шугнанского и близкородственного ему бартангского приводятся в рабочей латинской транскрипции, примеры из других языков~— в орфографиях, используемых в соответствующих источниках.}:

\begin{table}[h]
 \centering
 \caption{Основные формы четырёх шугнанских глаголов}
 \smallskip
 \label{tab:evid1}
 \begin{tabular}{cc|cccc} \toprule
 \multicolumn{2}{c|}{форма глагола} & ‘бросить’ & ‘дать’ & ‘быть’ & ‘уйти’ \\ \midrule
 \multirow{2}{*}{\makecell{основа\\Презенса {[}{\sc prs}{]}}} & {\small обычная} & \multirow{2}{*}{\i{питêw}-} & \i{δāδ}- & \multirow{2}{*}{\i{ви}-} & \i{ти}- \\
  & {\sc 3sg} &  & \i{δӣ}- &  & \i{тӣз}- \\ \midrule
 \multirow{2}{*}{\makecell{основа\\Претерита {[}{\sc pst}{]}}} & {\sc m.sg} & \multirow{2}{*}{\i{питêw-д}} & \multirow{2}{*}{\i{δо.д}} & \i{ву.д} & \i{тӯй.д} \\
  & {\sc f.sg/pl} &  &  & \i{ва.д} & \i{той.д} \\ \midrule
 \multirow{2}{*}{\makecell{основа\\Перфекта {[}{\sc pf}{]}}} & {\sc m.sg} & \multirow{2}{*}{\i{питêw-ҷ}} & \multirow{2}{*}{\i{δоδ.ҷ}} & \i{вуδ.ҷ} & \i{тӯй.ҷ} \\
  & {\sc f.sg/pl} &  &  & \i{ви.ц} & \i{тӣ.ц} \\ \bottomrule
 \end{tabular}
\end{table}

В форме Презенса глаголы согласуются с субъектом клаузы с помощью лично-числовых окончаний (\gethref{exevid1})\fn{Здесь и далее примеры без указания источника получены от носителей путем элицитации.}. В Претерите и Перфекте основа используется без окончаний, но глагол согласуется с субъектом с помощью лично-числовых энклитик, которые обычно присоединяются к первой именной группе клаузы\fn{Показатель 3 лица единственного числа =\i{и} используется только при переходных и псевдопереходных глаголах [\hyperref[chapter-chist-clitic]{Чистякова 2022b}].} (\gethref{exevid2}); эти же энклитики используются с именными предикатами (\gethref{exevid24}) \parencite[22–31]{chistiakova2021}. Лично-числовые суффиксы и энклитики перечислены в Таблице \ref{tab:evid2}.

\ex<exevid1>
\begingl
\gla Йā ар руз китоб \b{х̌ойд}.//
\glc {\sc d3.f.sg} каждый день книга читать.{\sc prs.3sg}//
\glft ‘Она \b{читает} книгу каждый день.’//
\endgl \xe

\ex<exevid2>
\begingl
\gla Йā=йи wам китоб \b{х̌êй-ҷ}.//
\glc {\sc d3.f.sg=3sg} {\sc d3.f.sg.o} книга читать-{\sc pf}//
\glft ‘[Я вижу, что закладка лежит в самом конце книги:] Она \b{прочитала} эту книгу.’//
\endgl \xe

\begin{table}[H]
 \centering
 \caption{Шугнанские лично-числовые показатели}
 \smallskip
 \label{tab:evid2}
 \begin{tabular}{l|cccccc} \toprule
  & {\sc 1sg} & {\sc 2sg} & {\sc 3sg} & {\sc 1pl} & {\sc 2pl} & {\sc 3pl} \\ \midrule
 \makecell[c]{лично-числовые\\суффиксы в~Презенсе} & -\i{ум} & -\i{и} & -\i{т} / -\i{д} & -\i{āм} & -\i{ет} & -\i{ен} \\ \midrule
 \makecell[c]{лично-числовые\\энклитики, использующиеся\\с~Претеритом и~Перфектом} & =\i{ум} & =\i{ат} & =\i{и} / $\emptyset$ & =\i{āм} & =\i{ет} & =\i{ен} \\ \bottomrule
 \end{tabular}
\end{table}

От основы Перфекта с помощью суффикса -\i{ат} регулярно образуется форма Плюсквамперфекта, при этом лицо и число выражается так же, как и в Перфекте~— энклитиками. В современном языке Плюсквамперфект не употребляется в прототипическом для этой категории значении предшествования в прошлом и сохранил исключительно модальные употребления (\gethref{exevid3})\fn{О семантике формы Плюсквамперфекта см.~более позднюю работу \parencite{melenchenko2025_pluperfect}~— \i{прим.~переиздания}.}.

\ex<exevid3>
\begingl
\gla Кошга=йāм мāш йак-ҷо зиндаги \b{чӯɣ̌ҷ-ат}!//
\glc если\_бы={\sc 1pl} {\sc pron.1pl} один-место жизнь делать.{\sc pf-pqp}//
\glft ‘Вот бы мы \b{жили} вместе!’//
\endgl \xe

Кроме того, от основы Перфекта образуются два причастия: результативное с суффиксом -\i{ин} (\gethref{exevid4}) и пассивное с суффиксом -\i{ак} (\gethref{exevid5}) \parencite[798–799]{edelman_dodykhudoeva2009_shughni}.

\ex<exevid4>
\begingl
\gla <…> Йу̊д-анд ик=дис \b{нивиш-ч-ин}: <…>.//
\glc ~~~~~~ {\sc d1-loc} {\sc emph}=такой писать-{\sc pf-ptcp1} ~~~~~~//
\glft ‘<И сказал им:> так \b{написано}, <и так надлежало пострадать Христу, и воскреснуть из мёртвых в третий день.>’ \trailingcitation{[Лк. 24:46]}//
\endgl \xe

\ex<exevid5>
\begingl
\gla Wāδ=ен ту-рд \b{δоδҷ-ак} сат.//
\glc {\sc d3.pl=3pl} {\sc pron.2sg-dat} дать.{\sc pf-ptcp2} стать.{\sc pst.f/pl}//
\glft ‘Они [уже] \b{даны} тебе.’ \trailingcitation{\parencite{karamshoev1988}}//
\endgl \xe

\pagebreak[4]

С диахронической точки зрения основы Претерита с суффиксом -\i{т}/-\i{д} восходят к древнеиранским причастиям на *-\i{ta}, *-\i{tī} \parencite[77]{dodykhudoeva1988}. Они превратились в «простое» прошедшее время и, в свою очередь, послужили основами для развития вторичных причастий, которые превратились в современный Перфект. Показатели Перфекта -\i{ч}/-\i{ҷ} и -\i{ц}/-\i{ӡ} происходят от суффиксов *-\i{ka} и *-\i{čī}, которые образовывали эти вторичные причастия \parencites[193–203]{pakhalina1989}[290–291]{jugel2020}. По предположению Эдельман [\cite*[358–372]{edelman1975_tense}], формы, соответствующие современному Претериту, могли функционировать как перфекты~— до тех пор, пока появление вторичных причастий на *-\i{ka} не вытеснило их в сферу «простого» прошедшего. Древнеиранские причастия, вероятно, прошли через «перфектный цикл», в котором процесс развития перфекта и последующий «аористический сдвиг» повторяются несколько раз \parencite[24]{plungian2016}. Таким образом, с точки зрения происхождения шугнанский Перфект весьма типичен~— он образовался из результативного причастия. Вместе с тем, по сравнению со многими другими перфектами Большого эвиденциального пояса он выделяется тем, что не задействует связку (глагол бытия)\fn{Впрочем, как справедливо заметил редактор выпуска [журнала «Вопросы языкознания»~— \i{прим.~переиздания}] О.~И.~Беляев, лично-числовые клитики можно считать связками, хотя этот вопрос требует отдельного рассмотрения.}~— а например, в таких языках, как таджикский, болгарский или багвалинский, перфекты в разной степени её задействуют. Однако исторически, по-видимому, Перфект имел по крайней мере нулевую связку. Это косвенно подтверждается устройством формы Плюсквамперфекта в шугнано-рушанских языках, которая происходит из конструкции Перфекта и глагола ‘быть’ в форме Претерита. Например, в близкородственном бартангском языке Плюсквамперфект сохранился именно в таком виде: \i{аз=ум суч вуд} (\textsc{pron.1sg=1sg} идти.{\textsc{pf.m.sg} быть.{\textsc{pst.m.sg}) ‘я ходил’, однако в шугнанском формы \i{вуд}/\i{вад} превратились в суффикс -\i{ат}: \i{уз=ум суδҷ-ат} ({\textsc{pron.1sg=1sg} идти.{\textsc{pf.m.sg-pqp}) ‘я ходил бы’ \parencite[76]{dodykhudoeva1988}. Таким образом они стали синтетическими формами, не встраивающимися в парадигму относительно Презенса и Претерита.

Семантика видовременных форм шугнанского языка до недавнего времени была описана сравнительно слабо. Существующие грамматики и очерки не отличаются разнообразием мнений и формулировок по этой теме. Известно, что Презенс используется для описания событий в настоящем и будущем (для референции к будущему часто используется энклитика =\i{та} \parencite[337–342]{parker2023}). Две формы, Претерит и Перфект, используются для описания событий в прошлом. Претерит в научных работах обычно описывается как «стандартное» прошедшее, не маркированное ни по аспекту, ни по эвиденциальному статусу \parencites[154]{karamshoev1963}[370]{edelman1975_tense}[806]{edelman_dodykhudoeva2009_shughni} (исключение составляет предстоящая грамматика \parencite[346]{parker2023}\fn{Шугнанская грамматика К.~Паркера \parencite{parker2023} ещё не была опубликована на момент написания этой статьи~— \i{прим.~переиздания}.}). Перфект, с другой стороны, обычно считается формой с более специфической семантикой. Пять источников единогласно приписывают ему два значения: одно традиционное для перфекта («наличный результат», «текущая релевантность»), а другое~— эвиденциальное (иногда подразделяющееся на инферентив и репортатив) \parencites[161–162]{karamshoev1963}[39]{pakhalina1969_pamir}[370]{edelman1975_tense}[42–43]{bakhtibekov1979}[806]{edelman_dodykhudoeva2009_shughni}. Значение характеристик типа «результат» и «текущая релевантность», используемых в этих работах, не вполне ясно. На мой взгляд, вполне вероятно, что на эти описания влияли представления их авторов о том, какие значения перфекты выражают в других языках (в том числе в других иранских). Никакие другие «перфектные» значения, встречающиеся в типологической литературе, в этих источниках не упоминаются.

\section{Семантика шугнанского Перфекта} \label{evid-sem}

\subsection{Эвиденциальность} \label{evid-evid}

Для большинства глаголов Перфект, по-видимому, является в первую очередь «эвиденциальным» прошедшим временем. Претерит используется в контекстах прямой засвидетельствованности (\gethref{exevid6}), в то время как Перфект используется в инферентивных (\gethref{exevid7}) и репортативных (\gethref{exevid8}) контекстах. Важно отметить, что в примерах (\gethref{exevid7}) и (\gethref{exevid8}) использование формы Претерита носители считают некорректным. Это показывает, что Претерит всё же маркирован по эвиденциальному статусу события и выражает только прямой доступ говорящего к информации, а Перфект выражает косвенный доступ говорящего к информации (инферентивный или репортативный).

\ex<exevid6>
\begingl
\gla Уз=ум тар мактаб тоқ \#вирух̌-ч / \b{вирух̌-т}, <…>.//
\glc {\sc pron.1sg=1sg} {\sc eq} школа окно ломать-{\sc pf} ~~~~~~ ломать-{\sc pst} ~~~~~~//
\glft ‘Я \b{разбил} окно в спортзале, <и теперь моих родителей вызывают в школу>.’//
\endgl \xe

\ex<exevid7>
\begingl
\gla Ар.чāй.ца аз дев талабā-йен=ен тоқ \b{вирух̌-ч} / \#вирух̌-т.//
\glc каждый {\sc el} {\sc d2.pl.o} ученик-{\sc pl=3pl} окно ломать-{\sc pf} ~~~~~~ ломать-{\sc pst}//
\glft ‘[Учительница слышит звук, заходит в спортзал, видит разбитое стекло и идёт в кабинет директора, чтобы сообщить:] Один из этих учеников \b{разбил} окно.’//
\endgl \xe

\ex<exevid8>
\begingl
\gla Йу=йи тар мактаб хейх̌ā \b{вирух̌-ч} / \#вирух̌-т.//
\glc {\sc d3.m.sg=3sg} {\sc eq} школа стекло ломать-{\sc pf} ~~~~~~ ломать-{\sc pst}//
\glft ‘[Учительница звонит матери подростка и сообщает, что её сын разбил окно в школе. Положив трубку, мать говорит своему мужу:] Он \b{разбил} окно в школе.’//
\endgl \xe

По-видимому, Претерит и Перфект противопоставляют по эвиденциальному статусу не только предельные (\gethref{exevid7})–(\gethref{exevid8}), но и непредельные глаголы, что показывают примеры (\gethref{exevid9}) и (\gethref{exevid10}):

\ex<exevid9>
\begingl
\gla Уз=ум то вегā=йец тар дарго \b{ред}.//
\glc {\sc pron.1sg=1sg} {\sc lim1} вечер={\sc lim2} {\sc eq} двор остаться.{\sc pst}//
\glft ‘[Я забыла ключи от дома.] Я \b{простояла} у дома до вечера.’//
\endgl \xe

\ex<exevid10>
\begingl
\gla Йам δу соат тар дарго \b{реδҷ}.//
\glc {\sc d1.sg} два час {\sc eq} двор остаться.{\sc pf}//
\glft ‘[Он забыл ключи от дома. Как он рассказал мне,] он \b{простоял} у дома два часа.’//
\endgl \xe


Некоторые носители шугнанского сами предлагали объяснения разницы в семантике форм Претерита и Перфекта. Многие отвечали, что форма Перфекта используется в случаях, когда говорящий не был свидетелем события. В некоторых случаях предлагались другие ситуативные объяснения: лицо субъекта, уверенность/сомнение говорящего в произошедшем и удалённость события во времени. Однако все эти эффекты на самом деле являются лишь коррелятами эвиденциальности, а не самостоятельными факторами, влияющими на употребление форм. Например, объяснение через лицо субъекта связано с известным «эффектом первого лица», который появляется у эвиденциальных показателей в разных языках: в сочетании с субъектами первого лица такие формы дают значение неосознанности действия \parencite[220–223]{aikhenvald2004}. Он отмечен, например, у другого памирского языка~— сарыкольского \parencite[323]{kim2017}. Так же и в шугнанском форма Перфекта с субъектами первого лица интерпретируется обычно как действие, о совершении которого говорящий не помнит и узнаёт только в момент речи (\gethref{exevid11}). Один из носителей также предположил, что с Перфектом предложение (\gethref{exevid6}) будет означать ‘Я якобы разбил окно’ (см.~такие же употребления эвиденциального Перфекта в турецком \parencite[101]{lewis2000}). На самом деле лицо субъекта не влияет на использование Претерита и Перфекта~— обе формы возможны со всеми лицами.

\ex<exevid11>
\begingl
\gla Уз=ум бийор нош δу̊нҷ-ен \b{вирух̌-ч}…//
\glc {\sc pron.1sg=1sg} вчера абрикос семя-{\sc pl} ломать-{\sc pf}//
\glft ‘[Старик говорит:~— Оказывается,] я \b{размельчил} абрикосовые косточки ещё вчера… [Но я не помню этого!]’//
\endgl \xe

Несколько раз носители предлагали другие объяснения различия семантики двух форм: например, Перфект якобы подчёркивает неуверенность говорящего в том, что событие имело место. Проблема взаимоотношения эвиденциальности и эпистемической модальности (выражения сомнения) давно является предметом дискуссий. Так, например, В.~Фридман предлагал считать «неконфирмативность», то есть сомнение, основным значением эвиденциальных перфектов в языках Балкан и Кавказа [\cite{friedman1979}; \cite*{friedman2003}]. Вполне естественно, что незасвидетельствованные события вызывают большее сомнение, так как прямое наблюдение или участие являются более надёжными источниками информации, но это не означает, что сомнение является одним из значений категории \parencite[3]{arslan2020}. Для шугнанского Перфекта не удалось найти контексты, которые можно было бы объяснить эпистемической модальностью и нельзя~— эвиденциальностью. Вместе с тем, в некоторых контекстах носители предлагали интерпретации, которые необязательно коррелируют с эвиденциальностью напрямую. Обсуждая пример (\gethref{exevid8}), одна консультантка пояснила, что Претерит может использоваться, если мать абсолютно уверена в том, что сын разбил окно, или ожидала, что это произойдёт\fn{Похожие употребления есть в турецком, где прямая эвиденциальная форма может быть использована в контекстах незасвидетельствованности, чтобы подчеркнуть «ожидаемые новости» \parencite[13]{arslan2020}.}~— впрочем, не все носители поддерживают такие суждения. По-видимому, эпистемическая модальность может быть дискурсивным эффектом шугнанского Перфекта.

Третий альтернативный критерий, упомянутый носителями,~— удалённость события во времени. Некоторые утверждали, что Перфект используется для более давних событий. Это объясняется экспериенциальными контекстами, которые выражают опыт, имевший место в неопределённом прошлом, а не конкретное событие (см.~\hyperref[evid-exper]{подраздел~3.2}}), а также контекстами, в которых удалённость во времени коррелирует с эвиденциальностью (то, что было раньше, совпадает с тем, что говорящий не видел). Например, при обсуждении предложения (\gethref{exevid12}) двое носителей предложили такое объяснение: форма Перфекта \i{тӣц} использовалась бы, если бы подруга ушла ещё до того, как пришёл говорящий. Очевидно, такой контекст отличается от изначального стимула в примере (\gethref{exevid12}) не только удалённостью во времени, но и эвиденциальным статусом: говорящий не видел, как она уходила, так как пришёл позже.

\ex<exevid12>
\begingl
\gla Йā аз мāш хез=анд \b{тойд} / \#тӣц, <…>//
\glc {\sc d3.f.sg} {\sc el} {\sc pron.1pl} {\sc apud=loc} идти.{\sc pst.f/pl} ~~~~~~ идти.{\sc pf.f/pl} ~//
\glft ‘[Я сижу и разговариваю с друзьями. Одна из них неожиданно уходит. Я говорю:] Она от нас \b{ушла}, <даже не попрощавшись>.’//
\endgl \xe

\subsection{Экспериенциальность} \label{evid-exper}

Перфект также используется в экспериенциальных контекстах (\gethref{exevid13})–(\gethref{exevid14}), то есть выражает наличие или отсутствие опыта в прошлом без привязки к конкретному событию. В примере (\gethref{exevid13}) один из носителей отметил, что форма Претерита использовалась бы в том случае, «если кто-то дал мне Библию и через некоторое время спросил, прочёл ли я её». В экспериенциальных контекстах эвиденциальный статус события, по-видимому, не влияет на выбор формы\fn{Непонятно, могут ли вообще экспериенциальные употребления различаться по эвиденциальности, что также было подчеркнуто одним из редакторов выпуска [журнала «Вопросы языкознания»~— \i{прим.~переиздания}].}.

\ex<exevid13>
\begingl
\gla Уз=ум Бӣблийā \#х̌êй-д / \b{х̌êй-ҷ}.//
\glc {\sc pron.1sg=1sg} Библия читать-{\sc pst} ~~~~~~ читать-{\sc pf}//
\glft ‘[Ты когда-нибудь читал(а) Библию?~— Да.] Я \b{читал(а)} Библию.’//
\endgl \xe

\ex<exevid14>
\begingl
\gla Уз=ум ачаθ вино \b{на-бирох̌-ч}.//
\glc {\sc pron.1sg=1sg} совсем вино {\sc neg}-пить-{\sc pf}//
\glft ‘Я никогда \b{не пил(а)} вина.’//
\endgl \xe

Экспериенциальное значение широко распространено у перфектов в разных языках, но оно совсем не обязательно выражается перфектом. К примеру, в родственном сарыкольском языке экспериенциальное значение выражается конструкцией со вторичным причастием на -\i{енҷ} \parencite[91]{palmer2016}, которое соответствует шугнанскому результативному причастию на -\i{ин} \parencite[371]{edelman1975_tense}.

\subsection{Результативность и стативность} \label{evid-result}

Для многих языков с «эвиденциальными перфектами» исследователи отдельно выделяют результативное значение, однако смысл этого ярлыка и распределение между эвиденциальной и традиционной не-эвиденциальной интерпретациями зачастую неясны. В некотором смысле результативность можно найти и в шугнанском Перфекте. У нескольких шугнанских глаголов имеется два отличительных свойства: форма Перфекта (а)~даёт интерпретацию состояния в настоящем времени и при этом (б)~не маркирована по эвиденциальному статусу. На настоящий момент таких глаголов известно четыре: \i{х̌офц}:\i{х̌овд} ‘лежать, спать, ложиться’, \i{ниθ}:\i{нӯст} ‘сидеть, ждать, садиться’, \i{wирāфц}:\i{wирӯвд} ‘стоять, вставать’ и \i{пиниӡ}:\i{пинӯйд} ‘носить, надевать’\fn{Данное предложение со словом \i{ғал} ‘ещё’ получено как перевод стимула, в котором русского слова «ещё» не было.}:

\ex<exevid15>
\begingl
\gla Уз=ум астанофка=йанд ғал \b{нӣсц}.//
\glc {\sc pron.1sg=1sg} остановка={\sc loc} ещё сидеть.{\sc pf.f/pl}//
\glft ‘[Ты где?] Я \b{сижу} на остановке.’//
\endgl \xe

Существование отдельного класса глаголов со значением состояния отмечено во многих языках с эвиденциальным перфектом. Такие глаголы известны в других памирских языках~— как минимум в рушанском \parencite[112]{fayzov1966}, сарыкольском \parencite[92]{palmer2016} и язгулямском \parencite[57]{edelman1966}, а также, например, в таджикском \parencite[221–223]{perry2005} и прочих иранских \parencite[292–293]{jugel2020}, как и, например, в языках Дагестана \parencites[450]{tatevosov2001}[70]{verhees2019} и многих других. Несмотря на то, что такие употребления часто называют «результативными», я предлагаю называть их скорее «стативными», так как они, видимо, могут обозначать и состояния, не являющиеся результатом действия (в духе «статального перфекта» Ю.~С.~Маслова [\cite*[42]{maslov1983}]). Форма Претерита у таких глаголов может выражать не только значение состояния в прошедшем, но и инцептивное значение (‘пошла спать’, ‘надела’, ‘села’).

Интересно взаимодействие «результативного» и «эвиденциального» значений: например, в багвалинском существуют глаголы, для которых возможна только одна из двух интерпретаций, а есть такие, для которых возможны обе \parencite[355–357]{tatevosov2007}. В шугнанском проблема наличия эвиденциального значения у «стативно-перфектных» глаголов требует дальнейшего изучения. По умолчанию носители переводят Перфектную форму таких глаголов со стативным значением и референцией к настоящему времени. При этом при просьбе перевести на шугнанский контекст в прошедшем времени с непрямым эвиденциальным статусом наблюдается вариативность в выборе формы. Некоторые носители разрешают, а другие запрещают употребление Претерита и Перфекта в таких случаях. Однозначно допустимой в этой ситуации считается конструкция с результативным причастием и формой Перфекта глагола ‘быть’:

\ex<exevid16>
\begingl
\gla Ар.чāй.ца мам стӯл-ак=ти \b{нӯсч-ин} \b{вуδҷ} / ?нӯсч / ?нӯст.//
\glc каждый {\sc d1.f.sg.o} стул-{\sc dim=sup} сидеть.{\sc pf.m.sg-ptcp1} быть.{\sc pf.m.sg} ~~~~~~ сидеть.{\sc pf.m.sg} ~~~~~~ сидеть.{\sc pst.m.sg}//
\glft ‘[Когда я уходил, стул был очень пыльный, но сейчас на нём нет ни пылинки. Я говорю:] Кто-то \b{сидел} на этом стуле [пока меня не было].’//
\endgl \xe

Эти «стативные Перфекты» не просто реферируют к моменту речи, они практически грамматикализовались как форма настоящего времени для соответствующих глагольных лексем. В то время как у остальных глаголов форма Презенса может выражать актуальное событие, хабитуалис или событие в будущем, у «стативно-перфектных» глаголов Перфект вытеснил Презенс из значения актуального события (\gethref{exevid17}), оставив ему только хабитуальную (\gethref{exevid18}) и футуральную (\getfullhref{exevid19.b}) семантику. Например, хотя на стимул из примера (\gethref{exevid19}) некоторые носители выдавали форму Презенса, затем оказывалось, что на самом деле такая форма будет интерпретироваться с референцией к ближайшему будущему и, соответственно, такие предложения обычно считали неграмматичными без энклитики =\i{та}, которая обычно маркирует будущее время (\getfullhref{exevid19.b}). В этом смысле глагол \i{х̌офц}:\i{х̌овд} ‘лежать, спать, ложиться’ отличается от других трёх вариативностью: некоторые опрошенные носители разрешали актуальную интерпретацию в Презенсе, другие запрещали. Для других глаголов все консультанты единогласно выбирали Перфект.

\ex<exevid17>
\begingl
\gla Ку мā-раф, уз=ум \b{х̌êвӡ}! ~~~~~~~~~~~~~~~~~~~~~~~~~~~~~~~~~~~~~~~ / ?уз хофц-ум!//
\glc {\sc ptcl} {\sc proh}-трогать {\sc pron.1sg=1sg} спать.{\sc pf.f/pl} ~ ~~~~~~ {\sc pron.1sg} спать-{\sc prs.1sg}//
\glft ‘Не мешай мне, я \b{сплю}!’//
\endgl \xe

\ex<exevid18>
\begingl
\gla Ту хурд лап башāнд \b{х̌офц-и} х̌āб.//
\glc {\sc pron.2sg} оказывается очень хорошо спать-{\sc prs.2sg} ночь//
\glft ‘Ты, оказывается, очень крепко \b{спишь} по ночам.’//
\endgl \xe

\pex<exevid19>
\a<a> \begingl
\gla Уз=ум мам астанофка=йанд \b{wирӣвӡ}.//
\glc {\sc pron.1sg=1sg} {\sc d1.f.sg.o} остановка={\sc loc} стоять.{\sc pf.f/pl}//
\glft ‘[Я ищу тебя, ты где?] Я \b{стою} на остановке.’//
\endgl
\a<b> \begingl
\gla Уз *(=та) мам астанофка=йанд \b{wирāфц-ум}.//
\glc {\sc pron.1sg} ={\sc fut} {\sc d1.f.sg.o} остановка={\sc loc} стоять-{\sc prs.1sg}//
\glft ‘[Я ищу тебя, ты где?] Я (буду) на остановке [~= ты найдёшь меня там].’//
\endgl \xe

В шугнанском языке можно также найти лексикализованные стативные Перфекты. Ярким примером является глагол \i{жӣwҷ} ‘любить, хотеть’. Он описан в словаре Карамшоева [\cite*{karamshoev1988}] как «недостаточный»: указывается, что он утратил все формы, кроме Перфекта (\getfullhref{exevid20.a}), который обозначает состояние в настоящем времени. Состояние в прошедшем, как пишет Карамшоев, выражается формой Плюсквамперфекта \i{жӣwҷат}. В современном хорогском шугнанском, однако, форма \i{жӣwҷат} вышла из употребления, а прошедшее время выражается сложным глаголом \i{жӣwҷ кин}:\i{чӯд} ‘делать’\fn{Свойства недостаточного глагола \i{жӣwҷ} были подробно изучены в недавней статье \parencite{melenchenko2024_love}~— \i{прим.~переиздания}.} (\getfullhref{exevid20.b}).

\pex<exevid20>
\a<a> \begingl
\gla Уз=ум Саӣдā \b{жӣwҷ}.//
\glc {\sc pron.1sg=1sg} Саида любить//
\glft ‘Я \b{люблю} Саиду.’//
\endgl
\a<b> \begingl
\gla Дойим=ум уз Саӣдā лап \b{жӣwҷ} \b{чӯд}=ат, шич нāй.//
\glc раньше={\sc 1sg} {\sc pron.1sg} Саида очень любить делать.{\sc pst=and2} сейчас нет//
\glft ‘Раньше я очень \b{любил} Саиду, а теперь нет.’//
\endgl \xe

В.~П.~Недялков и С.~Е.~Яхонтов [\cite*[12]{nedialkov_yakhontov1983}] предложили в качестве одного из критериев разграничения граммем результатива и перфекта тест на сочетаемость с наречиями со значением ‘всё ещё’ / ‘ещё не’. Предполагается, что в языке, где есть и результатив, и перфект, результатив будет сочетаться с таким наречием, а перфект нет. В шугнанском языке функцию такого наречия выполняет многофункциональная частица \i{ғал}, которая свободно сочетается как с Перфектом (\gethref{exevid21}), так и с результативным причастием. Таким образом, этот тест для шугнанского не применим. Более того, частица \i{ғал}, по-видимому, вообще очень часто употребляется с Перфектом (особенно экспериенциальным или стативным). Её значение и её связь с семантикой Перфекта ещё только предстоит изучить. В примере (\gethref{exevid21}) один из носителей разрешил употребление и Претерита, и Перфекта, но употребление Перфекта оказывалось невозможным без \i{ғал} (с лимитативной энклитикой =\i{ец}, особенности употребления которого требуют отдельного исследования):

\ex<exevid21>
\begingl
\gla Му пуц *(ғал=ец) мис ғулā \b{на-суδҷ}.//
\glc {\sc pron.1sg.o} сын ещё={\sc lim2} уже большой {\sc neg}-стать.{\sc pf.m.sg}//
\glft ‘Мой сын так и \b{не стал} взрослым. [Он не умеет брать на себя ответственность.]’//
\endgl \xe

\subsection{Форма Перфекта глагола ‘быть’} \label{evid-be}

Форма Перфекта одного шугнанского глагола отличается от других по ряду семантических свойств~— это глагол \i{ви}:\i{вуд} ‘быть’. Его формы Перфекта \i{вуδҷ}/\i{виц} могут иметь значение непрямой эвиденциальности и миративности и реферировать к настоящему времени. Это свойство глагола \i{ви}:\i{вуд} отмечено в соответствующей статье словаря Карамшоева [\cite*{karamshoev1988}] и сохраняется в современном языке. Это явление, по-видимому, не упоминается в существующих описаниях шугнанского, но отмечено для близкородственного бартангского языка в грамматике \parencite[170]{karamkhudoev1973}. Пример из этой грамматики, использующий глагол бытия в форме Перфекта (\gethref{exevid22}), современные носители шугнанского переводят с аналогичной формой (\gethref{exevid23}).

\ex<exevid22> 
\begingl
\gla Тā ғулā вирод башāнд одам \b{вуҷ} <…>.//
\glc {\sc pron.2sg.o} большой брат хороший человек быть.{\sc pf.m.sg} ~//
\glft ‘[\b{Оказывается},] твой старший брат хороший человек, <а ты плохой>.’ \trailingcitation{бартангский \parencite[170]{karamkhudoev1973}}//
\endgl \xe

\ex<exevid23>
\begingl
\gla Ту ғулā вирод башāнд \b{вуδҷ}, ~~~~~~~~~~~~~~~~~~~~~~~~~~~~~~ ту=т гандā.//
\glc {\sc pron.2sg} большой брат хороший быть.{\sc pf.m.sg} ~ {\sc pron.2sg=2sg} плохой//
\glft ‘[\b{Оказывается},] твой старший брат хороший, а ты плохой.’ \trailingcitation{шугнанский (элицитация, стимул взят из \parencite[170]{karamkhudoev1973})}//
\endgl \xe

Существенно, что в примерах (\gethref{exevid22}) и (\gethref{exevid23}) инферентивная семантика вводного слова ‘оказывается’ не имеет лексического выражения и выражается исключительно Перфектной формой глагола \i{ви}:\i{вуд} ‘быть’. При опущении этого глагола примеры (\gethref{exevid22})–(\gethref{exevid23}) означали бы простую констатацию факта. Такая ситуация невозможна с другими глаголами. К примеру, при переводе похожего стимула (\gethref{exevid24}), который требует глагола \i{фāм}:\i{фāмт} ‘знать, понимать’, для выражения инферентивности требуется частица \i{хурд} ‘оказывается’, а глагол стоит в Презенсе, а не в Перфекте:

\ex<exevid24>
\begingl
\gla Ту хурд лап ар чӣз=аθ \b{фāм-и}!//
\glc {\sc pron.2sg} оказывается очень каждый что={\sc int} знать-{\sc prs.2sg}//
\glft ‘Оказывается, ты очень много всего \b{знаешь}!’//
\endgl \xe

Вместе с тем, Перфектная форма \i{вуδҷ}/\i{виц} может употребляться и как обычный глагол~— в контекстах с непрямым эвиденциальным статусом и с референцией к прошлому. Это показывают различные контексты, в которых состояние явным образом прекратилось до момента речи, как в примере (\gethref{exevid25}), где говорящий рассказывает про своих родственников\fn{В этом примере использован пример из корпуса. Названия текстов из шугнанского корпуса приводятся в соответствии с именованиями в корпусе.}:

\ex<exevid25>
\begingl
\gla Боб Шофтур муалим \b{вуδҷ}.//
\glc дед Шофтур учитель быть.{\sc pf.m.sg}//
\glft ‘Дед Шофтур \b{был} учителем.’ \trailingcitation{[текст \i{Old parties}, 62]}//
\endgl \xe

Важно, что употребления глагола бытия типа (\gethref{exevid23}) не просто реферируют к настоящему времени, они могут отсылать к событиям, которые говорящий непосредственно наблюдает,~— на первый взгляд, их уже нельзя назвать эвиденциальными. Для анализа таких употреблений часто используют понятие миративности или адмиративности\fn{См.~обсуждение понятий ‘миративность’ и ‘адмиративность’ в \parencite[192]{friedman2003}.}. \b{Миративность}~— грамматическая категория, выражающая удивление говорящего от сообщаемого им высказывания, новизну и неожиданность информации для самого говорящего \parencite{delancey2001}. В описании Перфекта в багвалинском, где у глагола бытия имеется такое же свойство, С.~Г.~Татевосов [\cite*[380]{tatevosov2007}] отмечает, что “большинство примеров на адмиративное значение, которые обычно приводятся в литературе,~— это адмиратив в контексте глагола в настоящем времени и/или стативного глагола, в особенности глаголов ‘быть’ или ‘иметь’”~— это контексты типа (\gethref{exevid23}). При этом существование миративности как отдельной категории, отличной от эвиденциальности, является дискуссионным вопросом (см.~обсуждение в \parencites{lazard1999}{hill2012}{delancey2012}).

Некоторые формы, которые описывались в литературе как эвиденциальные, могут использоваться в контекстах, которые как будто предполагают прямое наблюдение говорящего~— что и привело к предложению выделить категорию миративности. К примеру, турецкий Перфект может быть употреблён в предложении типа (\gethref{exevid26}). Считается, что такие контексты уже не имеют непрямого эвиденциального статуса, поэтому для их анализа часто прибегают к таким концепциям, как «[ад]миративность» или «медиативность» (см.~\parencite{lazard1999}). Эти концепции предполагают, что адмиративная/медиативная форма указывает не на тип источника информации, а подчёркивает само наличие такого источника (медиатора): наблюдение, инференция или рассказ с чужих слов.

\ex<exevid26>
\begingl
\gla Kız-ınız çok iyi piyano \b{çal-ıyor-muş}.//
\glc дочь-{\sc 2sg} очень хорошо фортепиано играть-{\sc prs-pf}//
\glft ‘[Как я вижу,] ваша дочь отлично \b{играет} на фортепиано!’ (произнесено после того, как говорящий наблюдал за её игрой) \trailingcitation{турецкий \parencite[197]{slobin_aksu1982}}//
\endgl \xe

Употребления глагола ‘быть’ в Перфекте для выражения миративности в настоящем засвидетельствованы в других памирских языках~— в частности, в сарыкольском (\gethref{exevid27}) и ваханском (\gethref{exevid28})\fn{В ваханском примере сохранены оригинальные глоссы и транскрипция~— \i{прим.~переиздания}.}.

\ex<exevid27>
\begingl
\gla Туҷик халг-хêйл=аф ыч быланд \b{веδҷ}.//
\glc таджик человек-{\sc pl.nom=3pl.pfv} очень высокий быть.{\sc pf}//
\glft ‘Таджики [\b{оказывается}] очень высокие!’ \trailingcitation{сарыкольский \parencite[94]{palmer2016}}//
\endgl \xe

\ex<exevid28>
\begingl
\gla Yem=i trešp cuan \b{tuetk}.//
\glc этот={\sc 3sg} кислый абрикос быть.{\sc pf}//
\glft ‘Этот абрикос [\b{оказывается}] кислый.’ \trailingcitation{ваханский \parencite[8]{bashir2006}}//
\endgl \xe

В сарыкольском аналогичная форма развила и другие функции. Она используется в специальной аналитической конструкции для выражения эвиденциальности у незавершённых событий и состояний, которая по сути стала имперфективным аналогом эвиденциального Перфекта \parencite[94–95]{palmer2016}. Кроме того, в сарыкольском и ваханском языках форма перфекта глагола ‘быть’ может присоединяться к концу клаузы, в которой уже есть форма Перфекта другого глагола. \parencite[323]{kim2017} сообщает для сарыкольского, что вспомогательный глагол ‘быть’ в таких случаях опционален. Э.~Башир [\cite*[7–8]{bashir2006}] сообщает, что в ваханском добавление ‘быть’ к другой Перфектной форме даёт не просто эвиденциальное, а миративное значение (впрочем, из переводов примеров не вполне ясно, что она имеет в виду под миративностью). Эту особенность ваханский, вероятно, перенял у своих южных соседей~— дардских языков, таких как калашский и кховар, в которых эвиденциальность и миративность выражаются добавлением особой формы вспомогательного глагола к клаузе [\cite{bashir2006}; \cite*{bashir2010}].

Вообще же употребление Перфекта вспомогательных глаголов с референцией к настоящему времени имеет место и в других языках с эвиденциальными Перфектами. К примеру, в таджикском глаголы \i{будан} ‘быть’ и \i{доштан} ‘иметь’ могут выражать события, происходящие в настоящем \parencite[87–90]{nilsson2022} (\gethref{exevid29}).

\ex<exevid29>
\begingl
\gla Пул=ам \b{на-буда=й}.//
\glc деньги={\sc 1sg} {\sc neg}-быть.{\sc pf=cop.3sg}//
\glft ‘[Ой,] у меня \b{нет} денег.’ \trailingcitation{варзобский таджикский \parencite[235]{perry2000}}//
\endgl \xe

\pagebreak[4]

Я предлагаю рассмотреть несколько иную интерпретацию контекстов типа (\gethref{exevid23}) и (\gethref{exevid26})–(\gethref{exevid29}), в которых употребление перфектов традиционно объясняются «миративностью» или «медиативностью». Такие контексты сообщают о ситуации, которая не просто имеет место в настоящем, но протяжена на некоторую дистанцию в прошлое. Предложения (\gethref{exevid23}), (\gethref{exevid26})–(\gethref{exevid28}) сообщают о длительных состояниях-свойствах (‘быть плохим’, ‘хорошо играть на фортепиано’ и так далее), которые, очевидно, имели место и в прошлом. Например, при анализе высказывания (\gethref{exevid26}) обычно считается, что речь идёт о наблюдаемом в момент речи действии (‘играет на фортепиано’), но на самом деле говорящий сообщает о хабитуальном состоянии (‘умеет играть’), о котором он узнаёт, делая вывод на основании текущего наблюдения. В предложении (\gethref{exevid29}) состояние более кратковременное, но оно также начинается в прошлом~— например, в момент, когда говорящий, выходя из дома, должен был взять кошелёк, но забыл это сделать. Таким образом, употребления предикатов состояния в Перфекте с референцией к настоящему времени могут объясняться инферентивным значением. Шугнанские данные подтверждают корректность такого анализа:

\ex<exevid30>
\begingl
\gla Ту=т аҷаб зӣрд (*вуδҷ)!//
\glc {\sc pron.2sg=2sg} {\sc ptcl} яркий быть.{\sc pf.m.sg}//
\glft [Муж был брюнетом, но вдруг покрасился в рыжий цвет, не сообщив об этом жене. Увидев его, она с удивлением восклицает:] ‘Какой ты рыжий!’//
\endgl \xe

Пример (\gethref{exevid30}) сконструирован так, чтобы описываемое состояние было очевидно новым~— здесь жена знает, что раньше у мужа был другой цвет волос; событие не протяжено во времени в прошлое. Использование глагола бытия в форме Перфекта в таком случае запрещается. Интересно, что и русское выражение ‘оказывается’ с трудом получается вставить в такой контекст. Как и шугнанский Перфект, оно отсылает не просто к новой для говорящего информации, а к информации, которая является новостью для говорящего, но сама по себе не является новой.

Такой подход позволяет не применять понятие «миратив» по отношению к рассмотренным употреблениям шугнанского глагола бытия, а вместо этого анализировать их как частный случай инферентивного значения. Вместе с тем, форма Перфекта глагола ‘быть’ может быть на пути грамматикализации в полноценный миративный показатель. В ваханском и сарыкольском аналогичная форма, по-видимому, грамматикализовалась в большей степени. На мой взгляд, анализ миративности как эвиденциальности, «протяжённой» в настоящее, может быть полезным и для других языков, в которых форма Перфекта от ‘быть’ или других глаголов состояния ведёт себя похожим образом\fn{Миративность в шугнанском и других языках стала предметом обсуждения более поздней работы \parencite{melenchenko2024_mirativity}~— \i{прим.~переиздания}.}.

\subsection{Модальные употребления Перфекта} \label{evid-modal}

Шугнанский Перфект может иметь модальную семантику. В частности, он часто употребляется в обеих частях условных предложений с контрфактическим условием (\gethref{exevid31}). Возможно употребление Перфекта только в одной из клауз (или в условной, или в матричной), но такие случаи требуют отдельного изучения. Также Перфект может употребляться в значении пожелания (\gethref{exevid32}) \parencite[813–814]{edelman_dodykhudoeva2009_shughni}. Карамшоев [\cite*[162]{karamshoev1963}] упоминает, что иногда Перфект может реферировать к событиям в настоящем и будущем, но примеры, которые он приводит, достаточно неоднородны и, по-видимому, объясняются разными факторами. Вероятно, многие такие употребления стоит отнести к модальным (\gethref{exevid33}).

\ex<exevid31>
\begingl
\gla Му-нд шич-ард лап су̊м ца \b{вуδҷ}, ~~~~~~~~~~~ уз=ум ху-рд мошӣн \b{зох̌-ч}.//
\glc {\sc pron.1sg.o-loc} сейчас-{\sc dat} очень деньги {\sc subd} быть.{\sc pf.m.sg} ~ {\sc pron.1sg=1sg} {\sc refl-dat} машина взять-{\sc pf}//
\glft ‘Если бы у меня \b{было} много денег [сейчас], я бы \b{купил} себе машину.’//
\endgl \xe

\ex<exevid32>
\begingl
\gla Ту даδ дис.на маркāб-ен \b{зох̌-ч}=ху, ~~~~~~~~~~~~~~~~~~~~~~~~~~~~~~~~~ wи ках̌т \b{тӣж-ҷ}.//
\glc {\sc pron.2sg} лучше бы осёл-{\sc pl} взять-{\sc pf=and1} ~ {\sc d3.m.sg.o} зерно тянуть-{\sc pf}//
\glft ‘Ты \b{взял} бы ослов да и \b{привёз} бы зерно.’ \trailingcitation{\parencite[162]{karamshoev1963}}//
\endgl \xe

\ex<exevid33>
\begingl
\gla <…> дигā=м ту-рд \b{δоδҷ}.//
\glc ~~~~~~ другой={\sc 1sg} {\sc pron.2sg-dat} дать.{\sc pf}//
\glft ‘<Когда разделим, я возьму себе лишь одно, а> остальное \b{отдам} тебе.’ \trailingcitation{\parencite[162]{karamshoev1963}}//
\endgl \xe

\subsection{Использование Перфекта в нарративах} \label{evid-narr}

Традиционно считается, что перфекты не могут использоваться как нарративное время, то есть для обозначения повествования из последовательных событий \parencites[138]{dahl1985}[366]{lindstedt2000}. Вместо этого в нарративах им обычно отводится роль «фоновых» употреблений: они обозначают события, происходящие вне основного сюжета~— например, до его начала \parencite[62]{bybee_etal1994}. Однако для языков с эвиденциальными перфектами это ограничение, по-видимому, не так строго: выбор нарративного времени в них может зависеть от эвиденциального статуса рассказа. Так, рассказы о давних исторических событиях, а также сказки или анекдоты в некоторых языках могут использовать перфекты как основную глагольную форму для повествования \parencites[151–152]{dahl1985}[99]{lazard1999}.

Детальное исследование выбора видовременных форм в шугнанских нарративах ещё только предстоит, но уже можно сделать некоторые предварительные выводы на основании анализа текстов из корпуса, который включает в себя в основном фольклорные тексты и фрагменты перевода Евангелия от Луки. Основным нарративным временем является Претерит, а Перфект, как и ожидается, часто имеет «фоновые» функции. Также как нарративное время может использоваться Презенс~— по-видимому, только в незасвидетельствованных историях. Похожая стратегия выбора нарративного времени, в которой Презенс является «эвиденциальной» повествовательной формой, описана для ваханского \parencite[53–63]{obrtelova2019_text}. Выбор Перфекта для каких-либо событий в нарративе обычно определяется эвиденциальностью и её коррелятами (удалённостью во времени или новизной события). Дейктический центр обычно совпадает с протагонистами~— время и эвиденциальный статус определяются с их точки зрения относительно текущего момента в сюжете. Встречаются и не-эвиденциальные употребления Перфекта~— экспериенциальные и стативные:

\ex<exevid34>
\begingl
\gla <…> Wи Мāбад=анд вирӯд. Йу оху̊н-ен дарӯн \b{нӯсч}=атā, ниғу̊ɣ̌-д wев=ат пех̌с-т.//
\glc ~~~~~~ {\sc d3.m.sg.o} храм={\sc loc} найти.{\sc pst} {\sc d3.m.sg} учитель-{\sc pl} внутри сидеть.{\sc pf.m.sg=and3} слушать-{\sc prs.3sg} {\sc d3.pl.o=and2} спросить-{\sc prs.3sg}//
\glft ‘<Через три дня> нашли Его в храме, сидящего посреди учителей, слушающего их и спрашивающего их’ (буквально: ‘Его в храме нашли. Он среди учителей \b{сидит} и слушает их, и спрашивает.’) \trailingcitation{[Лк. 2:46; \cite{dodixudoev2001}]}//
\endgl \xe

Перфект может использоваться в интродуктивной части нарратива, чтобы описать положение до начала сюжета. Существует специальная дискурсивная формула \i{вуδҷ на-вуδҷ} [быть.{\sc pf.m.sg} {\sc neg}-быть.{\sc pf.m.sg}] ‘было, не было’ которая часто служит зачином сказочного сюжета (\gethref{exevid35}). Аналогичные конструкции ‘было, не было’, использующие формы Перфекта, есть в сарыкольском \parencite[98]{palmer2016} и ваханском \parencite[29]{obrtelova2017}, а также во многих языках Кавказа \parencite[341]{maisak2016}. Как только начинается повествование, Перфект сменяется нарративным временем (обычно Претеритом).

\ex<exevid35>
\begingl
\gla \b{Вуδҷ}\textsubscript{[PF]} \b{навуδҷ}\textsubscript{[PF]} йи потх̌о \b{вуδҷ}\textsubscript{[PF]}. Wинд=ен хоɣ̌ ɣ̌ин \b{виц}\textsubscript{[PF]}. Ас~дефанд йи нутфā мис wинд \b{навуδҷ}\textsubscript{[PF]}. Потх̌о=йи хойих̌ \b{чӯɣ̌ҷ}\textsubscript{[PF]} wӯвдум ɣ̌ин вӣрт…//
\glft ‘Жил-был царь [буквально: \b{Был}\textsubscript{[PF]}, \b{не~был}\textsubscript{[PF]}~— \b{был}\textsubscript{[PF]} царь]. У~него \b{было}\textsubscript{[PF]} шесть жён. У~него \b{не~было}\textsubscript{[PF]} от~них детей. Царь \b{решил}\textsubscript{[PF]} взять себе седьмую жену…’ \trailingcitation{[текст \i{The black-skinned servant}, 1–4]}//
\endgl \xe

«Фоновые» употребления затем появляются уже внутри нарратива, дополняя сюжет необходимым контекстом или отсылая к событиям, произошедшим до текущего момента в сюжете. В этой функции может использоваться и Претерит. По-видимому, Претерит в таких случаях подчёркивает событийность (герои нарратива видели это событие / оно произошло ранее в нарративе), а Перфект фокусируется на результате (событие произошло до начала нарратива). Выбор формы во многих случаях может зависеть от желания рассказчика акцентировать внимание на одном из этих аспектов. К примеру, отрывок (\gethref{exevid36}) отсылает к оживлению оборотня, которое описывалось в тексте раньше, но внимание акцентируется на результате этого действия. Кроме того, возможно, использование Перфекта связано с точкой зрения героя (юноши), который, как тут же и подчёркивается, не знал об этом:

\ex<exevid36>
\begingl
\gla …Ба кӯтойи мухтасар анҷāвāм, йā wи ғиδā нāн лозаки \b{сат}\textsubscript{[PST]}, амо йу ғиδā ас ху нāн кор бехабар, \b{наwзент}\textsubscript{[PST]} диди wи жиндӯрвак=и wи нāн зиндā \b{чӯɣ̌ҷ}\textsubscript{[PF]}.//
\glft ‘Короче говоря, мать этого юноши \b{забеременела}\textsubscript{[PST]}, но юноша о~делах своей матери понятия не~имел и~\b{не~знал}\textsubscript{[PST]}, что~она оборотня \b{оживила}\textsubscript{[PF]}’ \trailingcitation{[текст \i{White mountain goat}, 69]}//
\endgl \xe

Пример (\gethref{exevid37}) хорошо демонстрирует, как эвиденциальность влияет на выбор формы. Шугнанский текст здесь явно указывает, как события соотносятся друг с другом: женщина не видела, как ребёнок заплакал, она вошла в тот момент, когда старший сын уже держал его на руках~— Перфект используется для описания событий, которые произошли в её отсутствие.

\ex<exevid37>
\begingl
\gla Хêр, йилāв йāм ɣ̌иник тар ми жиндӯрвак хез \b{ност}\textsubscript{[PST]} ху \b{сат}\textsubscript{[PST]} тар чӣд, диди йу wих̌так=и йилāв \b{нӣwҷ}\textsubscript{[PF]} ху, йу ғулā пуц мис \b{суδҷ}\textsubscript{[PF]} агā ху, wи wих̌так=и \b{зох̌ч}\textsubscript{[PF]} ху пи бататā, нақл wи қатӣр \b{ких̌т}\textsubscript{[PRS]}.//
\glft ‘Она немного \b{посидела}\textsubscript{[PST]} с оборотнем и потом \b{пошла}\textsubscript{[PST]} в свою комнату, а там [как оказалось,] ребёнок \b{плакал}\textsubscript{[PF]}, и её старший сын \b{проснулся}\textsubscript{[PF]} и \b{взял}\textsubscript{[PF]} ребёнка на руки и [теперь] \b{разговаривает}\textsubscript{[PRS]} с ним.’ \trailingcitation{[текст \i{White mountain goat}, 119]}//
\endgl \xe

В некоторых случаях Перфект, видимо, подчёркивает неожиданность и/ли неконтролируемость события. В примере (\gethref{exevid38}) событие ‘увидела’ находится в нарративе и выражается Претеритом, а ‘влюбилась’ выражается Перфектом и будто бы выпадает из нарратива~— вероятно, потому, что является неожиданным для героини, неконтролируемым результатом предыдущего события\fn{Об использовании видовременных глагольных форм в шугнанских нарративах см.~более позднюю работу \parencite{melenchenko2025_diploma}~— \i{прим.~переиздания}.}.

\ex<exevid38>
\begingl
\gla Йā потх̌о нозийу̊н ɣ̌ин=и wи \b{wӣнт}\textsubscript{[PST]} ху, ошиқ wи-ти \b{сиц}\textsubscript{[PF]}.//
\glft ‘Любимая жена короля \b{увидела}\textsubscript{[PST]} его [слугу] и \b{влюбилась}\textsubscript{[PF]} в него.’ \trailingcitation{[текст \i{The black-skinned servant}, 9]}//
\endgl \xe

\pagebreak[2]

\section{Заключение} \label{evid-conclusion}

Проведённое исследование позволяет сделать следующие выводы: у большинства глаголов форма Перфекта используется для выражения (а)~незасвидетельствованных событий в прошлом, (б)~экспериенциальной семантики и (в)~модальной семантики (контрфактичность и пожелание). Существенно, что Претерит выражает прямое свидетельство и в этом противопоставлен Перфекту. У ограниченного числа глаголов, которые я здесь называю «стативно-перфектными», Перфект обозначает актуальные состояния в настоящем времени, в то время как Презенс обозначает хабитуальные состояния и состояния в будущем времени. Вопрос выражения экспериенциальной и ирреальной семантики у таких глаголов требует дальнейшего изучения. Условно к похожим на стативно-перфектные глаголы можно отнести застывшие формы Перфекта типа \i{жӣwҷ} ‘любить’. Перфектная форма глагола \i{вуδҷ}/\i{виц} ‘быть’ может употребляться с референцией к настоящему времени. Такие употребления глагола бытия часто считают «миративными», но я предлагаю анализировать их как инферентивные. В нарративах Перфект используется не как нарративное время, а в «фоновой» функции.

В общем шугнанский Перфект можно охарактеризовать как «неопределённое прошедшее» (формулировка предложена носительницей). Он используется для событий, которые говорящий не наблюдал, для выражения абстрактного опыта (экспериенциальность) или абстрактного состояния (форма Перфекта глагола ‘быть’)~— в противовес Претериту, употребляемому для описания конкретных событий, которые говорящий видел. Стативные Перфекты являются отклонениями от этого описания. По-видимому, их стоит считать итогом альтернативного развития семантики бывшего результативного причастия. В то время как у других глаголов оно развилось в эвиденциальную форму, у «стативно-перфектных» глаголов оно стало выражать стативность.

С типологической точки зрения шугнанский Перфект~— «эвиденциальный перфект» (также «перфектоид» \parencite[14–15]{plungian2016} или «эвиденциальная стратегия» \parencite[276]{aikhenvald2004}). Это не позволяет чётко встроить его в традиционную типологию перфектов, которая не включает в себя эвиденциальность. К примеру, неясно, относится ли шугнанский Перфект к межъязыковой категории «перфектов» по \parencite{dahl_velupillai2013}, где одним из критериев является наличие значения результата, но не сообщается о случаях, когда это значение выражается перфектом только у незасвидетельствованных событий. Что касается типологии эвиденциальных систем, шугнанский язык имеет систему типа А1 по классификации \parencite{aikhenvald2004}. Системы А1 различают два значения эвиденциальности: прямая и непрямая. По семантике перфекта и по географическому расположению шугнанский однозначно принадлежит к Большому эвиденциальному поясу. Многие его семантические свойства типичны для языков этого ареала (и особенно для памирского ареала)~— в частности, значение непрямой эвиденциальности у большинства глаголов и «миративные» употребления глагола бытия. При этом он выделяется тем, что, вероятно, достаточно далеко продвинулся на пути грамматикализации в эвиденциальную форму. Это можно проследить и в его морфологической структуре (синтетическая форма, не использующая связку), и в семантике (небольшой, чётко выделенный класс стативно-перфектных глаголов~— реликтов).