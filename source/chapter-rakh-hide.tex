\chapter*{К~истории понятий: лингвистический ракурс. Шугнанское \fakesc{ПРЯТАТЬ}}
\addcontentsline{toc}{chapter}{\textit{Е.~Рахилина, Ш.~Некушоева, Е.~Арманд}. \textbf{К~истории понятий: лингвистический ракурс. Шугнанское \fakesc{ПРЯТАТЬ}}}
\setcounter{section}{0}
\chaptermark{К~истории понятий: лингвистический ракурс. Шугнанское \fakesc{ПРЯТАТЬ}}
\label{chapter-rakh-hide}

\begin{customauthorname}
Екатерина Рахилина, Шахло Некушоева, Елена Арманд
\end{customauthorname}

\begin{englishtitle}
\i{On the history of concepts: a linguistic perspective. The verb ‘to hide’ in Shughni\\{\small Ekaterina Rakhilina, Shahlo Nekushoeva, Elena Armand}}
\end{englishtitle}

\begin{abstract}
Статья посвящена исследованию лексической семантики глаголов со значением ʽпрятатьʼ в одном из малых восточноиранских языков Памира — шугнанском, на котором говорят в Горно-Бадахшанской АО Таджикистана. Материалом для статьи послужили словарные данные шугнанского языка [\cite{karamshoev1988}–\cite*{karamshoev1999}], а также полевые материалы 2021~года (г.~Хорог), собранные авторами по типологической анкете, которая была создана Московской лексико-типологоической группой в рамках работы над проектом «Проблема семантической непрерывности в лексико-типологическом аспекте». В работе последовательно рассматриваются все шугнанские лексемы со значением ʽпрятатьʼ и проводится анализ примеров их употреблений. Проанализированный материал показывает, что исходно лексическая система шугнанского языка была практически лишена поля \fakesc{ПРЯТАТЬ}. Формирование этого поля происходило постепенно путём заимствования соответствующих лексем из доминантного для этого региона таджикского языка. Парадоксальным образом, в результате шугнанская система не стала идентичной таджикской в этой лексической зоне. Дело в том, что в качестве основного глагола для обозначения самых разных ситуаций прятания шугнанский язык выбрал таджикский глагол со значением ʽкласть, помещатьʼ и со временем развил у него значение ʽпрятатьʼ. Основные же для таджикского языка глаголы со значением ‘прятать’ — образованные на базе признаков ʽскрытый, тайныйʼ — в шугнанской системе оказались на переферии. В работе прослеживаются семантические переходы, как те, что свойственны глаголам со значением ‘прятать’, так и те, что приводят к образованию предикатов с таким значением.
\end{abstract}

\begin{keywords}
лексическая типология, \fakesc{ПРЯТАТЬ}, памирские языки, шугнанский язык.
\end{keywords}

\begin{eng-abstract}
In this paper, we undertake a study of the lexical semantics of verbs with the meaning ‘hideʼ in one of the small Eastern Iranian languages of the Pamirs — Shughni, spoken in the Gorno-Badakhshan Autonomous Province of Tajikistan. Data for the paper were extracted from the published dictionary of the Shughni language [Karamshoev \cite*{karamshoev1988}–\cite*{karamshoev1999}], as well as field materials collected by the authors in 2021 (Khorog) on the basis of a typological questionnaire, created by the Moscow Lexico-typological group as part of the work on the project “The problem of semantic continuity in the lexico-typological aspect”. The paper consistently examines all Shughni lexemes with the meaning ʽhideʼ and analyzes examples of their use. The analyzed material shows that the original lexical system of the Shughni language was practically devoid of the semantic field ‘hide’. The formation of this field took place gradually by borrowing the corresponding lexemes from the dominant language for this region (Tajik). Paradoxically, the Shughni system did not, as a result, become identical to Tajik in this lexical area. In reality, the Shughni language chose the Tajik verb with the meaning ʽto put, to placeʼ as the main verb to designate a variety of situations of hiding, and over time developed for it the general meaning of ʽhideʼ. At the same time, the principal Tajik verbs for the the meaning ʽto hideʼ — formed on the basis of the semantic features ʽhidden, secretʼ — in the Shughni system turned out to be on the periphery. Our work traces semantic shifts — both those typical of verbs with the meaning ‘hide’ and those that lead to the formation of predicates with such a meaning.
\end{eng-abstract}

\begin{eng-keywords}
lexical typology, \i{{\sc to~hide}}, Pamir languages, Shughni.
\end{eng-keywords}

\begin{acknowledgements}
Работа выполнена при поддержке гранта РФФИ 20-012-00240 «Проблема семантической непрерывности в лексико-типологическом аспекте».
\label{hide-acknow}
\end{acknowledgements}

\begin{initialprint}
\fullcite{rakhilina_etal2021}\end{initialprint}

\section{Введение} \label{hide-intro}

Исходная задача нашей статьи — сугубо лингвистическая: исследование лексической семантики глаголов со значением \fakesc{ПРЯТАТЬ} в шугнанском языке\fn{\label{hide-fn1}Шугнанский язык относится к шугнано-рушанской группе памирских языков (восточноиранские языки). Шугнанский язык играет роль лингва франка для памирцев, он распространён на территории Горно-Бадахшанской автономной области Республики Таджикистан, а также в прилегающей с юго-запада провинции Бадахшан Исламской Республики Афганистан. О шугнанском языке см.~\parencites[225–226]{edelman_dodykhudoeva2009_shughni}.}. В этом качестве она вкладывается, с одной стороны, в более широкий типологический проект исследования глаголов с недоопределённой семантикой, к которым относятся глаголы этой небольшой группы (см.~выше \hyperref[hide-acknow]{раздел «Благодарность»}). С другой стороны, этот сюжет вкладывается в более общее и сугубо практическое исследование, уточняющие в семантическом плане словарное описание шугнанских глаголов. Мы проводим его в рамках совместной работы Школы лингвистики НИУ~ВШЭ и Института гуманитарных наук им.~Б.~Искандарова Национальной академии наук Таджикистана по редактированию знаменитого шугнанско-русского словаря Д.~Карамшоева [\cite*{karamshoev1988}, \cite*{karamshoev1991}, \cite*{karamshoev1999}], основная работа над которым велась полвека назад. Для практически бесписьменного языка (литература на шугнанском крайне скудна) такая временная дистанция оказывается очень значительной — особенно в условиях существенных социальных перемен последних десятилетий. Пересмотр данных Д.~Карамшоева важен и для мониторинга динамики лексико-семантических изменений в языках мира.

Однако, как мы покажем, осмысление лексического материала может выходить за рамки простого типологического или лексикографического описания. Представленные здесь данные отражают сложное и динамичное взаимодействие двух языков — шугнанского и таджикского, сосуществующих в общем географическом и политическом пространстве. Это взаимодействие затрагивает и процесс формирования абстрактных понятий: приобретение новых значений через заимствование слов и калькирование, но одновременно и постепенную утрату «лишних» единиц, и упрощение фрагмента лексической системы. Отследить такой сложный процесс своего рода лингвистической «истории понятий» \parencites{zhivov1996} обычно очень трудно, но кажется, в данном случае это может получиться.

Выбор глаголов прятанья как объекта данного исследования определялся их семантической спецификой, представляющей для нас особый интерес. В отличие от обычных процессных глаголов, скажем, глаголов с семантикой движения (‘идти’, ‘ползти’, ‘катиться’), смены посессора (‘красть’, ‘давать’, ‘меняться’) или, например, поглощения пищи (‘есть’, ‘пить’, ‘прихлёбывать’), глаголы прятанья сами по себе не описывают никакой конкретной ситуации, предсказуемым образом развивающейся во времени. Действительно, ни для какого глагола этой группы способ прятанья никогда не определён: можно прятать, перемещая лицо или объект, можно — закрывая его от посторонних глаз, можно даже комбинировать эти действия — важно только, чтобы всегда присутствовала общая цель: \b{сокрытие объекта}. Именно эта цель, а не жёсткая последовательность конкретных ситуаций (ср.:~перенести центр тяжести на одну ногу, поднять другую, переместить её вперед, опустить до соприкосновения с поверхностью… — как в случае \fakesc{ИДТИ}) объединяет все возможные ситуации прятанья. Заметим, что глаголы прятанья безоценочны. Их особенность именно в недоопределённости значения, как бы исключающей суть основного процесса\fn{Ср.~близкий класс интерпретативов (термин Ю.~Д.~Апресяна) как оценочных слов с «пропущенным» процессом, типа \i{баловать}, \i{грешить}, \i{нарушать закон} \parencite{apresian2014}, ср.~также \parencite{kustova2000}.}, которое не мешает им существовать и образовывать нетривиальные системы со сложными оппозициями в самых разных языках мира \parencite{reznikova2022}.

Важно, что ввиду абстрактности недоопределённого значения оно может передаваться не непосредственно — специализированной лексемой или группой лексем, а переносно, с опорой на какие-то уже существующие в языке конкретные значения. Эта особенность ставит задачу исследования семантических источников недоопределённых предикатов (ср.~в их числе \fakesc{НАХОДИТЬ}) — и среди них источников для семантики ‘прятать’. Предварительные\fn{Во время выхода данной статьи упомянутая работа Т.~И.~Резниковой была в печати; впоследствии она вышла в журнале «Вопросы языкознания» — \i{прим.~переиздания}.} исследования в этом направлении \parencite{reznikova2022} довольно хорошо соотносятся с данными базы CLICS~3 \parencite{clics2020} (а в некоторых позициях дополняют её) и выделяют несколько важных сдвигов, которые оказываются релевантны и для шугнанского языка:

\begin{itemize}
  \item ‘хранить / \i{keep}’ $\rightarrow$ ‘прятать’
  \item ‘накрывать / укрывать / \i{cover}’ $\rightarrow$ ‘прятать’
  \item ‘помещать’ $\rightarrow$ ‘прятать’
\end{itemize}

С семантической точки зрения эти сдвиги вполне ожидаемы, потому что каждый из них представляет конкретное действие, которое можно совершить, чтобы скрыть некоторый объект: переместить этот объект куда-то, накрыть его чем-то, хранить / сохранять его — не используя и тем самым не демонстрируя другим. Именно этому принципу подчиняются источники и других недоопределённых глаголов, ср.~семантические переходы в зону ‘искать’, описанные в \parencite{tolstaya2011} для славянских языков, и в [ЕВРика! \cite*{eureka2018}] в широкой типологической перспективе: они описывают какой-то частный вид деятельности для поиска объекта. Ср.:

\begin{itemize}
  \item ‘смотреть’ $\rightarrow$ ‘искать’
  \item ‘ощупывать руками (например, в темноте)’ $\rightarrow$ ‘искать’
  \item ‘ходить (в поисках)’ $\rightarrow$ ‘искать’
  \item ‘копать(ся), рыть(ся)’ $\rightarrow$ ‘искать’
  \item ‘охотиться’ $\rightarrow$ ‘искать’ и~другие
\end{itemize}

Поэтому статья будет строиться следующим образом. В разделах \ref{hide-joy}, \ref{hide-pano} и \ref{hide-rare} мы подробно рассмотрим шугнанскую систему глаголов прятанья. Порядок описания системы будет соответствовать порядку только что обозначенных источников для ‘прятать’. Мы начнём с обсуждения центрального глагола с наиболее широкой сочетаемостью и потом перейдём к рассказу о более маргинальных глаголах с нетривиальной семантикой. В \hyperref[hide-conclusion]{разделе~5} мы подведём итог этому исследованию и обсудим его в более широкой перспективе. Все примеры даются либо по материалам [\cite{karamshoev1988}; \cite*{karamshoev1991}; \cite*{karamshoev1999}], проверенным нами с точки зрения их актуальности для сегодняшнего узуса, либо собраны авторами статьи по анкете \parencite{reznikova2022} в рамках работы над проектом «Проблема семантической непрерывности в лексико-типологическом аспекте»\fn{Все примеры в статье, для которых источник не указан отдельно, были получены в ходе анкетирования.}.

\section{Доминантный глагол \i{ҷо(й) чӣдоw}} \label{hide-joy}

Шугнанскую систему поля прятанья можно отнести к доминантным: в ней доминирует сложный глагол \i{ҷо(й) чӣдоw} на базе существительного \i{ҷо(й)} со значением ‘место, определённое пространство; край, область, местность; местонахождение, местожительство’ \parencite[556]{karamshoev1999}. Таким образом, буквальным значением основного глагола прятанья в шугнанском оказывается ‘делать место’, то есть ‘помещать’\fn{По-видимому, в этом значении он был калькирован из таджикского \i{ҷо кардан} ʽпомещать, размещать, укладывать, наполнятьʼ [Таджикско-русский словарь \cite*[1057]{mirboboev2006}].}. В шугнанском этот же глагол развивает и связанное с прятаньем значение ‘хранить, беречь’. Таким образом, в отношении глагола \i{ҷо(й) чӣдоw} шугнанская система хорошо вкладывается в общую лексико-типологическую картину.

Как мы уже сказали, сам этот глагол играет в шугнанском ключевую роль как маркер значения ‘прятать’: он применим и к ситуации сокрытия людей (\gethref{exhide1}), и к ситуации тайного хранения артефактов, представляющих ценность (\gethref{exhide2}) или опасность (\gethref{exhide3}), и к ситуации хранения объекта в специально отведённом месте (\gethref{exhide4}), и в ситуации не-использования объекта при бережном с ним обращении (\gethref{exhide5}), и наконец, при укрывании части тела от внешнего воздействия или наблюдения (\gethref{exhide6}):

\ex<exhide1>
\begingl
\gla Wуз=та йи кунҷ=ард ту \b{ҷой} \b{кин-ум}.//
\glc {\sc pron.1sg=fut} один угол={\sc dat} {\sc pron.2sg} место делать-{\sc prs.1sg}//
\glft ‘Я \b{спрячу} тебя в каком-нибудь уголке.’ \trailingcitation{\parencite[146]{karamshoev1991}}//
\endgl \xe

\ex<exhide2>
\begingl
\gla циф-ч-ин \b{ҷой} \b{чӣд-оw}//
\glc красть-{\sc pf-ptcp1} место делать.{\sc inf-purp}//
\glft ‘\b{прятать} украденное’//
\endgl \xe

\ex<exhide3>
\begingl
\gla \b{Чӯд}=и \b{ҷой} wам чêд.//
\glc делать.{\sc pst=3sg} место {\sc d3.f.sg.o} нож//
\glft ‘\b{Спрятал} [он] тот нож.’ \trailingcitation{\parencite[556]{karamshoev1999}}//
\endgl \xe

\ex<exhide4>
\begingl
\gla Дам wих̌ӣӡ пи пилес бӣр \b{ҷой} \b{кин}.//
\glc {\sc d2.f.sg.o} ключ {\sc up} палас под место делать[{\sc imp}]//
\glft ‘\b{Положи} (спрячь) тот ключ под паласом.’//
\endgl \xe

\ex<exhide5>
\begingl
\gla Йид ху курта-йен зорδ на-виред, ~~~~~~~~~~~~~~~~~~~~~~~~~~~~~~ \b{ҷой} wев \b{ких̌т}.//
\glc {\sc d2.sg} {\sc refl} платье-{\sc pl} сердце {\sc neg}-найти.{\sc prs.3sg} ~ место {\sc d3.pl.o} делать.{\sc prs.3sg}//
\glft ‘Ему жалко (надевать) рубашки, \b{бережёт} их.’ \trailingcitation{\parencite[148]{karamshoev1991}}//
\endgl \xe

\ex<exhide6>
\begingl
\gla Му пуц=и ху δуст-ен ас ху зибо \b{ҷой} \b{чӯд}.//
\glc {\sc pron.1sg.o} сын={\sc 3sg} {\sc refl} рука-{\sc pl} {\sc el} {\sc refl} зад место делать.{\sc pst}//
\glft ‘Мой сын \b{спрятал} руки за спиной.’//
\endgl \xe

В значении ‘прятать’ может также употребляться вариант \i{бар ҷо чӣдоw} ‘положить, помещать, определить на место’, где перед именной частью глагола добавляется предлог \i{бар} ‘на’:

\ex<exhide7>
\begingl
\gla \b{Бар} \b{ҷо(й)} ди \b{кин}, лāк уз ди мā-вирӣм.//
\glc на место {\sc d2.m.sg.o} делать[{\sc imp}] чтобы {\sc pron.1sg} {\sc d2.m.sg.o} {\sc proh}-найти.{\sc prs.1sg}//
\glft ‘\b{Положи} это куда-нибудь, чтоб я не нашёл.’//
\endgl \xe

\ex<exhide8>
\begingl
\gla Ар цу̊нд=ат йу̊д=анд=ат йам=анд wи ~~~~~~~~~~~~~~~~~~~ \b{бар} \b{ҷой} \b{чӯд}, уз=ум wи вирӯд.//
\glc каждый сколько={\sc 2sg} {\sc d1=loc=and2} {\sc d3=loc} {\sc d3.m.sg.o} ~ на место делать.{\sc pst} {\sc pron.1sg=1sg} {\sc d3.m.sg.o} найти.{\sc pst}//
\glft ‘Сколько ты ни \b{перепрятывал} это то тут, то там, я это нашёл.’//
\endgl \xe

\i{\b{Примечание}}. Работа над анкетой показала, что часть ситуаций, в принципе релевантных для зоны прятанья, в шугнанском по разным причинам выходит за её пределы. Например, ситуация, которая могла бы интерпретироваться как ‘спрятать волосы под платок’ может передаваться исключительно конверсно, как ‘покрыть волосы платком’, с помощью глагола \i{биɣ̌ӣн чӣдоw} в его прямом значении ‘накрыть, покрыть’ (см.~\hyperref[hide-rare]{раздел~4}). Ситуация ‘поджать / спрятать хвост (о собаке)’ культурно не значима и не маркируется лексически потому, что собака считается нечистым животным.

\section{Близкая семантика, разная судьба: \i{пано} \& \i{пину̊н}} \label{hide-pano}

Помимо \i{ҷо(й) чӣдоw}, в словаре Д.~Карамшоева для перевода русского ‘прятать’ приводится два похожих сложных глагола — \i{пано чӣдоw} и \i{пину̊н чӣдоw}, оба таджикизмы\fn{\i{Панаҳ кардан}~— а) прикрывать, заслонять что-л.; б) прятать, укрывать; в) скрывать; таить; утаивать [Таджикско-русский словарь \cite*[463]{mirboboev2006}]. \i{Пинҳон}~— скрытый, тайный, секретный, \textasciitilde~\i{кардан}~— прятать, скрывать, утаивать [Таджикско-русский словарь \cite*[478]{mirboboev2006}].}, на базе прилагательных с уже изначально абстрактным значением ‘скрытый’ или ‘тайный’ (\gethref{exhide71})–(\gethref{exhide81}). Тем самым, для этих глаголов речь о внешнем семантическом источнике ‘прятать’ не идёт. Обратим внимание, однако, что в параллель к абстрактному значению базового признака центральными употреблениями этих глаголов являются контексты с абстрактным именем ситуации как объектом сокрытия, ср.~(\gethref{exhide9}).

\ex<exhide71>
\begingl
\gla Йу му-рд \b{пано} вуд.//
\glc {\sc d3.m.sg} {\sc pron.1sg.o-dat} скрытый быть.{\sc pst.m.sg}//
\glft ‘Он был от меня \b{скрыт} [мне не виден].’ \trailingcitation{\parencite[372]{karamshoev1991}}//
\endgl \xe

\ex<exhide81>
\begingl
\gla Му-нд йи чӣз ас ту \b{пину̊н} нист.//
\glc {\sc pron.1sg.o-loc} {\sc indef} вещь {\sc el} {\sc pron.2sg} скрытый есть.{\sc neg}//
\glft ‘У меня нет ничего \b{тайного} от тебя.’ \trailingcitation{\parencite[414]{karamshoev1991}}//
\endgl \xe

\ex<exhide9>
\begingl
\gla Wуз ху кор ас wи ~~~~~~~~~~~~~~~~~~~~~~~~~~~~~~~~~~~~~~~~~~~~~~~~~~~~~~ \b{пано~/~пину̊н} \b{на-кин-ум}.//
\glc {\sc pron.1sg} {\sc refl} поступок {\sc el} {\sc d3.m.sg.o} ~ скрытый {\sc neg}-делать-{\sc prs.1sg}//
\glft ‘Я \b{не таю} от него свои поступки.’//
\endgl \xe

При этом глагол \i{пано чӣдоw} воспринимается в шугнанском как устаревший и, в частности, не выдерживает конкуренции с \i{ҷой чӣдоw} в сочетании с объектом-лицом или предметным именем типа (\gethref{exhide1}) (‘я спрячу тебя’) или (\gethref{exhide3}) (‘спрятал [он] тот нож’). Как часть глагола, продуктивным признаковое слово \i{пано} осталось в основном в составе непереходного сложного глагола \i{пано δêдоw} ‘скрываться, становиться не видным’:

\ex<exhide10>
\begingl
\gla Уз=ум \b{пано} \b{δод}=ат, wāδ=ен мис фирӣп-ч.//
\glc {\sc pron.1sg=1sg} скрытый ударить.{\sc pst=and2} {\sc d3.pl=3pl} тоже дойти-{\sc pf}//
\glft ‘Я \b{скрылся} (из глаз), и они тоже добрались.’//
\endgl \xe

Другое частотное употребление \i{пано} — в составе сакральных формул типа \i{пано бар Хуδой} ‘Да хранит [тебя, нас, их, вас] Бог’.

В этом отношении второй глагол, \i{пину̊н чӣдоw} встроен в современный шугнанский гораздо лучше — в частности, он допустим с объектами-артефактами (\gethref{exhide11}), хотя чаще — с абстрактными объектами (\gethref{exhide12}):

\ex<exhide11>
\begingl
\gla Кампӣр=и ху пӯл-ен \b{пину̊н} \b{чӯд}=ху, ~~~~~~~~~~~ шич wеф на-виред.//
\glc старушка={\sc 3sg} {\sc refl} деньги-{\sc pl} скрытый делать.{\sc pst=and1} ~ сейчас {\sc d3.pl.o} {\sc neg}-найти.{\sc prs.3sg}//
\glft ‘Старая женщина \b{спрятала} свои деньги и не может теперь их найти.’//
\endgl \xe

\ex<exhide12>
\begingl
\gla Ди кор=ат \b{пину̊н} ца \b{на-чӯɣ̌ҷ-ат}, ~~~~~~~~~~~~~~~~~~ wахт=анд=ам ту-рд йордам чӯɣ̌ҷ-ат.//
\glc {\sc d2.m.sg.o} работа={\sc 2sg} скрытый {\sc subd} {\sc neg}-делать.{\sc pf-pqp} ~ время={\sc loc=1pl} {\sc pron.2sg-dat} помощь делать.{\sc pf-pqp}//
\glft ‘Если бы ты \b{не скрыл} это [дело] от нас, мы бы тебе вовремя помогли.’//
\endgl \xe

И всё-таки узус \i{пину̊н чӣдоw} тоже ограничен. Недоступной для него оказывается зона одушевлённых объектов — никто не скажет по-шугнански:

\ex<exhide13>
\begingl
\gla \ljudge*Wāδ=та ту \b{пину̊н} \b{кин-ен}.//
\glc {\sc d3.pl=fut} {\sc pron.2sg} скрытый делать-{\sc prs.3pl}//
\glft {\small ожидаемое значение:} ‘Они \b{спрячут} тебя.’//
\endgl \xe

Это не случайное для \i{пину̊н чӣдоw} ограничение: по нашему мнению, сочетаемость с именами лиц всегда «последний рубеж» на пути освоения словом новой лексической зоны. Сочетаемость с природными объектами (к которым относятся в том числе люди) осваивается позже, чем область абстрактных понятий или артефактов, к которым легче применить новый предикат или оператор. В нашей лексикологической практике это обстоятельство подтверждалось на множестве примеров — начиная с цветообозначений (см.~\parencite[175–179]{rakhilina2008} о судьбе русского \i{коричневый}, а также других заимствований — например, \i{проблема} \parencite{lyashevskaya_rakhilina2010}, и заканчивая процессами квазиграмматикализации на примере \i{куча} и его синонимов в \parencite{rakhilina_sukhoen2010}).

\section{Редкие и исчезнувшие лексемы} \label{hide-rare}

Помимо описанных, в словаре Д.~Карамшоева упоминается ещё пять глаголов со значением ʽпрятать’: \i{туптā чӣдоw}, \i{рӯпӯх̌ чӣдоw}, \i{биɣ̌ӣн чӣдоw}, \i{ғарқ чӣдоw}, \i{риӡен} / \i{ризӣн}, так что в целом поле выглядит очень объёмным. Это, однако, не так: все эти глаголы в сегодняшнем шугнанском либо утрачены, по крайней мере в нужном нам значении, либо маргинальны. Например, сложный глагол \i{туптā чӣдоw}, образованный на базе прилагательного \i{туптā} ‘спрятанный, укрытый, покрытый’, возможно, остался только в баджувском диалекте и не опознаётся носителями хорогского шугнанского.

Между тем сам переход ‘закрыть, укрыть (cover)’ $\rightarrow$ ‘спрятать’, как мы уже говорили, вполне стандартен и засвидетельствован, в частности, в романских языках \parencite{reznikova2022}. Поэтому неудивительно, что в шугнанском похожей исходной семантикой, согласно Карамшоеву [\cite*{karamshoev1988}], обладают ещё два признака, задействованные в структуре глаголов прятанья: заимствованное из таджикского прилагательное \i{рӯпӯх̌} ‘закрытый; покрытый’ и собственная шугнанская лексема \i{биɣ̌ӣн} ‘накрытый, покрытый; крытый, застланный’. Оба они образуют сложные глаголы \i{рӯпӯх̌ чӣдоw} и \i{биɣ̌ӣн чӣдоw} с исходным значением ‘покрывать, накрывать’ и производным ‘скрывать’. Между тем для \i{рӯпӯх̌ чӣдоw} эта лексикографическая информация в современном шугнанском устарела: он в довольно большой степени вышел из употребления и остался прежде всего в сочетаниях с \i{гуно} ‘грех, недостаток’ (\gethref{exhide14})\fn{В этом значении существует ещё одно частотное выражение \i{ғаθ=ти сит чӣдоw}, букв. ‘экскременты сыпать песком’ в значении ‘покрывать кого-то’, ср.:

\ex[belowexskip=4pt]<exhide14fn1>
\begingl
\gla Wи ғаθ=ти муду̊м \b{сит} \b{кин-и}.//
\glc {\sc d3.m.sg.o} экскременты={\sc sup} всегда песок делать-{\sc prs.2sg}//
\glft ‘Его грехи всегда \b{покрываешь}.’//
\endgl \xe

Ср.~похожую русскую метафору \i{заметать следы} в значении ‘скрывать последствия проступков и преступлений’. Судя по данным НКРЯ [Савчук и~др. \cite*{nkrya2024}], в буквальном значении субъектом выступают природные явления: снег, ветер, которые приносят снег или пыль и скрывают под ними дороги и следы:

\ex[belowexskip=4pt]<exhide14fn2>
\i{<…> ветер взвизгивал, касаясь земли, заметал следы и выл протяжно, грустно.} \trailingcitation{[Максим Горький. Злодеи (1901)]}
\xe

Однако соответствующая русская метафора, как в:

\ex[belowexskip=4pt]<exhide14fn3>
\i{Возможно, там попросту «заметали следы» — убирали то, что может вызвать подозрение?} \trailingcitation{[Известия, 2009.10.26]}
\xe

принципиально отлична от шугнанской: в русской субъект, как бы наметая снег или пыль на следы, уничтожает свидетельства исключительно \b{своих} преступлений — а в шугнанском он, как бы насыпая песок на экскременты, покрывает \b{чужие} грехи. Чужие следы стереть гораздо легче, чем свои, которые возникают за спиной идущего: ему пришлось бы идти задом наперёд, чтобы воплотить такую метафору. Единственный способ оправдать русскую метафору по сравнению с шугнанской в том, чтобы возвести её не к ветру или вьюге, а к лисе с пушистым хвостом.}, а \i{биɣ̌ӣн чӣдоw} остался основным глаголом для прямого значения ‘покрывать’, ср.~(\gethref{exhide15})–(\gethref{exhide16}):

\ex<exhide14>
\begingl
\gla Wи гуно-йен муду̊м \b{рӯпӯх̌-и}.//
\glc {\sc d3.m.sg.o} грех-{\sc pl} всегда покрытый-{\sc prs.2sg}//
\glft ‘Его грехи всегда \b{покрываешь}.’//
\endgl \xe

\ex<exhide15>
\begingl
\gla Тар чӣд ди пāй риб-и, ~~~~~~~~~~~~~~~~~~~~~~~~~~~~~~~~~ \b{биɣ̌ӣн} ди \b{ки}=ху, лāк.//
\glc {\sc eq} дом {\sc d2.m.sg.o} кислое\_молоко ставить-{\sc prs.2sg} ~ покрытый {\sc d2.m.sg.o} делать[{\sc imp}]={\sc and1} оставить[{\sc imp}]//
\glft ‘Поставь кислое молоко в доме, \b{покрой} его и оставь.’ \trailingcitation{\parencite[260]{karamshoev1988}}//
\endgl \xe

\ex<exhide16>
\begingl
\gla Wам қалā=йен \b{биɣ̌ӣн} \b{чӯɣ̌ҷ-ат}.//
\glc {\sc d3.f.sg.o} крепость={\sc 3pl} покрытый делать.{\sc pf-pqp}//
\glft ‘Ту крепость давно \b{покрыли} крышей.’ \trailingcitation{\parencite[261]{karamshoev1988}}//
\endgl \xe

Однако как переносное \i{биɣ̌ӣн чӣдоw} употребляется в предельно малом числе абстрактных контекстов, с объектом-ситуацией, очень напоминающих контексты русского \i{покрывать}, ср.~(\gethref{exhide17}):

\ex<exhide17>
\begingl
\gla Йā ху чор гуно-йен \b{биɣ̌ӣн} \b{ких̌т}.//
\glc {\sc d3.f.sg} {\sc refl} муж грех-{\sc pl} покрытый делать.{\sc prs.3sg}//
\glft ‘Она \b{скрывает} проделки своего мужа.’ \trailingcitation{\parencite[261]{karamshoev1988}}//
\endgl \xe

— правда, без возможности метонимического переноса на человека. Это значит, что естественный в русском смысл, возникающий благодаря метонимическому переносу с деятельности на субъекта деятельности:

\begin{itemize}
  \item ‘она скрывает проделки своего мужа’ $\rightarrow$ ‘она его [мужа] покрывает’
\end{itemize}

(то есть ‘скрывает его проступки или преступления’), в шугнанском не возникает для \i{биɣ̌ӣн чӣдоw}. Соответствующее предложение с этим глаголом будет пониматься буквально, ср.:~‘она мужа [чем-то] покрывает / укрывает (например, одеялом)’.

Четвёртый глагол — тоже исходно таджикский — \i{ғарқ чӣдоw}, необычайно интересен с точки зрения семантического источника. Он строится на базе прилагательного \i{ғарқ} ‘потонувший, погружённый (в жидкость, в грязь)ʼ, и согласно данным Д.~Карамшоева, подтверждённым опросами носителей, имеет переносное значение ʽукрывать, скрывать’ — а прямое значение у него уже утрачено (и даже Д.~Карамшоев не даёт на него примеров). Ни наши данные, ни данные CLICS~3 перехода:

\begin{itemize}
  \item ‘погружать в жидкость / грязь’ $\rightarrow$ ‘прятать’
\end{itemize}

не дают. Вместе с тем, по своей семантической природе он близок к известному

\begin{itemize}
  \item ‘bury’ $\rightarrow$ ‘hide’,
\end{itemize}

который встречается в удмуртском, ненецком, мари, гунзибском, кофан, отоми и других языках (данные CLICS~3 \parencite{clics2020}), — разница только в субстанции, в которую погружается объект: земля это или жидкость / грязь. Интересно, что результирующий смысл оказывается интенсифицирован по сравнению с обычным ‘прятать’: \i{ғарқ чӣдоw} больше похоже на русское \i{запрятать}, \i{задевать} или устаревшее \i{запропастить} — то есть спрятать так, что никто не может найти\fn{Примеры (\gethref{exhide18})–(\gethref{exhide20}) взяты из Национального корпуса русского языка, см.~[Савчук и~др. \cite*{nkrya2024}] — \i{прим.~переиздания}.}:

\ex<exhide18>
\i{Бабы, куда рукавицы-то \b{запропастили}?} \trailingcitation{[Артём Весёлый. Россия, кровью умытая (1924–1932)]} \xe

\ex<exhide19>
\i{На заставе меня спутают, а увидя золотые деньги, \b{запропастят} навеки.} \trailingcitation{[В.~Т.~Нарежный. Гаркуша, малороссийский разбойник (1825)]} \xe

\ex<exhide20>
\i{Значит, в мае теперь? ― и смотрит с улыбкой, и обиды нет, а если обида, то \b{запрятанная}.} \trailingcitation{[Владимир Маканин. Отдушина (1977)]} \xe

Ср.~русский фразеологизм для некаузативной ситуации \i{как в воду канул}, который довольно точно отражает суть дела: в качестве русского эквивалента \i{ғарқ чӣдоw} следовало бы придумать что-то вроде *\i{кануть кого-то / что-то как в воду}, ср.~(\gethref{exhide21}):

\ex<exhide21>
\begingl
\gla Wи мардум мол=и \b{ғарқ} \b{чӯд}.//
\glc {\sc d3.m.sg.o} народ скот={\sc 3sg} потонувший делать.{\sc pst}//
\glft ‘(Так) он \b{спрятал} скотину людей [так, что они не могут её найти — может быть, даже продал].’//
\endgl \xe

По-видимому, это как раз хороший пример оценочного интерпретативного глагола, о которых мы говорили в \hyperref[hide-intro]{первом разделе}. Об обязательности оценки свидетельствует, в частности, то, что говорящий не может употребить этот глагол применительно к себе, даже если он так спрятал, например, деньги, что теперь сам не может ничего найти.

\ex<exhide22>
\begingl
\gla Уз=ум вегā ху пӯл \b{ҷой} / *ғарқ \b{чӯд}, ғал=аθ вирêд-оw на-вāрδӣм.//
\glc {\sc pron.1sg=1sg} вчера {\sc refl} деньги место ~~~~~~ потонувший делать.{\sc pst} ещё={\sc int} найти.{\sc inf-purp} {\sc neg}-мочь.{\sc prs.1sg}//
\glft ‘Я вчера деньги \b{спрятал}, но до сих пор найти не могу.’//
\endgl \xe

Наконец, последний глагол — собственно шугнанский, простой и единственный из всех не имеющий внешних семантических источников — \i{риӡен} (бадж.~\i{\i{ризӣн}})\fn{В тексте статьи допущена ошибка: слово \i{риӡен} названо простым глаголом. На самом деле оно является именем, которое используется как «именной» элемент сложного глагола. Нетривиальным, однако, является тот факт, что, согласно словарю Карамшоева, оно не употреблялось самостоятельно и встречалось исключительно в составе сложных глаголов — \i{прим.~переиздания}.}. По Д.~Карамшоеву, он имеет очень узкое значение ‘припрятывать после охоты’ и одновременно другое значение — ‘поджаривать мясо после охоты в раскалённых камнях’. Сегодня он вышел из употребления — или, может быть, сохранился исключительно как профессиональный среди охотников, но в принципе шугнанцы помнят, что когда-то мясо зверя действительно оставляли среди камней — а поскольку камни в горах сильно накаляются под солнцем, такое хранение означает одновременно и поджаривание, так что связь этих двух значений вовсе не случайна. Однако примеры (\gethref{exhide23})–(\gethref{exhide24}), которые приводит Д.~Карамшоев, уже не опознаются обычными носителями:

\ex<exhide23>
\begingl
\gla Нахчӣр=ум δод=ху, даδ=ум \b{риӡен} \b{wеδд}.//
\glc киик={\sc 1sg} ударить.{\sc pst=and1} потом={\sc 1sg} спрятанный класть.{\sc pst}//
\glft ‘Я подстрелил киика и его тушу \b{спрятал}.’ \trailingcitation{\parencite[497]{karamshoev1991}}//
\endgl \xe

\ex<exhide24>
\begingl
\gla Δу цицӯ=м δод, wи йӣw=ум сêд=тӣр \b{ризӣн} \b{чӯд}.//
\glc два улар={\sc 1sg} ударить.{\sc pst} {\sc d3.m.sg.o} один={\sc 1sg} плоский\_камень={\sc sup} спрятанный делать.{\sc pst}//
\glft ‘Я убил двух уларов, одного \b{изжарил} среди камней.’ \trailingcitation{\parencite[497]{karamshoev1991}}//
\endgl \xe

\section{Заключение} \label{hide-conclusion}

История с полем ʽпрятать’ в шугнанском интересна сама по себе, потому что вносит определённый вклад в типологию соответствующего поля. Шугнанский подтверждает и дополняет уже замеченные в других языках семантические переходы и стратегии колексификации, такие как:

\begin{itemize}
 \item ‘помещать’ $\rightarrow$ ‘прятать’ (доминантный глагол \i{ҷо(й) чӣдоw})
 \item ‘накрывать / укрывать / cover’ $\rightarrow$ ‘прятать’ (исходно таджикские глаголы \i{туптā чӣдоw}, \i{рӯпӯх̌ чӣдоw} и шугнанский \i{биɣ̌ӣн чӣдоw})
 \item ‘хранить / keep’ $\rightarrow$ ‘прятать’ (доминантный глагол \i{ҷо(й) чӣдоw})
\end{itemize}

— и даже обогащает их, добавляя переход:

\begin{itemize}
 \item ‘погружать в жидкость / грязь’ $\rightarrow$ ‘прятать’ (\i{ғарқ чӣдоw})
\end{itemize}

симметричный засвидетельствованному:

\begin{itemize}
 \item ‘bury’ $\rightarrow$ ‘hide’
\end{itemize}

с точностью до субстанции, в которую при этом тайно погружают объект.

Однако есть и ещё одна интересная особенность шугнанской системы, а именно способ её взаимодействия с конкурирующей и усиливающей своё влияние таджикской — и отчасти русской. В шугнанском нет собственного глагола с семантикой ‘прятать’. Фактически, все действующие предикаты этой группы: \i{ҷо(й) чӣдоw}, \i{пину̊н чӣдоw}, \i{ғарқ чӣдоw}, практически утраченные \i{туптā чӣдоw}, \i{пано чӣдоw} и \i{рӯпӯх̌ чӣдоw} — это сложные глаголы, заимствованные из таджикского.

Единственный «настоящий» и собственно шугнанский глагол прятанья\fn{Имеется в виду: глагол с «именной» частью \i{риӡен}.} — \i{риӡен} — имеет крайне специфическую семантику: это прятанье добычи (причём очень особенное — среди камней, ср.~дополнительное метонимическое значение ‘жарить на камнях’) от зверей, а не от людей.

Заметим, что собственно шугнанский глагол \i{биɣ̌ӣн чӣдоw} с исходным значением ‘закрывать, покрыватьʼ, которое представляет собой известный семантический источник для ʽпрятать’, фактически не развивает этого значения (хотя мог бы!) — если не считать очень скромного набора контекстов, вероятнее всего, поздних и калькированных с русского. Таджикские заимствования со значением прятанья с этим источником (\i{туптā чӣдоw} и \i{рӯпӯх̌ чӣдоw}) в шугнанском, как мы видели, тоже почему-то не прижились.

Таким образом, основной шугнанский глагол для ‘прятать’ — \i{ҷо(й) чӣдоw} — тоже заимствование. Его значение описывается широко распространённым в языках переходом от конкретного ‘класть / хранить’ к более абстрактному ввиду недоопределённости ‘прятать’. Однако в таджикском этот глагол не играет такой важной роли, там он находится скорее на периферии системы. В таджикском основным для прятанья является глагол \i{пинҳон кардан} на базе прилагательного \i{пинҳон} с абстрактным значением ‘тайный, скрытый’, так что семантика основного таджикского глагола не метафорична. Это известный нам когнат шугнанского \i{пину̊н чӣдоw} (вспомним, что таджикизм \i{пано чӣдоw}, сохранившийся в шугнанском только в формуле \i{пано бар Хуδой} ‘Да хранит Бог’, имел практически то же значение).

Между тем в шугнанском этот доминантный для таджикского глагол, когнат \i{пину̊н чӣдоw}, имеет довольно ограниченную сочетаемость — в частности, он не применим к объекту-человеку и в целом тяготеет к абстрактным контекстам (хотя уже употребляется с артефактами). То есть массовое заимствование таджикской лексики для освоения семантики прятанья произошло, и произошло в отношении почти всех глаголов — но в результате системы все равно не оказались идентичными: новая шугнанская построена на именной метафоре, а уже освоенная таджикская — на сложном глаголе с абстрактным признаком в его составе.

Ясно, что исходная система в шугнанском была практически лишена поля \fakesc{ПРЯТАТЬ}, и то, что мы видим на разобранном нами лексическом материале — это процесс формирования поля, который мы до некоторой степени можем отслеживать. Мы видим, что в нашем случае формирование поля происходит путём заимствования лексем из таджикского, и происходит оно, во-первых, постепенно, а во-вторых, с некоторым своего рода «сопротивлением» шугнанской лексической системы. Оно проявляется в периферийности основных для таджикского глаголов на базе абстрактных признаков ‘скрытый, тайный’ и в выборе другого, совершенно конкретного источника для семантической доминанты: ‘класть / хранить’.
