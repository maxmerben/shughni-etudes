\chapter*{Система глаголов движения вниз в~шугнанском языке}
\addcontentsline{toc}{chapter}{\textit{Е.~Рахилина, Ш.~Некушоева}. \textbf{Система глаголов движения вниз в~шугнанском языке}}
\setcounter{section}{0}
\chaptermark{Система глаголов движения вниз в~шугнанском языке}
\label{chapter-rakh-down}

\begin{customauthorname}
Екатерина Рахилина, Шахло Некушоева
\end{customauthorname}

\begin{englishtitle}
\i{Falling verbs in Shughni\\{\small Ekaterina Rakhilina, Shahlo Nekushoeva}}
\end{englishtitle}

\begin{abstract}
Статья описывает систему глаголов падения в одном из памирских языков (восточно-иранских) — шугнанском. Она опирается как на первичные данные, специально собранные для этой статьи, так и на данные словаря Д.~Карамшоева [\cite*{karamshoev1988}], проверенные во время полевых исследований. Показано, что в шугнанском действует доминантный глагол \i{wêх̌тоw}, который в целом покрывает основные ситуации падения. Параллельно в этом языке имеется несколько «малых» глаголов, для описания резкого обрушения (\i{чук δêдоw}), особой траектории падения (фразовый глагол \i{оле ситтоw} ‘падать кубарем’) и нарушения целостности (\i{нихих̌тоw} ‘разваливаться’). В качестве глаголов падения в шугнанском функционируют и глаголы других, семантически близких, полей — поля вращения (\i{гāх̌тоw} ‘поворачиваться’), поля прыгания (\i{зибидоw} ‘прыгать’) и поля удара (\i{δêдоw} ‘падать, ударяться’). Типологически материал шугнанского ценен тем, что позволяет выявить прототипические фреймы, обеспечивающие семантическое пересечение выделенных глаголов с полем падения. Для вращения это падающие деревья, для прыгания — открепление функционально связанных друг с другом объектов и частей от целых, для удара — падение с акцентом на результат (в первую очередь, падение человека с указанием части тела, приходящей в контакт с твёрдой поверхностью).
\end{abstract}

\begin{keywords}
лексическая типология, метафора, метонимия, падение, памирские языки, топология предметов, шугнанский язык.
\end{keywords}

\vfill

\begin{eng-abstract}
The article deals with the system of falling verbs in Shughni, which is one of the Pamir languages of the Southeastern Iranian group. It presents the original data collected from native speakers and the data from Karamshoev’s dictionary [\cite*{karamshoev1988}], checked during our field work.

The paper argues that the Shughni system of falling verbs, though not dominant in the proper sense of the term, has a central specialized verb \i{wêx̌tow} covering the main situations of falling: falling from an elevated surface (a cup from the table), falling of a person, falling of a vertically oriented artifacts like road poles, etc.

There are also several “minor” verbs of falling: for falling with non-vertical trajectory (etymologically opaque phrasal verb \i{ole sittow}), for collapsing (\i{čuk δêdow}) or falling accompanied with disintegration, like falling into pieces / fragments or falling of a pipe / heap of objects (\i{nixix̌tow}). In addition, there are non-falling verbs which are used for lexification of some important falling frames. For example, the verb of rotation \i{gāx̌tow} with the meaning ‘turn’ is used for trees falling because of the strong wind; the verb of upward motion \i{zibidow} ‘jump’ is used to denote different situations of detachment of one object from another including parts from wholes, like a damaged wheel being detached from the car during the trip. The causative verb of destruction \i{δêdow} ‘hit’ (which by default denotes aggressive physical effect of one person to another), when applied to falling situations, means falling of an object from above with the clear accent on the result.

The Shughni system reveals the main oppositions relevant for falling verbs cross-linguistically. However, this system is quite abnormal, because apart from the central verb \i{wêx̌tow} it does not have dedicated verbs of falling (with some very marginal exceptions): all the other markers are borrowed from other semantic fields. It means that Shughni data may serve as an important source for lexical typology illuminating the points of intersection of {\sc falling} with other semantic domains.
\end{eng-abstract}

\begin{eng-keywords}
lexical typology, metaphor, metonymy, falling, Pamir languages, topological features, Shughni.
\end{eng-keywords}

\begin{initialprint}
\fullcite{rakhilina_nekushoeva2020}\end{initialprint}

\section{Введение} \label{down-intro}

Шугнанский язык является одним из самых крупных памирских языков. Он относится к восточно-иранской группе иранских языков. Шугнанцы (численностью около 90~тысяч) живут в основном в Горно-Бадахшанской автономной области Таджикистана (административный центр — город Хорог, примерно в пятистах километрах от Душанбе, у слияния рек Пяндж и Гунд), в Шугнанском и Рошткалинском районах. Есть шугнанцы и в Бадахшанской провинции Афганистана, на другой стороне Пянджа, но наши данные собраны на территории Таджикистана.

Основные описания шугнанского изданы во второй половине прошлого века. Это, прежде всего, небольшое собрание текстов со словарем \parencite{zarubin1960}, а также замечательный словарь \parencite{karamshoev1988}, который во многом служил для нас точкой отсчёта.

Из грамматических особенностей шугнанского, важных при чтении нашей статьи, отметим, что в шугнанском (как и в других индо-иранских языках) предикатные значения в основном выражаются так называемыми сложными глаголами. Однако поскольку статья касается самой базовой лексики, здесь нам в основном встретятся как раз простые глаголы. Им свойственна нетривиальная глагольная морфология, характерная для шугнанского, и прежде всего, многоосновность. Сложные для анализа фрагменты мы старались снабжать необходимыми морфологическими комментариями помимо глосс.

История нашего описания падения в шугнанском довольно парадоксальна — и в определенном отношении показательна. Дело в том, что языковой материал собирался в два этапа: сначала в Москве с квалифицированным носителем, а затем в условиях совместной Памирской экспедиции Школы лингвистики НИУ~ВШЭ и Института памирских языков. При этом наши первоначальные, московские данные свидетельствовали о том, что в шугнанском действует доминантная система глаголов падения. Другими словами, эти данные говорили о том, что там есть очень общий (то есть доминирующий) глагол \i{wêх̌тоw} (с основой настоящего времени \i{wох̌}- и формой прошедшего времени \i{wêх̌т}), который соотносится с самыми разными типами падения, покрывая и падение человека, и камня в реку, и моста, провалившегося под тяжестью машины, и веревки, соскочившей с гвоздя, и выпавших зубов, и прочее.

Интересно то, что все полученные в ходе этой работы примеры при последующей проверке экспертами полностью подтвердились — однако их интерпретация изменилась: со сбором новых данных, наше исходное представление о шугнанской системе падения подверглось существенной коррекции. Обнаружилось, что на самом деле в зоне падения в шугнанском языке действует дистрибутивная, а не доминантная система, со сложным распределением всего пространства поля между как минимум шестью предикатами, специфику паденческого значения каждого из которых мы будем подробно обсуждать в этой статье, а именно: \i{wêх̌тоw}, \i{гāх̌тоw}, \i{нихих̌тоw}, \i{зибидоw}, \i{δêдоw}, а также фразеологизованной конструкцией \i{оле ситтоw} — если не считать глаголов движения жидкости, которые мы здесь рассмотреть не сможем ввиду объемности имеющегося материала\fn{Заметим, что все это глаголы моментальные, практически всегда они употребляются в перфективных контекстах, соответственно, в формах прошедшего времени: \i{wêх̌т}, \i{гāх̌т}, \i{нихух̌т} [{\sc m.sg}] / \i{нихах̌т} [{\sc f/pl}], \i{зибуд} [{\sc m.sg}] / \i{зибад} [{\sc f/pl}], \i{δод}, \i{оле сут} [{\sc m.sg}] / \i{оле сат} [{\sc f/pl}].}.

В этой системе работают следующие типологически релевантные противопоставления, которые мы подробно рассмотрим в соответствующих разделах статьи:

\begin{itemize}
  \item по топологии падающего объекта (\hyperref[down-topology]{Раздел~2})
  \item по топологии (траектории) самого падения, включая так называемое «рефлексивное» падение (\hyperref[down-geometry]{Раздел~3})
  \item по значимости начальной / конечной точки (\hyperref[down-endpoint]{Раздел~4})
\end{itemize}

В \hyperref[down-conclusion]{Разделе~5} мы подведем итоги анализа этого материала.

\section{Топология падающего субъекта: \i{wêх̌тоw} и~\i{гāх̌тоw}} \label{down-topology}

Как обычно, по этому геометрическому параметру противопоставлены два типа ситуаций: падение с приподнятой поверхности, для которого не важна форма и ориентация объекта (см.~\hyperref[down-wextow]{Раздел~2.1}), и падение жёсткого вертикально ориентированного объекта, в результате которого этот объект оказывается на той же плоскости, но в горизонтальном положении (см.~\hyperref[down-vertical]{Раздел~2.2}).

\subsection{\i{Wêх̌тоw}: падение сверху, сопутствующие значения и метафорика} \label{down-wextow}

Падение с приподнятой поверхности обозначается частотным шугнанским глаголом \i{wêх̌тоw}, который сочетается с предлогом \i{аз} с широким аблативным значением ‘с / из / сверху’\fn{В основном наши примеры собраны во время полевой работы; в словарных примерах указывается источник.}:

\ex<exdown1>
\begingl
\gla Қалам аз му сӯмка=нд \b{wêх̌-т}.//
\glc карандаш {\sc el} {\sc pron.1sg.o} сумка={\sc loc} падать-{\sc pst}//
\glft ‘Из моей сумки \b{выпал} карандаш.’//
\endgl \xe

\ex<exdown2>
\begingl
\gla Жиниҷ аз дишӣд=ти \b{wêх̌-т}.//
\glc снег {\sc el} крыша={\sc sup} падать-{\sc pst}//
\glft ‘Снег \b{упал} с крыши.’//
\endgl \xe

\ex<exdown3>
\begingl
\gla Жӣр аз тӣр=ти \b{wêх̌-т}.//
\glc камень {\sc el} верх={\sc sup} падать-{\sc pst}//
\glft ‘Сверху \b{упал} камень.’//
\endgl \xe

\ex<exdown4>
\begingl
\gla Wиδич-буц аз ху рêӡ=анд \b{wêх̌-т}.//
\glc птица-детёныш.{\sc m} {\sc el} {\sc refl} гнездо={\sc loc} падать-{\sc pst}//
\glft ‘Птенец \b{выпал} из своего гнезда.’//
\endgl \xe

\ex<exdown5>
\begingl
\gla Йу кӯдак аз стӯл=ти \b{wêх̌-т}.//
\glc {\sc d3.m.sg} ребёнок {\sc el} стул={\sc sup} падать-{\sc pst}//
\glft ‘Ребёнок \b{упал} со стула.’//
\endgl \xe

Как мы уже говорили, помимо падения с поверхности, этим глаголом может описываться падение человека (\gethref{exdown6}), но также и обвал ветхого моста, падение самолёта, птицы из гнезда, ключа из кармана, человека с балкона, машины, мяча, шляпы, верёвки, соскочившей с гвоздика, и многое другое (подробнее см.~в следующих разделах).

\ex<exdown6>
\begingl
\gla Ɣ̌иник ху поδ=и пи жӣр=анд ~~~~~~~~~~~~~~~~~~~~~~~~~~~~~~~~~~~~~ ҷук-т=ху \b{wêх̌-т}.//
\glc женщина {\sc refl} нога={\sc 3sg} {\sc up} камень={\sc loc} ~ удариться-{\sc pst=and1} падать-{\sc pst}//
\glft ‘Женщина \b{споткнулась} о камень и упала.’//
\endgl \xe

Глагол \i{wêх̌тоw}, по \parencite[360–361]{karamshoev1988}, имеет довольно широкий круг метафорических значений. Часть из них может быть описана как обычная метафора, то есть переход из семантической зоны физического движения к абстрактному, ср.~‘лишиться должности’ (\gethref{exdown8}) или ‘выпасть’ как ‘родиться’ (\gethref{exdown9})\fn{Для примеров со значением ‘родиться’ возможна и другая трактовка: не исключено, что глагол \i{wêх̌тоw} в таких случаях используется в качестве табу. Например, кто-то из домочадцев входит в дом с известием о рождении телёнка, но для того чтобы потусторонние силы не услышали и не навредили только что родившемуся телёнку, используется глагол \i{wêх̌тоw}, как если бы телёнок упал и умер.}:

\ex<exdown7>
\begingl
\gla Бало аз осму̊н ца \b{wох̌т}, ~~~~~~~~~~~~~~~~~~~~~~~~~~~~~~~~~~~~~~~~~~~~~~~~~~~~~~~~~~~~ бардохт чӣд-оw даркор.//
\glc беда {\sc el} небо {\sc subd} падать.{\sc prs.3sg} ~ терпение делать.{\sc inf-purp} нужно//
\glft ‘Если \b{случится} беда, нужно смириться.’//
\endgl \xe

\ex<exdown8>
\begingl
\gla Йу аз ху амал \b{wêх̌-т}.//
\glc {\sc d3.m.sg} {\sc el} {\sc refl} должность падать-{\sc pst}//
\glft ‘Он лишился (буквально ‘\b{выпал} из’) своего чина.’ \trailingcitation{\parencite[84]{karamshoev1988}}//
\endgl \xe

\ex<exdown9>
\begingl
\gla Шӣг-буц бехилӣwанд-аθ аз ху нāн \b{wêх̌-т}.//
\glc телёнок-детёныш.{\sc m} неожиданный-{\sc adv} {\sc el} {\sc refl} мама падать-{\sc pst}//
\glft ‘Телёнок \b{родился} сразу.’//
\endgl \xe

Другие можно рассматривать как результат частичной грамматикализации, то есть превращения \i{wêх̌тоw} в лёгкий глагол со значением ‘попадать в ситуацию Р / оказываться в Р’, вызванного идеей спонтанности и неконтролируемости, свойственной падению:

\ex<exdown10>
\begingl
\gla Дис гандā wазийат=анд=ум \b{wêх̌-т}.//
\glc такой плохой положение={\sc loc=1sg} падать-{\sc pst}//
\glft ‘Я \b{попал} в тяжёлое положение.’//
\endgl \xe

Эти сдвиги глаголов с семантикой падения широко распространены в языках мира\fn{См. другие статьи в специальном выпуске журнала \i{Acta Linguistica Petropolitana}, в котором была напечатана эта статья: \parencite{rakhilina_etal2020}~— \i{прим.~переиздания}.}. Вместе с тем шугнанский даёт любопытные дополнения к общему стандарту. Действительно, как и в других языках, шугнанское \i{wêх̌тоw}, имеющее довольно общую семантику, используется для выражения потери функциональности, ср. аналогичные шугнанским русские примеры типа \i{трон пал} / \i{власть пала} (= ‘перестала функционировать’). Особенность шугнанского в том, что идея потери функциональности конкретных артефактов (а не только абстрактных сущностей), по крайней мере в некоторых фразеологизмах (возможно, отражающих некоторое предшествующее состояние его лексической системы) тоже может выражаться через апелляцию к идее падения. Ср.~\parencites[531]{karamshoev1991}[360]{karamshoev1988}.

\ex<exdown11>
\begingl
\gla Му рӯшт куртā \b{wêх̌-ч}, ā йу сāвӡ ~~~~~~~~~~~~~ ғал наw-аθ.//
\glc {\sc pron.1sg.o} красный платье падать-{\sc pf} а {\sc d3.m.sg} зеленый ~ ещё новый-{\sc adv}//
\glft ‘Моё красное платье \b{износилось}, а то зелёное ещё совсем новое.’//
\endgl \xe

\ex<exdown12>
\begingl
\gla Йу̊д-ирд ачаθ сандāл нист, фук=аθ=ен \b{wêх̌-ч}.//
\glc {\sc d1-dat} совсем старая\_обувь быть.{\sc neg} все={\sc int=3pl} падать-{\sc pf}//
\glft ‘Здесь нет даже никаких потрёпанных сапог, все \b{порвались}.’//
\endgl \xe

Пример (\gethref{exdown13}) представляет контекст другого рода, где \i{wêх̌тоw} применён к части тела и имеет уникальное значение ‘мёрзнуть, стыть’ с неясной историей, однако он тоже может быть проинтерпретирован как выражение потери функциональности:

\ex<exdown13>
\begingl
\gla Тар дарго дис шито, му δуст-ен=ен тар \b{wêх̌-т}.//
\glc {\sc eq} двор такой холодный {\sc pron.1sg.o} рука-{\sc pl=3pl} {\sc eq} падать-{\sc pst}//
\glft ‘На улице так холодно, мои руки замёрзли (=~букв. мои руки \b{упали})’//
\endgl \xe

\subsubsection*{Примечание}

В связи с только что сказанным обратим внимание на любопытную фразему \i{wêх̌ч-питêwҷ одам}, не отмеченную у Карамшоева: так говорят о неуклюжем, никчёмном человеке. По структуре это конструкция повтора — два глагола близкой семантики идут подряд в одной и той же форме: \i{wêх̌ч} — перфектная форма глагола \i{wêх̌тоw} ‘падать’, а \i{питêwҷ} — перфект от \i{питêwдоw} — ‘бросать, кидать’.

Карамшоев приводит примеры такой последовательности, но в глагольном контексте, где она приобретает значение ‘заброшенный, брошенный’: \i{йу wêх̌ч-питêwҷ ред} — он остался без присмотра (то есть ‘брошенный, безнадзорный, оставленный всеми’ \parencite[335]{karamshoev1988}). См.~там же в значении ‘брошенный’:

\ex<exdown14>
\begingl
\gla Wам пуц [\b{wêх̌-ч} \b{питêw-ҷ}] вуд=ат, ~~~~~~~~~~~~ шич ғулā сут.//
\glc {\sc d3.f.sg.o} сын падать-{\sc pf} бросать-{\sc pf} быть.{\sc pst.m.sg=and2} ~ сейчас большой стать.{\sc pst.m.sg}//
\glft ‘Её сын был \b{безнадзорным}, да вырос теперь.’ \trailingcitation{\parencite[335]{karamshoev1988}}//
\endgl \xe

Значения ‘никчёмный / неуклюжий’ данная фразема в глагольных контекстах не имеет, но, по-видимому, приобретает его в именных — и не исключено, что в этом случае такой метафорический эффект тоже связан с идеей потери функциональности.

\ex<exdown14b>
\begingl
\gla Йи [\b{wêх̌-ч} \b{питêw-ҷ}] одам йу вуд.//
\glc {\sc indef} падать-{\sc pf} бросать-{\sc pf} человек {\sc d3.m.sg} быть.{\sc pst.m.sg}//
\glft ‘Он был таким \b{неуклюжим} человеком.’//
\endgl \xe

Именно широкая сочетаемость \i{wêх̌тоw} и создает иллюзию существования на его основе доминантной системы. Однако, как мы увидим дальше, в шугнанском есть и другие возможности выразить идею падения.

\subsection{Падение вертикальных объектов} \label{down-vertical}

Одна из таких возможностей связана с ситуацией падения вертикального вытянутого предмета, которую в шугнанском часто выражает глагол \i{гāх̌тоw}. Именно он и только он применяется для описания деревьев, вывернутых из земли сильным ветром, или столбов — при похожих обстоятельствах. Применительно к этим ситуациям \i{wêх̌тоw} малоприемлем:

\ex<exdown15>
\begingl
\gla Хах̌ х̌ӯӡ сут=ху, йā дирахт ~~~~~~~~~~~~~~~~ \b{гāх̌-т} / *wêх̌-т.//
\glc твёрдый ветер стать.{\sc pst.m.sg=and1} {\sc d3.f.sg} дерево ~ повернуться-{\sc pst} ~~~~~~ падать-{\sc pst}//
\glft ‘От сильного ветра то дерево \b{упало}.’//
\endgl \xe

В случае срубленных / сломанных деревьев / столбов возможна вариативность в ответах носителей языка, связанная с архаизацией глагола \i{гāх̌тоw}. Старшее поколение для описания падения дерева обязательно использует глагол \i{гāх̌тоw}, однако молодое поколение употребляет глагол \i{wêх̌тоw}. При этом соответствующая каузативная ситуация выражается глаголом \i{гарδентоw}, производным именно от \i{гāх̌тоw}:

\ex<exdown16>
\begingl
\gla Столба ар пу̊нд \b{wêх̌-т} / \b{гāх̌-т}.//
\glc столб {\sc down} дорога падать-{\sc pst} ~~~~~~ повернуться-{\sc pst}//
\glft ‘Столб \b{упал} на дорогу.’//
\endgl \xe

\ex<exdown17>
\begingl
\gla Wуз=ум дам дарахт тавāр қати \b{гарδ-ен-т}.//
\glc {\sc pron.1sg=1sg} {\sc d2.f.sg.o} дерево топор {\sc com} повернуться-{\sc caus-pst}//
\glft ‘Я \b{срубил} это дерево.’//
\endgl \xe

Заметим, что сам класс прототипически вертикальных объектов, которые выделяются особой глагольной лексемой в ситуации падения, в шугнанском не совсем обычен. В частности, падение людей описывается не этим выделенным глаголом, а исключительно общим глаголом \i{wêх̌тоw}, см.~выше пример (\gethref{exdown6})\fn{К ситуации падения других вертикальных природных объектов, таких как цветы или высокая трава, \i{гāх̌тоw} тоже не применяется — чего, впрочем, вполне можно ожидать: топология их падения несколько другая: это нежёсткие объекты, хотя и вертикальные.}. Это удивительно, потому что было бы естественно полагать, что соответствующий топологический тип объектов формируется как антропоцентричный, и все предметы, которые в него попадают (как деревья или столбы), лишь уподобляются человеческим существам. Шугнанский материал показывает, что это предположение далеко не всегда верно.

Падение крупных артефактов, имеющих вертикальную ориентацию (ср.~‘дом’ или ‘шкаф’), допускает вариативность в выборе глагола, но здесь некоторые эксперты разрешают глагол \i{гāх̌тоw}\fn{О других лексических маркерах падения для дома / стены и подобных объектов — см.~ниже.}.

\ex<exdown18>
\begingl
\gla Хах̌ заминҷунби сут=ху, ~~~~~~~~~~~~~~~~~~~~~~~~~~~~~~~~~~~~~~~~~ мāш чӣд \b{гāх̌-т}.//
\glc твёрдый землетрясение стать.{\sc pst.m.sg=and1} ~ {\sc pron.1pl} дом повернуться-{\sc pst}//
\glft ‘Наш дом \b{рухнул} от сильного землетрясения.’//
\endgl \xe

Ещё один характерный для \i{гāх̌тоw} тип употреблений в зоне падения артефактов — это вертикальные контейнеры с содержимым, как например, ведро с водой, см.~пример (\gethref{exdown19})\fn{Ср.~другой вариант того же предложения с глаголом \i{тис ситтоw} ‘сыпаться, литься’: \i{йā х̌ац чалак қати \b{тис сат}} [{\sc d3.f.sg} вода ведро {\sc com} разлитый стать.{\sc pst.f/pl}]. Контекст опрокинутого сосуда достаточно частотный, а варианты с \i{тис ситтоw} и \i{гāх̌тоw} практически синонимичны: фактически глагол \i{гāх̌тоw} в этих контекстах импликативно получает семантику ‘проливаться’. Именно поэтому в \parencite{karamshoev1988} для него отмечается и это значение тоже, хотя строго говоря, это не так.}. Заметим, что и в русском этот тип артефактов лексически выделен: в частности, к ним применяется особый предикат \i{опрокинуться}, совмещающий этот тип падения только с падением людей навзничь.

\ex<exdown19>
\begingl
\gla Х̌ац \b{гāх̌-т} чалак қати.//
\glc вода повернуться-{\sc pst} ведро {\sc com}//
\glft ‘Ведро с водой \b{опрокинулось}.’ \trailingcitation{\parencite[387]{karamshoev1988}}//
\endgl \xe

В отличие от русского, где пустой контейнер может в отношении глаголов падения вести себя точно так же, как и полный, в шугнанском падение пустого контейнера описывается исключительно общим глаголом \i{wêх̌тоw}. Точно так же падение человека навзничь (например, сидящего на стуле), которое в русском объединено с контейнерами, глагол \i{гāх̌тоw} тоже не покрывает: человек вообще исключён из зоны действия \i{гāх̌тоw}.

\pagebreak[4]

Однако падение самого стула назад, на спинку, выражается именно с помощью \i{гāх̌тоw} — а глагол \i{wêх̌тоw}, как мы и ожидаем, соответствует его падению в любую другую сторону. Формально такое распределение вполне мотивировано: ведь только со спинкой стул нам «виден» как вертикальный объект, но в целом избирательность сочетаемости \i{гāх̌тоw} удивляет. Однако объяснение такому распределению легко найти, если учесть, что по своему исходному значению \i{гāх̌тоw} — это \b{доминантный глагол вращения объекта} \parencite[387]{karamshoev1988}, который применим к вращению и вокруг своей, и вокруг внешней оси, и в том числе, по свидетельству автора словаря, описывает множественные обороты. Правда, в качестве подтверждения последнему, Карамшоев приводит только один пример:

\ex<exdown20>
\begingl
\gla Лāк йид хидорҷ жӣр \b{гāрδд}=ат, ~~~~~~~~~~~~~~ wуз=та ди хидорҷ илат wизу̊н-ум.//
\glc пусть {\sc d2.sg} мельница камень повернуться.{\sc prs.3sg=and2} ~~~~~~~~~~~~~~ {\sc pron.1sg=fut} {\sc d2.m.sg.o} мельница секрет знать-{\sc prs.1sg}//
\glft ‘Пусть жернов мельницы \b{вращается}, и я определю дефект мельницы.’//
\endgl \xe

Носители современного шугнанского такого рода употребления не признают: они считают, что у \i{гāх̌тоw} есть только значение одиночного неполного оборота, поворота в сторону или переворота, ср.~также более идиоматичные конструкции: ‘подвернул ногу’, ‘переправился на другую сторону реки’ и другие подобные\fn{Возможно, это исходно отымённый глагол, восходящий к семантике круга, ср.~\i{гарδā} ‘лепёшка’ (только круглая), \i{гарδов} ‘водоворот’.}:

\ex<exdown21>
\begingl
\gla Йā дивӯск пāли \b{гāх̌-т}.//
\glc {\sc d3.f.sg} змея бок повернуться-{\sc pst}//
\glft ‘Змея \b{повернула} в другую сторону.’//
\endgl \xe

\ex<exdown22>
\begingl
\gla Wêδ нихах̌т=ху, йā х̌ац ~~~~~~~~~~~~~~~~~~~~~~~~~~~~~~~~ ар wи тараф \b{гāх̌-т}.//
\glc арык разрушиться.{\sc pst.f/pl=and1} {\sc d3.f.sg} вода ~ {\sc down} {\sc d3.m.sg.o} сторона повернуться-{\sc pst}//
\glft ‘Арык разрушился, и вода \b{изменила направление}.’//
\endgl \xe

\ex<exdown23>
\begingl
\gla Му поδ \b{гāх̌-т}.//
\glc {\sc pron.1sg.o} нога повернуться-{\sc pst}//
\glft ‘У меня нога \b{подвернулась} / Я подвернул ногу.’//
\endgl \xe

\ex<exdown24>
\begingl
\gla Йу=йи йакборāθ гāз жақ-т=ху, ~~~~~~~~~~~~~~~~~~~~~~~~~~~~~~~~~~~~~~~~~~~~ wи мошӣн \b{гāх̌-т}.//
\glc {\sc d3.m.sg=3sg} сразу.{\sc adv} газ давить-{\sc pst=and1} ~ {\sc d3.m.sg.o} машина повернуться-{\sc pst}//
\glft ‘Он сразу нажал на газ, и его машина \b{перевернулась}.’//
\endgl \xe

Ср.~нетривиальный метафорический сдвиг значения этого глагола в зону боли и неприятных физиологических ощущений в примерах типа: \i{му зорδ пāли \b{гāх̌-т}} [{\sc pron.1sg.o} сердце бок повернуться-{\sc pst}]: ‘моё сердце перевернулось’ — о недомогании в области сердца или общем недомогании. (Здесь возможен и дальнейший семантический сдвиг в зону эмоций: то же предложение может пониматься в значении ‘почувствовать что-то’, ср.~аналогичную полисемию между болью и эмоцией в русском для контекстов, подобных \i{сердце щемит}). Заметим, что обычно глаголы вращения описывают физиологические ощущения в области живота или головы, как в примере (\gethref{exdown25}) с глаголом \i{нêɣ̌доw} (см.~подробнее \parencite{reznikova_etal2012}).

В качестве (доминантного) глагола \b{вращения с множественными оборотами} — и вокруг самого объекта, и вокруг внешнего ориентира — в современном шугнанском используется глагол \i{нêɣ̌доw}\fn{Ср.~здесь характерные для исходной семантики множественных оборотов переносные значения, отмеченные в \parencite{krugliakova2010}, свойственные \i{нêɣ̌доw}, которые сохраняют идею множественного движения:

\ex<exdown26>
\begingl
\gla Х̌āр=ард то қишлоқ=ард=ум \b{нêɣ̌-д}.//
\glc город={\sc loc} {\sc lim} кишлак={\sc loc=1sg} кружиться-{\sc pst}//
\glft ‘По городу, по кишлаку я \b{гулял} (бродил).’//
\endgl \xe

\ex<exdown27>
\begingl
\gla Чӣз му кинорā \b{ноɣ̌-и}?//
\glc что {\sc pron.1sg.o} вокруг кружиться-{\sc prs.2sg}//
\glft ‘Что ты \b{вертишься} вокруг меня?’//
\endgl \xe

\ex<exdown28>
\begingl
\gla Йу пис wам=аθ \b{ноɣ̌д}.//
\glc {\sc d3.m.sg} {\sc goal} {\sc d3.f.sg.o=int} кружиться.{\sc prs.3sg}//
\glft ‘Он с ней \b{встречается}.’ \trailingcitation{\parencite[333–334]{karamshoev1991}}//
\endgl \xe

Ср.~также производное \i{ноɣ̌ӣҷ} ‘непоседливый, любящий бродить’:

\ex<exdown29>
\begingl
\gla Йā дис \b{ноɣ̌-ӣҷ} одам.//
\glc {\sc d3.f.sg} такой кружиться-{\sc agn} человек//
\glft ‘Она очень любит гулять / Она \b{непоседа}.’ \trailingcitation{\parencite[334]{karamshoev1991}}//
\endgl \xe}:

\ex<exdown25>
\begingl
\gla Му кāл \b{ноɣ̌д}.//
\glc {\sc pron.1sg.o} голова кружиться.{\sc prs.3sg}//
\glft ‘У меня \b{кружится} голова.’//
\endgl \xe

С точки зрения типологии исторических изменений, приобретение доминантным глаголом вращения значения падения очень интересно в свете его дальнейшего метафорического развития. Дело в том, что именно глагол \i{гāх̌тоw}, совмещающий значение поворота и падения вертикальных объектов (то есть их \i{пере}ворота по вертикальной оси), даёт в шугнанском языке \b{метафору превращения}. Иллюстрацией этому служат, в частности, примеры из \parencite{karamshoev1988}, касающиеся приобретения объектом новых свойств, как (29)\fn{Ср.~также в дополнение к ним пример (\gethref{exdown30}), в котором шугнанский глагол каузации вращения \i{гарδентоw} описывает превращение одного объекта в другой:

\ex<exdown30>
\begingl
\gla Бāд=и δеw ху дивуск \b{гарδ-ен-т}=ху, ~~~~~~~~~~~~~~~~~~~~~ хойих̌=и чӯд wам виро жирих̌-т-оw.//
\glc потом={\sc 3sg} чёрт {\sc refl} змея повернуться-{\sc caus-pst=and1} ~ просьба={\sc 3sg} делать.{\sc pst} {\sc d3.f.sg.o} брат кусать-{\sc inf-purp}//
\glft ‘Потом чёрт \b{превратился} в змею и хотел укусить её брата.’//
\endgl \xe}.

По нашим данным, такая метафора встречается в разных языках, причём она свойственна, с одной стороны, глаголам падения вертикальных объектов (как в коми, см.~\parencite[93]{kashkin2017}), а с другой — глаголам вращения (ср.~англ.~\i{turn into}, рус.~\i{[пре]вращаться} и так далее). В шугнанском этот метафорический сдвиг соединяет оба перехода, поскольку глагол \i{гāх̌тоw} сам совмещает семантику падения и поворота / переворота. Его результирующее значение более специфицированно: метафорически он покрывает большинство наблюдаемых \b{естественных} изменений — прежде всего, касающихся цвета и света:

\ex<exdown31>
\begingl
\gla Мāш=āм ар мам дӯс=га завāрка чӯд=ху, wам чой рāнг \b{гāх̌-т}.//
\glc {\sc pron.1pl=1pl} {\sc down} {\sc d1.f.sg.o} мало={\sc add1} заварка делать.{\sc pst=and1} {\sc d3.f.sg.o} чай цвет повернуться-{\sc pst}//
\glft ‘Мы добавили сюда немного заварки, и цвет чая \b{изменился}.’//
\endgl \xe

\ex<exdown32>
\begingl
\gla Wи ранг \b{гāх̌-т}.//
\glc {\sc d3.m.sg.o} цвет повернуться-{\sc pst}//
\glft ‘У него цвет [лица] \b{изменился} (например, если человеку плохо или если луна осветила его лицо).’//
\endgl \xe

\ex<exdown33>
\begingl
\gla Ди гāп х̌ӣ-д-оw қати, ~~~~~~~~~~~~~~~~~~~~~~~~~~~~~~~~~~~~~~~~~~~~~~~~~~~~ йу \b{гилгӯн} \b{гāх̌-т}.//
\glc {\sc d2.m.sg.o} слово слышать-{\sc inf-purp} {\sc com}, ~ {\sc d3.m.sg} розовый повернуться-{\sc pst}//
\glft ‘Услышав такую речь, он \b{покраснел} (от стыда или от злости).’//
\endgl \xe

\ex<exdown34>
\begingl
\gla Йу лап хах̌ касал сут=ху, ~~~~~~~~~~~~~~~~~~~~~~~~~~ wи рāнг \b{гāх̌-т}.//
\glc {\sc d3.m.sg} очень твёрдый больной стать.{\sc pst.m.sg=and1} ~ {\sc d3.m.sg.o} цвет повернуться-{\sc pst}//
\glft ‘Он тяжело заболел, и у него \b{изменился} цвет лица.’//
\endgl \xe

Метонимически по-шугнански можно сказать и \i{сурат гāх̌т}, с опущением признака: ‘Лицо [у него] изменилось’ (буквально ‘лицо повернулось’) — но тоже только если лицо изменилось \b{самопроизвольно}, а не потому что кто-то приложил к этому усилия: покрасил волосы, отрастил усы и так далее. Никакие искусственные, целенаправленные зрительные изменения объекта не покрываются этим метафорическим значением \i{гāх̌тоw}\fn{В связи со всем сказанным, заметим, что в шугнанском глагол с семантикой бросания (=~каузации падения) метафорически может выражать притворство (то есть осознанное, контролируемое превращение).}. Хорошим примером естественных изменений может быть старение: оно наблюдаемо (появляется седина, морщины, меняется цвет лица и прочее) — и как раз для ситуаций такого рода глагол \i{гāх̌тоw} хорошо применим:

\ex<exdown35>
\begingl
\gla Йу дис \b{пӣр} мис \b{гāх̌-ч}.//
\glc {\sc d3.m.sg} такой старый {\sc add2} повернуться-{\sc pf}//
\glft ‘Он тоже очень \b{постарел}.’//
\endgl \xe

\ex<exdown36>
\begingl
\gla Йу лап хах̌ касал сут=ху, ~~~~~~~~~~ \b{қоқ-и-зор} \b{гāх̌-т}.//
\glc {\sc d3.m.sg} очень твёрдый больной стать.{\sc pst.m.sg=and1} ~ худой-{\sc subst-place} повернуться-{\sc pst}//
\glft ‘Он сильно заболел и очень сильно \b{похудел}.’//
\endgl \xe

Что касается других, не-зрительных каналов восприятия изменений объекта: в принципе, изменение вкуса тоже может обозначаться этим глаголом, но значительно реже, а смену запаха или тактильных ощущений \i{гāх̌тоw}, видимо, не описывает:

\ex<exdown37>
\begingl
\gla Ди аwқот маззā \b{гāх̌-т}.//
\glc {\sc d2.m.sg.o} еда вкус повернуться-{\sc pst}//
\glft ‘У этого блюда \b{изменился} вкус.’//
\endgl \xe

Такого рода метафоры возможны в более абстрактных семантических зонах, с абстрактными же субъектами:

\ex<exdown38>
\begingl
\gla Wазийат wазмин \b{гāх̌-т}.//
\glc ситуация тяжёлый повернуться-{\sc pst}//
\glft ‘Ситуация \b{ухудшилась}.’//
\endgl \xe

Ср.~здесь характерные метафорические употребления \i{гāх̌тоw} в контексте ситуации изменения мнения / отношения / настроения. Допустимость таких контекстов подтверждает и \parencite[387]{karamshoev1988} (\gethref{exdown39}); ср. также (\gethref{exdown40}–\gethref{exdown41}):

\ex<exdown39>
\begingl
\gla Йу йак-ум-аθ мāш қати раwу̊н вуд=ху, бāд wи фикри \b{гāх̌-т}.//
\glc {\sc d3.m.sg} один-{\sc ord-adv} {\sc pron.1sg.o} {\sc com} отправляющийся быть.{\sc pst.m.sg=and1} после {\sc d3.m.sg.o} мысль повернуться-{\sc pst}//
\glft ‘Сначала он собирался поехать с нами, но потом \b{передумал}.’//
\endgl \xe

\ex<exdown40>
\begingl
\gla Wев муносибат wазмин \b{гāх̌-т}.//
\glc {\sc d3.pl.o} отношения тяжёлый повернуться-{\sc pst}//
\glft ‘Их отношения \b{ухудшились}.’//
\endgl \xe

\ex<exdown41>
\begingl
\gla Wам му̊них \b{гāх̌-т}.//
\glc {\sc d3.f.sg.o} настроение повернуться-{\sc pst}//
\glft ‘Она обиделась / У неё настроение испортилось (букв. \b{переменилось}).’//
\endgl \xe

\section{Геометрия движения вниз и рефлексивное падение} \label{down-geometry}

\subsection{Вращение в процессе движения} \label{down-gyrate}

Особая траектория падения может быть связана с резким началом и непроизвольным вращением субъекта в процессе движения, ср.~рус.~\i{свалиться кубарем}. Обычно такое падение происходит в контакте с поверхностью, поэтому чаще русское \i{кубарем} выступает при глаголе \i{скатиться}, который в этом контексте акцентирует ненамеренность и фактически обозначает особый тип падения\fn{Точно так же особый тип падения безусловно представляют собой кружащиеся в воздухе листья: как видим, пересечение между зоной вращения и падения не случайно и достаточно существенно.}.

В шугнанском языке есть специальное выражение \i{оле ситтоw}. По Карамшоеву [\cite*[237]{karamshoev1991}], идеофон \i{оле-оле} значит ‘катясь, кубарем’, но в современном языке это междометное наречие почти исчезло. Правда, засвидетельствовано слово \i{олейак} — в частности, его используют, рассказывая о том, как переворачивается младенец (ср.~русское \i{оп-ля!}). Основные значения глагола \i{ситтоw} — ‘идти’ и ‘стать’. Для \i{оле ситтоw} прототипической ситуацией является падение камней с гор, ср.~также:

\ex<exdown42>
\begingl
\gla Δорг-ен=ен \b{оле} \b{сат}.//
\glc палка-{\sc pl=3pl} катясь стать.{\sc pst.f/pl}//
\glft ‘Дрова \b{покатились}.’//
\endgl \xe

\subsection{«Рефлексивное движение»} \label{down-reflexive}

Особая геометрия падения объекта свойственна и так называемой «ситуации рефлексивного движения», когда сам он остается неподвижен, а перемещаются только его части — все или некоторые.

К рефлексивному падению можно отнести такое быстрое и резкое падение, обычное следствие которого в том, что предмет разваливается на части, ср.~рус.~\i{рухнуть}. В шугнанском оно обозначается глаголом \i{нихих̌тоw} ‘разрушаться, спускаться’.

В соответствующей ситуации этот глагол применим к дому, стене, камням, мосту и подобным объектам. Со всеми этими объектами (в зависимости от контекстной ситуации) возможен ещё и глагол \i{wêх̌тоw} ‘падать’. Кроме того, уже как единственно возможный, \i{нихих̌тоw} описывает падение / разрушение дорог, лавин, стопок матрасов\fn{Надо признаться, что «стопка» в данном случае не вполне удачный термин. Лучше было бы сказать \i{гора} или \i{груда}, потому что в высоту эти матрасы, сложенные в шугнанском доме в специальной комнате и предназначенные как для хозяев и всех родственников, так и для гостей, могут достигать почти человеческого роста. Однако матрасы никогда не сваливают в кучу, они всегда сложены очень аккуратно, как бумаги в стопку.} и даже стопок книг. С ними глагол \i{wêх̌тоw} не употребляется, поскольку они не подпадают ни под прототип падения с высоты (ср.~мост, который попадает в зону вариативности \i{нихих̌тоw} / \i{wêх̌тоw}), ни под прототип вертикального падения (как дом, стена — они тоже в зоне вариативности). Ср.:

\ex<exdown43>
\begingl
\gla Йā йед лап кӣнā вад=ху, wазмин мошӣн йам=ти наɣ̌ҷӣд=ху, йā \b{wêх̌-т}.//
\glc {\sc d3.f.sg} мост очень старый быть.{\sc pst.f/pl=and1} тяжёлый машина {\sc d1.sg=sup} проходить.{\sc pst.m.sg=and1} {\sc d3.f.sg} падать-{\sc pst}//
\glft ‘Мост был очень старый и \b{рухнул}, когда проехала тяжёлая машина.’//
\endgl \xe

\ex<exdown44>
\begingl
\gla Йу чӣд / йу бурҷ / йā йед ~~~~~~~~~~~~~~~~~~~~~~~~~~~~~~ пиδид=ху \b{нихух̌т}.//
\glc {\sc d3.m.sg} дом ~~~~~~ {\sc d3.m.sg} стена ~~~~~~ {\sc d3.f.sg} мост ~~~~~~~~~~~~~~~~~~~~~~~~~~~~~~ гореть.{\sc pst=and1} разрушиться.{\sc pst.m.sg}//
\glft ‘Дом / стена / мост загорелся и \b{рухнул} (в контексте землетрясения).’//
\endgl \xe

\ex<exdown45>
\begingl
\gla Йу бӣреҷ \b{нихух̌т} / *wêх̌-т.//
\glc {\sc d3.m.sg} постель разрушиться.{\sc pst.m.sg} ~~~~~~ падать-{\sc pst}//
\glft ‘Стопка матрасов \b{рухнула}.’//
\endgl \xe

\ex<exdown46>
\begingl
\gla Кӯ / бар \b{нихух̌т}.//
\glc гора ~~~~~~ обрыв разрушиться.{\sc pst.m.sg}//
\glft ‘Гора / обрыв \b{обвалился}.’//
\endgl \xe

\ex<exdown47>
\begingl
\gla Йā сêр \b{нихах̌т}.//
\glc {\sc d3.f.sg} умолот разрушиться.{\sc pst.f/pl}//
\glft ‘Та [гора] обмолоченного и провеянного зерна (ссыпанного горой = ‘умолот’) \b{развалилась}.’//
\endgl \xe

Метафорически глагол \i{нихих̌тоw} описывает, как «разваливается» сыр / творог (не собирается в твёрдое тело, а распадается на части):

\ex<exdown48>
\begingl
\gla Му δу̊ғ \b{нихах̌т}.//
\glc {\sc pron.1sg.o} пахтанье разрушиться.{\sc pst.f/pl}//
\glft ‘У меня творог \b{развалился}.’//
\endgl \xe

\subsubsection*{Примечание}

Карамшоев [\cite*[314]{karamshoev1991}] даёт ещё одно значение \i{нихих̌тоw}: ‘спуститься (о людях) с горы пешком’ и приводит следующий пример:

\ex<exdown49>
\begingl
\gla Мардум фук=аθ ас баройи видӣрм \b{нихух̌т} ~~~~~~~~~~~ ар Сох̌чāрв.//
\glc люди все={\sc int} {\sc el} для веник разрушиться.{\sc pst.m.sg} ~ {\sc down} Сохчарв//
\glft ‘Все [люди] \b{спустились} в Сохчарв за веником.’//
\endgl \xe

Между тем не все современные носители подтверждают такого рода употребления \i{нихих̌тоw}. Некоторые согласны принять этот пример исключительно как ироничный: дикорастущий веник (некоторое специальное растение) раньше можно было найти везде, потому что в быту такой веник необходим, люди даже сажали это растение во дворе — а тут они вдруг все спустились из-за какого-то веника в Сохчарв (возможный русский аналог, тоже с оттенком иронии: \i{все повалили в Сохчарв за вениками}).

В нейтральном контексте со значением ‘спуститься’ используются другие глаголы, \i{хāвдоw} ‘спуститься’ или \i{йатоw} ‘прийти’, одинаково применимые и к людям, и к животным:

\ex<exdown50>
\begingl
\gla Йā жоw аз пух̌тā \b{хāв-д} / \b{йат}.//
\glc {\sc d3.f.sg} корова {\sc el} холм спуститься-{\sc pst} ~~~~~~ прийти.{\sc pst}//
\glft ‘Корова \b{спустилась} с горы.’//
\endgl \xe

\ex<exdown51>
\begingl
\gla Мāш=ам ар тагов \b{хāв-д}.//
\glc {\sc pron.1pl=1pl} {\sc down} вниз спуститься-{\sc pst}//
\glft ‘Мы \b{спустились} вниз.’//
\endgl \xe

Что касается \i{нихих̌тоw}, то для значения, близкого к ‘спуститься / опуститься’ уверенно можно говорить только о вторичных, метафорических употреблениях этого глагола:

\ex<exdown52>
\begingl
\gla Бало мāш тора \b{нихух̌т}.//
\glc беда {\sc pron.1pl} макушка разрушиться.{\sc pst.m.sg}//
\glft ‘Беда \b{свалилась} на нашу голову.’//
\endgl \xe

\ex<exdown53>
\begingl
\gla Wāδ=ен мāш тора \b{нихах̌т}.//
\glc {\sc d3.pl=3pl} {\sc pron.1sg.o} макушка разрушиться.{\sc pst.f/pl}//
\glft ‘Они \b{свалились} на нашу голову.’//
\endgl \xe

\subsection{Отделение частей} \label{down-parting}

Продолжая разговор об отпадении частей, остановимся на отделении части объекта, сопровождающейся её падением вниз: так, например, может упасть колесо во время движения машины. В шугнанском эта ситуация не описывается общим глаголом \i{wêх̌тоw}, для неё требуется особый глагол — \i{зибидоw} с исходным значением ‘прыгать’ (\gethref{exdown54}), ср.~русское \i{от}- / \i{соскакивать}:

\ex<exdown54>
\begingl
\gla Йу мис ар х̌ац \b{зибуд}.//
\glc {\sc d3.m.sg} {\sc add2} {\sc down} вода прыгать.{\sc pst.m.sg}//
\glft ‘Он тоже \b{прыгнул} в воду.’//
\endgl \xe

\ex<exdown55>
\begingl
\gla Му велик балу̊н \b{зибуд}.//
\glc {\sc pron.1sg.o} велик колесо прыгать.{\sc pst.m.sg}//
\glft ‘У моего велосипеда отскочило (букв. \b{отпрыгнуло}) колесо.’//
\endgl \xe

Ср. также примеры из \parencite[451]{karamshoev1991}:

\ex<exdown56>
\begingl
\gla Wам бӯт пох̌нā \b{зибуд}.//
\glc {\sc d3.f.sg.o} ботинок каблук прыгать.{\sc pst.m.sg}//
\glft ‘У её ботинка \b{отлетел} каблук.’//
\endgl \xe

\ex<exdown57>
\begingl
\gla Му чиллā=нд=та к=ам к=у̊=ва ~~~~~~~~~~~~~~ нāла \b{зибӣнт}.//
\glc {\sc pron.1sg.o} перстень={\sc loc=fut} {\sc emph=d1.sg} {\sc emph=d1=prol} ~ {\sc quot} прыгать.{\sc prs.3sg}//
\glft ‘Говорят, что стержень на моём перстне \b{отвалится}.’//
\endgl \xe

Во всех приведённых примерах фигурируют части\fn{Интересно, что точно так же, как прототипические части, в шугнанском себя ведут брызги: если в реку бросить камень, то брызги как бы отскакивают (\i{зибуд}) от «цельной» поверхности воды. Сюда же относятся капли-слёзы (ср.~рус.~\i{брызнули слёзы}), ср.~следующий пример из \parencite[549]{karamshoev1988}:

\ex<exdown58>
\begingl
\gla А ҷу̊н ку чис, дам маркāб ~~~~~~~~~~~~~~~~~~~~~~~~~~~~~~~~~~~~~~~~ йӯх̌к-ен=ен \b{зибад}=о?//
\glc {\sc voc} душа {\sc ptcl} смотреть[{\sc imp}] {\sc d2.f.sg.o} осёл ~ слеза-{\sc pl=3pl} прыгать.{\sc pst.f/pl=q}//
\glft ‘Дорогой, взгляни-ка, \b{капают} ли у осла слёзы?’//
\endgl \xe}, однако и ситуация отскочившей подковы у лошади, и отлетевшей пуговицы, и соскочившей с петель двери, которые выходят за пределы отношения часть–целое, допускают \i{зибидоw} — наряду с наиболее общим (доминантным) глаголом падения \i{wêх̌т}. В то же время такие ситуации, как падение лошади с обрыва или полотенца с верёвки (если оно там сохло), исключают \i{зибидоw} и требуют только глагола \i{wêх̌тоw}. По-видимому, прототипически «отскакивают» (\i{зибидоw}) действительно прежде всего части целого или похожие на них независимые (как подкова и лошадь), но функционально тесно и достаточно постоянно связанные с «целым» объекты (в нашей терминологии, «дополнители», см.~\parencite{rakhilina2000}). Мгновенное неконтролируемое нарушение вре́менной пространственной связи между объектами (такой, которая возникает, например, у обрыва и лошади) описывается дефолтным глаголом \i{wêх̌тоw}, а не более специальным \i{зибидоw}. Ср.~пример неконтролируемого движения человека, именно с глаголом \i{зибидоw}:

\ex<exdown59>
\begingl
\gla Аз ху ҷой=ти=йум \b{зибуд}.//
\glc {\sc el} {\sc refl} место={\sc sup=1sg} прыгать.{\sc pst.m.sg}//
\glft ‘Я вскочил (более буквально, видимо: \b{соскочил}) с места (чаще всего от испуга, от неожиданности).’//
\endgl \xe

Однако если в ситуации открепления (так же как и в любой другой ситуации падения) конечная точка оказывается по какой-то причине более значима, чем начальная, используется ещё один глагол поля падения, \i{δêдоw}, о котором речь пойдёт в следующем разделе.

\section{Значимость конечной точки} \label{down-endpoint}

Глагол \i{δêдоw} (непереходный; с основой настоящего времени \i{δи}- и формой прошедшего времени \i{δод}), о котором пойдёт речь в этом разделе, имеет значения ‘падать, ударяться’ и ‘стучать’ (например, ‘стучать в дверь’) (\gethref{exdown60}); ср.~также переходный глагол \i{δêдоw} ‘дать; ударить’ (\gethref{exdown61}–\gethref{exdown62}):

\ex<exdown60>
\begingl
\gla Пи диви=йен тақ-тақ \b{δод}.//
\glc {\sc up} дверь={\sc 3pl} стук-{\sc redup} упасть.{\sc pst}//
\glft ‘В дверь \b{постучались}.’//
\endgl \xe

\ex<exdown61>
\begingl
\gla Wуз=ум диви йет чӯд, wи=йум на-wӣн-т=ху, диви қати=йум wи \b{δод}.//
\glc {\sc pron.1sg=1sg} дверь открытый делать.{\sc pst} {\sc d3.m.sg.o=1sg} {\sc neg}-видеть-{\sc pst=and1} дверь {\sc com=1sg} {\sc d3.m.sg.o} ударить.{\sc pst}//
\glft ‘Когда я открывал дверь, я не увидел его и \b{ударил} его дверью.’//
\endgl \xe

\ex<exdown62>
\begingl
\gla Йу қāр=анд йат=ху, ~~~~~~~~~~~~~~~~~~~~~~~~~~~~~~~~~~~~~~~~~~~~~~~~~~~~~~ бāд=и мут wи \b{δод}.//
\glc {\sc d3.m.sg} гнев={\sc loc} приходить.{\sc pst=and1} ~ потом={\sc 3sg} кулак {\sc d3.m.sg.o} ударить.{\sc pst}//
\glft ‘Он в гневе пришёл и \b{ударил} его кулаком.’//
\endgl \xe

В то же время этот глагол можно считать принадлежащим полю падения. Действительно, именно \i{δêдоw} используется в шугнанском для обозначения падения осадков, и, по понятным причинам, это самое частотное его употребление. Вообще говоря, переход из зоны удара в зону падения можно трактовать как исходно метонимический, основывающийся на смежности ситуаций: например, падение капель дождя всегда связано со звуком удара — как если бы дождь бил или стучал по поверхности, но в современном шугнанском этот сдвиг давно уже лексикализован:

\ex<exdown63>
\begingl
\gla Бору̊н \b{δод}.//
\glc дождь упасть.{\sc pst}//
\glft ‘\b{Шёл} дождь.’//
\endgl \xe

Другая область семантической смежности для удара и падения возникает не на базе результирующего звукового эффекта падения (сопровождающего зрительный), который совпадает со звуком удара, а на базе совпадения физиологического ощущения (боли) при восприятии человеком удара о внешний объект или при падении такого объекта на человека. В обоих случаях возможен \i{δêдоw}, с некоторой разницей в моделях управления, ср.:

\pex<exdown64>
\a \begingl
\gla Му кāл пи жӣр=анд \b{δод}.//
\glc {\sc pron.1sg.o} голова {\sc up} камень={\sc loc} ударить.{\sc pst}//
\glft ‘Я \b{ударился} головой об камень.’//
\endgl
\a \begingl
\gla Жӣр му кāл=ти \b{δод}.//
\glc камень {\sc pron.1sg.o} голова={\sc sup} упасть.{\sc pst}//
\glft ‘Камень \b{упал} мне на голову.’//
\endgl \xe

Между тем в значении падения \i{δêдоw} используется гораздо шире, чем только для выпадения осадков. Этот глагол можно встретить и в самых обычных прототипических контекстах падения с приподнятой поверхности, таких как: ‘ребёнок упал с крыши’, ‘козлёнок упал с горы’ (\gethref{exdown64}), где он, по свидетельству носителей языка, может конкурировать с глаголом \i{wêх̌тоw}.

\ex<exdown65>
\begingl
\gla Кирпӣч му кāл=ти \b{δод}.//
\glc кирпич {\sc pron.1sg.o} голова={\sc sup} упасть.{\sc pst}//
\glft ‘Кирпич \b{упал} мне на голову.’//
\endgl \xe

Конкуренция \i{δêдоw} / \i{wêх̌тоw} не свободная, выбор между ними семантически мотивирован и связан с идеей выделенности конечной точки падения или акцента на среду, в которой в результате оказывается объект. Профилирование \b{конечной точки / среды} задаётся глаголом \i{δêдоw}; глагол \i{wêх̌тоw} в этом отношении более нейтрален, но, как мы увидим из примеров, склонен к выделению начальной точки падения. Поэтому именно \i{δêдоw} используется для обозначения таких ситуаций, как ‘проваливаться в снег’, ‘окунаться в воду’, ‘падать в ущелье’:

\pex<exdown66>
\a \begingl
\gla Ар жиниҷ=ум \b{δод}.//
\glc {\sc down} снег={\sc 1sg} упасть.{\sc pst}//
\glft ‘Я \b{свалился/ась} в снег.’//
\endgl
\a \begingl
\gla Ар х̌ац=ум \b{δод}.//
\glc {\sc down} вода={\sc 1sg} упасть.{\sc pst}//
\glft ‘Я \b{упал(а)} в воду.’//
\endgl
\a \begingl
\gla Ар дарā=йум \b{δод}.//
\glc {\sc down} ущелье={\sc 1sg} упасть.{\sc pst}//
\glft ‘Я \b{упал(а)} в ущелье.’//
\endgl \xe

При этом глагол \i{δêдоw} предпочитает предлог \i{ар} ‘к (вниз)’; глагол \i{wêх̌тоw} — предлог \i{аз} ‘от’. В случае, если сами контексты специально акцентирует начальную или конечную точку, семантическое распределение \i{wêх̌тоw} / \i{δêдоw} в них подтверждается:

\pex<exdown67>
\a \begingl
\gla Му wих̌ӣӡ-ен=ен аз му ҷебак=ард \b{wêх̌-т}.//
\glc {\sc pron.1sg.o} ключ-{\sc pl=3pl} {\sc el} {\sc pron.1sg.o} карман={\sc dat} падать-{\sc pst}//
\glft ‘У меня ключи из кармана \b{выпали}.’//
\endgl 
\a \begingl
\gla Му wих̌ӣӡ-ен=ен зимāδ=ард \b{δод}.//
\glc {\sc pron.1sg.o} ключ-{\sc pl=3pl} земля={\sc dat} упасть.{\sc pst}//
\glft ‘У меня ключи на землю \b{упали}.’//
\endgl \xe

Дополнительным аргументом в пользу такого распределения служит то, что \i{δêдоw} (но не \i{wêх̌тоw}) используется для выражения ситуации, так сказать, контактного падения, когда эксплицируется точка результирующего контакта упавшего объекта и поверхности, на которую он упал, что особенно важно в отношении живых существ, ср.~\i{упал на колени}, \i{на спину}, \i{на бок} и так далее\fn{Обратим внимание, что в контексте (\gethref{exdown70}) можно употребить и глагол \i{гāх̌тоw}, исходно вращения (см.~\hyperref[down-vertical]{Раздел~2.2}), ведь упав, машина перевернулась.}.

\ex<exdown69>
\begingl
\gla Wи кāл=и нêɣ̌-д=ху, ~~~~~~~~~~~~~~~~~~~~~~~~~~~~~~~~~~~~ йу чи дāм / қӣч \b{δод} / *wêх̌-т.//
\glc {\sc d3.m.sg.o} голова={\sc 3sg} кружиться-{\sc pst=and1} ~ {\sc d3.m.sg} {\sc cont} спина ~~~~~~ живот упасть.{\sc pst} ~~~~~~ падать-{\sc pst}//
\glft ‘У него голова закружилась, и он \b{упал} на спину (букв. на живот).’//
\endgl \xe

\ex<exdown70>
\begingl
\gla Йā мошӣн=и пи жӣр=анд ҷук-т=ху, ~~~~~~~~~~~~ чи пāли \b{δод} / *wêх̌-т.//
\glc {\sc d3.f.sg} машина={\sc 3sg} {\sc up} камень={\sc loc} удариться-{\sc pst=and1} ~ {\sc cont} бок упасть.{\sc pst} ~~~~~~ падать-{\sc pst}//
\glft ‘Машина врезалась в камень и \b{упала} на бок.’//
\endgl \xe

\ex<exdown71>
\begingl
\gla Йу лап мот сут=ху, ~~~~~~~~~~~~~~~~~~~~~~~~~~~~~~~~~~~~ чи зу̊н \b{δод} / *wêх̌-т.//
\glc {\sc d3.m.sg} очень уставший стать.{\sc pst.m.sg=and1} ~ {\sc cont} колено упасть.{\sc pst} ~~~~~~ падать-{\sc pst}//
\glft ‘Он очень устал и \b{упал} на колени.’//
\endgl \xe

Таким образом, если мы хотим просто констатировать, что человек или неодушевлённый предмет упал, то в этом случае всегда используется глагол \i{wêх̌тоw}, но если упоминается часть тела или предмета как место его контакта с поверхностью, представляющей нижний предел движения, то нужен глагол \i{δêдоw}.

Ещё одна характерная (и вполне предсказуемая) группа употреблений \i{δêдоw} может быть аналогом русского ‘расшибиться’: ‘упасть вниз с высоты с серьёзными повреждениями’. Соответственно, наиболее естественные примеры экспертов касаются детей и детёнышей — ср.~‘ребёнок упал с крыши’ (из специального отверстия в крыше, это часто бывает в быту шугнанцев ввиду конструктивных особенностей их домов), ‘козлёнок свалился с горы’ и так далее.

\ex<exdown72>
\begingl
\gla Йā ғāц ар ру̊ӡ \b{δод}.//
\glc {\sc d3.f.sg} девочка {\sc down} окно\_в\_крыше упасть.{\sc pst}//
\glft ‘Девочка \b{упала} из отверстия в крыше [дома].’//
\endgl \xe

\ex<exdown73>
\begingl
\gla Йу ворҷ ар дарā \b{δод}.//
\glc {\sc d3.m.sg} конь {\sc down} ущелье упасть.{\sc pst}//
\glft ‘Тот конь \b{сорвался} вниз с обрыва.’//
\endgl \xe

\ex<exdown74>
\begingl
\gla Wи пуц ар тāх \b{δод}.//
\glc {\sc d3.m.sg.o} сын {\sc down} скала упасть.{\sc pst}//
\glft ‘Его сын \b{сорвался} со скалы.’//
\endgl \xe

Среди артефактов ярким аналогом этой ситуации является ‘разбиться’ — о посуде, которая тоже требует глагола \i{δêдоw}, в отличие от падения вниз камня, для которого прототипическим оказывается не \i{δêдоw}, а \i{wêх̌тоw}.

Все описанные ситуации, в которых выбор делается в пользу \i{δêдоw}, а не \i{wêх̌тоw}, характеризует специальное внимание к результирующей ситуации, которая сопровождается:

\begin{itemize}
  \item звуком (‘осадки’)
  \item значимым или наблюдаемым результатом удара-падения (‘расшибся’ / ‘ощутил боль’)
  \item выделенным местом падения (‘в воду’)
  \item или части объекта, которая соприкасается с этим местом (‘[упал] на колени’)
\end{itemize}

Значимость \i{δêдоw} для зоны падения подтверждается тем, что именно этот глагол — в сочетании с «лексическим» компонентом \i{чук} — участвует в образовании сложного глагола падения \i{чук δêдоw}, с достаточно узким и с семантической точки зрения действительно сложным значением падения частей (или целого объекта по частям) ‘(про)валиться, (об)рушиться; упасть внутрь’, см. \parencite[388]{karamshoev1999} Семантика падения частей (проваливание) не покрывается стандартным \i{wêх̌тоw}, так что пересечение с этим глаголом здесь периферийно.

\ex<exdown75>
\begingl
\gla Чӣд дишӣд чук \b{δод}.//
\glc дом потолок обвал упасть.{\sc pst}//
\glft ‘Потолок дома \b{провалился}.’//
\endgl \xe

(Ср.~также: \i{қāбар чук δод} ‘могила (=~традиционная для шугнанцев надгробная плита) провалилась’; \i{пурхи чук δод} ‘кизяк (в очаге) провалился’).

\ex<exdown76>
\begingl
\gla Бе ситан-аθ=та ғиҷӣδ чук \b{δед}.//
\glc без столб-{\sc adv=fut} хлев обвал упасть.{\sc prs.3sg}//
\glft ‘Без столба хлев \b{рухнет}.’ \trailingcitation{\parencite[388]{karamshoev1999}}//
\endgl \xe

По-видимому, именно присутствие вспомогательного \i{δêдоw} в составе \i{чук δêдоw} придает оттенок результативности всей ситуации — за счёт акцента на конечной точке падения.

Заметим, что метафорические употребления \i{δêдоw} также профилируют конечную точку: метафорически \i{δêдоw} приобретает значение неконтролируемого и неожиданного контакта со средой или объектом, ср.~‘попасть’ (\gethref{exdown77}–\gethref{exdown78}) или ‘наткнуться’ (\gethref{exdown79}):

\ex<exdown77>
\begingl
\gla Wӯрҷ пи мол=анд \b{δод}.//
\glc волк {\sc up} стадо={\sc loc} упасть.{\sc pst}//
\glft ‘Волк \b{попал} в стадо.’//
\endgl \xe

\ex<exdown78>
\begingl
\gla Хāт тар му δуст \b{δод}.//
\glc письмо {\sc eq} {\sc pron.1sg.o} рука упасть.{\sc pst}//
\glft ‘Письмо \b{попало} мне в руки.’//
\endgl \xe

\ex<exdown79>
\begingl
\gla Йу пи му-нди \b{δод}.//
\glc {\sc d3.m.sg} {\sc up} {\sc pron.1sg.o-loc} упасть.{\sc pst}//
\glft ‘Он \b{наткнулся} на меня.’//
\endgl \xe

Ср.~также любопытные фразеологизации с этим глаголом: \i{рух δод} ‘наступило утро’ (букв. ‘свет выпал’) или (с полностью противоположным значением):

\ex<exdown80>
\begingl
\gla Хӣр ар ку / абри \b{δод}.//
\glc солнце {\sc down} гора ~~~~~~ туча упасть.{\sc pst}//
\glft ‘Солнце \b{зашло} за гору / тучу.’//
\endgl \xe

Помимо этого, у глагола \i{δêдоw} засвидетельствована хорошо известная метафора ‘впасть в состояние’. Существенно, что в шугнанском она выводится (как того и требует в данном случае последовательный семантический переход) из фрейма, который соответствует глаголу падения (а не удара). Этот фрейм профилирует конечную точку падения, которая легко метафоризуется в конечное состояние субъекта:

\ex<exdown81>
\begingl
\gla тар хӯδм δêд-оw//
\glc {\sc eq} сон упасть.{\sc inf-purp}//
\glft ‘заснуть’//
\endgl \xe

\ex<exdown82>
\begingl
\gla фāнд δêд-оw//
\glc обман упасть.{\sc inf-purp}//
\glft ‘обмануть’//
\endgl \xe

Ср.~также другие подобные сложные глаголы с именной частью: \i{δар} (‘далеко’) \i{δêдоw} ‘удаляться’, буквально ‘падать в даль’; \i{ғалт} (‘кувырок’) \i{δêдоw} ‘кувыркаться’, буквально ‘падать в кувырок’, \i{дāрδ δêдоw} ‘заболеть’ (о каком-то органе или части тела), буквально ‘падать в боль’ и так далее\fn{Интересен случай некомпозициональной метафоры (уже устаревшей) с участием \i{δêдоw}, которая возникает при метафоризации конструкции в целом, а не только одного глагола, ср.:

\ex<exdown84>
\begingl
\gla бӣw=ат δод ар лой//
\glc рыдать={\sc and2} упасть.{\sc pst} {\sc down} грязь//
\glft букв. ‘с рыданием упала в грязь’ — использовалось для выражения состояния, когда, что называется, «на душе погано»//
\endgl \xe}. Интересно, что составной глагол \i{чук δêдоw} тоже метафоризуется, причем по стандартной дейктической модели, когда отрицательные события падают на говорящего-наблюдателя. Для глаголов с семантикой ‘провалиться’, которую передаёт этот глагольный комплекс в целом, такой перенос не типичен. Однако для глагола с акцентом на конечной точке — которым является \i{δêдоw} как «несущий» элемент в этой конструкции — этот сдвиг вполне естественен (подробнее см.~вводную статью к сборнику \parencite{rakhilina_etal2020}):

\ex<exdown83>
\begingl
\gla Бало / қӣни-гар-и му тора ~~~~~~~~~~~~~ чук \b{δод}.//
\glc беда ~~~~~~ трудность-{\sc agn-subst} {\sc pron.1sg.o} вершина.{\sc dat} ~ обвал упасть.{\sc pst}//
\glft ‘Беда обрушилась / трудности \b{обрушились} на меня.’//
\endgl \xe

\section{Итоги} \label{down-conclusion}

Структура поля падения в шугнанском представляет несомненный теоретический интерес. Дело в том, что в нём действительно, как мы и полагали в начале исследования, параллельно с другими предикатами действует достаточно сильный глагол \i{wêх̌тоw} ‘падать’, с широкой семантикой. Такая конфигурация поля достаточно распространена — в частности, так устроено поле падения в русском языке. Однако шугнанское поле обладает гораздо более мощной периферией, чем русское.

В русском (как это часто бывает) наряду с доминантным \i{падать} / \i{упасть} в поле присутствуют узко специальные глаголы типа \i{рухнуть} или \i{опрокинуться}. Их семантика полностью покрывается значением падения и создаёт частные противопоставления внутри поля, привнося идею мгновенного перемещения с отделением всех частей объекта, как \i{рухнуть} / \i{обрушиться} или движения вертикально ориентированного объекта ввиду внезапной потери устойчивости, как \i{опрокинуться}. Важно, что такая система нестабильна и меняется в сторону упрощения: глаголы узкой семантики исчезают. Падение частотности хорошо видно на графике НКРЯ для \i{опрокинуться} или \i{обрушиться}. Сложнее устроен этот график для \i{рухнуть}: до 1940~года — ожидаемое падение, а потом внезапный рост. Рост связан с усилением метафорических контекстов типа: \i{режим} / \i{государство} / \i{экономика} / \i{финансовая система} и особенно — \i{валюта} (ср.~\i{крона рухнула}). Фактически этот глагол тоже постепенно покидает поле падения: из конкретного предиката, который обозначает физическое движение, он постепенно становится абстрактным. Эту метаморфозу претерпел в свое время главный глагол поля — двувидовой \i{пасть}, который остался в русском языке исключительно в виде фиксированных и стилистически маркированных фразем (\i{пал на колени}, \i{пал ниц}) и метафор (\i{низко пал}, \i{пал в бою} и другие). Его заместил глагол \i{падать} и в совершенном виде — приставочный дериват \i{упасть} \parencite{plungian2017}.

На этих примерах видно, как дробная система постепенно стремится к более простой доминантной за счёт редукции «малых» глаголов. В основном мы в своей практике сталкивались именно с такими системами.

В шугнанском тоже есть такие «малые» глаголы — к ним можно отнести \i{нихих̌тоw} ‘разрушаться’, \i{оле ситтоw} ‘падать кубарем’ и \i{чук δêдоw} ‘резко обрушиться’. Они имеют очень узкую сочетаемость и в принципе, видимо, обречены на исчезновение. Этим объясняется их отсутствие в нашем первоначальном материале. Однако в остальном конкуренцию доминантному \i{wêх̌тоw} составляют полноценные частотные лексемы: именно они создают в шугнанском семантическое разнообразие поля падения.

Особенным для шугнанского оказывается устройство зоны падения вертикальных объектов — лексически выделены в нём (и не конкурируют с доминантным \i{wêх̌тоw}) только два довольно периферийных фрейма: полные контейнеры и вывороченные с корнем деревья. (В силу своей периферийности, они вначале и ускользнули от нашего внимания.) В то же время маркером для них выступает глагол \i{гāх̌тоw}, который обозначает в шугнанском все повороты и перевороты, а следовательно, имеет большой круг употреблений далеко за пределами поля падения. Такой предикат, конечно, совершенно не склонен к исчезновению из языка. Теоретически, он мог бы постепенно «захватывать» всё бо́льшую семантическую территорию, «вбирая», например, контексты падения человека и сужая применимость \i{wêх̌тоw}. Такое развитие усиливало бы дистрибутивность системы. В реальном шугнанском ситуация, по-видимому, иная: система упрощается, и в следующем поколении говорящих заметна экспансия и доминантность \i{wêх̌тоw} в зоне вертикально ориентированных объектов.

Ещё одна нетривиальная семантическая область — падение закреплённых объектов, то есть падение как открепление. Её тоже интересно сравнить с русским: и в русском, и в шугнанском такой тип падения маркируется лексикой поля со значением ‘прыгать’. Таким образом, для обоих языков это значение оказывается пограничным полю падения. Однако в русском в зону падения проникает второстепенный для современного языка глагол прыгания \i{отскочить} с корнем \i{скок}- (основным в этом поле является совсем другой глагол — \i{прыгать}), тогда как шугнанское \i{зибидоw} — стандартный предикат для прыжков самого разного свойства. В то же время, в русском доминантный глагол \i{упасть} / \i{падать} практически невозможен ни для какого контекста этой зоны, тогда как в шугнанском \i{wêх̌тоw} повсеместно конкурирует с \i{зибидоw}, за исключением отпадения / открепления частей целого. Понятно, что с точки зрения (исходной) прикреплённости одного объекта к другому это и есть самые центральные контексты, но, по понятным причинам, части и целые отделяются друг от друга гораздо реже других пар, состоящих из временно сопряжённых между собой объектов. Поэтому с точки зрения ситуации падения (от)падающие части как раз являются сугубой периферией (и снова — именно по этой причине могут не попасться на глаза исследователю).

Наконец, последний, самый главный игрок на периферии поля падения — \i{δêдоw} с прототипическим значением ‘падать, ударяться’. Он обслуживает самую идиоматичную зону падения — осадки. Эта зона обычно заполняется какой-то специально предназначенной для неё застывшей метафорой, как в русском — \i{дождь / снег идёт}. Поэтому естественно, что на начальном этапе нам показалось, что и здесь прошла простая метафоризация, ср.~русск.~\i{дождь стучал по крыше}, и мы не считали этот глагол значимым.

На деле, глагол \i{δêдоw} как глагол падения оказался много шире, чем результат метонимического переноса со звука падающих капель на ситуацию выпадения осадков целиком. Он работает и в основной части поля, составляя существенную конкуренцию доминантному \i{wêх̌тоw}, и покрывает очень значимую с семантической точки зрения область результативного падения. Понятно, что акцент на результат падения может быть совмещен с ударом (эта метонимия и есть семантический источник присутствия \i{δêдоw} в «чужом» поле). Однако ситуация в шугнанском такова, что независимо от удара / звука удара при указании результирующей точки падения выбирается \i{δêдоw}, а не \i{wêх̌тоw}. Можно с уверенностью прогнозировать, что \i{δêдоw} не исчезнет как глагол падения в ближайшее время, так что поле удара останется пограничным для поля падения.

Как видим, шугнанская система, о нестабильности которой мы говорили в начале, как и русская, тоже обнаруживает динамику, хотя и несколько иного рода. Она обогатилась почти исключительно за счёт внедрения когнитивно смежных полей: это поля вращения, прыгания и стука / удара. Понятно, что их смежность с падением семантически нагружена, и так или иначе будет проявляться и в других языках, но обнаружить её «лексические следы» можно, только зная заранее о её существовании и о том, в каких точно контекстах её нужно искать. Так, в русском мы действительно говорим \i{дождь / град стучит} (а не: *\i{падает}), \i{деревья выворачивает с корнем от сильного ветра} (а не: *\i{деревья падают с корнем от сильного ветра}) или: \i{ножка стула отскочила} (а не: *\i{упала}). Поиск аналогов в других языках был бы здесь очень показателен.